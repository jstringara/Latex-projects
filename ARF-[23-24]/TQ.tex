\documentclass[a4paper]{report} %document class and format


%>=====< Language  and bibliography packages >=====<%

\usepackage[T1]{fontenc} % font encoding
\usepackage[utf8]{inputenc} % input encoding
\usepackage[style=numeric-comp,useprefix,hyperref,backend=bibtex]{biblatex}

%>=====< Math packages >=====<%

\usepackage{amsmath, amssymb, amsthm, mathrsfs, mathtools}
\usepackage{centernot} %for centered negation
\usepackage[makeroom]{cancel} %to cancel out stuff

%>=====< Other packages >=====<%

\usepackage{framed} % for framing text
\usepackage{titling} % for customizing title
\usepackage{titlesec} %for customizing sections title
\usepackage[ top=2cm, bottom=2cm, left=2cm, right=2cm]{geometry} % for smaller margins
\usepackage{xifthen} %for conditional statements
\usepackage{enumerate} %to modify lists
\usepackage{hyperref} %to insert hyperlinks
%\usepackage[open]{bookmark} %to have bookmarks open by default
\usepackage{titletoc} %to modify the toc
\usepackage{tikz}
\usetikzlibrary{tikzmark}

%>=====< Chapter >=====<% 

%redefine chapters
\titleformat{\chapter}[display]{\normalfont\Huge\bfseries} % format
{Sheet n. \thechapter} % label
{0ex} % sep
{ \vspace{1ex} \centering } % before-code
[ \vspace{-2.5ex} ] % after-code

%modify chapter in toc
\titlecontents{chapter}[0em]{\smallskip\bfseries}%\vspace{1cm}%
{}%
{\itshape\bfseries}%numberless%
{\hfill\contentspage}[\medskip]%

%new sheet command for toc
\newcommand{\sheet}{\chapter[Sheet n.\thechapter]{}}

%>=====< Section >=====<%

%redefine section
\titleformat{\section}[display]{\normalfont\huge\bfseries} % format
{Question \thesection} % label
{0ex} % sep
{ \vspace{0ex} } % before-code
[ \vspace{-1.5ex} ] % after-code

%modify chapter in toc
\titlecontents{section}[1.5em]{\smallskip\bfseries}%\vspace{1cm}%
{}%
{\itshape\bfseries}%numberless%
{\hfill\contentspage}[\medskip]%

%new question command for toc
\newcommand{\question}{\section[Question \thesection]{}}

%>=====< Subsection >=====<%

%redefine subsection
\titleformat{\subsection}[hang]{\normalfont\large\bfseries} % format
{} % label
{0pt} % sep
{} % before-code
[ \vspace{0.5ex} ] % after-code

%redefine subsection*
\titleformat{name=\subsection, numberless}[hang]{\normalfont\large\bfseries} % format
{} % label
{0pt} % sep
{} % before-code
[ \vspace{0.5ex} ] % after-code

%>=====< Subsubsection >=====<%

%redefine subsubsection
\titleformat{\subsubsection}[hang]{\normalfont\large\bfseries} % format
{} % label
{0pt} % sep
{} % before-code
[ \vspace{0.5ex} ] % after-code

%make it appear in toc
\setcounter{tocdepth}{4}
\setcounter{secnumdepth}{4}

%>=====< Math sets >=====<%

\newcommand{\N}{\mathbb{N}} %natural
\newcommand{\R}{\mathbb{R}} %real
\newcommand{\Rpos}{\R_{+}} %positive real
\newcommand{\Rcomp}{\overline{\R}} %complete real
\newcommand{\Rcomppos}{\Rcomp_{+}} %complete positive real
\newcommand{\Q}{\mathbb{Q}} %rational
\newcommand{\A}{\mathcal{A}} %calligraphic A
\newcommand{\B}[1][]{
  \ifthenelse{\isempty{#1}}{\mathcal{B}}{\mathcal{B}\left(#1\right)}
}%Borel sigma-algebra
\newcommand{\Parts}[1]{\mathcal{P}\left(#1\right)} %set of all parts
\renewcommand{\complement}{\mathsf{c}} %complement
\newcommand{\Leb}{\mathcal{L}} %Lebesgue sigma-algebra
\newcommand{\Mes}{\mathcal{M}} %measurable functions
\newcommand{\Smes}{\mathcal{S}} %measurable simple functions

%>=====< Math symbols >=====<%

\renewcommand{\epsilon}{\varepsilon} %epsilon
\renewcommand{\theta}{\vartheta} %theta
\renewcommand{\phi}{\varphi} %phi
\DeclareMathOperator*{\esssup}{ess\,sup} %esssup
\DeclareMathOperator*{\essinf}{ess\,inf} %essinf
\DeclareMathOperator*{\interior}{int} %interior 
\DeclareMathOperator*{\dist}{dist} %distance 
\DeclareMathOperator{\spn}{span} %span
\DeclareMathOperator{\supp}{supp} %support

%>=====< Math jargon >=====<%

\newcommand{\salg}{\sigma-\text{algebra}} %sigma-algebra
\newcommand{\s}[1]{\sigma-\text{#1}}%sigma-smth
\newcommand{\provdef}[1][]{
  \ifthenelse{\isempty{#1}}{Let us define:}{Let us define #1 as:}
} %smart definition
%define #1
\newcommand{\provdefs}{Let us define the following:} %multiple definitions
\newcommand{\seq}[1]{ \{ #1_n \} } %sequence
\newcommand{\subseq}[1]{ \{ #1_{n_k} \} } %subsequence
\newcommand{\tfae}{the following are equal:}  %the following are equal
\newcommand{\toin}[1]{\xrightarrow[n\to\infty]{#1} } %to in 
\newcommand*{\cvrule}{\smash{\rule{0.4pt}{2ex}}}

%>=====< Math commands added for compatibility >=====<%

%\newcommand{\proj}{\mbox{Proj_{H}}} this was the original, I changed it for consistency and good practice
\DeclareMathOperator*{\proj}{Proj} %projection
\newcommand{\wc}{\rightharpoonup}
\newcommand{\wsc}{\overset{\ast}{\rightharpoonup}}
\newcommand{\toi}{n\to\infty}

%>=====< Custom commands >=====<%

\newcommand{\cancelnum}[2]{\overset{\text{\normalsize#1}}{\xcancel{#2}}}

%>=====< Document >=====<%

\begin{document}
%==== Title ====% see https://www.overleaf.com/learn/latex/How_to_Write_a_Thesis_in_LaTeX_(Part_5)%3A_Customising_Your_Title_Page_and_Abstract for reference
\begin{titlepage}
    \begin{center}
        \vspace*{\fill}
        \Huge
        Answers to the Theory Questions\\
        \vspace{0.5em}
        \Large
        of the course of Real and Functional Analysis\\
        taught by prof. Fabio Punzo \\
        \vspace{0.5em}
        by Jacopo Stringara \\
        \vspace{0.5em}
        \href{mailto:jstringara@gmail.com}{jstringara@gmail.com} \\
        \vspace{0.5em}
        \today
        \vfill
        The source code for this document (and many more) can be found at:\\
        \href{https://github.com/jstringara/Latex-projects/tree/master/ARF}{https://github.com/jstringara/Latex-projects/tree/master/ARF}
    \end{center}
\end{titlepage}


%>=====< Table of contents >=====<%

\tableofcontents
\newpage

%>=====< Sheets >=====<%

\sheet

% include the appropriate files
%>=====< Question 1.1 >=====<%

\question
Write the definitions of: sequence of sets $\seq{E}$; increasing and decreasing sequence of sets $\seq{E}$;
$\limsup_{n\to\infty} E_n$, $\liminf_{n\to\infty} E_n$, $\lim_{n\to\infty} E_n$.

\subsection*{Solution}

\provdefs
\begin{itemize}
    \item \subsection{Sequence of sets} A family (or collection) of sets $\{E_i\}_{i\in I}$ is called a sequence of sets if $I=\N$ (i.e. it is indexed by the set of natural numbers $\N$)
    \item \subsection{Increasing sequence of sets} a sequence of sets $\seq{E}$ is said to be increasing (or ascending) if:
    \[
        E_n \subseteq E_{n+1} \quad \forall n\in\N    
    \]
    \item \subsection{Decreasing sequence of sets}
    A sequence of sets $\seq{E}$ is said to be decreasing (or descending) if:
    \[
        E_n \supseteq E_{n+1} \quad \forall n\in\N    
    \]
    \item \subsection{Limsup for a sequence of sets} for a sequence of sets $\seq{E}$ we define:
    \[
        \limsup_{n\to\infty} E_n \coloneqq \bigcap_{k=1}^{\infty} \bigcup_{n=k}^\infty E_n
    \]
    \item \subsection{Liminf for a sequence of sets} analogously:
    \[
        \liminf_{n\to\infty} E_n \coloneqq \bigcup_{k=1}^{\infty} \bigcap_{n=k}^\infty E_n
    \]
    \item \subsection{Limit for a sequence of sets} as for a sequence of real numbers if the limsup and liminf coincide we may define:
    \[
        \lim_{n\to\infty} E_n \coloneqq \liminf_{n\to\infty} E_n = \limsup_{n\to\infty} E_n    
    \]
\end{itemize}
%>=====< Question 1.2 >=====<%

\question
Write the definitions of: cover (or covering) of a set; subcover.

\subsection*{Solution}
\provdefs
\begin{itemize}
    \item \subsection{Cover of a set} a family of sets $\{E_i\}_{i\in I}$ is called a cover (or covering) of $X$ if:
    \[
        X \subseteq \bigcup_{i\in I} E_i    
    \]
    \item \subsection{Subcover} a sub-family of a cover $\{E_i\}_{i\in J}$ ($J\subseteq I$) which forms a cover is called a subcover.  
\end{itemize}
%>=====< Question 1.3 >=====<%

\question
Write the definitions of: equivalence relation, equivalence class, quotient set.

\subsection*{Solution}
\provdefs
\begin{itemize}
    \item \subsection{Equivalence relation} \label{equivrel} a relation $R$ in $X$ (i.e. a subset $R\subseteq X\times X$) is an equivalence relation if:
    \begin{enumerate}[i)]
        \item $(x,x) \in R$ $\forall x\in X$ (\textbf{reflexivity})
        \item $(x,y) \in R \implies (y,x)\in R$ (\textbf{simmetry})
        \item $(x,y) \in R, \, (y,z)\in R \implies (x,z)\in R$ (\textbf{transitivity})
    \end{enumerate}
    \item \subsection{Equivalence class} we define an equivalence class for $x$ w.r.t. $R$ as:
    \[
        E_x \coloneqq \{y\in X : yRx\}
    \]
    i.e. the set of all elements equivalent to $x$ for $R$
    \item \subsection{Quotient set} we define the quotient set of $X$ over $R$ as:
    \[
        X / R \coloneqq \{E_x: x\in X \}    
    \]
    i.e. it is the set of all equivalence classes.
\end{itemize}
%>=====< Question 1.4 >=====<%

\question
Write the definitions of: infinite set, finite set, countable set, uncountable set. Provide examples.

\subsection*{Solution}
\provdefs
\begin{itemize}
    \item \subsection{Finite sets} a set $X$ is finite if $\exists n\in\N$ such that there is a bijection:
    \[
        f:X\to {1,\dots,n}    
    \]
    \textbf{Example:} $\{\frac{1}{1}, \dots, \frac{1}{n}\}$
    \item \subsection{Infinite sets} $X$ is infinite if it is not finite.\\
    \textbf{Example:} $\N$ is clearly infinite
    \item \subsection{Countable sets} $X$ is countable if $X$ is equipotent to $\N$\\
    \textbf{Example:} $\Q$ can be put in bijection with $\N$
    \item \subsection{Uncountable sets} $X$ is uncountable if it is infinite and not countable.\\
    \textbf{Example:} $\R$ is clearly infinite and not countable since it has the cardinality of continuum.
\end{itemize}
%>=====< Question 1.5 >=====<%

\question
Write the definitions of: algebra, $\salg$, measurable space, measurable set. Show that if $\A$
is a $\salg$ and $\{E_k\}\subset \A$, then $\bigcap^{+\infty}_{k=1} E_k \in \A $.

\subsection*{Solution}
\provdefs
\begin{itemize}
    \item \subsection{Algebra} A family $\A\subseteq\Parts{X}$ is an algebra if:
    \begin{enumerate}[i)]
        \item $\emptyset \in \A$
        \item $E\in\A \implies E^\complement \in\A$
        \item $A,B\in\A \implies A\cup B\in\A$
    \end{enumerate}
    \item \subsection{\texorpdfstring{$\salg$}{Sigma-algebra}} A family $\A\subseteq\Parts{X}$ is a $\salg$ if:
    \begin{enumerate}[i)]
        \item $\emptyset \in \A$
        \item $E\in\A \implies E^\complement \in\A$
        \item $\seq{E}_{n\in\N} \subseteq\A \implies \bigcup_{n=1}^\infty E_n \in\A$
    \end{enumerate}
    \item \subsection{Measurable space} The couplet $(X,\A)$ where $\A$ is a $\salg$ is called a measurable space.
    \item \subsection{Measurable set} the elements of the $\salg$ of a measurable space are called measurable sets.
\end{itemize}

Clearly $\bigcap^{+\infty}_{k=1} E_k = \left(\bigcup^{+\infty}_{k=1} E_k^\complement\right)^\complement$
and since $\A$ is a $\salg$ we have that $\bigcup^{+\infty}_{k=1} E_k^\complement \in \A$ and thus $\bigcap^{+\infty}_{k=1} E_k \in \A$.

%>=====< Question 1.6 >=====<%

\question
State the theorem concerning the existence of the $\salg$ generated by a given set. Give an
idea of the proof.

\subsection*{Solution}

\subsection{Minimal \texorpdfstring{$\salg$}{sigma-algebra}}
Let $S\subseteq\Parts{X}$, then there exists a $\salg$ $\sigma_0(S)$ such that:\\
\begin{enumerate}
    \item $S\subseteq \sigma_0(S)$
    \item $\forall \salg \A\subseteq\Parts{X}$ such that $S\subseteq\A$ we have $\sigma_0(S) \subseteq\A$
\end{enumerate}
thus $\sigma_0(S)$ is the minimal $\salg$ generated by $S$.\\

\subsection*{Sketch of Proof}
We construct the set:
\[
    \mathcal{V} \coloneqq \{\A \subseteq \Parts{X} \| \A \supseteq S, \; \A \quad \salg \}    
\]
we may define:
\[
    \sigma_0(S)\coloneqq \bigcap \{\A: \: \A\in\mathcal{V}\}    
\]

%>=====< Question 8 >=====<%

\question
Write the definition of the Borel $\salg$ in a metric space. Provide classes of Borel sets.
Characterize $\B[\R], \B[\Rcomp]$ and $\B[\R^N]$.

\subsection*{Solution}

\subsection{Borel \texorpdfstring{$\salg$}{sigma-algebra}}
Let $(X,d)$ be a metric space and let $\mathcal{G}$ be the family of open sets of $X$, then we define the Borel $\salg$ as:
\[
    \B[X] \coloneqq \sigma_0(\mathcal{G})    
\]
The elements of $\mathcal{G}$ are called Borel sets,   let us enumerate some classes of them:

\subsection{Classes of Borel sets} 
\begin{enumerate}[i)]
    \item open sets
    \item closed sets (they are the complementary of open sets and this is a $\salg$)
    \item countable intersections of open sets, known as the family $G_\delta$
    \item countable union of closed sets, known as the family $F_\delta$.
\end{enumerate} 
Lastly, let us characterize the Borel $\salg$s $\B[\R],\B[\Rcomp]$ and $\B[\R^N]$:
\subsection{Characterization of \texorpdfstring{$\B[\R],\B[\Rcomp]$}{B(R), B(Rcomp)} and \texorpdfstring{$\B[\R^N]$}{B(Rn)}}
\begin{enumerate}
    \item $\B[\R]=\sigma_0(I)=\sigma_0(I_1)=\sigma_0(I_2)=\sigma_0(I_0)=\sigma_0(\hat{I})$\\
    where:
    \begin{align*}
        I&=\{ (a,b): a,b \in \R, a\leq b \} \\
        I_1&=\{ [a,b]: a,b \in \R, a\leq b \} \\
        I_2&=\{ (a,b]: a,b \in \R, a\leq b \} \\
        I_0&=\{ (a,b): -\infty\leq a < b <\infty \} \cup \{ (a,\infty): a\in\R \} \\
        \hat{I}&=\{ (a,\infty): a\in\R \}
    \end{align*}
    \item $\B[\Rcomp]=\sigma_0(\tilde{I})=\sigma_0(\tilde{I}_1)$\\
    where:
    \begin{align*}
        \tilde{I} &= \{ (a,b): a,b \in \R, a < b \} \cup \{ [-\infty,b): b\in\R \} \cup \{ (a,+\infty] : a\in\R \} \\
        \tilde{I}_1 &= \{ (a,+\infty] : a\in\R \}
    \end{align*}
    \item $\B[\R^N] = \sigma_0(K_1)=\sigma_0(K_2)$\\
    where:
    \begin{align*}
        K_1 &= \{ \text{n-dimensional closed rectangles}\}\\
        K_2 &= \{ \text{n-dimensional open rectangles }\}
    \end{align*}
\end{enumerate}

%>=====< Question 9 >=====<%

\question
Write the definitions of: measure, finite measure, $\s{finite measure}$, measure space, probability
space. Provide some examples of measures.

\subsection*{Solution}
\provdefs
\begin{itemize}
    \item \subsection{Measure} Let $X$ be a set and $\mathcal{C}\subseteq\Parts{X}$, then a function $\mu$:
    \[
        \mu:\mathcal{C}\to\Rcomppos    
    \]
    is a measure if:
    \begin{enumerate}
        \item $\mu(\emptyset)=0$
        \item \textbf{$\s{additivity}$:}\\
        $\forall \seq{E}\subseteq\mathcal{C}$ disjoint ($E_i\cap E_j\quad \forall i\neq j$) such that $\bigcup_{k=1}^{\infty} E_k \in \mathcal{C}$ we have that:
        \[
            \mu \left( \bigcup_{k=1}^{\infty} E_k \right) = \sum_{k=1}^{\infty} \mu(E_k) 
        \]
    \end{enumerate}
    \item \subsection{Finite measure} a measure $\mu$ defined as above is said to be finite if:
    \[
        \mu(X) < + \infty    
    \]
    \item \subsection{\texorpdfstring{$\s{finite measure}$}{Sigma-finite measure}} a measure $\mu$ is said to be $\s{finite}$ if there exists a sequence $\seq{E}$ such that:
    \[
        X = \bigcup_{k=1}^\infty E_k, \quad \mu(E_k) <+\infty
    \]
    \item \subsection{Measure space} Let $\A\subseteq\Parts{X}$ be a $\salg$ and $\mu:\A\to\Rcomppos$ a measure, then the triplet $(X,\A,\mu)$ is called  a measure space.
    \item \subsection{Probability space} if $\mu(X)=1$ then we say that $(X,\A,\mu)$ is a probability space.
\end{itemize}

%>=====<  Question 10 >=====<%

\question
State and prove the theorem regarding properties of measures. Why the two continuity properties
are called in this way? For what concerns continuity w.r.t. a descending sequence ${E_k}$, show that
the hypothesis $\mu(E_1) < +\infty$ is essential.

\subsection*{Solution}
\subsection{Properties of measures}
Let us state and prove the properties of a measure $\mu$ on a set $X$ and $\salg$ $\A$:
\begin{enumerate}[i)]
    \item \label{meas:add} \textbf{Additivity:} \\
    $\forall \{ E_1, \dots, E_n \} \subseteq \A$ disjoint we have:
    \[
        \mu\left( \bigcup_{k=1}^{n} E_k \right) = \sum_{k=1}^{n} \mu(E_k)
    \]
    \begin{proof}
    indeed if we define a sequence such that:
    \[
        \seq{E} = \left\{ \begin{array}{l}
            B_k = E_k \quad \forall k \leq n\\
            B_k = \emptyset \quad \forall k > n
        \end{array} \right.
    \]
    this sequence is also disjoint ($E\cap\emptyset = \emptyset$ $\forall E\in X$), thus we may write:
    \[
        \mu \underbrace{ \left( \bigcup_{k=1}^{\infty} E_k \right) }_{=\bigcup_{k=1}^{n} E_k \cup \emptyset}
         = \sum_{k=1}^{\infty} \mu(E_k) = \sum_{k=1}^{n} \mu(E_k) + \sum_{k=n+1}^{\infty} \underbrace{\cancelnum{}{\mu(E_k)}}_{=0}
    \]
    \end{proof}

    \item \label{meas:mono} \textbf{Monotonicity:} \\
    $\forall E,F \in \A$ we have:
    \[
        E\subseteq F \implies \mu(E) \leq \mu(F)    
    \]
    \begin{proof}
    We may write $F$ in the following way:
    \[
        F=E\cup (F \setminus E) 
    \]
    and since these two sets are obviously disjoint we may use (\ref{meas:add}) to write:
    \[
        \mu(F) = \mu(E)+\underbrace{\mu(E \setminus F)}_{\geq 0} > \mu(E)
    \]
    \end{proof}
    
    \item \label{meas:sub} \textbf{$\s{subadditivity}$:} \\
    $\forall \seq{E}\subseteq\A$ (\textbf{not} disjoint) we have:
    \[
        \mu\left( \bigcup_{k=1}^{\infty} E_k \right) \leq \sum_{k=1}^{\infty} \mu(E_k)
    \]
    \begin{proof}
    \provdef{}
    \[
        \left\{ \begin{array}{l}
            F_1 \coloneqq E_1 \\
            F_n \coloneqq E_n \setminus \bigcup_{k=1}^{n-1} E_k \quad \forall n > 1
        \end{array} \right.  
    \]
    Clearly $\seq{F} \subseteq \A$ and $\seq{F}$ is a disjoint sequence and:
    \[
        \begin{array}{l}
            F_k \subseteq E_k \quad \forall k\in\N \implies \mu(F_k) \leq \mu(E_k) \text{ by (\ref{meas:mono})} \\
            \bigcup_{k=1}^{\infty} F_k = \bigcup_{k=1}^{\infty} E_k    
        \end{array}
    \]
    thus we may write:
    \[
        \mu\left( \bigcup_{k=1}^{\infty} E_k \right) = \mu\left( \bigcup_{k=1}^{\infty} F_k \right) = \sum_{k=1}^{\infty} \mu(F_k) \leq \sum_{k=1}^{\infty} \mu(E_k)   
    \]
    \end{proof}

    \item \label{meas:contbel} \textbf{Continuity from below:} \\
    $\forall \seq{E}\subseteq\A, \; E_k \nearrow$ we have:
    \[
        \mu \left( \lim_{k\to\infty}E_k \right)  = \lim_{k\to\infty} \mu(E_k)    
    \]
    \begin{proof}
    \provdef[a new sequence $\seq{F}$]
    \[
        \left\{ \begin{array}{l}
            F_k \coloneqq E_k \setminus E_{k-1} \quad \forall k \in\N \text{ and } E_0 \coloneqq \emptyset \\
            \implies \bigcup_{k=1}^n F_k = E_n ,\; \bigcup_{k=1}^{\infty} F_k = \bigcup_{k=1}^{\infty} E_k
        \end{array} \right.
    \]
    and since $\seq{F}$ is a disjoint sequence (we may visually think of it as a set of ever increasing rings) we may use (\ref{meas:add}) to write:
    \begin{align*}
        \mu\left( \lim_{n\to\infty} E_n \right) &\tikzmarknode{eq1}{=} \mu\left( \bigcup_{k=1}^{\infty} E_k \right) = \mu\left( \bigcup_{k=1}^{\infty} F_k \right) \\
        & \tikzmarknode{eq2}{=} \lim_{n\to\infty}\sum_{k=1}^{n} \mu(F_k) = \lim_{n\to\infty} \mu \left( \bigcup_{k=1}^{n} F_k \right) \\
        & \tikzmarknode{eq3}{=} \lim_{n\to\infty} \mu(E_n)
    \end{align*} \tikz[overlay,remember picture]{\draw[shorten >=1pt,shorten <=1pt] (eq1) -- (eq2) -- (eq3);}
    \end{proof}

    \item \label{meas:contab} \textbf{Continuity from above:} \\
    $\forall \seq{E} \subseteq \A$, $E_k\searrow$, $\mu(E_1) < +\infty$ we have:
    \[
        \mu\left( \lim_{n\to\infty} E_n \right) = \lim_{n\to\infty} \mu(E_n)
    \]
    \begin{proof}
    Like we did above \provdef{a new sequence $\seq{F}$}
    \[
            F_k \coloneqq E_1 \setminus E_k \quad \forall k \in\N
    \]
    let us note that $\seq{F}$ is an increasing sequence thus by (\ref{meas:contbel}) we can write:
    \[
        \mu\left( \bigcup_{k=1}^{\infty} F_k \right) = \lim_{k\to\infty} \mu(F_k) = \mu(E_1) - \lim_{k\to\infty}  (E_k)   
    \]
    because by (\ref*{meas:mono}) $\mu(F \setminus E) = \mu(F)-\mu(E)$, moreover:
    \begin{align*}
        \bigcup_{k=1}^{\infty} F_k & = \bigcup_{k=1}^{\infty} (E_1\cap E_k^\complement) = E_1\cap \left( \bigcup_{k=1}^{\infty} E_k^\complement \right) = E_1 \setminus \left( \bigcap_{k=1}^{\infty} E_k \right) \\
        &\implies \mu\left( \bigcup_{k=1}^{\infty} F_k \right) = \mu(E_1) - \mu \left( \bigcap_{k=1}^{\infty} E_k \right)
    \end{align*}
    thus combining these two and canceling the $\mu(E_1)$ on both sides we obtain:
    \[
        \lim_{k\to\infty} \mu(E_k) = \mu \left( \bigcap_{k=1}^{\infty} E_k \right)
    \]
    \end{proof}
    let us note that for this last, crucial, step $\mu(E_1)$ must be finite, otherwise we would not be able to cancel it out from both sides.
\end{enumerate}

\sheet

% include the appropriate files
%>=====<  Question 11 >=====<%

\question
Write the definitions of: sets of zero measure; negligible sets. What is meant by saying that a property holds a.e.? Provide typical properties that can be true a.e. .

\subsection*{Solution}

\provdefs
\begin{itemize}
    \item \subsection{Sets of zero measure} 
    Given a measure space $(X,\A,\mu)$, we say that a set $E\subseteq X$ has zero measure if $E\in\A$ and $\mu(E)=0$. We denote the set of all sets of zero measure by $\mathcal{N}_{\mu}$
    \item \subsection{Negligible sets} a set $E\subseteq X$ is negligible if:
    \[
        \exists N\in\A \text{ s.t. } E \subseteq N, \; \mu(N)=0
    \]
    So any subset of a set of zero measure is negligible, we denote the collection of all negligible sets by $\tau_{\mu}$. Moreover let us note that $E$ doesn't need to be an element of $\A$ ($E\notin\A$)
    \item \subsection{Almost Everywhere} a property $P$ on $X$ is said to hold almost everywhere if:
    \[
       \mu( \{ x\in X: P(x) \text{ is false } \} ) = 0    
    \]
    We may also say that $\{ x\in X: P(x) \text{ is false } \} \in \mathcal{N}_{\mu}$\\
    \subsection*{Examples} typical properties that can be true a.e. are: equality, continuity, monotonicity, etc. etc.
\end{itemize}
%>=====<  Question 12 >=====<%

\question
Write the definition of complete measure space. Exhibit an example of a measure space which is not complete.

\subsection*{Solution}
\subsection{Complete measure space}
A measure space $(X,\A,\mu)$ is said to be complete if $\tau_{\mu}\subseteq\A$
\subsubsection{Counterexample}
Let $X=\{a,b,c\}$, $\A = \sigma (\{\emptyset, \{a\}, \{b,c\}, X \})$ and $\mu\equiv 0$, clearly here we have:
\[
    \tau_{\mu} \setminus \mathcal{N}_{\mu}= \{ \{b\}, \{c\} \}    
\]
and clearly $\{b\}, \{c\} \notin \A$. So this measure space is not complete.
%>=====< Question 1 >=====<%

\question
Write the definition of complete measure space. State the theorem concerning the existence of the completion of a measure space. Give just an idea of the proof.

\subsection*{Solution}

\subsection{Complete measure space}
A measure space $(X,\A,\mu)$ is said to be complete if $\tau_{\mu}\subseteq\A$

\subsection{Existence of the completion}
Let $(X,\A,\mu)$ be a measure space. \provdef{$\bar{\A}, \bar{\mu}$}
\begin{align*}
    \bar{\A}  & =\{ E\subseteq X: \exists F,G \in \A \text{ s.t. } F\subseteq E \subseteq G \; \mu(G \setminus F) =0 \} \\
    \bar{\mu} & : \bar{\A}\to\Rcomppos,\quad \bar{\mu}(E) \coloneqq \mu(F)
\end{align*}
then:
\begin{enumerate}
    \item $\bar{\A}$ is a $\salg$ , $\bar{\A} \supseteq \A$
    \item $\bar{\mu}$ is a complete measure, $\bar{\mu}|_{\A}=\mu$
\end{enumerate}
and the triplet $(X,\bar{A}, \bar{\mu})$ is a complete measure space and is called the completion of $(X,\A,\mu)$, i.e. it the smallest (w.r. to inclusion) complete measure space that cointains $(X,\A,\mu)$
\subsection*{Sketch of proof} \footnote{
    This is a partial proof of my own making. It has been review by the TA and professor Punzo and stated to be correct.
}
We  must prove two things:
\begin{itemize}
    \item \textbf{First:} that $\bar{\A}$ is a $\salg$ and that it contains $\A$, the latter is trivial since $\forall A\in\A \quad A\subseteq A\subseteq A \implies A\in\bar{\A}$ while the former is quite hardous so we shall just assume it to be true.
    \item \textbf{Second:} that $\bar{\mu}$ is a complete measure and $\bar{\mu}|_{\A}=\mu$.\\
          The latter is trivial (see above). We can also easily prove that it is a measure:
          \begin{enumerate}[i)]
              \item $\bar{\mu}(\emptyset)=\mu(\emptyset)=0$ since the only set contained inside $\emptyset$ is $\emptyset$ itself, as the container set we may take any zero set measure inside $\A$.
              \item that $\s{additivity}$ holds is clear since for any disjoint sequence $\seq{E}\subseteq\bar{\A}$ we may construct two sequences:
                    \[
                        \left\{ \begin{array}{l}
                            \seq{F}, \; F_k \subseteq E_k \\
                            \seq{G}, \; G_k \supseteq E_k
                        \end{array} \right. \forall k\in\N \text{ s.t. } \mu(G_k\setminus F_k) = 0
                    \]
                    Let us note the following:
                    \begin{itemize}
                        \item $\seq{F}$ is also disjoint because $\seq{E}$ is disjoint.
                        \item Moreover:
                              \begin{align*}
                                   & \bigcup_{k=1}^{\infty} F_k \subseteq \bigcup_{k=1}^{\infty} E_k \subseteq \bigcup_{k=1}^{\infty} G_k                                                                                                 \\
                                   & \bigcup_{k=1}^{\infty} G_k \setminus \bigcup_{k=1}^{\infty} F_k \subseteq \bigcup_{k=1}^{\infty} (G_k \setminus F_k )                                                                                \\
                                   & \mu\left(\bigcup_{k=1}^{\infty} G_k \setminus \bigcup_{k=1}^{\infty} F_k \right) \leq \mu\left(\bigcup_{k=1}^{\infty} (G_k \setminus F_k )\right) \leq \sum_{k=1}^{\infty} \mu(G_k\setminus F_k) = 0
                              \end{align*}
                              The last inequality is true thanks to the $\s{subadditivity}$ and monotonicty of $\mu$.
                    \end{itemize}
                    Thus we can say that:
                    \[
                        \bar{\mu}\left( \bigcup_{k=1}^{\infty} E_k \right) = \mu \left( \bigcup_{k=1}^{\infty} F_k \right) = \sum_{k=1}^{\infty} \mu(F_k) = \sum_{k=1}^{\infty} \bar{\mu}(E_k)
                    \]
          \end{enumerate}
          thus $\bar{\mu}$ is a measure.\\
          Let us prove that $\bar{\mu}$ is complete.
          Let $E_1 \in X$ and $E_2 \in \bar{\A}$ such that $\bar{\mu}(E_2)=\mu(F_2)=0$ and $E_1 \subseteq E_2$, let us note that:
          \[
              \left\{ \begin{array}{l}
                  \mu(G_2) = \cancelnum{0}{\mu(G_2\setminus F_2)} + \cancelnum{0}{\mu(F_2)}\\
                  \mu(G_2 \setminus \emptyset) = \mu(G_2) - 0                       \\
                  \emptyset \subseteq E_1 \subseteq G_2
              \end{array} \right. \implies E_1\in \bar{\A}, \; \bar{\mu}(E_1)=\mu(\emptyset) = 0
          \]
          thus any negligible set is also a zero measure set and $\bar{\mu}$ is complete.
\end{itemize}

%>=====< Question 2 >=====<%

\question
Write the definition of outer measure. State and prove the theorem concerning generation of
outer measure on a general set $X$, starting from a set $K \in\Parts{X}$, containing $\emptyset$, and a function
$\nu : K \to \Rcomppos, \; \nu(\emptyset) = 0$. Intuitively, which is the meaning of $(K, \nu)$?

\subsection*{Solution}

\subsection{Outer measure}\label{outer:def}
We say that a function: $\mu^*:\Parts{X}\to\Rcomppos$ (where $X$ is any set) is an outer measure if:
\begin{enumerate}[i)]
    \item $\mu^*(\emptyset)=0$
    \item \label{outer:mono}$E_1\subseteq E_2 \implies \mu^*(E_1) \leq \mu^*(E_2)$
    \item \label{outer:sub}$\mu^*\left( \bigcup_{k=1}^{\infty} E_k \right) \leq \sum_{k=1}^{\infty} \mu^*(E_k)$
\end{enumerate}

\subsection{Generation of an outer measure} \label{outer:gen}
Let $K\subseteq\Parts{X}, \, \emptyset\in K, \: \nu:K\to\Rcomppos, \; \nu(\emptyset)=0$, then we can generate an outer measure $\mu^*$ on $X$ defined as:
\[
    \left\{ \begin{array}{l}
        \mu^*(E) \coloneqq \inf \left\{ \sum_{k=1}^{\infty} \nu(I_k) : E\subseteq \bigcup_{k=1}^{\infty} I_k,\; \seq{I}\subseteq K \right\} , \text{ if } E \text{ can be covered by a countable union of sets } I_n\in K. \\
        \mu^*(E) \coloneqq +\infty, \text{ otherwise.}
    \end{array} \right.
\]

\begin{proof}
    Let us verify that such a $\mu^*$ meets the definition of outer measure (\ref{outer:def}):
    \begin{enumerate}[i)]
        \item $\emptyset\in K$, $0\leq\mu^*(\emptyset)\leq\nu(\emptyset)=0$ by the definition of $\mu^*$.
        \item $E_1\subseteq E_2$, we have two possible cases
              \begin{itemize}
                  \item if there exists a countable covering of $E_2$ then it is also a covering of $E_1$ and from the definitio of $\mu^*$ it follows that:
                        \[
                            \mu^*(E_1) \leq \mu^*(E_2)
                        \]
                  \item if there is no countable covering of $E_2$ then:
                        \[
                            \mu^*(E_1) \leq \mu^*(E_2) = +\infty
                        \]
              \end{itemize}
        \item this condition is obviously met if:
              \[
                  \sum_{k=1}^{\infty} \mu^*(E_k) = +\infty
              \]
              otherwise if we suppose that:
              \[
                  \sum_{k=1}^{\infty} \mu^*(E_k) < +\infty
              \]
              thus $\mu^*(E_k)<+\infty$ $\forall k\in\N$, by the definition of $\mu^*$ and $\inf$:
              \[
                  \forall \epsilon>0, \; \forall n\in\N \quad \exists \{ I_{n,k} \} \subseteq K
              \]
              such that:
              \[
                  E_n \subseteq \bigcup_{k=1}^{\infty} I_{n,k} \quad \text{ and } \quad \mu^*(E_n)+\frac{\epsilon}{2^n} > \sum_{k=1}^{\infty} \nu(I_{n,k})
              \]
              Now, since:
              \[
                  \bigcup_{n=1}^{\infty} E_n \subseteq \bigcup_{n,k=1}^{\infty} I_{n,k}, \quad \{ I_{n,k} \} \subseteq K
              \]
              it clearly follows that:
              \[
                  \mu^*(\bigcup_{n=1}^{\infty} E_n) \leq \sum_{n=1}^{\infty} \sum_{k=1}^{\infty} \nu(I_{n,k}) < \sum_{n=1}^{\infty}\mu^*(E_n) + \epsilon\cdot\cancelnum{1}{\sum_{n=1}^{\infty} \frac{1}{2^n}}
              \]
              because $\epsilon$ is arbitrary, we have the cocnlusion.
    \end{enumerate}
\end{proof}

The intuitive meaning $(K,\nu)$ is that $K$ is a special class of sets in $X$ and $\nu$ is a function that assigns a value to each set in $K$. On the other hand $\nu$ can be any real valued positive function, thus it is not necessary to be a measure.

%>=====< Question 3 >=====<%

\question
What is the Caratheodory condition? How can it be stated in an equivalent way? Prove it.

\subsection*{Solution}

\subsection{Caratheodory condition} \label{CarEq}
Let $\mu^*$ be an outer measure on a set $X$, then we say that $E\subset X$ is $\mu^*$-measurable if:
\[
    \mu^*(Z) = \mu^*(Z\cap E) + \mu^*(Z\setminus E) \quad \forall Z\subset X
\]

\subsection{Equivalent statement}\label{CarIneq}
Let $\mu^*$ be an outer measure on a set $X$, then we say that $E\subset X$ is $\mu^*$-measurable if and only if:
\[
    \mu^*(Z) \geq \mu^*(Z\cap E) + \mu^*(Z\setminus E) \quad \forall Z\subset X
\]
\begin{proof}
    It is enough to note that $\forall E\subseteq X$ we have:
    \[
        Z = (Z\cap E) \cup (Z \cap E^\complement) \quad \forall Z\subset X
    \]
    and thus by the subadditivity of $\mu^*$ (\ref{outer:sub}) we get:
        \[
        \mu^*(Z) \leq \mu^*(Z\cap E) + \mu^*(Z\setminus E) \quad \forall Z\subset X
    \]
    and we may combine this inequality with the other to yield an equality.
\end{proof}


%>=====< Question 4 >=====<%

\question
Can it exist a set of zero outer measure, which does not fulfill the Caratheodory condition? Prove it.

\subsection*{Solution}

\subsection{All zero measure sets are in \texorpdfstring{$\Leb$}{L}} \label{zerosetsaremeas}
There cannot exist such a set $E$ because all sets of zero aouter measure meet the Caratheodory Inequality (\ref{CarIneq}).

\begin{proof}\label{outer:zeromeas}
    Indeed $\forall Z \subseteq X$ by the monotonicty of $\mu^*$ (\ref{outer:mono}) we have:
    \[
        \mu^*(\underbrace{Z\cap E}_{\subseteq E}) + \mu^*(\underbrace{Z\setminus E}_{\subseteq Z}) \leq \cancelnum{0}{\mu^*(E)} + \mu^*(Z)
    \]
\end{proof}

%>=====< Question 5 >=====<%

\question
State the theorem concerning generation of a measure as a restriction of an outer measure.

\subsection*{Solution}
\subsection{Generation of a measure from an outer measure}\label{meas:gen}
\provdef[$\mathcal{L}$]
\[
    \mathcal{L} \coloneqq \{ E\subseteq X : \; E \text{ is } \mu^*-\text{measurable } \}
\]
where $\mu^*$ is an outer measure on $X$, then:
\begin{enumerate}[i)]
    \item the collection $\Leb$ is a $\salg$
    \item $\mu^* |_{\Leb}$ is a complete measure on $\Leb$
\end{enumerate}
%>=====< Question 6 >=====<%

\question
Show that the measure induced by an outer measure on the $\salg$ of all sets fulfilling the
Caratheodory condition is complete.

\subsection*{Solution}
\subsection{Generation of a measure from an outer measure (proof of completeness)}
Let us see that such a measure as the one described in the previous question is complete. Let $\mu^*$ be an outer measure on $X$ and $\Leb$ the $\salg$ of all sets fulfilling the Caratheodory condition. Let $\mu$ be the measure induced by $\mu^*$ on $\Leb$ ($\mu=\mu^* |_{\Leb}$).
\begin{proof}
    Let $N\in\Leb$ such that $\mu(N)=\mu^*(N)=0$ and let $E\subseteq N$.\\
    By monotonicty of $\mu^*$ (\ref{outer:mono}):
    \[
        0\leq \mu^*(E)\leq \mu^*(N)=0 \implies \mu^*(E)=0
    \]
    thus by the lemma seen in \ref{outer:zeromeas} we get that $E\in\Leb$ and so $\Leb$ is complete.
\end{proof}



%>=====< Question 7 >=====<%

\question
Describe the construction of the Lebesgue measure in $\R$ and in $\R^n$.

\subsection*{Solution}

\subsection{Construction of the Lebesgue measure on \texorpdfstring{$\R$}{R}}
Let $I$ be a family of open, bounded intervals in $\R$:
\[
    I \coloneqq \{ (a,b) : a,b\in\R, a\leq b \}
\]
Let us note that $\emptyset\in I$.\\
Now let us consider a function $\lambda_0$:
\begin{align*}
     & \lambda_0 : I \to \R_+   \\
     & \lambda_0 (\emptyset) =0 \\
     & \lambda_0 ((a,b)) = b-a
\end{align*}
Here we take $X=\R$, $(K,\nu)=(I,\lambda_0)$ and construct the outer Lebesgue measure $\lambda^*$ as seen above (\ref{outer:gen}):
\[
    \lambda^*(E) \coloneqq \left\{ \begin{array}{l}
        \inf \left\{ \sum_{n=1}^{\infty} \lambda_0(I_n) \, : \quad E\subseteq \bigcup_{n=1}^{\infty} I_n ,\; \seq{I}\subseteq I \right\}, \quad \forall E\subseteq\R \text{ s.t. } E \text{ has a countable covering }\seq{I}\subseteq I \\
        +\infty, \text{ otherwise}
    \end{array}\right.
\]
The corresponding $\salg$ is the Lebesgue $\salg$ $\Leb(\R)$ and now we define the Lebesgue measure $\lambda$ as the measure generated by the outer Lebesgue measure (as seen in \ref{meas:gen}):
\[
    \lambda \coloneqq \lambda^*|_{\Leb(\R)}
\]

\subsection{Construction of the Lebesgue measure on \texorpdfstring{$\R^n$}{Rn}}
Analogously to what we have seen above we first define an outer measure and then a (complete) measure but we take:
\[
    I^n = \left\{ \bigtimes_{k=1}^{n} (a_k,b_k): \; a_k,b_k\in\R, \; a_k\leq b_k  \right\}
\]
and accordingly we define:
\begin{align*}
     & \lambda_0^n : I^n \to \R_+                                                         \\
     & \lambda_0^n (\emptyset) = 0                                                        \\
     & \lambda_0^n \left( \bigtimes_{k=1}^{n} (a_k,b_k) \right) = \prod_{k=1}^n (b_k-a_k)
\end{align*}
and therefore we take $X=\R^n$ and $(K,\nu)=(I^n,\lambda_0^n)$, we define the outer Lebesgue measure $\lambda^{*,n}$ on $\R^n$ and the Lebesgue $\salg$ $\Leb(\R^n)$ and finally we construct the n-dimensional Lebesgue measure as:
\[
    \lambda^n \coloneqq \lambda^{*,n} |_{\Leb(\R^n)}
\]


%>=====< Question 8 >=====<%

\question
Prove that any countable subset $E\subset\R$ is Lebesgue measurable and $\lambda(E) = 0$.

\subsection*{Solution}

\subsection{All countable sets are \texorpdfstring{$\Leb$}{L}-measurable and \texorpdfstring{$\lambda(E)=0$}{l(E)=0}}
Any countable subset $E\subset\R$ is $\Leb$-measurable and $\lambda(E)=0$
\begin{proof}
    Let $a\in\R$, clearly $\{ a \} \subseteq (a-\epsilon, a]$ $\forall \epsilon >0$, thus by the definition of $\lambda^*$:
    \[
        \lambda^*(\{ a \}) \leq \lambda^*( (a-\epsilon, a] )  = \epsilon \to 0  \implies \{ a \}\in\Leb
    \]
    Now if E is countable we may write as follows:
    \[
        E = \bigcup_{n=1}^{\infty} \{ a_n \} \quad a_n\in\R, \; n\in\N
    \]
    and so by monotonicty (\ref{outer:mono}):
    \[
        0 \leq \lambda^*(E) = \lambda^*\left( \bigcup_{n=1}^{\infty} \{ a_n \} \right) \leq \sum_{n=1}^{\infty} \lambda^*(a_n) = 0
    \]
    thus $\lambda^*(E)=0 \implies E\in \Leb$ by the lemma seen above (\ref{outer:zeromeas})
\end{proof}


%>=====< Question 9 >=====<%

\question
Show that $\B[\R] \subseteq \Leb(\R)$. Is the inclusion strict? Which is the relation between $(\R,\Leb(\R), \lambda)$ and
$(\R, \B[\R], \lambda)$?

\subsection*{Solution}

\subsection{\texorpdfstring{$\B[\R] \subseteq \Leb(\R)$}{B(R) is included in L(R)}}
\begin{proof}
    Since $\B[\R]=\sigma_0((a,+\infty))$ it is enough to show that $(a,+\infty)\in\Leb(\R)$. We already know from above that all bounded intervals belong to $\Leb(\R)$. \\
    Now, let $A\subseteq\R$ be any set. We assume $a\notin A$, otherwise we would replace $A$ with $A\setminus\{a\}$ and this would leave the Lebesgue outer measure unchanged. Furthermore $(a,+\infty)\in\Leb(\R) \iff (a,+\infty)$ satisfies the Caratheodory Condition (\ref{CarIneq}):
    \[
        \lambda^*(A_1)+\lambda^*(A_2)\leq\lambda^*(A) \label{a}
    \]
    where $A_1 = A\cap (-\infty,a)$ and $A_2 = A \cap (a,+\infty)$.\\
    Since $\lambda^*(A)$ is defined as an $\inf$, to verify the above, it is necessary and sufficient to show that for \textbf{any countable collection} $\seq{I}$ of \textbf{open bounded} intervals that \textbf{covers} $A$ we have that:
    \[
        \lambda^*(A_1)+\lambda^*(A_2)\leq \sum_{k=1}^{\infty} \lambda_0(I_k)
    \]
    For every $k\in\N$ we define:
    \begin{align*}
        I_k' \coloneqq I_k \cap (-\infty, a) \\
        I_k'' \coloneqq I_k \cap (a,+\infty)
    \end{align*}
    then:
    \[
        I_k' \cap I_k'' = \emptyset (\text{disjoint}) \implies \lambda_o(I_k) = \lambda_0(I_k') + \lambda_0 (I_k'')
    \]
    Let us note that $\seq{I'}$ is a countable cover for $A_1$ and $\seq{I''}$ is a countable cover for $A_2$.
    Hence:
    \begin{align*}
        \lambda^*(A_1)=\sum_{k=1}^{\infty} \lambda_0(I_k') \\
        \lambda^*(A_2)=\sum_{k=1}^{\infty} \lambda_0(I_k'')
    \end{align*}
    therefore:
    \[
        \lambda^*(A_1)+\lambda^*(A_2)\leq \sum_{k=1}^{\infty} \lambda_0(I_k') + \sum_{k=1}^{\infty} \lambda_0(I_k'') = \sum_{k=1}^{\infty} \lambda_0(I_k)
    \]
    which equivalento to the condition above.
\end{proof}

\subsection{\texorpdfstring{$\B[\R] \subsetneqq \Leb(\R)$}{B(R) is strictly included in L(R)}}
The inclusion demonstrated above can be shown to be strict. A counterexample can be produced (see \href{https://math.stackexchange.com/questions/141017/lebesgue-measurable-set-that-is-not-a-borel-measurable-set}{\color{blue}{here}}) but it is quite pathological.

\subsection{Relation between \texorpdfstring{$(\R,\Leb(\R),\lambda)$}{(R,L(R),l)} and \texorpdfstring{$(\R,\B[\R],\lambda)$}{(R,B(R),l)}}
$(\R,\Leb(\R),\lambda)$ is the completion of $(\R,\B[\R],\lambda|_{\B[\R]})$. Indeed as we have shown above $\B[\R]$ is not a complete $\salg$ while $\Leb(\R)$ is.

\sheet

% include the appropriate files
% 22 - 38

%>=====< Question 11 >=====<%

\question
Write the excision property and prove it. Write and prove (partially) the theorem concerning
the regularity of the Lebesgue measure on $\R$.

\subsection*{Solution}
\subsection{Excision property}\label{ExcProp}
If $A\in\Leb(\R)$, $\lambda^*(A)\leq +\infty$ and $A\subseteq B$, then:
\[
    \lambda^*(B\setminus A) = \lambda^* (B) - \lambda^*(A)
\]
\begin{proof}
    Since $A\in\Leb(\R)$ we can use the Caratheodory equality (\ref{CarEq}) using $Z=B$, $E=A$:
    \[
        \lambda^*(B) = \lambda^*(\underbrace{B\cap A}_{=A \; (A\subseteq B)}) + \lambda^* (B\setminus A)
    \]
    so, since $\lambda^*(A)\leq +\infty$ we may write:
    \[
        \lambda^*(B\setminus A) = \lambda^*(B)-\lambda^*(A)
    \]
\end{proof}

\subsection{Regularity of the Lebesgue Measure}
Let $E\subseteq\R$, \tfae
\begin{enumerate}[i)]
    \item\label{LebReg:1} $E\in\Leb(\R)$
    \item\label{LebReg:2} $\forall \epsilon >0$ $\exists A \subseteq \R$ open s.t.
          \[
              E\subseteq A \quad \lambda^*(A\setminus E) < \epsilon
          \]
    \item\label{LebReg:3} $\exists G \subseteq \R$ in the class $G_{\delta}$ (countable intersections of open sets) s.t.
          \[
              E\subseteq G \quad \lambda^*(G\setminus E)=0
          \]
    \item\label{LebReg:4} $\forall \epsilon >0$ $\exists C \subseteq \R$ closed s.t.
          \[
              C\subseteq E \quad \lambda^*(E\setminus C) < \epsilon
          \]
    \item\label{LebReg:5} $\exists F \subseteq \R$ in the class $F_{\delta}$ (countable unions of closed sets) s.t.
          \[
              F\subseteq E \quad \lambda^*(E\setminus F)=0
          \]
\end{enumerate}
\begin{proof}
    Let us give a (partial) proof:\\
    \begin{itemize}
        \item
              $(\ref{LebReg:1})\implies(\ref{LebReg:2})$: if $E\in\Leb(\R)$, $\lambda(E)<+\infty$ then by definition of outer measure (\ref{outer:def}):
              \[
                  \forall \epsilon >0 \; \exists\seq{I}\text{ that covers } E \text{ and } \sum_{k=1}^{\infty} \lambda_0(I_k) < \lambda^*(E)+\epsilon
              \]
              Let us now define the set $O$:
              \[
                  O\coloneqq \bigcup_{k=1}^{\infty} I_k, \; O \text{ is open}, \; E\subseteq O
              \]
              and so we may write:
              \begin{align*}
                   & \lambda^*(O) \overset{sub-add \; (\ref{outer:sub})}{\leq} \sum_{k=1}^{\infty} \lambda_0(I_k) < \lambda^*(E)+\epsilon \\
                   & \implies \lambda^*(O)-\lambda^*(E) < \epsilon
              \end{align*}
              and by the Excision property (\ref{ExcProp}) ($E\in\Leb(\R), \; \lambda^*(E)<+\infty$):
              \[
                  \lambda^*(O\setminus E) = \lambda^*(O)-\lambda^*(E) < \epsilon
              \]
              and so we have obtained the second statement (\ref{LebReg:2}).
        \item
              $(\ref{LebReg:2}) \implies (\ref{LebReg:3})$, $\forall k\in\N$ we choose $O_k \supseteq E$ open for which:
              \[
                  \lambda^*(O_k\setminus E) < \frac{1}{k}
              \]
              and then define:
              \[
                  G = \bigcap_{k=1}^{\infty} O_k \implies G\in G_{\delta}, \; G\supseteq E
              \]
              Moreover $\forall k\in \N$:
              \[
                  G\setminus E \subseteq O_k\setminus E
              \]
              so by monotonicty (\ref{outer:mono}):
              \[
                  \lambda^*(G\setminus E) \leq \lambda^*(O_k\setminus E) < \frac{1}{k}
              \]
              let us apply a limit $k\to\infty$ to both sides:
              \[
                  \lambda^*(G\setminus E) = 0
              \]
        \item
              $(\ref{LebReg:3}) \implies (\ref{LebReg:1})$, let us note that $G\setminus E\in\Leb(\R)$ since $\lambda^*(G\setminus E)=0 $ by lemma \ref{zerosetsaremeas} and:
              \begin{align*}
                   & G\in\Leb(\R) \text{ since } G\in G_{\delta} \subseteq \B[\R] \subseteq \Leb(\R)                  \\
                   & \implies E = \underset{\in\Leb}{G} \cap (\underset{\in\Leb}{G \setminus E})^\complement \in \Leb
              \end{align*}

    \end{itemize}
\end{proof}


%>=====< Question 12 >=====<%

\question
Is it true that any subset $E\subseteq\R$ is $\Leb$-measurable? Is it possibile to find two disjoint sets
$A,B\subset\R$ for which $\lambda^*(A \cup B) < \lambda^*(A) + \lambda^*(B)$? Why?

\subsection*{Solution}

\subsection{Vitali's non-measurable sets}
Any measurable set $E\subseteq \R$ with $\lambda(E)>0$ contains a subset that fails to be measurable. \\
Therefore there exist subsets of $\R$ that are not $\Leb$-measurable.

\subsection{Disjoints sets for which \texorpdfstring{$\lambda^*(A\cup B) < \lambda^*(A)+\lambda^*(B)$}{the measure of the union is less than the sum of the measure}}
There are disjoint sets $A,B\subseteq\R$ for which:
\[
    \lambda^*(A\cup B) < \lambda^*(A)+\lambda^*(B)
\]

\begin{proof}
    Assume by contradiction that:
    \[
        \lambda^*(A\cup B) = \lambda^*(A)+\lambda^*(B) \quad \forall A,B \subseteq\R, \; A\cap B = \emptyset
    \]
    Now $\forall E,Z \subseteq \R$ we write:
    \[
        \lambda^*(\underbrace{Z\cap E}_{=A}) + \lambda^*(\underbrace{Z\cap E^{\complement}}_{=B}) = \lambda^*(\underbrace{Z}_{=A\cup B})
    \]
    thus any set $E$ would satisfy the Caratheodory condition (\ref{CarEq}) and be $\Leb$-measurable which is absurd since we know that Vitali's sets exist.
\end{proof}

%>=====< Question 1 >=====<%

\question

Write the definition of measurable function. Show the measurability of the composite function.

\subsection*{Solution}

\subsection{Measurable function}

Let $(X,\A)$ and $(X',\A')$ be two measurable spaces and $f$ a function:
\[
    f:X\to X'
\]
$f$ is said to be measurable if:
\[
    f^{-1}(A)\in\A\quad\forall A\in\A'
\]

\subsection{Measurability of the composite function} \label{meas:comp}
Let $(X,\A)$, $(X',\A')$ and $(X'',\A'')$ be three measurable spaces and $f:X\to X'$ and $g:X'\to X''$ two measurable functions. Then the composite function $g\circ f:X\to X''$ is measurable.

\begin{proof}
    \begin{align*}
         & \forall E \in \A' \quad f^{-1}(E)\in\A   \\
         & \forall F \in \A'' \quad g^{-1}(F)\in\A' \\
    \end{align*}
    thus:
    \[
        \forall F\in \A'' \quad (g\circ f)^{-1}(F) = f^{-1} \left[ \underbrace{g^{-1}(F)}_{\coloneqq E \in \A'} \right] \in\A
    \]
\end{proof}


%>=====< Question 2 >=====<%

\question

Characterize measurability of functions and prove it.

\subsection*{Solution}

\subsection{Characterization of Measurability}\label{CharMeas}

Let $(X,\A)$ and $(X',\A')$ be two measurable spaces and $\mathcal{C} ' \subseteq \Parts{X'}$ such that $\sigma_0(\mathcal{C}')=\A'$ then:
\[
    f:X\to X' \text{ measurable } \iff f^{-1}(E)\in\A \quad \forall E\in\mathcal{C}'
\]

\begin{proof}
    Let us prove both sides of the implication:
    \begin{itemize}
        \item \textbf{$(\implies)$:} Suppose $f$ be measurable $\implies$ $\mathcal{C}'\subseteq \A'$ and so we get the thesis.
        \item \textbf{$(\impliedby)$:} Let us define the following:
              \[
                  \Sigma \coloneqq \{ E\subseteq X': \; f^{-1}(E)\in\A \}
              \]
              We can easily see that $\Sigma$ is a $\salg$ so $\mathcal{C}'\subseteq \Sigma$ and thus:
              \[
                  \A'=\sigma_0(\mathcal{C}')\subseteq\Sigma
              \]
              and we get the thesis.
    \end{itemize}
\end{proof}

%>=====< Question 3 >=====<%

\question

Write the definitions of:
\begin{enumerate}[a)]
    \item \label{Bmeas} Borel measurable functions;
    \item \label{Lmeas} Lebesgue measurable functions.
\end{enumerate}

\subsection*{Solution}

\subsection{\ref{Bmeas}) Borel measurable functions}
Let $(X,d), (X,\B)$ and $(X',d'), (X', \B')$ be couples of metric spaces and measurable spaces. A function $f$:
\[
    f:X\to X' \text{ measurable}
\]
is called Borel-measurable or $\B$-meaurable.

\subsection{\ref{Lmeas}) Lebesgue measurable functions}
Let $(X,\Leb)$ be a measurable space and $(X',d')$ a metric space, $(X',\B')$ a measurable space, then:
\[
    f:X\to X' \text{ measurable}
\]
is called Lebesgue-measurable or $\Leb$-measurable.

%>=====< Question 4 >=====<%

\question

Prove that continuous functions are both Borel and Lebesgue measurable.

\subsection*{Solution}

\subsection{Continuous functions are \texorpdfstring{$\B$}{Borel}-measurable}
A continuous function $f:X\to X'$ is $\B$-measurable.

\begin{proof}
    Let $\mathcal{C}'$ be the class of open sets of $X'$ and $\mathcal{C}$ the class of open sets of $X$. We have:
    \[
        \forall E \in \mathcal{C}' \quad f^{-1}(E)\in\mathcal{C} \subseteq \B \; (\text{ by definition of continuity })
    \]
    and $\B'=\sigma_0(\mathcal{C}')$ so we get the thesis.
\end{proof}

\subsection{Continuous functions are \texorpdfstring{$\Leb$}{Lebesgue}-measurable}
A continuous function $f:X\to X'$ is $\Leb$-measurable.

\begin{proof}
    Since $\B \subset \Leb$ and the previous statement has been proven true, the thesis follows trivially.
\end{proof}

%>=====< Question 5 >=====<%

\question

Characterize Lebesgue measurability of functions and prove it.

\subsection*{Solution}

\subsection{Characterization of Lebesgue measurability}
All we must do is apply the Characterization of Measurability (\ref{CharMeas}) taking $(X,\A=\Leb)$, $(X', \A'=\B')$ and $\mathcal{C}'$ the class of open sets of $X'$, since $\B'=\sigma_0(\mathcal{C}')$. We then can write:
\[
    f:X\to X' \text{ Lebesgue measurable } \iff f^{-1}(E)\in\Leb \quad \forall E\in\mathcal{C}'
\]

\begin{proof}
    Let us prove both sides of the implication:
    \begin{itemize}
        \item \textbf{$(\implies)$:} Suppose $f$ be Lebesgue measurable $\implies$ $\mathcal{C}'\subseteq \B'$ and so we get the thesis.
        \item \textbf{$(\impliedby)$:} Let us define the following:
              \[
                  \Sigma \coloneqq \{ E\subseteq X': \; f^{-1}(E)\in\Leb \}
              \]
              We can easily see that $\Sigma$ is a $\salg$ so $\mathcal{C}'\subseteq \Sigma$ and thus:
              \[
                  \B'=\sigma_0(\mathcal{C}')\subseteq\Sigma
              \]
              and we get the thesis.
    \end{itemize}
\end{proof}


%>=====< Question 6 >=====<%

\question

Establish and show all equivalent statements to the fact that $f : X \to \Rcomp$ is measurable.

\subsection*{Solution}

\subsection{Equivalent statements of measurability}
Let $(X,\A)$ be a measurable space and $f:X\to\Rcomp$ a function, \tfae

\begin{enumerate}[i)]
    \item \label{statomeas:1} $f$ is measurable;
    \item \label{statomeas:2} $\{ f>\alpha \}\in\A$ $\forall\alpha\in\R$;
    \item \label{statomeas:3} $\{ f\geq\alpha \}\in\A$ $\forall\alpha\in\R$;
    \item \label{statomeas:4} $\{ f<\alpha \}\in\A$ $\forall\alpha\in\R$;
    \item \label{statomeas:5} $\{ f\leq\alpha \}\in\A$ $\forall\alpha\in\R$.
\end{enumerate}

\begin{proof}
    Let us prove all the coimplications:\\
    \textbf{(\ref{statomeas:1}) $\iff$ (\ref{statomeas:3}):}
    \begin{align*}
         & \A'=\B[\Rcomp]=\sigma_0( \overbrace{ \{(\alpha,+\infty]:\; \alpha\in\R\} }^{\mathcal{C}'} )                   \\
         & f \text{ is measurable } \iff f^{-1} ( \underbrace{(\alpha,+\infty]}_{E} ) \in \A \quad \forall \alpha \in \R
    \end{align*}
    \textbf{(\ref{statomeas:2}) $\implies$ (\ref{statomeas:3}):}
    \[
        \{ f\geq \alpha \} = \cap_{n=1}^\infty \overbrace{\{ f > \alpha-\frac{1}{n} \}}^{\in\A} \in\A
    \]
    \textbf{(\ref{statomeas:3}) $\implies$ (\ref{statomeas:4}):}
    \[
        \{ f<\alpha \} = \{ f\geq\alpha \}^\complement \in\A
    \]
    \textbf{(\ref{statomeas:4}) $\implies$ (\ref{statomeas:5}):}
    \[
        \{ f \leq \alpha \} = \cap_{n=1}^\infty \overbrace{\{ f < \alpha+\frac{1}{n} \}}^{\in\A} \in\A
    \]
    \textbf{(\ref{statomeas:5}) $\implies$ (\ref{statomeas:2}):}
    \[
        \{ f>\alpha \} = \{ f\leq\alpha \}^\complement \in\A
    \]
\end{proof}

%>=====< Question 7 >=====<%

\question

Let $f, g \in \Mes(X, \A)$. What can we say about measurability of $\{f < g\},\; \{f \leq g\},\; \{f = g\}$? Justify the answer.

\subsection*{Solution}

\subsection{Measurability of \texorpdfstring{$\{f < g\},\; \{f \leq g\},\; \{f = g\}$}{ \{f less than g\}, \{f less or equal to g\}, \{f equal to g\}}}
Let $f, g \in \Mes(X, \A)$, we have:
\begin{enumerate}[i)]
    \item $\{f < g\}\in\A$
    \item $\{f \leq g\}\in\A$
    \item $\{f = g\}\in\A$
\end{enumerate}

\begin{proof}
    \hspace*{\fill} %leave a blank line
    \begin{enumerate} [i)]
        \item $\{ f < g \} = \bigcup_{r\in\Q} \left[ \underbrace{\overbrace{\{ f < r \}}^{\in\A} \cap \overbrace{\{ r < g\}}^{\in\A}}_{\in\A} \right]$
        \item $ \{ f\leq g \} = \{ f > g \}^\complement\in\A$ by the previous point.
        \item $\{f=g\} = \underset{\in\A}{\{ f\leq g \}} \cap \underset{\in\A}{\{ f\geq g \}} \in \A$
    \end{enumerate}
\end{proof}

%>=====< Question 8 >=====<%

\question

Let $\{f_n\} \subset \Mes(X, \A)$. Show that $\sup_n f_n, \inf_n f_n, \limsup_n f_n, \liminf_n f_n \in \Mes(X, \A)$. Can there exist two functions $f, g \in \Mes(X, \A)$ such that $\max\{f, g\} \notin \Mes(X, \A)$? Why?

\subsection*{Solution}

\subsection{Measurability of \texorpdfstring{$\sup_n f_n, \inf_n f_n, \limsup_n f_n, \liminf_n f_n$}{sup fn, inf fn, limsup fn, liminf fn}} \label{meas:extremes}
Let $\{f_n\} \subset \Mes(X, \A)$, we have:
\begin{enumerate}[i)]
    \item $\sup_n f_n \in \Mes(X, \A)$
    \item $\inf_n f_n \in \Mes(X, \A)$
    \item $\limsup_n f_n \in \Mes(X, \A)$
    \item $\liminf_n f_n \in \Mes(X, \A)$
\end{enumerate}

\begin{proof}
    \hspace*{\fill} %leave a blank line
    \begin{enumerate} [i)]
        \item $\forall \alpha \in \R \quad \{ \sup_{n\in\N} f_n > \alpha \} = \bigcup_{n=1}^\infty \{ f_n > \alpha \} \in \A \implies \sup_{n\in\N} f_n \in \Mes$
        \item $\inf_n f_n = - \sup_{n\in\N} (-f_n) \in \Mes(X,\A)$
        \item $\limsup_n f_n = \inf_{k\geq 1} \sup_{n \geq k} f_n \in \Mes(X, \A)$
        \item $\liminf_n f_n = \sup_{k\geq 1} \inf_{n\geq k} f_n \in \Mes(X, \A)$
    \end{enumerate}
\end{proof}

\subsection{Corollary, \texorpdfstring{$\max\{f,g\}\in\Mes$}{max(f,g) is measurable}}
There cannot exist two functions $f, g \in \Mes(X, \A)$ such that $\max\{f, g\} \notin \Mes(X, \A)$.
Indeed we can write:
\[ \max(f,g) = f\cdot \chi_{\{f\geq g\}} + g \cdot \chi_{\{g>f\}} \]
In other words the max can be written as the sum and product of measurable functions, hence it is measurable itself.

%>=====< Question 9 >=====<%

\question

Let $f, g \in \Mes(X, \A)$. Show that $f + g, f\cdot g \in \Mes(X, \A)$.

\subsection*{Solution}

\subsection{Measurability of \texorpdfstring{$f + g, f\cdot g$}{sum f and g, product f and g}} \label{meas:sumprod}
Let $f,g:X\to\R$ and $f,g\in\Mes(X,\A)$, we have that $f+g, f\cdot g \in \Mes(X, \A)$.

\begin{proof}
    Let us define a few new functions $\phi, \psi$ and $\chi$:
    \[
        \left\{ \begin{array}{l}
            \phi(x) = X\to\R^2 \quad \phi(x) \coloneqq \left( f(x), g(x) \right) \\
            \psi(x) = \R^2\to\R \quad \psi(s,t) \coloneqq s+t                    \\
            \chi(x) = \R^2\to\R \quad \chi(s,t) \coloneqq s\cdot t
        \end{array} \right.
        \implies
        \left\{ \begin{array}{l}
            \psi \circ \phi = f + g \\
            \chi \circ \phi = f \cdot g
        \end{array} \right.
    \]
    Now, clearly $\psi,\chi \in C^0(\R^2)$ (hence measurable), let us prove that $\phi $ is also measurable. We use the Characterization of Measurability (\ref{CharMeas}):
    \[
        \phi: (X,\A)\to (\R^2,\B[\R^2]) \text{ is measurable } \iff \forall E \subseteq \R^2 \text{ open } \phi^{-1}(E) \in \A
    \]
    We take:
    \begin{align*}
        E            & = R \coloneqq (a,b) \times (c,d)                                                         \\
        \phi^{-1}(R) & \tikzmarknode{eq1}{=} \{ x\in X: \; (f(x),g(x))\in R \}                                  \\
                     & \tikzmarknode{eq2}{=} \{ x\in X: \; f(x)\in (a,b) \} \cap \{ x\in X: \; g(x)\in (c,d) \} \\
                     & \tikzmarknode{eq3}{=} f^{-1}(a,b) \cap g^{-1}(c,d) \in \A
    \end{align*} \tikz[overlay,remember picture]{\draw[shorten >=1pt,shorten <=1pt] (eq1) -- (eq2) -- (eq3);}
    Thus $\forall E \subseteq \R^2$ open, we may write:
    \begin{align*}
        E         & = \bigcup_{k=1}^\infty R_k, \quad R_k = (a_k,b_k) \times (c_k,d_k)                                       \\
        \phi^{-1} & = \bigcup_{k=1}^\infty \phi^{-1}(R_k) = \bigcup_{k=1}^\infty f^{-1}(a_k,b_k) \cap g^{-1}(c_k,d_k) \in \A
    \end{align*}
    Hence $\phi \in \Mes(X,\A)$, and we have:
    \[
        \psi \circ \phi, \; \chi \circ \phi \in \Mes(X,\A)
    \]
\end{proof}


%>=====< Question 10 >=====<%

\question

Prove that A is measurable if and only if $\chi_A$ is a measurable function.

\subsection*{Solution}

\subsection{A is measurable if and only if \texorpdfstring{$\chi_A$}{the indicator function of A} is a measurable function} \label{AinA:chi}

Let $A\subseteq X$ and $\chi_A$ be the indicator function of $A$. We have:
\[
    \chi_A \in \Mes(X, \A) \iff \quad A \in \A
\]

\begin{proof}
    \[
        \{ \chi_A > \alpha \}  = \left\{ \begin{array}{l}
            X \quad \alpha <0         \\
            A \quad 1 > \alpha \geq 0 \\
            \emptyset \quad \alpha \geq 1
        \end{array} \right.
    \]
    Now, $X,\emptyset\in\A$ by definition, so:
    \[
        A\in\A \iff \chi_A \in \Mes
    \]
\end{proof}

%>=====< Question 11 >=====<%

\question

Prove or disprove the following statements:
\begin{enumerate}[a)]
    \item\label{fmeas:fpmmeas} $f \in \Mes(X, \A) \iff f_{\pm} \in \Mes_+(X, \A)$;
    \item\label{fmeas:fabsmeas} $f \in \Mes(X, \A) \iff |f| \in \Mes(X, \A)$.
\end{enumerate}

\subsection*{Solution}

\subsection{Measurability of \texorpdfstring{$f_{\pm}$}{f positive, f negative} and \texorpdfstring{$|f|$}{absolute value of f}}

Let $f:X\to\R$, we have:
\begin{enumerate}[i)]
    \item $f \in \Mes(X, \A) \iff f_{\pm} \in \Mes_+(X, \A)$
    \item $f \in \Mes(X, \A) \iff |f| \in \Mes(X, \A)$
\end{enumerate}

\begin{proof}
    \hspace*{\fill} %leave a blank line
    \begin{enumerate} [i)]
        \item \begin{itemize}
                  \item \textbf{($\implies$):} if $f \in \Mes(X, \A)$, then we define $f_+$ as:
                        \[
                            f_+(x) = \max\{f(x),0\} \geq 0 \quad \forall x\in X
                        \]
                        and since $f,0 \in \Mes(X,\A)$ and $\max$ is a measurable function we have that $f_+ = \max \circ (f,0) \in \Mes_+(X,\A)$ by (\ref{meas:comp}). We may analogously prove the same for $f_-$.
                  \item \textbf{($\impliedby$):} if $f_+ \in \Mes_+(X, \A)$, then we define $f = f_+ - f_-$, and since $f_+,f_-,f \in \Mes(X,\A)$ we have that $f \in \Mes(X,\A)$ by (\ref{meas:sumprod}).
              \end{itemize}
        \item $f\in\Mes \implies f_{+}, f_{-} \in \Mes$ by the previous point $\implies |f|=f_+ + f_- \in \Mes$ by (\ref{meas:sumprod}).
    \end{enumerate}
\end{proof}


%>=====< Question 12 >=====<%

\question

Write the definition of simple function. What is its canonical form? How can we characterize measurability of a simple function? Write the definition of step function.

\subsection*{Solution}

\subsection{Definition of simple function}
Let $X$ be a set and $s:X\to\R$ a function. We say that $s$ is a simple function if $s(X)$ is a finite set. \\
Furthermore we define the two sets:
\begin{align*}
     & \Smes(X, \A) \coloneqq \{ \text{ measurable simple functions}  \}                           \\
     & \Smes_+(X, \A) \coloneqq \{ \text{ measurable simple functions with non-negative values} \}
\end{align*}

\subsection{Canonical form of simple function} \label{simple:canon}
The canonical form of a simple function is:
\[
    s(x) = \sum_{i=1}^n c_i \chi_{E_i}(x)
\]
where:
\begin{align*}
     & c_i \in \R \; \forall i=1,\dots,n                                      \\
     & E_i = \{ x\in X : \; s(x)=c_i \} \; \forall i=1,\dots,n                \\
     & X = \bigcup_{i=1}^n E_i, \; E_k\cap E_l = \emptyset \; \forall k\neq l
\end{align*}
i.e. $E_i$ is a partition of $X$.

\subsection{Measurability of simple function}
A simple function is measurable if and only if we have the following:
\[
    E_i \in \A \; \forall i=1,\dots,n
\]
i.e. :
\[
    s(x)\in \Mes(X,\A) \iff E_i \in \A \; \forall i=1,\dots,n
\]
this is because $s(x)$ is a linear combination of indicator functions.

\subsection{Step Functions}
Let $I=[a_0,a_1)$ be an interval and $P=\{ a_o \equiv x_0 < x_1 < \dots < x_n \equiv a_1 \}$ a partition of $I$. A function $f:I\to\R$ is a step function if:
\[
    f \coloneqq \sum_{k=0}^{n-1} c_k \chi_{[x_k,x_{k+1})} (x)
\]

%>=====< Question 13 >=====<%

\question

State and give a sketch of the proof of the Simple Approximation Theorem.

\subsection*{Solution}

\subsection{Simple Approximation Theorem}\label{SAT}
Let $(X,\A)$ be a measurable space and $f:X\to\Rcomp$. Then there exists a sequence of simple functions $\seq{s}$ such that:
\[
    s_n \xrightarrow{n\to\infty} f \text{ in } X \text{ (pointwise)}
\]
\textbf{Furthermore:}
\begin{enumerate}[i)]
    \item if $f\in\Mes(X,\A)$, then $\seq{s}\subseteq\Smes(X,\A)$;
    \item if $f\geq 0 \implies \seq{s} \uparrow$, $0\leq s_n \leq f$;
    \item $f$ bounded $\implies s_n \xrightarrow{n\to\infty} f$ uniformly in $X$.
\end{enumerate}

\subsection*{Sketch of proof}
Let $f\geq0$, bounded and $0 \leq f \leq 1$ $\forall x\in X$.
\[
    f : X \to [0,1]
\]
Let us divide $[0,1]$ in $2^n$ intervals of equal length $\forall n \in \N$, then we define:
\begin{align*}
     & E_k^{(n)} \coloneqq \left\{ x\in X: \; \frac{k}{2^n} \leq f(x) \leq \frac{k+1}{2^n} \right\} \quad k = 0,\dots,2^n-1 \\
     & s_n \coloneqq \sum_{k=0}^{2^n-1} \frac{k}{2^n} \chi_{E_k^{(n)}}(x) \quad \forall n\in\N
\end{align*}
Clearly $\seq{s}$ has the desired properties.

%>=====< Question 14 >=====<%

\question

Write the definitions of $\esssup_X f$ and $\essinf_X f$. State their properties and prove some of them.

\subsection*{Solution}

\subsection{Definition of \texorpdfstring{$\esssup_X f$}{essup f}}
Let $(X, \A, \mu)$ be a measure space and $f$ a function on $X$. We define:
\[
    \esssup_X f(x) \coloneqq \inf \left\{ \sup_{x\in N^\complement} f(x) : \; N \in \mathcal{N}_{\mu} \right\}
\]

\subsection{Definition of \texorpdfstring{$\essinf_X f$}{essinf f}}
Let $(X, \A, \mu)$ be a measurable space and $f$ a function on $X$. We define:
\[
    \essinf_X f(x) \coloneqq \sup \left\{ \inf_{x\in N^\complement} f(x) : \; N \in \mathcal{N}_{\mu} \right\}
\]

\subsection{Properties of \texorpdfstring{$\esssup_X f$}{essup f} and \texorpdfstring{$\essinf_X f$}{essinf f}}
Let $(X, \A, \mu)$ be a measure space and $f,g\in\Mes(x,\A)$ two functions on $X$. We have that:
\begin{enumerate}[i)]
    \item $\exists N \in \mathcal{N}_\mu$ such that $\esssup_X f = \sup_{x\in N^\complement} f$ and $f\leq \esssup_X f$ almost surely $x\in X$;
    \item $\esssup_X f = -\essinf_X -f$;
    \item $\esssup_X k\cdot f  = k \cdot \esssup_X f$;
    \item $f\leq g \text{ a.e. in } X \implies \esssup_X f \leq \esssup_X g$;
    \item $\esssup_X (f+g) \leq \esssup_X f + \esssup_X g$;
    \item $f = g$ almost everywhere in $X$ $\implies \esssup_X f = \esssup_X g$;
    \item $g \geq 0$ almost everywhere in $X$ $\implies f\cdot g \leq (\esssup_X f)\cdot g$ almost everywhere in $X$.
\end{enumerate}

\begin{proof}
    Let us give a partial proof:
    \begin{enumerate}[i)]
        \item Suppose $\esssup_X f < +\infty$, $\forall k\in\N$ $\exists N_k \in \mathcal{N}_\mu$ such that:
              \[
                  \sup_{x\in N_k} f < \esssup_X f + \frac{1}{k}
              \]
              We define $N \coloneqq \bigcup_{k=1}^\infty N_k$. Then $N \in \mathcal{N}_\mu$ and:
              \begin{align*}
                   & N^\complement = \bigcap_{k=1}^\infty N^\complement_k \subseteq N^\complement_k \quad \forall k \in \N                              \\
                   & \implies  \esssup_X f \leq \sup_{N^\complement} f \leq \sup_{N^\complement_k} f < \esssup_X f + \frac{1}{k} \quad \forall k \in \N
              \end{align*}
              Now we pass apply a limit $k\to+\infty$ and we get:
              \begin{align*}
                   & \sup_{N^\complement} f = \esssup_X f                                                             \\
                   & N \supseteq \bar{N} \coloneqq \{ x\in X: \; f(x) > \esssup_X f(x) \} \in\A                       \\
                   & \implies \bar{N} \in \mathcal{N}_\mu \implies f \leq \esssup_X f \text{ almost everywhere in } X
              \end{align*}
    \end{enumerate}
\end{proof}

%>=====< Question 15 >=====<%

\question

What is $\Leb^\infty$? Which is the relation between functions finite a.e. and essentially bounded functions? Justify the answer.

\subsection*{Solution}

\subsection{Definition of \texorpdfstring{$\Leb^\infty$}{the set of essentially bounded functions}}
Let $(X, \A, \mu)$ be a measure space. A function $f\in\Mes(X,\A)$ is said to be essentially bounded if:
\[
    \esssup_X f < +\infty
\]
and we define the set of essentially bounded functions as:
\[
    \Leb^\infty (X,\A,\mu) \coloneqq \{ f:X\to\Rcomp : \; \text{ f is essentially bounded }\}
\]

\subsection{Relation between functions finite a.e. and essentially bounded functions}
We have that:
\begin{enumerate}
    \item $f\in\Leb^\infty \implies f$ is finite a.e. in $X$;
    \item in general if $f$ is finite a.e. in $X$ $\centernot\implies f\in\Leb^\infty$.
\end{enumerate}

\begin{proof}
    \begin{enumerate}
        \item We can easily see that:
              \[
                  |f| \leq \esssup |f| < + \infty \text{ almost everywhere in } X
              \]
              thus $f$ is finite almost everywhere in $X$;
        \item Let us assume that:
              \[
                  f \text{ is finite a.e. in }  X \implies f\in\Leb^\infty
              \]
              and let us see a clear counterexample of this, take:
              \[
                  f(x): \R \to \Rcomp \coloneqq \begin{cases}
                      \frac{1}{|x|} & x \neq 0 \\
                      + \infty      & x = 0
                  \end{cases}
              \]
              Clearly $f$ is finite in $E=\R\setminus\{0\}$, i.e. $f$ is finite a.e. in $\R$. Let us note that $\lambda(\{0\}) = 0 $. Thus:
              \[
                  \esssup_X |f| = +\infty \implies f \notin \Leb^\infty
              \]
    \end{enumerate}
\end{proof}
% 41 - 45

%>=====< Question 3 >=====<%

\question

Write the definitions of the Lebesgue integral of a nonnegative measurable simple 
function over $X$ and over a measurable subset $E \subseteq X$. Write the main properties of the integral and prove some of them.

\subsection*{Solution}

\subsection{Lebesgue integral of nonnegative simple functions}
Let $(X,\A,\mu)$ be a measure space and $s\in\Smes_+(X,\A)$ a nonnegative simple function with canonical form as in (\ref{simple:canon}). We define the Lebesgue integral of $s$ over $X$ as:
\[
    \int_X s\,d\mu \coloneqq \sum_{k=1}^n c_k \mu(E_k)
\]
And its integral over a measurable subset $E\in \A$ as:
\[
    \int_E s\,d\mu \coloneqq \int_X s\cdot \chi_E \, d\mu = \sum_{k=1}^n c_k \mu(E_k\cap E)
\]

\subsection{Properties of the Lebesgue integral}
\begin{enumerate}[i)]
    \item 
        \[
            \int_X \chi_E\,d\mu = \mu(E) \quad \forall E\in \A
        \]
    \item \label{LebInt:null}
        \[
            \int_N s\,d\mu = 0 \quad \forall N \in \mathcal{N}_\mu
        \]
    \item Let $s\in\Smes_+(X,\A)$, $c\geq 0$, then:
        \[
            \int_X c\cdot s \,d\mu = c\cdot\int_X  s \,d\mu   
        \]
    \item $s,t\in \Smes_+(X,\A)$, then:
        \[
            \int_X (s+t) \,d\mu = \int_X s \,d\mu + \int_X t \,d\mu
        \]
    \item \label{LebInt:monofunc}$s,t \in \Smes_+(X,\A)$, such that $s\leq t$ then:
        \[
            \int_X s \,d\mu \leq \int_X t \,d\mu
        \]
    \item \label{LebInt:monoset}$s\in\Smes_+(X,\A)$, $E\subseteq F \in\A$ then:
        \[
            \int_E s \,d\mu \leq \int_F s \,d\mu
        \]
\end{enumerate}

\begin{proof}
    \hspace*{\fill} %leave a blank line
    \begin{enumerate}[i)]
        \item We may write:
            \[
                \chi_E = \sum_{k=1}^2 c_k \chi_{E_k} = \begin{cases}
                    c_1 = 1 & \;E_1 = E \\
                    c_2 = 0 & \; E_2 = E^\complement
                \end{cases}
            \]
            thus by applying the definition of the Lebesgue integral we get:
            \[
                \int_X \chi_E\,d\mu = \sum_{k=1}^2 c_k \mu(E_k) = 1\cdot\mu(E) + 0 \cdot \mu(E^\complement) = \mu(E)
            \]
        \item Let us apply the definition:
            \[
                \int_N s\,d\mu = \sum_{k=1}^n c_k \mu(E_k\cap N)
            \]
            but by the monotonicity of $\mu$ (\ref{meas:mono}) we have:
            \[
                E_k\cap N \subseteq N \implies \mu(E_k\cap N) \leq \mu(N) = 0      
            \]
            so the previous sum is equal to $0$.
    \end{enumerate}
\end{proof}

%>=====< Question 4 >=====<%

\question

Let $s \in \Smes_+(X, \A)$. For any $E \in \A$, let $\phi(E) \coloneqq \int_E s d\mu$. Prove that $\phi$ is a measure.

\subsection*{Solution}

\subsection{Measure induced by a function}
Let $(X,\A,\mu)$ be a measure space and $s\in\Smes_+(X,\A)$ a nonnegative simple function. We define the measure $\phi$ induced by $s$ as:
\[
    \phi(E) \coloneqq \int_E s\,d\mu \quad \forall E\in \A
\]

\begin{proof}
    Let us see that $\phi$ meets the definition of a measure:\\
    Clearly:
    \[
        \phi : \A \to \Rcomppos    
    \]
    Furthermore:
    \begin{enumerate}[i)]
        \item $\phi(\emptyset) = 0$ by property 2 of the Lebesgue integral (\ref{LebInt:null}).
        \item Let $\seq{E}\subseteq\A$ disjoint and $E=\bigcup_{k=1}^\infty E_k$, let us write:
            \[
                s \coloneqq \sum_{l=1}^m d_l \chi_{F_l} \quad F_l \in \A    
            \]
            thus:
            \begin{align*}
                \phi(E) &\tikzmarknode{eq1}{=} \int_E s\,d\mu = \sum_{l=1}^m d_l \mu(F_l\cap E) \\
                &\tikzmarknode{eq2}{=} \sum_{l=1}^m d_l \sum_{k=1}^\infty \mu(F_l\cap E_k) \text{ by the $\s{additivity}$ of $\mu$ (\ref{meas:mono})}\\
                &\tikzmarknode{eq3}{=} \sum_{k=1}^\infty \sum_{l=1}^m d_l \mu(F_l\cap E_k) \\
                &\tikzmarknode{eq4}{=} \sum_{k=1}^\infty \int_{E_k} s \, d\mu = \sum_{k=1}^\infty \phi(E_k)
            \end{align*} \tikz[overlay,remember picture]{\draw[shorten >=1pt,shorten <=1pt] (eq1) -- (eq2) -- (eq3)--(eq4);}
    \end{enumerate}
\end{proof}


%>=====< Question 5 >=====<%

\question

Write the two possible equivalent definitions of Lebesgue integral of a measurable nonnegative function.

\subsection*{Solution}

Let $f:X\to\Rcomppos$ be a measurable nonnegative function ($f\in\Mes_+(X,\A))$. Let us define the set $\Smes_f$:
\[
    \Smes_f = \{ s\in \Smes_+ : \; s\leq f \text{ in } X\}    
\]
We then have two possible and equivalent definitions of the Lebesgue integral of $f$.

\subsection{Definition by \texorpdfstring{$\sup$}{the supremum}}
We define the integral of $f$ as:
\[
    \int_X f \,d\mu = \sup_{s\in\Smes_f} \int_X s \,d\mu
\]

\subsection{Definition by \texorpdfstring{$\lim$}{the limit}}
Thanks to the Simple Approximation Theorem (\ref{SAT}) we know:
\[
    \exists \seq{s} \subseteq \Smes_f \quad s_n\leq s_{n+1} \; n\in\N, \; s_n\xrightarrow{n\to\infty} s \text{ in } X
\]
So we can define the integral as:
\[
    \int_X f \,d\mu = \lim_{n\to\infty} \int_X s_n \,d\mu    
\]
Let us note that the integral must be independent of the choice of the sequence $\seq{s}$.

%>=====< Question 6 >=====<%

\question

State and prove the Chebychev inequality.

\subsection*{Solution}

\subsection{Chebychev inequality}\label{ChebIneq}

Let $f\in\Mes_+(X,\A)$ then $\forall c>0$ we have:
\[
    \mu( \{  f\geq c \}) \leq \frac{1}{c} \int_{ \{ f \geq c \}} f\; d\mu \leq \frac{1}{c} \int_X f\; d\mu    
\]
\begin{proof}
    Clearly
    \[
        E_C \coloneqq \{ f\geq c \} \in \A \text{ since } f\in\Mes_+(X,\A) \text{ (see (\ref{statomeas:3}))} 
    \]
    and we have that:
    \[
        c\cdot \chi_{E_C} \leq f \cdot \chi_{E_C}    
    \]
    thus by the monotonicity of the integral for functions (\ref{LebInt:monofunc}) and for sets (\ref{LebInt:monoset}) we have:
    \[
        c\cdot \mu(E_C) = \int_ X c\cdot \chi_{E_C} \, d\mu \leq \int_ X f \cdot \chi_{E_C} \, d\mu = \int_{E_C} f \, d\mu \leq \int_X f \, d\mu
    \]
    so we have the Chebychev inequality.
\end{proof}

%>=====< Question 7 >=====<%

\question

Let $f \in \Mes_+(X, \A)$ be such that $\int_X f \, d\mu < +\infty$. Show that $f$ is finite a.e. in $X$.

\subsection*{Solution}

\subsection{f is finite a.e. in X if \texorpdfstring{$\int_X f d\mu < +\infty$}{ its integral is finite}} \label{intfin:ffin}
Let $f \in \Mes_+(X, \A)$ be such that $\int_X f \, d\mu < +\infty$, then $f$ is finite a.e. in $X$.

\begin{proof}
    Let us note that the thesis is equivalent to $\mu(\{ f = +\infty\}) = 0$. Let us define:
    \[
        \{ f = + \infty \} = \bigcap_{n=1}^\infty \{ f > n \}    
    \]
    Clearly we have that:
    \begin{enumerate}[a)]
        \item $\{ f > n\} \downarrow \{ f = +\infty \}$
        \item $\mu (\{ f > n \}) \leq \frac{1}{n}\cdot \int_X f\, d\mu$ $\forall n\in\N$ by the Chebychev inequality (\ref{ChebIneq}).
    \end{enumerate}
    So since $\mu(\{f > 1\}) \leq \frac{1}{1}\cdot \int_X f \, d\mu < +\infty$ we may apply the continuity from above of $\mu$ (\ref{meas:contab}):
    \[
        \mu (\{ f= +\infty \}) = \mu \left( \bigcap_{n=1}^\infty \{f>n\} \right) = \lim_{n\to\infty} \mu(\{f>n\}) = \lim_{n\to\infty} \frac{1}{n}\cdot \underbrace{\int_X f \, d\mu}_{<+\infty} \longrightarrow 0  
    \]
\end{proof}

%>=====< End Document >=====<%

\end{document}












