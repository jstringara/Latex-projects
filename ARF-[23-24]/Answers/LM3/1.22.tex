
%>=====< Question 11 >=====<%

\question
Write the excision property and prove it. Write and prove (partially) the theorem concerning
the regularity of the Lebesgue measure on $\R$.

\subsection*{Solution}
\subsection{Excision property}\label{ExcProp}
If $A\in\Leb(\R)$, $\lambda^*(A)\leq +\infty$ and $A\subseteq B$, then:
\[
    \lambda^*(B\setminus A) = \lambda^* (B) - \lambda^*(A)
\]
\begin{proof}
    Since $A\in\Leb(\R)$ we can use the Caratheodory equality (\ref{CarEq}) using $Z=B$, $E=A$:
    \[
        \lambda^*(B) = \lambda^*(\underbrace{B\cap A}_{=A \; (A\subseteq B)}) + \lambda^* (B\setminus A)
    \]
    so, since $\lambda^*(A)\leq +\infty$ we may write:
    \[
        \lambda^*(B\setminus A) = \lambda^*(B)-\lambda^*(A)
    \]
\end{proof}

\subsection{Regularity of the Lebesgue Measure}
Let $E\subseteq\R$, \tfae
\begin{enumerate}[i)]
    \item\label{LebReg:1} $E\in\Leb(\R)$
    \item\label{LebReg:2} $\forall \epsilon >0$ $\exists A \subseteq \R$ open s.t.
          \[
              E\subseteq A \quad \lambda^*(A\setminus E) < \epsilon
          \]
    \item\label{LebReg:3} $\exists G \subseteq \R$ in the class $G_{\delta}$ (countable intersections of open sets) s.t.
          \[
              E\subseteq G \quad \lambda^*(G\setminus E)=0
          \]
    \item\label{LebReg:4} $\forall \epsilon >0$ $\exists C \subseteq \R$ closed s.t.
          \[
              C\subseteq E \quad \lambda^*(E\setminus C) < \epsilon
          \]
    \item\label{LebReg:5} $\exists F \subseteq \R$ in the class $F_{\delta}$ (countable unions of closed sets) s.t.
          \[
              F\subseteq E \quad \lambda^*(E\setminus F)=0
          \]
\end{enumerate}
\begin{proof}
    Let us give a (partial) proof:\\
    \begin{itemize}
        \item
              $(\ref{LebReg:1})\implies(\ref{LebReg:2})$: if $E\in\Leb(\R)$, $\lambda(E)<+\infty$ then by definition of outer measure (\ref{outer:def}):
              \[
                  \forall \epsilon >0 \; \exists\seq{I}\text{ that covers } E \text{ and } \sum_{k=1}^{\infty} \lambda_0(I_k) < \lambda^*(E)+\epsilon
              \]
              Let us now define the set $O$:
              \[
                  O\coloneqq \bigcup_{k=1}^{\infty} I_k, \; O \text{ is open}, \; E\subseteq O
              \]
              and so we may write:
              \begin{align*}
                   & \lambda^*(O) \overset{sub-add \; (\ref{outer:sub})}{\leq} \sum_{k=1}^{\infty} \lambda_0(I_k) < \lambda^*(E)+\epsilon \\
                   & \implies \lambda^*(O)-\lambda^*(E) < \epsilon
              \end{align*}
              and by the Excision property (\ref{ExcProp}) ($E\in\Leb(\R), \; \lambda^*(E)<+\infty$):
              \[
                  \lambda^*(O\setminus E) = \lambda^*(O)-\lambda^*(E) < \epsilon
              \]
              and so we have obtained the second statement (\ref{LebReg:2}).
        \item
              $(\ref{LebReg:2}) \implies (\ref{LebReg:3})$, $\forall k\in\N$ we choose $O_k \supseteq E$ open for which:
              \[
                  \lambda^*(O_k\setminus E) < \frac{1}{k}
              \]
              and then define:
              \[
                  G = \bigcap_{k=1}^{\infty} O_k \implies G\in G_{\delta}, \; G\supseteq E
              \]
              Moreover $\forall k\in \N$:
              \[
                  G\setminus E \subseteq O_k\setminus E
              \]
              so by monotonicty (\ref{outer:mono}):
              \[
                  \lambda^*(G\setminus E) \leq \lambda^*(O_k\setminus E) < \frac{1}{k}
              \]
              let us apply a limit $k\to\infty$ to both sides:
              \[
                  \lambda^*(G\setminus E) = 0
              \]
        \item
              $(\ref{LebReg:3}) \implies (\ref{LebReg:1})$, let us note that $G\setminus E\in\Leb(\R)$ since $\lambda^*(G\setminus E)=0 $ by lemma \ref{zerosetsaremeas} and:
              \begin{align*}
                   & G\in\Leb(\R) \text{ since } G\in G_{\delta} \subseteq \B[\R] \subseteq \Leb(\R)                  \\
                   & \implies E = \underset{\in\Leb}{G} \cap (\underset{\in\Leb}{G \setminus E})^\complement \in \Leb
              \end{align*}

    \end{itemize}
\end{proof}
