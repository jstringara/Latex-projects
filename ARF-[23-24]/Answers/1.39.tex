
%>=====< Question 1 >=====<%

\question

Define the Cantor set. State its main properties and prove some of them.

\subsection*{Solution}

\subsection{Definition of the Cantor set}
The Cantor set is defined iteratively, let us illustrate the first two steps:
\begin{enumerate}[Step 1:]
    \item We start with the interval $[0,1]$ and remove from it the open interval $(1/3, 2/3)$. We define the following sets:
        \[
            I_{1,1} = \left(\frac{1}{3}, \frac{2}{3} \right) \quad J_{1,1} = \left[0, \frac{1}{3} \right] \quad J_{1,2} = \left[\frac{2}{3}, 1 \right]
        \]
        and:
        \[
            C_1 = \bigcup_{k=1}^2 J_{1,k} \quad \lambda(C_1) = 2\cdot \frac{1}{3} = \frac{2}{3}
        \]
    \item We now remove the open set (1/9, 2/9) from $J_{1,1}$ and the open set (7/9, 8/9) from $J_{1,2}$. We define the following sets:
        \begin{align*}
            & I_{2,1} = \left(\frac{1}{9}, \frac{2}{9} \right) \quad J_{2,1} = \left[0, \frac{1}{9} \right] \quad J_{2,2} = \left[\frac{2}{9}, \frac{1}{3} \right] \\
            & I_{2,2} = \left(\frac{7}{9}, \frac{8}{9} \right) \quad J_{2,3} = \left[\frac{2}{3}, \frac{7}{9} \right] \quad J_{2,4} = \left[\frac{8}{9}, 1 \right]
        \end{align*}
        and:
        \[
            C_2 = \bigcup_{k=1}^4 J_{2,k} \quad \lambda(C_2) = 4\cdot \frac{1}{9} = \frac{4}{9}
        \]
\end{enumerate}
So at the $n$-th step we will have:
\[
    C_n = \bigcup_{k=1}^{2^n} J_{n,k} \quad \lambda(C_n) = 2^n \cdot \frac{1}{3^n} = \left( \frac{2}{3} \right)^n  
\]
Thus we can finally define the Cantor set $\mathcal{C}$ as:
\[
    \mathcal{C} = \bigcup_{n=1}^\infty C_n
\]
let us note that since the endpoints of all the closed intervals are always preserved at each step we have that $C_n \supseteq C_{n+1}$ and thus $C_n \downarrow \mathcal{C}$.

\subsection{Properties of the Cantor set}
\begin{enumerate}[i)]
    \item $\mathcal{C}$ is closed since it is the countable intersection of closed sets ($C_n$ closed $\forall n\in\N$);
    \item $\mathcal{C} \in \B[\R] \subseteq \Leb(\R)$ by virtue of its closedness;
    \item $\lambda(\mathcal{C}) = \lim_{n\to\infty} \lambda(C_n) = \lim_{n\to\infty} (2/3)^n = 0$ since $\lambda(C_1) = 1/3 < +\infty$ and $\lambda$ is continuos from above (\ref{meas:contab}).
    \item $int(\mathcal{C}) = \emptyset$ \\
        \begin{proof}
            \[
            int (\mathcal{C}) \subseteq \mathcal{C} \quad \lambda(\mathcal{C}) = 0 \implies \lambda(int(\mathcal{C})) = 0    
            \]
            by the monotonicity of $\lambda$ (\ref{meas:mono}). Now, since $int(\mathcal{C})$ is open ($\mathcal{C}$ is closed) it must contain an interval, but intervals have positive measure (this holds true only in $\Leb(\R)$) and thus $int(\mathcal{C}) = \emptyset$.
        \end{proof}
        Alternatively:
        \begin{proof}
            Let us assume that $int(\mathcal{C}) \neq \emptyset$, then:
            \[
                \exists J \text{ open } \subseteq int(\mathcal{C})    
            \]
            now, since $\lambda(J)=l >0$ we may write that:
            \[
                \lambda(J) = l > \left( \frac{2}{3} \right)^n = \lambda(C_n) \quad \exists n \in\N 
            \]
            in other words $\exists n \in\N$ such that $J \supseteq C_n \implies J \centernot \subseteq C_n$ which is absurd since we assumed that $J \subseteq int (\mathcal{C}) \implies J \subseteq C_n$ $\forall n\in\N$. Thus $int(\mathcal{C}) = \emptyset$.
        \end{proof}
    \item $\mathcal{C}$ is uncountable, indeed each of its elements can be written as an alternating series of 0s and 2s divided by $3^n$. This would be equal to approximanting each element by going right or left through the sets $J_{n,k}$ where 0 represents a choice to go right and 2 a choice to go left. We can write this as follows:
        \[
            \mathcal{C} = \left\{ x\in [0,1] | \; x = \sum_{n=1}^\infty \frac{x_n}{3^n}, \; x_n \in \{0,2\} \right\}
        \]
        and thus $\mathcal{C}$ can be put into a bijection with $\{0,2\}^\N$ which is uncountable.
\end{enumerate}
