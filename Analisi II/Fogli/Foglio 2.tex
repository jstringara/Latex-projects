\chapter{Foglio \ \thechapter}


\section*{Quesito 1}
\addcontentsline{toc}{section}{Quesito 1}
Dire cosa si intende per equazione differenziale ordinaria lineare del primo
ordine. Scrivere la formula risolutiva e dimostrarla. Inoltre, scrivere la formula risolutiva
del problema di Cauchy associato e dimostrarla

\medskip
\begin{large}
\textbf{Soluzione}
\end{large}


\section*{Quesito 2}
\addcontentsline{toc}{section}{Quesito 2}
Dire cosa si intende per equazioni differenziale ordinaria lineare del secondo
ordine a coefficienti costanti omogenea. Qual è il polinomio caratteristico? Dire qual è
l’integrale generale e spiegare come si ricava.



\medskip
\begin{large}
\textbf{Soluzione}
\end{large}


\section*{Quesito 3}
\addcontentsline{toc}{section}{Quesito 3}
Qual è la struttura dell’integrale generale di una equazione differenziale ordinaria lineare del secondo ordine completa? Illustrare il metodo di somiglianza per la
determinazione di una soluzione particolare, nei casi in cui il termine noto è un polinomio
o una funzione esponenziale o una funzione trigonometrica.


\medskip
\begin{large}
\textbf{Soluzione}
\end{large}


\section*{Quesito 4}
\addcontentsline{toc}{section}{Quesito 4}
Dare le definizioni di: punto interno, punto esterno, punto di frontiera (o di
bordo), interno di un insieme, frontiera (o bordo), chiusura. Fornire esempi.


\medskip
\begin{large}
\textbf{Soluzione}
\end{large}


\section*{Quesito 5}
\addcontentsline{toc}{section}{Quesito 5}
Dire cosa si intende per insieme aperto, per insieme chiuso e per insieme
limitato. Caratterizzare gli insiemi aperti in termini del loro interno, e gli insiemi chiusi
in termini della loro chiusura. Fornire esempi.


\medskip
\begin{large}
\textbf{Soluzione}
\end{large}


\section*{Quesito 6}
\addcontentsline{toc}{section}{Quesito 6}
Dare la definizione di insieme compatto. Enunciare il teorema di Heine-Borel.
Dare la definizione di insiemi connesso e le definizioni equivalenti di insieme connesso per
archi e di insieme connesso per poligonali. Dare la definizione di insieme convesso. Fornire
esempi.

\medskip
\begin{large}
\textbf{Soluzione}
\end{large}


\section*{Quesito 7}
\addcontentsline{toc}{section}{Quesito 7}
Dire cosa si intende per insieme di definizione di una funzione di più variabili
$f(x_1, x_2, \dots , x_n)$ e per insieme di livello. Scrivere la definizione di limite.

\medskip
\begin{large}
\textbf{Soluzione}
\end{large}


\section*{Quesito 8}
\addcontentsline{toc}{section}{Quesito 8}
Spiegare come le restrizioni di una funzione $f(x, y)$ lungo successioni, grafici
di funzioni e curve permettono di dedurre la non esistenza del limite o di determinare un
candidato limite.

\medskip
\begin{large}
\textbf{Soluzione}
\end{large}


\section*{Quesito 9}
\addcontentsline{toc}{section}{Quesito 9}
Illustrare come si può utilizzare il teorema del confronto (o dei due carabinieri)
per il calcolo di limiti di funzioni di più variabili. Enunciare il criterio che permette il
1
calcolo di limiti per funzioni di due variabili tramite le coordinate polari. E’ vero che se
\[
f(x_0+\rho\cos \theta, y_0+\rho\sin\theta) \xrightarrow[\rho\to 0]{} L \quad \forall\theta\in [0,2\pi],   
\]
allora
\[
    \lim_{(x,y)\to(x_0,y_0)} f(x,y)=L \; ?
\]

\medskip
\begin{large}
\textbf{Soluzione}
\end{large}


\section*{Quesito 10}
\addcontentsline{toc}{section}{Quesito 10}
Dare la definizione di funzione di più variabili continua in un punto e di
funzione Lipschitziana in un sottoinsieme. Enunciare il teorema di Weierstrass e il teorema
di esistenza dei valori intermedi.

\medskip
\begin{large}
\textbf{Soluzione}
\end{large}
