\chapter{Foglio \ \thechapter}


\section*{Quesito 1}
\addcontentsline{toc}{section}{Quesito 1}
Enunciare e dimostrare il teorema dell’inversione dei limiti per successioni di
funzioni e il corollario sulla continuità della funzione limite. In che modo il corollario si
può usare per lo studio della convergenza uniforme?

\medskip
\begin{large}
\textbf{Soluzione}
\end{large}


\section*{Quesito 2}
\addcontentsline{toc}{section}{Quesito 2}
Enunciare i teoremi di passaggio al limite sotto il segno di integrale e sotto il
segno di derivata. Discutere, mediante esempi, l’importanza delle ipotesi.


\medskip
\begin{large}
\textbf{Soluzione}
\end{large}


\section*{Quesito 3}
\addcontentsline{toc}{section}{Quesito 3}
Dare le definizioni di serie di funzioni, di somma, di convergenza puntuale e
uniforme.


\medskip
\begin{large}
\textbf{Soluzione}
\end{large}


\section*{Quesito 4}
\addcontentsline{toc}{section}{Quesito 4}
Enunciare i criteri di Cauchy puntuale e uniforme per le serie di funzioni. Dare
la definizione di convergenza totale per una serie di funzioni. Enunciare e dimostrare il
criterio di Weierstrass.

\medskip
\begin{large}
\textbf{Soluzione}
\end{large}


\section*{Quesito 5}
\addcontentsline{toc}{section}{Quesito 5}
Enunciare i teoremi sulla continuità della somma, di integrazione per serie, di
derivazione per serie.

\medskip
\begin{large}
\textbf{Soluzione}
\end{large}


\section*{Quesito 6}
\addcontentsline{toc}{section}{Quesito 6}
Dare le definizioni di distanza (o metrica) su un insieme e di spazio metrico.
Fornire esempi. Dare le definizioni di successione convergente e di successione di Cauchy
in uno spazio metrico. Dare la definizione di spazio metrico completo.

\medskip
\begin{large}
\textbf{Soluzione}
\end{large}


\section*{Quesito 7}
\addcontentsline{toc}{section}{Quesito 7}
Caratterizzare la convergenza e la condizione di Cauchy per successioni nello
spazio metrico $C^0([a, b])$ equipaggiato con la metrica Lagrangiana. Dimostrare che tale
spazio metrico è completo. Esibire una metrica che rende $C^0([a, b])$ uno spazio metrico
non completo.

\medskip
\begin{large}
\textbf{Soluzione}
\end{large}


\section*{Quesito 8}
\addcontentsline{toc}{section}{Quesito 8}
Dare la definizione di insieme chiuso in uno spazio metrico. Dimostrare che un
insieme chiuso in uno spazio metrico completo è esso stesso uno spazio metrico completo.

\medskip
\begin{large}
\textbf{Soluzione}
\end{large}


\section*{Quesito 9}
\addcontentsline{toc}{section}{Quesito 9}
Per una funzione definita in uno spazio metrico e a valori in uno spazio metrico
dare le definizioni di: continuità in un punto, Lipschitzianità. Cosa si intende per contrazione? Enunciare e dimostrare il teorema delle contrazioni (o di Banach-Caccioppoli).

\medskip
\begin{large}
\textbf{Soluzione}
\end{large}


\section*{Quesito 10}
\addcontentsline{toc}{section}{Quesito 10}
Dire cosa si intende: per equazione differenziale ordinaria di ordine $n$ e per
soluzione; per equazione differenziale ordinaria di ordine $n$ in forma normale; per sistema
di equazioni differenziali del primo ordine di $n$ equazioni e per soluzione. Un sistema
differenziale quando si dice lineare? Quando si dice autonomo? Dimostrare l’equivalenza
tra equazioni differenziali ordinarie di ordine $n$ in forma normale e sistemi di equazioni
differenziali del primo ordine di $n$ equazioni.


\medskip
\begin{large}
\textbf{Soluzione}
\end{large}
