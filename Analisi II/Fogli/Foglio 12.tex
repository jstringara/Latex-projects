\chapter{Foglio \ \thechapter}


\section*{Quesito 1}
\addcontentsline{toc}{section}{Quesito 1}
Enunciare il teorema di esistenza Peano per il problema di Cauchy per equazioni
differenziali ordine del primo ordine. Mostrare con un esempio che in generale, sotto le
ipotesi del teorema, non vale l’unicità.

\medskip
\begin{large}
\textbf{Soluzione}
\end{large}


\section*{Quesito 2}
\addcontentsline{toc}{section}{Quesito 2}
Cosa si intende per analisi qualitativa di soluzioni di equazioni differenziali ordinarie del primo ordine? Illustrare i vari punti in cui è solitamente articolata. Enunciare:
il criterio dell’asintoto, il teorema del confronto e il teorema di monotonìa.


\medskip
\begin{large}
\textbf{Soluzione}
\end{large}


\section*{Quesito 3}
\addcontentsline{toc}{section}{Quesito 3}
Dire cosa si intende per sistema differenziale lineare del primo ordine di n
equazioni, per sistema omogeneo e per sistema autonomo. Introdurre l’applicazione associata e dimostrare che è lineare. Quali conseguenze ha la linearità sulla somma di
soluzioni? Qual è il sistema lineare associato a una equazione differenziale lineare di
ordine $n$? Perché?


Enunciare e dimostrare il teorema di esistenza ed unicità globale per sistemi differenziali lineari del primo ordine di $n$ equazioni.


\medskip
\begin{large}
\textbf{Soluzione}
\end{large}


\section*{Quesito 4}
\addcontentsline{toc}{section}{Quesito 4}
Enunciare e dimostrare il teorema di struttura dell’insieme delle soluzioni di
sistemi differenziali lineari del primo ordine di $n$ equazioni omogenei e completi.

\medskip
\begin{large}
\textbf{Soluzione}
\end{large}


\section*{Quesito 5}
\addcontentsline{toc}{section}{Quesito 5}
Dare le definizioni di soluzioni, di un sistema differenziale lineare omogeneo,
linearmente indipendenti e linearmente dipendenti. Per quali $x \in I$ si richiede che le
dovute condizioni siano verificate? Perchè? Dare la definizione di sistema fondamentale
di soluzioni per un sistema differenziale lineare omogeneo.


\medskip
\begin{large}
\textbf{Soluzione}
\end{large}


\section*{Quesito 6}
\addcontentsline{toc}{section}{Quesito 6}
Dare la definizione di matrice Wronskiana e di determinante Wronskiano per
sistemi lineari omogenei. Come si caratterizzano, in termini di determinante Wronskiano,
$n$ soluzioni linearmente indipendenti di un sistema omogeneo di n equazioni? Qual è
la matrice Wronskiana per una equazione lineare omogenea di ordine $n$? Enunciare e
dimostrare le proprietà della matrice Wronskiana. Enunciare il teorema di Liouville.

\medskip
\begin{large}
\textbf{Soluzione}
\end{large}


\section*{Quesito 7}
\addcontentsline{toc}{section}{Quesito 7}
Dire qual è l’integrale generale di un sistema differenziale del primo ordine
omogeneo a coefficienti costanti di 2 equazioni e dimostrarlo.


\medskip
\begin{large}
\textbf{Soluzione}
\end{large}


\section*{Quesito 8}
\addcontentsline{toc}{section}{Quesito 8}
Dire qual è l’integrale generale di una equazione differenziale del secondo
ordine omogenea a coefficienti costanti e dimostrarlo.

\medskip
\begin{large}
\textbf{Soluzione}
\end{large}


\section*{Quesito 9}
\addcontentsline{toc}{section}{Quesito 9}
Scrivere e dimostrare la formula risolutiva di un sistema differenziale del primo
ordine completo di $n$ equazioni a coefficienti continui, mediante il metodo di variazione
delle costanti.

\medskip
\begin{large}
\textbf{Soluzione}
\end{large}


\section*{Quesito 10}
\addcontentsline{toc}{section}{Quesito 10}
Illustrare, fornendone la dimostrazione, il metodo di variazione delle costanti
per equazioni differenziali del secondo ordine complete a coefficienti costanti.


\medskip
\begin{large}
\textbf{Soluzione}
\end{large}
