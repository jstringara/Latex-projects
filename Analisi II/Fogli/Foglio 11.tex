\chapter{Foglio \ \thechapter}


\section*{Quesito 1}
\addcontentsline{toc}{section}{Quesito 1}
Siano $A\subseteq\R^{n+1}$ un aperto, $f : A \to R^n$. Per $(x, y) \in \R^{n+1}$ poniamo
$(x, y) \equiv (x, y^1, \dots , y^n)$. Dire cosa si intende per funzione $f$: (i) Lipischitziana nell'insieme
$A$ nella variabile $y$ uniformemente rispetto alla variabile $x$; (ii) localmente Lipischitziana
nell'insieme $A$ nella variabile $y$ uniformemente rispetto alla variabile $x$.


\medskip
\begin{large}
\textbf{Soluzione}
\end{large}


\section*{Quesito 2}
\addcontentsline{toc}{section}{Quesito 2}
Enunciare e dimostrare il teorema sull'equivalenza tra problema di Cauchy ed
equazione integrale di Volterra.


\medskip
\begin{large}
\textbf{Soluzione}
\end{large}


\section*{Quesito 3}
\addcontentsline{toc}{section}{Quesito 3}
Enunciare e dimostrare il teorema di esistenza ed unicità locale per il problema
di Cauchy.


\medskip
\begin{large}
\textbf{Soluzione}
\end{large}


\section*{Quesito 4}
\addcontentsline{toc}{section}{Quesito 4}
Enunciare il teorema di esistenza ed unicità globale per il problema di Cauchy.

\medskip
\begin{large}
\textbf{Soluzione}
\end{large}


\section*{Quesito 5}
\addcontentsline{toc}{section}{Quesito 5}
Illustrare, mediante esempi, cosa può succedere quando non è soddisfatta
l'ipotesi di sublinearità richiesta nel teorema di esistenza ed unicità globale per il problema
di Cauchy.


\medskip
\begin{large}
\textbf{Soluzione}
\end{large}


\section*{Quesito 6}
\addcontentsline{toc}{section}{Quesito 6}
Spiegare come si risolvono equazioni differenziali lineari del primo ordine: a
variabili separabili, ad esse riconducibili, omogenee, esatte, di Bernoulli.

\medskip
\begin{large}
\textbf{Soluzione}
\end{large}


\section*{Quesito 7}
\addcontentsline{toc}{section}{Quesito 7}
Due soluzioni di una equazione differenziale ordinaria del primo ordine, con
$f(x, y)$ verificante le ipotesi del teorema di esistenza ed unicità locale, possono essere
uguali in un solo punto $x_0$? Perché?


\medskip
\begin{large}
\textbf{Soluzione}
\end{large}


\section*{Quesito 8}
\addcontentsline{toc}{section}{Quesito 8}
Dare la definizione di prolungamento di una soluzione, di prolungamento
massimale e di intervallo massimale di esistenza. Mostrare che la soluzione data dal
teorema di esistenza ed unicità locale ammette un prolungamento. Cosa succede se si
itera la costruzione del prolungamento? Sia $I = (\alpha, \beta)$ l'intervallo massimale di esistenza;
cosa può succedere alla soluzione $y(x)$ per $x \to \beta^-$ e per $x \to \alpha^+$?

\medskip
\begin{large}
\textbf{Soluzione}
\end{large}


\section*{Quesito 9}
\addcontentsline{toc}{section}{Quesito 9}
Enunciare e dimostrare il lemma di Gronwall. Come si può dimostrare l'unicità
di soluzioni con il lemma di Gronwall?

\medskip
\begin{large}
\textbf{Soluzione}
\end{large}


\section*{Quesito 10}
\addcontentsline{toc}{section}{Quesito 10}
Enunciare e dimostrare i teoremi sulla dipendenza continua dal dato iniziale
e da parametri.

\medskip
\begin{large}
\textbf{Soluzione}
\end{large}
