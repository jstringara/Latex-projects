\chapter{Foglio \ \thechapter}


\section*{Quesito 1}
\addcontentsline{toc}{section}{Quesito 1}
Dare la definizione di superficie regolare, di equazioni parametriche di una
superficie regolare, di sostegno di una superficie regolare. Inoltre, dare le definizioni di
superficie, di superficie semplice, di punti regolari di una superficie.

\medskip
\begin{large}
\textbf{Soluzione}
\end{large}


\section*{Quesito 2}
\addcontentsline{toc}{section}{Quesito 2}
Qual è il piano tangente a una superficie regolare in un punto? Qual è il
versore normale?


\medskip
\begin{large}
\textbf{Soluzione}
\end{large}


\section*{Quesito 3}
\addcontentsline{toc}{section}{Quesito 3}
Qual è l’area di una superficie regolare? Come si definisce l’integrale di superficie di una funzione continua definita sul sostegno della superficie?


\medskip
\begin{large}
\textbf{Soluzione}
\end{large}


\section*{Quesito 4}
\addcontentsline{toc}{section}{Quesito 4}
Dare la definizione di superficie orientabile. Spiegare il significato geometrico
di superficie orientabile e di superficie non orientabile. Come si costruisce il nastro di
Moebius?


\medskip
\begin{large}
\textbf{Soluzione}
\end{large}


\section*{Quesito 5}
\addcontentsline{toc}{section}{Quesito 5}
Dare la definizione di superficie regolare con bordo. Inoltre, dire come si
può definire, in maniera geometrica, il bordo del sostegno di una superficie orientabile.
Una superficie orientabile quando si dice chiusa? Dare la definizione di orientazione
positiva del bordo di una superficie con bordo. Come si può spiegare geometricamente
tale orientazione?


\medskip
\begin{large}
\textbf{Soluzione}
\end{large}


\section*{Quesito 6}
\addcontentsline{toc}{section}{Quesito 6}
Dare la definizione di flusso di un campo vettoriale attraverso una superficie;
cosa si intende per flusso entrante o uscente? Dare la definizione di circuitazione di un
campo vettoriale lungo il bordo di una superficie.


\medskip
\begin{large}
\textbf{Soluzione}
\end{large}


\section*{Quesito 7}
\addcontentsline{toc}{section}{Quesito 7}
Enunciare il teorema di Stokes.

\medskip
\begin{large}
\textbf{Soluzione}
\end{large}


\section*{Quesito 8}
\addcontentsline{toc}{section}{Quesito 8}
Qual è la divergenza di un campo vettoriale $C^1$ ? Enunciare il teorema della
divergenza nello spazio. Ricavare le formule di integrazione per parti che seguono dal
teorema della divergenza nello spazio. Enunciare e dimostrare il teorema della divergenza
nel piano.


\medskip
\begin{large}
\textbf{Soluzione}
\end{large}


\section*{Quesito 9}
\addcontentsline{toc}{section}{Quesito 9}
Dare le definizioni di convergenza puntuale e di convergenza uniforme di una
successione di funzioni. Dimostrare che se $f_n \underset{n\to +\infty}{\longrightarrow} f$ uniformemente in 
$I\subseteq\R$, allora $f_n \underset{n\to +\infty}{\longrightarrow} f$ puntualmente in $I$;
inoltre, mostrare che il viceversa non è vero in generale.

\medskip
\begin{large}
\textbf{Soluzione}
\end{large}


\section*{Quesito 10}
\addcontentsline{toc}{section}{Quesito 10}
Scrivere le condizioni di Cauchy puntuale ed uniforme per una successione
di funzioni. Enunciare il criterio di Cauchy puntuale. Enunciare e dimostrare il criterio
di Cauchy uniforme.

\medskip
\begin{large}
\textbf{Soluzione}
\end{large}
