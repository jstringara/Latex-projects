\chapter{Foglio \ \thechapter}


\section*{Quesito 1}
\addcontentsline{toc}{section}{Quesito 1}
Dare la definizione di funzione $k$ volte differenziabile in un punto. Sotto quali
ipotesi su $f$, vale la formula di Taylor di ordine $k$ (cioè $f$ è $k$ volte differenziabile)?


\medskip
\begin{large}
\textbf{Soluzione}
\end{large}


\section*{Quesito 2}
\addcontentsline{toc}{section}{Quesito 2}
Cosa si intende per matrice Jacobiana di una funzione 
$f : A \subseteq \R^n \to \R^m$ ?
Scrivere la formula sulla matrice Jacobiana di funzione composta.


\medskip
\begin{large}
\textbf{Soluzione}
\end{large}


\section*{Quesito 3}
\addcontentsline{toc}{section}{Quesito 3}
Dare le definizioni di punto: di minimo relativo, di massimo relativo, di sella,
critico o stazionario. Enunciare e dimostrare il teorema di Fermat.


\medskip
\begin{large}
\textbf{Soluzione}
\end{large}


\section*{Quesito 4}
\addcontentsline{toc}{section}{Quesito 4}
Enunciare e dimostrare il criterio della matrice Hessiana per la classificazione
di punti critici.


\medskip
\begin{large}
\textbf{Soluzione}
\end{large}


\section*{Quesito 5}
\addcontentsline{toc}{section}{Quesito 5}
Enunciare e dimostrare la condizione necessaria perché un punto sia di massimo o di minimo relativo.

\medskip
\begin{large}
\textbf{Soluzione}
\end{large}


\section*{Quesito 6}
\addcontentsline{toc}{section}{Quesito 6}
Illustrare, anche mediante degli esempi, come si procede per classificare un
punto critico quando la matrice Hessiana nel punto è semidefinita negativa o positiva.

\medskip
\begin{large}
\textbf{Soluzione}
\end{large}


\section*{Quesito 7}
\addcontentsline{toc}{section}{Quesito 7}
Enunciare e dimostrare il teorema del Dini

\medskip
\begin{large}
\textbf{Soluzione}
\end{large}


\section*{Quesito 8}
\addcontentsline{toc}{section}{Quesito 8}
Enunciare il teorema del Dini nel caso in cui $F(x_0,y_0)=0, \: F_x(x_0,y_0)\neq 0$

\medskip
\begin{large}
\textbf{Soluzione}
\end{large}


\section*{Quesito 9}
\addcontentsline{toc}{section}{Quesito 9}
Enunciare e dimostrare la versione globale del teorema della funzione implicita.

\medskip
\begin{large}
\textbf{Soluzione}
\end{large}


\section*{Quesito 10}
\addcontentsline{toc}{section}{Quesito 10}
Sotto quali ipotesi su $F$, la funzione definita implicitamente $f$ è di classe $C^k$?
Nel caso esista, come si trova $f''(x)$?

\medskip
\begin{large}
\textbf{Soluzione}
\end{large}
