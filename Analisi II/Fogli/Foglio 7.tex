\chapter{Foglio \ \thechapter}


\section*{Quesito 1}
\addcontentsline{toc}{section}{Quesito 1}
Dare le definizioni di forma differenziale e di campo vettoriale.

\medskip
\begin{large}
\textbf{Soluzione}
\end{large}


\section*{Quesito 2}
\addcontentsline{toc}{section}{Quesito 2}
Come si definisce l’integrale curvilineo di seconda specie di una forma differenziale? E quello di un campo vettoriale?


\medskip
\begin{large}
\textbf{Soluzione}
\end{large}


\section*{Quesito 3}
\addcontentsline{toc}{section}{Quesito 3}
L’integrale curvilineo di seconda specie di una forma differenziale su due curve
equivalenti cambia? Motivare la risposta.


\medskip
\begin{large}
\textbf{Soluzione}
\end{large}


\section*{Quesito 4}
\addcontentsline{toc}{section}{Quesito 4}
Dare la definizione di forma differenziale esatta e di campo vettoriale conservativo. Cosa si intende per primitiva di una forma differenziale e per potenziale di un
campo vettoriale?

\medskip
\begin{large}
\textbf{Soluzione}
\end{large}


\section*{Quesito 5}
\addcontentsline{toc}{section}{Quesito 5}
Come si calcola l’integrale curvilineo di seconda specie di una forma differenziale esatta o di un campo vettoriale conservativo?

\medskip
\begin{large}
\textbf{Soluzione}
\end{large}


\section*{Quesito 6}
\addcontentsline{toc}{section}{Quesito 6}
Quanto vale l’integrale curvilineo di seconda specie di una forma differenziale
esatta lungo una curva chiusa? Motivare la risposta. Come si riformula lo stesso risultato
per i campi vettoriali conservativi?

\medskip
\begin{large}
\textbf{Soluzione}
\end{large}


\section*{Quesito 7}
\addcontentsline{toc}{section}{Quesito 7}
Enunciare e dimostrare il teorema sulla caratterizzazione delle forme differenziali esatte.

\medskip
\begin{large}
\textbf{Soluzione}
\end{large}


\section*{Quesito 8}
\addcontentsline{toc}{section}{Quesito 8}
Dare la definizione di forma differenziale chiusa e di campo vettoriale irrotazionale.

\medskip
\begin{large}
\textbf{Soluzione}
\end{large}


\section*{Quesito 9}
\addcontentsline{toc}{section}{Quesito 9}
Dimostrare che una forma differenziale esatta è anche chiusa.

\medskip
\begin{large}
\textbf{Soluzione}
\end{large}


\section*{Quesito 10}
\addcontentsline{toc}{section}{Quesito 10}
Esibire una forma differenziale chiusa che non sia esatta.

\medskip
\begin{large}
\textbf{Soluzione}
\end{large}
