\chapter{Foglio \ \thechapter}


\section*{Quesito 1}
\addcontentsline{toc}{section}{Quesito 1}
Enunciare il teorema del Dini nel caso in cui $F:A\subseteq \R^{n+1}\to\R$.

\medskip
\begin{large}
\textbf{Soluzione}
\end{large}


\section*{Quesito 2}
\addcontentsline{toc}{section}{Quesito 2}
Enunciare il teorema del Dini nel caso in cui $F:A\subseteq \R^{n+h}\to\R^h$.


\medskip
\begin{large}
\textbf{Soluzione}
\end{large}


\section*{Quesito 3}
\addcontentsline{toc}{section}{Quesito 3}
Mostrare che, applicando il teorema precedente in un intorno di un punto $P_0$
ad una funzione $F:A\subseteq \R^3\to\R^2$, si vede che l’insieme degli zeri di
$F$ in un intorno di $P_0$ è il sostegno di una curva regolare, semplice. Ricavare un vettore parallelo al vettore
tangente alla curva nel punto.


\medskip
\begin{large}
\textbf{Soluzione}
\end{large}


\section*{Quesito 4}
\addcontentsline{toc}{section}{Quesito 4}
Illustrare come si studia il grafico di una funzione $y = f(x)$ definita implicitamente globalmente da una equazione del tipo $F(x, y) = 0$.


\medskip
\begin{large}
\textbf{Soluzione}
\end{large}


\section*{Quesito 5}
\addcontentsline{toc}{section}{Quesito 5}
Utilizzando il teorema del Dini, dimostrare che $\nabla F$ è ortogonale all’insieme
degli zeri di una funzione $F(x, y)$. Come si ottiene lo stesso risultato nel caso di un insieme
di livello generale?


\medskip
\begin{large}
\textbf{Soluzione}
\end{large}


\section*{Quesito 6}
\addcontentsline{toc}{section}{Quesito 6}
Dire cosa si intende per vincolo di uguaglianza. Dare le definizioni di: punto
di minimo vincolato, punto di massimo vincolato, punto di estremo vincolato, punto di
minimo relativo vincolato, punto di massimo relativo vincolato, punto di estremo relativo
vincolato.


\medskip
\begin{large}
\textbf{Soluzione}
\end{large}


\section*{Quesito 7}
\addcontentsline{toc}{section}{Quesito 7}
Cosa si intende per derivata tangenziale al vincolo? Dare la definizione per
punto critico (o singolare) vincolato.

\medskip
\begin{large}
\textbf{Soluzione}
\end{large}


\section*{Quesito 8}
\addcontentsline{toc}{section}{Quesito 8}
Illustrare come si determinano punti di massimo e di minimo vincolati nel
caso di vincolo esplicitabile tramite il sostegno di una curva o l’unione di un numero finito
di grafici di funzioni di una variabile.


\medskip
\begin{large}
\textbf{Soluzione}
\end{large}


\section*{Quesito 9}
\addcontentsline{toc}{section}{Quesito 9}
Dare le definizioni di punto regolare e di punto singolare di un vincolo. Come
si definisce la Lagrangiana? Enunciare e dimostrare il teorema dei moltiplicatori di Lagrange. Quale parte del teorema dei moltiplicatori di Lagrange può essere vista come una
versione vincolata del teorema di Fermat?


\medskip
\begin{large}
\textbf{Soluzione}
\end{large}


\section*{Quesito 10}
\addcontentsline{toc}{section}{Quesito 10}
Illustrare come si determinano i punti di minimo e di massimo assoluto di
una funzione $f(x, y)$ continua in un insieme compatto $K\subseteq\R^2$.

\medskip
\begin{large}
\textbf{Soluzione}
\end{large}
