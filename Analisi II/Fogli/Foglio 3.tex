\chapter{Foglio \ \thechapter}


\section*{Quesito 1}
\addcontentsline{toc}{section}{Quesito 1}
Dare la definizione di derivate parziali, di funzione derivabile e di gradiente.
Può una funzione di due variabili essere derivabile in un punto $(x_0, y_0)$
 e non essere continua in $(x_0, y_0)$?

\medskip
\begin{large}
\textbf{Soluzione}
\end{large}


\section*{Quesito 2}
\addcontentsline{toc}{section}{Quesito 2}
Dare la definizione di derivata direzionale. In che modo le derivata parziali
sono collegate alle derivate direzionali? Può una funzione di due variabili essere derivabile
in un punto $(x_0, y_0)$ lungo ogni direzione e non essere continua in $(x_0, y_0)$?



\medskip
\begin{large}
\textbf{Soluzione}
\end{large}


\section*{Quesito 3}
\addcontentsline{toc}{section}{Quesito 3}
Dare la definizione di funzione differenziabile in un punto. Scrivere l’equazione
del piano tangente in un punto al grafico di una funzione di due variabili; sotto quale
ipotesi su $f$ esiste? Dimostrare che una funzione differenziabile in un punto $(x_0, y_0)$ è
continua e derivabile in $(x_0, y_0)$. Enunciare e dimostrare la formula del gradiente.


\medskip
\begin{large}
\textbf{Soluzione}
\end{large}


\section*{Quesito 4}
\addcontentsline{toc}{section}{Quesito 4}
Enunciare e dimostrare il teorema del differenziale totale.

\medskip
\begin{large}
\textbf{Soluzione}
\end{large}


\section*{Quesito 5}
\addcontentsline{toc}{section}{Quesito 5}
Enunciare e dimostrare il teorema sulle direzioni di massima e di minima
crescita.

\medskip
\begin{large}
\textbf{Soluzione}
\end{large}


\section*{Quesito 6}
\addcontentsline{toc}{section}{Quesito 6}
Enunciare il teorema di derivazione di funzioni composte. Dimostrare che il
gradiente di una funzione di due variabili è ortogonale alle curve di livello.

\medskip
\begin{large}
\textbf{Soluzione}
\end{large}


\section*{Quesito 7}
\addcontentsline{toc}{section}{Quesito 7}
Definire le derivate parziali seconde e la matrice Hessiana. Esibire una funzione
per cui $f_{xy}(x_0,y_0)\neq f_{yx}(x_0,y_0)$, per qualche $(x_0, y_0)$.

\medskip
\begin{large}
\textbf{Soluzione}
\end{large}


\section*{Quesito 8}
\addcontentsline{toc}{section}{Quesito 8}
Enunciare il teorema di Schwarz. Sotto quale ipotesi su $f$ la matrice Hessiana
è simmetrica?

\medskip
\begin{large}
\textbf{Soluzione}
\end{large}


\section*{Quesito 9}
\addcontentsline{toc}{section}{Quesito 9}
Enunciare e dimostrare il teorema di Lagrange e la formula di Taylor con resto
di Lagrange per funzioni di più variabili.


\medskip
\begin{large}
\textbf{Soluzione}
\end{large}


\section*{Quesito 10}
\addcontentsline{toc}{section}{Quesito 10}
Enunciare e dimostrare la formula di Taylor con resto di Peano.

\medskip
\begin{large}
\textbf{Soluzione}
\end{large}
