\chapter{Foglio \ \thechapter}


\section*{Quesito 1}
\addcontentsline{toc}{section}{Quesito 1}
Dare le definizioni di misura interna ed esterna di un insieme limitato di $\R^n$,
di insieme misurabile secondo Peano-Jordan e di misura di un insieme.

\medskip
\begin{large}
\textbf{Soluzione}
\end{large}


\section*{Quesito 2}
\addcontentsline{toc}{section}{Quesito 2}
Esibire delle classi di insiemi misurabili in $\R^2$ e in $\R^3$.

\medskip
\begin{large}
\textbf{Soluzione}
\end{large}


\section*{Quesito 3}
\addcontentsline{toc}{section}{Quesito 3}
Esibire, motivando la risposta, un insieme di $\R^2$ non misurabile secondo PeanoJordan.

\medskip
\begin{large}
\textbf{Soluzione}
\end{large}


\section*{Quesito 4}
\addcontentsline{toc}{section}{Quesito 4}
Dare le definizioni di somme inferiori e superiori di Riemann, di funzione integrabile secondo Riemann e di integrale di Riemann. Quale classe di funzioni è integrabile
secondo Riemann su insiemi compatti misurabili?

\medskip
\begin{large}
\textbf{Soluzione}
\end{large}


\section*{Quesito 5}
\addcontentsline{toc}{section}{Quesito 5}
Scrivere le formule di riduzione (o di Fubini-Tonelli) per integrali doppi. Dare
una giustificazione approssimativa della formula.

\medskip
\begin{large}
\textbf{Soluzione}
\end{large}


\section*{Quesito 6}
\addcontentsline{toc}{section}{Quesito 6}
Dare le definizioni di dominio normale regolare e di dominio regolare. Scrivere
il teorema sulla formula di cambiamento di coordinate negli integrali doppi. Spiegare come
va intesa l’ipotesi $\phi$ di classe $C^1$
sul dominio regolare $\Omega '$, che è chiuso.

\medskip
\begin{large}
\textbf{Soluzione}
\end{large}


\section*{Quesito 7}
\addcontentsline{toc}{section}{Quesito 7}
Perché non si può applicare il teorema del Quesito 6 per ricavare il passaggio
da coordinate cartesiane a polari negli integrali doppi? Quale estensione del precedente
teorema permette di ottenere tale cambio di coordinate? Motivare le risposte.

\medskip
\begin{large}
\textbf{Soluzione}
\end{large}


\section*{Quesito 8}
\addcontentsline{toc}{section}{Quesito 8}
Scrivere le formule di integrazione per fili e per strati negli integrali tripli.
Dare una giustificazione approssimativa della formula.

\medskip
\begin{large}
\textbf{Soluzione}
\end{large}


\section*{Quesito 9}
\addcontentsline{toc}{section}{Quesito 9}
Enunciare e dimostrare il teorema di Pappo-Guldino.

\medskip
\begin{large}
\textbf{Soluzione}
\end{large}


\section*{Quesito 10}
\addcontentsline{toc}{section}{Quesito 10}
Spiegare come si definiscono gli integrali impropri (o generalizzati) in $\R^n$,
considerando il segno della funzione. Calcolare l’integrale della Gaussiana.

\medskip
\begin{large}
\textbf{Soluzione}
\end{large}
