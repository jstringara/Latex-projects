\chapter{Foglio \ \thechapter}


\section*{Quesito 1}
\addcontentsline{toc}{section}{Quesito 1}
Dare la definizione di insieme semplicemente connesso. Fornire esempi di
insiemi di insiemi semplicemente connessi e di insiemi non semplicemente connessi, in $\R^2$ e in $\R^3$.

\medskip
\begin{large}
\textbf{Soluzione}
\end{large}


\section*{Quesito 2}
\addcontentsline{toc}{section}{Quesito 2}
Dare le definizioni di insieme convesso e di insieme stellato rispetto a un
punto. In che relazione sono con gli insiemi semplicemente connessi? Fornire un esempio
di insiemi semplicemente connesso che non sia convesso e un altro che non sia stellato.

\medskip
\begin{large}
\textbf{Soluzione}
\end{large}


\section*{Quesito 3}
\addcontentsline{toc}{section}{Quesito 3}
Sotto quale ipotesi sull’insieme $A$, una forma differenziale chiusa in $A$ (un
campo vettoriale irrotazionale) è anche esatta in $A$ (conservativo)? Che succede se si
elimina tale ipotesi su $A$?

\medskip
\begin{large}
\textbf{Soluzione}
\end{large}


\section*{Quesito 4}
\addcontentsline{toc}{section}{Quesito 4}
Come si può rappresentare il bordo di un dominio regolare di $\R^2$ ? Sul bordo
sono ben definiti i versori tangente e normale?

\medskip
\begin{large}
\textbf{Soluzione}
\end{large}


\section*{Quesito 5}
\addcontentsline{toc}{section}{Quesito 5}
Cosa si intende per orientazione positiva della frontiera di un dominio regolare
di $\R^2$ ? E’ corretto dire che coincide con la frontiera percorsa in verso antiorario? Motivare
la risposta con degli esempi.

\medskip
\begin{large}
\textbf{Soluzione}
\end{large}


\section*{Quesito 6}
\addcontentsline{toc}{section}{Quesito 6}
Enunciare e dimostrare le formule di Gauss-Green in domini normali regolari
rispetto all’asse $x$ o rispetto all’asse $y$.

\medskip
\begin{large}
\textbf{Soluzione}
\end{large}


\section*{Quesito 7}
\addcontentsline{toc}{section}{Quesito 7}
Dedurre le formule di Gauss-Green in domini regolari che sono normali sia
rispetto all’asse $x$ che all’asse $y$.


\medskip
\begin{large}
\textbf{Soluzione}
\end{large}


\section*{Quesito 8}
\addcontentsline{toc}{section}{Quesito 8}
Enunciare le formule di Gauss-Green in domini regolari.

\medskip
\begin{large}
\textbf{Soluzione}
\end{large}


\section*{Quesito 9}
\addcontentsline{toc}{section}{Quesito 9}
Scrivere e dimostrare le formule che permettono di calcolare l’area di una
regione piana mediante le formule di Gauss-Green.

\medskip
\begin{large}
\textbf{Soluzione}
\end{large}


\section*{Quesito 10}
\addcontentsline{toc}{section}{Quesito 10}
Come si calcola l’area di una regione del piano racchiusa dal sostegno di una
curva di cui è data l’equazione polare? Perché?

\medskip
\begin{large}
\textbf{Soluzione}
\end{large}
