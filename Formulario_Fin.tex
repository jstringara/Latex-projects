\documentclass[a4paper,notitlepage]{report}%imposto la classe la dimensione


\usepackage[T1]{fontenc} % codifica dei font per l'italiano
\usepackage[utf8]{inputenc} % lettere accentate da tastiera
\usepackage[italian]{babel} % lingua del documento
\usepackage[shortlabels]{enumitem}%per fare elenchi ad hoc
\usepackage{amsmath}% per avere le formule matematiche fatte bene
\usepackage{amssymb}% per avere le formule matematiche fatte bene
\usepackage{amsthm}%per fare i teoremi in modo ordinato
\usepackage[big]{layaureo} %per avere i margini più stretti
\usepackage[table]{xcolor}%per colorare le tabelle
\usepackage{framed}% per riquadrare il testo
\usepackage{calrsfs}% per avere le lettere calligrafiche
\usepackage{empheq} %per fare le equazioni per bene
\usepackage{titling}% per modificare il titolo
\usepackage{graphicx} %per inserire le immagini 
\usepackage{dirtytalk}% per mettere più facilmente le virgolette
\usepackage{titlesec}
\usepackage{multicol}
\usepackage{blindtext}
\setlength{\columnsep}{1cm}


%ridefinisco i set di numeri
\newcommand{\C}{\mathbb{C}}
\newcommand{\N}{\mathbb{N}}%naturali
\newcommand{\R}{\mathbb{R}}%reali
\newcommand{\Z}{\mathbb{Z}} %relativi 

\setlist[itemize]{leftmargin=*}
\setlength\parindent{0pt}


\begin{document}

\section*{Formule}

\begin{multicols*}{2}


\section*{Indici Temporali}

    \[
        p(t,x)=\sum_{n=1}^N x_n(1+i(t,T_n))^{-(T_n-t)}    
    \]

    \subsection*{Duration e Convexity}
    \begin{align*}
        & DU(t,x) = \frac{ \sum^N (T_n-t) x_n (1+i(t,T_n))^{-(T_n-t)} }{p(t,x)} \\
        & DU^2(t,x) = \frac{ \sum^N (T_n-t)^2 x_n (1+i(t,T_n))^{-(T_n-t)} }{p(t,x)} \\
        & C(t,x) = \\
        & \quad  = \frac{ \sum (T_n-t) (T_n-t+1) x_n (1+i(t,T_n))^{-(T_n-t+2)} }{p(t,x)}
    \end{align*}
    Oppure con $e^{-y*(T_N-t)}$ al posto di \\ 
    $(1+i)^{-(Tn-t)}$\\
    Per portafoglio $\alpha,X$:
    \[
        DU(t,\alpha X) = \sum_{k=1}^K \frac{\alpha_k p_k}{\sum_{k=1}^K \alpha_k p_k} Du(t,x^k)
    \]

    \subsection*{Approssimazione}

        \subsubsection*{Composta}
        \begin{align*}
            & \Delta p(t,x) = -\frac{1}{1+i} DU(t,x) p(t,x) \Delta i + \\
            & \quad + 1/2 p(t,x) C(t,x) \Delta i^2
        \end{align*}

        \subsubsection*{Esponenziale}
        \begin{align*}
            & \Delta p(t,x) = -DU(t,x) p(t,x) \Delta y + \\
            & \quad + 1/2 p(t,x) DU^2(t,x) \Delta y^2
        \end{align*}
    \subsection*{Immunizzazione}

        \subsubsection*{Fisher-Weil}
        Unica uscita $L$ in $H>t$, per immunizzare $(V(t^+,x)\geq V(t^+,L))$ basta:
        \[
            \left\{\begin{array}{l}
                V(t,x) = V(t,L)\\
                DU(t,x)=H-t
            \end{array}\right.
        \]

        \subsubsection*{Redington}
        Sia $V(t,x) = V(t,y)$, per immunizzare $(V(t^+,x)\geq V(t^+,L))$ basta:
        \[
            \left\{\begin{array}{l}
                DU(t,x)=DU(t,y) \\
                DU^2(t,x) \geq DU^2(t,y)
            \end{array}\right.
        \]

\section*{Scelte in rischio}

    \subsection*{Premio per il rischio}
    $\rho_u(\tilde{x})$ t.c. $u(\underbrace{E[\tilde{x}]-\rho_u(\tilde{x})}_{Certo Equivalente})=E[u(\tilde{x})]$

    \subsection*{Coefficiente di avversione}
    $r^a_u(x)=-\frac{u''(x)}{u'(x)}$ e moltiplico per $x$

    \subsection*{Funzioni utilità}
    \begin{itemize}
        \item \textbf{Esponenziale} $-\frac{1}{a}e^{-xa}$ e $r_u^a(x)=a$
        \item \textbf{Quadratica} $x-\frac{b}{2}x^2$ e $r_u^a(x)=\frac{b}{1-bx}$
    \end{itemize}

\section*{Media-Varianza}

    \subsection*{Per due titoli rischiosi}
    $(w,1-w)$
    \[
        \left\{\begin{array}{l}
            \tilde{r}=w\tilde{r}_1+(1-w)\tilde{r}_2 \\
            E[\tilde{r}]=wE[\tilde{r}_1]+(1-w)E[\tilde{r}_2] \\
            \sigma^2(\tilde{r}) = w^2\sigma^2(\tilde{r}_1) + (1-w)^2\sigma^2(\tilde{r}_2)\\
            \quad +2w(1-w)\rho \sigma_1\sigma_2
        \end{array}\right.
    \]

\section*{Scelte di Portafoglio}

    Massimiziamo utilità attesa $E[u(\tilde{x})]$, abbiamo un titolo sicuro $r_f$

    \begin{align*}
        & \tilde{W} =  xr_f +\sum_{n=1}^N w_n (\tilde{r}_n-r_f) \\
        & E[u'(xr_f+w(\tilde{r}-r_f))(\tilde{r}-r_f)] = 0
    \end{align*}
    
    \subsection*{Con un titolo}

    \subsubsection*{Quadratica}
    \begin{align*}
        & xr_f+w(E[\tilde{r}]-r_f)-\frac{b}{2} [(xr_f+w(E[\tilde{r}]-r_f))^2+\\
        &\quad w^2\sigma^2(\tilde{r})] \\
        & E[(1-b(xr_f + w (\tilde{r}-r_f) ) (\tilde{r}-r_f)] \\
        & w^* = \frac{(1-bxr_f) E[\tilde{r}-r_f]}{ b (\sigma^2(\tilde{r} + (E[\tilde{r}]-r_f)^2))}
    \end{align*}

    \subsubsection*{Esponenziale + Normali}
    Massimizzare $E[u(\tilde{w})]$ equivale a massimizzare l'esponente (lognormale)
    \begin{align*}
        & E[\tilde{W}]-\frac{a}{2}\sigma^2(\tilde{W})\\
        & w^* = \frac{1}{a\sigma^2(\tilde{r})}(E[\tilde{r}]-r_f)
    \end{align*}

    \subsection*{Con più titoli}
    Vettore media $e$ e matrice Varianza $V$

    \subsubsection*{Quadratica}
    \begin{align*}
        & E[(1-b\tilde{W}) (\tilde{r}_n-r_f)]=0 \quad \forall n \in \{ 1,\dots,N \} \\
        & w^* = \frac{(1-bxE[\tilde{r}])}{b} V^{-1}(e-r_f\mathbf{1})
    \end{align*}

    \subsubsection*{Esponenziale + Normali}
    Massimizzare $E[u(\tilde{w})]$ equivale a massimizzare l'esponente (lognormale)
    \begin{align*}
        & xr_f + w\cdot (e-r_f\mathbf{1})-\frac{1}{2} aw^TVw\\
        & w^* = \frac{1}{a} V^{-1}(e-r_f\mathbf{1})
    \end{align*}


\section*{Frontiera dei portafogli}

    \subsection*{Solo rischiosi}
        Vettore delle medie $e$ e matrice varianza $V$

        Bisogna calcolare i \textbf{coefficienti}:
        \begin{align*}
            & A = \mathbf{1}^T V^{-1} e \\
            & B = e^T V^{-1} e \\
            & C = \mathbf{1}^T V^{-1} \mathbf{1} \\
            & D = BC-A^2
        \end{align*}

        Ora ci scriviamo i vettori di frontiera $g$ e $h$:
        \begin{align*}
            & g = \frac{(BV^{-1}\mathbf{1} - AV^{-1}e)}{D} \\
            & h = \frac{(CV^{-1}e - AV^{-1}\mathbf{1})}{D}
        \end{align*}

        Possiamo scrivere i ptf della frontiera come:
        \[
            w^p = g+hE[\tilde{r}^p]   
        \]
    
    \subsubsection*{Covarianza tra portafogli in FP}
    \[
        Cov(\tilde{r}^q, \tilde{r}^p ) = \frac{C}{D} \left(E[\tilde{r}^p]-\frac{A}{C}\right) \left(E[\tilde{r}^q]-\frac{A}{C}\right) + \frac{1}{C}
    \]
    Da cui posso trovare la varianza

    \subsubsection*{Portafoglio a Varianza minima globale}
    \[
        w*=\frac{V^{-1}\mathbf{1}}{C}
    \]
    Da cui posso trovare la varianza

    \subsection*{Con titolo privo di rischio}
        Rendimento del titolo privo di rischio $r_f$\\
        Calcolo il coefficiente $H$:
        \[
            H = (e-r_f\mathbf{1})^T V^{-1} (e-r_f\mathbf{1}) = B-2Ar_f+Cr_f^2>0
        \]
        La frontiera dei portafogli $FP^*$ sono tutti i portafogli:
        \[
            w^p = V^{-1} (e-r_f\mathbf{1}) \frac{E[\tilde{r}^p]-r_f}{H}
        \]
        Per ottenere la varianza:
        \[
            \sigma^2(w^p) = { w^p}^T V w^p = \frac{(E[\tilde{r}^p]-r_f)^2}{H}
        \]

    \subsection*{Portafoglio tangente}
    E' il portafoglio comune a FP e FP*, o il portafoglio
    a Covarianza nulla con $w^p\in FP$ con $E[\tilde{w}^p] = r_f$
    \begin{align*}
        & w^e = V^{-1} \frac{(e-r_f\mathbf{1})}{\mathbf{1}^TV^{-1}(e-r_f\mathbf{1})}  \\
        & E[\tilde{r}^e] = \frac{Ar_f-B}{Cr_f-A}  
    \end{align*}


\section*{Modelli di Equilibrio}
    $N$ titoli e $S$ possibili stati. \\
    Vettore prezzi $q=(q_1,\dots,q_n)$,
    matrice dei dividendi $D\in\R^{S\times N}$,
    ricchezza negli stati $y=Dw\in\R^S$

    \subsection*{Completezza}
        Un mercato si dice completo se $rank(D)=S$ (con $N\geq S$).\\
        Di solito si usa $\det(D)\neq0$.

    \subsection*{State price}
    Prezzo $m_s$ che paga 1 nello stato $s$ e 0 negli altri.\\
    Dalla loro somma otteniamo il prezzo del titolo sicuro $m_0 = {\sum_{s=1}^S m_s}$
    ed il suo rendimento $r_f = \frac{1}{m_0}$\\
    Se il mercato è completo, possiamo invertire $D$ e fare:
    \[
        m=(D^T)^{-1} q   
    \]
    Per replicarle mi basta creare il portafoglio
    il cui valore sia uguale ad $m$ facendo:
    \[
        Dw=\mathbf{1}   
    \]

    \subsection*{Assenza di Arbitraggio}
    In assenza di opportunità di Arbitraggio troveremo che:
    \[
        m  = D^{-T}q > 0     
    \]


\section*{Arbitrage Pricing Theory}
    N titoli e un titolo sicuro $r_f$\\
    I rendimenti sono generati da una media e da $k$ fattori
    di rischio $\tilde{x}_i$
    \[
        \tilde{r}_n = \alpha + \sum_{h=1}^k b_{nh} \tilde{x}_h \quad \forall n  
    \]

    \subsection*{Portafogli Puri}
    Il portafoglio puro $w_s$ è il portafoglio tale che:
    \[
        \left\{\begin{array}{l}
            \sum_{n=1}^N w_s b_{ns} = 1 \\
            \sum_{n=1}^N w_j b_{nj} = 0 \quad \forall j\neq s \\
            \sum_{n=1}^N w_n  = 1
        \end{array}\right.  
    \]

    \subsection*{Portafoglio privo di rischio}
    Il portafoglio puro $w_s$ è il portafoglio tale che:
    \[
        \left\{\begin{array}{l}
            \sum_{n=1}^N w_s b_{ns} = 0 \quad \forall s \\
            \sum_{n=1}^N w_n  = 1
        \end{array}\right.  
    \]

    \subsection*{Portafoglio di replica}
    Dato un titolo:
    \[
        \tilde{r}_{N+1} = \alpha + \sum_{h=1}^k b_{N+1 \, k}\tilde{x}_h
    \]
    Il suo portafoglio di replica è il portafoglio tale che:
    \[
        \left\{\begin{array}{l}
            \sum_{n=1}^N w_r b_{nk} = b_{N+1 \, k} \quad \forall k\\
            \sum_{n=1}^N w_n  = 1
        \end{array}\right.  
    \]
    Allora, se non c'è arbitraggio:
    \[
        E[\tilde{r}_r] = \alpha    
    \]

    \subsection*{Verifica APT}
    Definiamo $\lambda_h = E[\tilde{r}_h]-r_f$ per ogni portafoglio puro.
    Se non c'è arbitraggio allora:
    \[
        \alpha-r_f = b_{N+1} \cdot \lambda    
    \]


\section*{Put e Call Option}
    Abbiamo un titolo rischioso (Stock) e uno non rischioso (Bond)
    i cui valori di mercato sono $S(t), B(t)$. \\
    Abbiamo delle opzioni esercitabili sul sottostante $S$,
    con strike-price $K$.
    \subsection*{Call}
    La call option restituisce:
    \[
        max[S-K,0]
    \]
    \subsection*{Put}
    La put option restituisce:
    \[
        max[0,K-S]
    \]

    \subsection*{Albero Binomiale}
        \begin{itemize}
            \item \textbf{Bond}: passa dal valore $B(0)$ in $t=0$ al valore $B(0)r_f$ in $t=1$
            \item \textbf{Stock}: il suo valore è una bernoulli di ragione $p$:
                \[
                    \left\{\begin{array}{l}
                        S(0)u \quad \pi = p \\
                        S(0)d \quad \pi = 1-p
                    \end{array}\right.
                \]
                Dove $u,d$ stanno per \textit{up, down},
                e perchè non ci sia arbitraggio: $u>r_f>d$.\\
                E per non arbitraggio:
                \[
                    \pi_u = \frac{r_f - d}{u - d}
                \]
        \end{itemize}
    
        \subsubsection*{Put e Call Europee}
        Per ricostruire il loro valore iniziale $F(S(0))$,
        possiamo procedere in due modi:
        \begin{itemize}
            \item \textbf{Con le probabilità:}
                \begin{enumerate}
                    \item Ricavo i payoff dalle foglie dell'albero di S(t)
                    \item Iterativamente ricostruisco ogni nodo come:
                        \[
                            F(S(t)) = \frac{1}{r_f}[F(S(t)u)\pi_u + F(S(0)d)\pi_d]    
                        \]
                \end{enumerate}
            \item \textbf{Con gli state price:}
                \begin{enumerate}
                    \item Ricostruisco i vari state price come:
                        \[
                            m = D^{-T}q  \text{ oppure } m_s = \frac{\pi_s}{r_f^T} 
                        \]
                    \item Ora prezzare le option equivale a prezzare
                    il portafoglio con gli stessi rendimenti, cioè a fare:
                        \[
                            F(S(0)) = \vec{F}(S(T)) \cdot m
                        \]
                \end{enumerate}
        \end{itemize}
        

        \subsubsection*{Put e Call Americane}
            Il loro valore iniziale $F(S(0))$ si può ricostruire
            solo nel caso binomiale
            \begin{enumerate}
                \item Ricostruire l'albero dei payoff per tutti i nodi
                \item Partendo dalle foglie ricostruisco il valore come media pesata
                \item Faccio il max tra questa e il payoff, e lo scrivo nell'albero e procedo
            \end{enumerate}

        \subsubsection*{Put-Call parity}
            Viene soddisfatta solo dalle Europee:
            \[
                P(t) + S(t) = C(t) + \frac{K}{ r_f^{T-t} }    
            \]
    
    
    \subsection*{Portafoglio di Replica}
    Cerchiamo un portafoglio costituito di B e S
    che replichi i payoff a $t=1$ del derivato. \\
    Nel caso di multiperiodale, prima dobbiamo ricostruire il valore
    a $t=1$. Poi basta risolvere il sistema lineare:
    \[
        \begin{bmatrix}
            r_f & S(0)u \\
            r_f & S(0)d  
        \end{bmatrix}
        \begin{bmatrix}
            x \\
            y  
        \end{bmatrix}
        =
        \begin{bmatrix}
            F(S(0)u) \\
            F(S(0)d)  
        \end{bmatrix}
    \]
    E il suo valore di mercato sarà:
    \[
        xB(0)+yS(0) = \frac{1}{r_f} \left[ F(S(0)u) \pi_u + F(S(0)d) \pi_d \right]
    \]


    \subsection*{Delta hedging}
    Voglio sapere quanto $\Delta$ devo vendere allo scoperto di S
    per neutralizzare le oscillazioni del Derivato. \\
    In pratica:
    \[
        \Delta = \frac{F(S(0)u) - F(S(0)d)}{S(0) (u-d)}    
    \]
    Anche questo modello nel caso di multiperiodale,
    necessita di ricondursi a $t=1$.
    
    \subsection*{Double checking}
    Notiamo che possiamo controllare di aver ottenuto dei risultati
    corretti se:
    \[
        \Delta = y
    \]

\section*{Value at Risk e Expected Shortfall}
    I dati che ci servono sono: \\
    $\mu$, media, $\sigma$, deviazione std (fai la radice),
    $h$ le unità di tempo passato e il livello $p$
    \begin{itemize}
        \item \textbf{Logaritmici Normali} 
            \begin{align*}
                &VaR = -h\mu+\sigma\sqrt{h}\Phi^{-1}(1-p) \\
                &ES = -h\mu+\frac{\sqrt{h}\sigma}{p} \phi(\Phi^{-1}(1-p))
            \end{align*}
        \item \textbf{Giornalieri}
            \begin{align*}
                &VaR = h(1-\mu)+\sigma\sqrt{h}\Phi^{-1}(1-p) \\
                &ES = h(1-\mu)+\frac{\sqrt{h}\sigma}{p} \phi(\Phi^{-1}(1-p))
            \end{align*}
        \item \textbf{Perdita}
            \begin{align*}
                &VaR = h\mu+\sigma\sqrt{h}\Phi^{-1}(1-p) \\
                &ES = h\mu+\frac{\sqrt{h}\sigma}{p} \phi(\Phi^{-1}(1-p))
            \end{align*}
    \end{itemize}

    \subsection*{Portafogli e Subadditività}
    Se abbiamo un portafoglio:\\
    $\mu = w\cdot e$ e $\sigma = \sqrt{w^TVw}$, inoltre, se vale
    la Subadditività:
    \[
        VaR(\tilde{R}) \leq \sum_{i=1}^N VaR(\tilde{R}_n)  
    \]
    L'ES è sempre subadditivo.

    \subsubsection*{Specchieto sulla normale}
    \begin{align*}
        & \Phi^{-1}(0.99) = 2.3263\\
        & \phi(\Phi^{-1}(0.99)) = 0.0266
    \end{align*}

    
\end{multicols*}











\end{document}




























