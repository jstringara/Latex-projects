\documentclass[a4paper,notitlepage]{report}%imposto la classe la dimensione


\usepackage[T1]{fontenc} % codifica dei font per l'italiano
\usepackage[utf8]{inputenc} % lettere accentate da tastiera
\usepackage[italian]{babel} % lingua del documento
\usepackage[shortlabels]{enumitem}%per fare elenchi ad hoc
\usepackage{amsmath}% per avere le formule matematiche fatte bene
\usepackage{amssymb}% per avere le formule matematiche fatte bene
\usepackage{amsthm}%per fare i teoremi in modo ordinato
\usepackage[big]{layaureo} %per avere i margini più stretti
\usepackage[table]{xcolor}%per colorare le tabelle
\usepackage{framed}% per riquadrare il testo
\usepackage{calrsfs}% per avere le lettere calligrafiche
\usepackage{empheq} %per fare le equazioni per bene
\usepackage{titling}% per modificare il titolo
\usepackage{graphicx} %per inserire le immagini 
\usepackage{dirtytalk}% per mettere più facilmente le virgolette
\usepackage{titlesec}
\usepackage{multicol}
\usepackage{blindtext}
\setlength{\columnsep}{1cm}


%ridefinisco i set di numeri
\newcommand{\C}{\mathbb{C}}
\newcommand{\N}{\mathbb{N}}%naturali
\newcommand{\R}{\mathbb{R}}%reali
\newcommand{\Z}{\mathbb{Z}} %relativi 
\newcommand{\p}{\mathbb{P}} %probabilità
\newcommand{\E}{\mathbb{E}} %media
\newcommand{\indep}{\perp \!\!\! \perp} %indipendenza

\setlist[itemize]{leftmargin=*}
\setlength\parindent{0pt}


\begin{document}

\chapter*{Formule e Teoremi utili}

\begin{multicols*}{2}

\section*{Probabilità}


    \subsection*{Distribuzione della trasformata}
    Sia $f_X(x)$ e $y=g(x)$.
    \begin{itemize}
        \item Se $g$ crescente: $F_Y(y) = F_X(g^{-1}(y))$
        \item Se $g$ decrescente: $F_Y(y) = 1 - F_X(g^{-1}(y))$
        \item Se $g^{-1}$ derivabile: \\
            $f_Y(y) = f_X(g^{-1}(y)) |\dot{g}^{-1}(y)| $
    \end{itemize}


    \subsection*{Varianza}
    \[
        Var[X] = \E[(X-\E[X])^2] = \E[X^2]-\E[X]^2    
    \]


    \subsection*{Marginali, Congiunte e Condizionate}
    \begin{align*}
        & f_{x,y} = f_{x|y}f_y \\
        & f_x = \int f_{x,y} dy
    \end{align*}


    \subsection*{Media e Varianza Condizionati}
    \begin{align*}
        & \E[X] = \E[\E[X|Y]] \\
        & Var[X] = \E[Var[X|Y]] + Var[\E[X|Y]]
    \end{align*}


    \subsection*{LFGN}
    Sia $X_1, \dots, X_n$ iid con $\mu, \sigma^2$, allora:\\
    $\bar{X}_n \overset{q.c.}{\to} \mu$


    \subsection*{Teorema centrale del limite}
    Sia $X_1, \dots, X_n$ iid con $\mu, \sigma^2$, allora:\\
    $\sqrt{n}(\bar{X}_n - \mu) \overset{L}{\to} N(0,\sigma^2)$


    \subsection*{Metodo delta}
    Sia $X_1, \dots, X_n$ iid con $\mu, \sigma^2$, tali che:\\
    $\sqrt{n}(\bar{X}_n - \mu) \overset{L}{\to} N(0,\sigma^2)$ \\
    Prendiamo una funzione $g(x)$:
    \begin{itemize}
        \item Se $g'(x)\neq0$: \\
            $\sqrt{n}(g(\bar{X}_n) - g(\mu)) \overset{L}{\to} N(0,\sigma^2g'(\mu)^2)$
        \item Se Se $g'(x)=0$: \\
        $\sqrt{n}(g(\bar{X}_n) - g(\mu)) \overset{L}{\to} \frac{\sigma^2}{2} g''(\mu)\chi^2(1)$
    \end{itemize}


\section*{Statistiche sufficienti}


    \subsection*{Definizione}
    Una statistica $T$ è sufficiente per $\theta$ se:\\
    $f(\vec{x}|T=t) \indep \theta \quad \forall t$


    \subsection*{Teorema di Fattorizzazione}
    Data la congiunta $f(\vec{x},\theta)$, $T(x)$ è suff se:\\
    $f(\vec{x},\theta) = h(\vec{x}) g(T(x),\theta)$\\
    Questo vale anche per trasformazioni \textbf{biunivoche} di $T$


    \subsection*{FE}
    Se ho una distribuzione della FE:\\
    $f(\vec{x},\theta) = h(\vec{x}) c(\theta) \exp\left\{ \sum_{i=1}^k w_i(\theta) t_i(x) \right\}$ \\
    Allora $T=(\sum_j t_1(X_j), \dots, \sum_j t_k(X_j) )$ è sufficiente


\section*{Statistiche sufficienti e minimali}







\end{multicols*}


\end{document}

























