\sheet


%==== Question 1 ====%

\question
Write the definition of complete measure space. State the theorem concerning the existence of the completion of a measure space. Give just an idea of the proof.

\subsection*{Solution}

\subsection{Complete measure space}
A measure space $(X,\A,\mu)$ is said to be complete if $\tau_{\mu}\subseteq\A$

\subsection{Existence of the completion}
Let $(X,\A,\mu)$ be a measure space. \provdef{$\bar{\A}, \bar{\mu}$}
\begin{align*}
    \bar{\A}  & =\{ E\subseteq X: \exists F,G \in \A \text{ s.t. } F\subseteq E \subseteq G \; \mu(G \setminus F) =0 \} \\
    \bar{\mu} & : \bar{\A}\to\Rcomppos,\quad \bar{\mu}(E) \coloneqq \mu(F)
\end{align*}
then:
\begin{enumerate}
    \item $\bar{\A}$ is a $\salg$ , $\bar{\A} \supseteq \A$
    \item $\bar{\mu}$ is a complete measure, $\bar{\mu}|_{\A}=\mu$
\end{enumerate}
and the triplet $(X,\bar{A}, \bar{\mu})$ is a complete measure space and is called the completion of $(X,\A,\mu)$, i.e. it the smallest (w.r. to inclusion) complete measure space that cointains $(X,\A,\mu)$
\subsection*{Sketch of proof}
We  must prove two things:
\begin{itemize}
    \item \textbf{First:} that $\bar{\A}$ is a $\salg$ and that it contains $\A$, the latter is trivial since $\forall A\in\A \quad A\subseteq A\subseteq A \implies A\in\bar{\A}$ while the former is quite hardous so we shall just assume it to be true.
    \item \textbf{Second:} that $\bar{\mu}$ is a complete measure and $\bar{\mu}|_{\A}=\mu$.\\
          The latter is trivial (see above). We can also easily prove that it is a measure:
          \begin{enumerate}[i)]
              \item $\bar{\mu}(\emptyset)=\mu(\emptyset)=0$ since the only set contained inside $\emptyset$ is $\emptyset$ itself, as the container set we may take any zero set measure inside $\A$.
              \item that $\s{additivity}$ holds is clear since for any disjoint sequence $\seq{E}\subseteq\bar{\A}$ we may construct two sequences:
                    \[
                        \left\{ \begin{array}{l}
                            \seq{F}, \; F_k \subseteq E_k \\
                            \seq{G}, \; G_k \supseteq E_k
                        \end{array} \right. \forall k\in\N \text{ s.t. } \mu(G_k\setminus F_k) = 0
                    \]
                    Let us note the following:
                    \begin{itemize}
                        \item $\seq{F}$ is also disjoint because $\seq{E}$ is disjoint.
                        \item Moreover:
                              \begin{align*}
                                   & \bigcup_{k=1}^{\infty} F_k \subseteq \bigcup_{k=1}^{\infty} E_k \subseteq \bigcup_{k=1}^{\infty} G_k                                                                                                 \\
                                   & \bigcup_{k=1}^{\infty} G_k \setminus \bigcup_{k=1}^{\infty} F_k \subseteq \bigcup_{k=1}^{\infty} (G_k \setminus F_k )                                                                                \\
                                   & \mu\left(\bigcup_{k=1}^{\infty} G_k \setminus \bigcup_{k=1}^{\infty} F_k \right) \leq \mu\left(\bigcup_{k=1}^{\infty} (G_k \setminus F_k )\right) \leq \sum_{k=1}^{\infty} \mu(G_k\setminus F_k) = 0
                              \end{align*}
                              The last inequality is true thanks to the $\s{subadditivity}$ and monotonicty of $\mu$.
                    \end{itemize}
                    Thus we can say that:
                    \[
                        \bar{\mu}\left( \bigcup_{k=1}^{\infty} E_k \right) = \mu \left( \bigcup_{k=1}^{\infty} F_k \right) = \sum_{k=1}^{\infty} \mu(F_k) = \sum_{k=1}^{\infty} \bar{\mu}(E_k)
                    \]
          \end{enumerate}
          thus $\bar{\mu}$ is a measure.\\
          Let us prove that $\bar{\mu}$ is complete.
          Let $E_1 \in X$ and $E_2 \in \bar{\A}$ such that $\bar{\mu}(E_2)=\mu(F_2)=0$ and $E_1 \subseteq E_2$, let us note that:
          \[
              \left\{ \begin{array}{l}
                  \mu(G_2) = \xcancel{\mu(G_2\setminus F_2)}^0+\xcancel{\mu(F_2)}^0 \\
                  \mu(G_2 \setminus \emptyset) = \mu(G_2) - 0                       \\
                  \emptyset \subseteq E_1 \subseteq G_2
              \end{array} \right. \implies E_1\in \bar{\A}, \; \bar{\mu}(E_1)=\mu(\emptyset) = 0
          \]
          thus any negligible set is also a zero measure set and $\bar{\mu}$ is complete.
\end{itemize}

%==== Question 2 ====%

\question
Write the definition of outer measure. State and prove the theorem concerning generation of
outer measure on a general set $X$, starting from a set $K \in\Parts{X}$, containing $\emptyset$, and a function
$\nu : K \to \Rcomppos, \; \nu(\emptyset) = 0$. Intuitively, which is the meaning of $(K, \nu)$?

\subsection*{Solution}

\subsection{Outer measure}\label{outer:def}
We say that a function: $\mu^*:\Parts{X}\to\Rcomppos$ (where $X$ is any set) is an outer measure if:
\begin{enumerate}[i)]
    \item $\mu^*(\emptyset)=0$
    \item \label{outer:mono}$E_1\subseteq E_2 \implies \mu^*(E_2) \leq \mu^*(E_2)$
    \item \label{outer:sub}$\mu^*\left( \bigcup_{k=1}^{\infty} E_k \right) \leq \sum_{k=1}^{\infty} \mu^*(E_k)$
\end{enumerate}

\subsection{Generation of an outer measure} \label{outer:gen}
Let $K\subseteq\Parts{X}, \, \emptyset\in K, \: \nu:K\to\Rcomppos, \; \nu(\emptyset)=0$, then we can generate an outer measure $\mu^*$ on $X$ defined as:
\[
    \left\{ \begin{array}{l}
        \mu^*(E) \coloneqq \inf \left\{ \sum_{k=1}^{\infty} \nu(I_k) : E\subseteq \bigcup_{k=1}^{\infty} I_k,\; \seq{I}\subseteq K \right\} , \text{ if } E \text{ can be covered by a countable union of sets } I_n\in K. \\
        \mu^*(E) \coloneqq +\infty, \text{ otherwise.}
    \end{array} \right.
\]

\begin{proof}
    Let us verify that such a $\mu^*$ meets the definition of outer measure (\ref{outer:def}):
    \begin{enumerate}[i)]
        \item $\emptyset\in K$, $0\leq\mu^*(\emptyset)\leq\nu(\emptyset)=0$ by the definition of $\mu^*$.
        \item $E_1\subseteq E_2$, we have two possible cases
              \begin{itemize}
                  \item if there exists a countable covering of $E_2$ then it is also a covering of $E_1$ and from the definitio of $\mu^*$ it follows that:
                        \[
                            \mu^*(E_1) \leq \mu^*(E_2)
                        \]
                  \item if there is no countable covering of $E_2$ then:
                        \[
                            \mu^*(E_1) \leq \mu^*(E_2) = +\infty
                        \]
              \end{itemize}
        \item this condition is obviously met if:
              \[
                  \sum_{k=1}^{\infty} \mu^*(E_k) = +\infty
              \]
              otherwise if we suppose that:
              \[
                  \sum_{k=1}^{\infty} \mu^*(E_k) < +\infty
              \]
              thus $\mu^*(E_k)<+\infty$ $\forall k\in\N$, by the definition of $\mu^*$ and $\inf$:
              \[
                  \forall \epsilon>0, \; \forall n\in\N \quad \exists \{ I_{n,k} \} \subseteq K
              \]
              such that:
              \[
                  E_n \subseteq \bigcup_{k=1}^{\infty} I_{n,k} \quad \text{ and } \quad \mu^*(E_n)+\frac{\epsilon}{2^n} > \sum_{k=1}^{\infty} \nu(I_{n,k})
              \]
              Now, since:
              \[
                  \bigcup_{n=1}^{\infty} E_n \subseteq \bigcup_{n,k=1}^{\infty} I_{n,k}, \quad \{ I_{n,k} \} \subseteq K
              \]
              it clearly follows that:
              \[
                  \mu^*(\bigcup_{n=1}^{\infty} E_n) \leq \sum_{n=1}^{\infty} \sum_{k=1}^{\infty} \nu(I_{n,k}) < \sum_{n=1}^{\infty}\mu^*(E_n) + \epsilon\cdot\xcancel{\sum_{n=1}^{\infty} \frac{1}{2^n}}^1
              \]
              because $\epsilon$ is arbitrary, we have the cocnlusion.
    \end{enumerate}
\end{proof}

The intuitive meaning $(K,\nu)$ is that $K$ is a special class of sets in $X$ and $\nu$ is a function that assigns a value to each set in $K$. On the other hand $\nu$ can be any real valued positive function, thus it is not necessary to be a measure.

%==== Question 3 ====%

\question
What is the Caratheodory condition? How can it be stated in an equivalent way? Prove it.

\subsection*{Solution}

\subsection{Caratheodory condition} \label{CarEq}
Let $\mu^*$ be an outer measure on a set $X$, then we say that $E\subset X$ is $\mu^*$-measurable if:
\[
    \mu^*(Z) = \mu^*(Z\cap E) + \mu^*(Z\setminus E) \quad \forall Z\in X
\]

\subsection{Equivalent statement}\label{CarIneq}
Let $\mu^*$ be an outer measure on a set $X$, then we say that $E\subset X$ is $\mu^*$-measurable if:
\[
    \mu^*(Z) \geq \mu^*(Z\cap E) + \mu^*(Z\setminus E) \quad \forall Z\in X
\]
\begin{proof}
    It is enough to note that $\forall E\subseteq X$ we have:
    \[
        Z = (Z\cap E) \cup (Z \cap E^\complement) \quad \forall Z\in X
    \]
    and thus by the subadditivity of $\mu^*$ (\ref{outer:sub}) we get:
    \[
        \mu^*(Z) \leq \mu^*(Z\cap E) + \mu^*(Z\setminus E) \quad \forall Z\in X
    \]
    and we may combine this inequality with the other to yield an equality.
\end{proof}

%==== Question 4 ====%

\question
Can it exist a set of zero outer measure, which does not fulfill the Caratheodory condition? Prove it.

\subsection*{Solution}

\subsection{All zero measure sets are in \texorpdfstring{$\Leb$}{L}} \label{zerosetsaremeas}
There cannot exist such a set $E$ because all sets of zero aouter measure meet the Caratheodory Inequality (\ref{CarIneq}).

\begin{proof}\label{outer:zeromeas}
    Indeed $\forall Z \subseteq X$ by the monotonicty of $\mu^*$ (\ref{outer:mono}) we have:
    \[
        \mu^*(\underbrace{Z\cap E}_{\subseteq E}) + \mu^*(\underbrace{Z\setminus E}_{\subseteq Z}) \leq \xcancel{\mu^*(E)}^0 + \mu^*(Z)
    \]
\end{proof}


%==== Question 5 ====%

\question
State the theorem concerning generation of a measure as a restriction of an outer measure.

\subsection*{Solution}
\subsection{Generation of a measure from an outer measure}\label{meas:gen}
\provdef[$\mathcal{L}$]
\[
    \mathcal{L} \coloneqq \{ E\subseteq X : \; E \text{ is } \mu^*-\text{measurable } \}
\]
where $\mu^*$ is an outer measure on $X$, then:
\begin{enumerate}[i)]
    \item the collection $\Leb$ is a $\salg$
    \item $\mu^* |_{\Leb}$ is a complete measure on $\Leb$
\end{enumerate}

%==== Question 6 ====%

\question
Show that the measure induced by an outer measure on the $\salg$ of all sets fulfilling the
Caratheodory condition is complete.

\subsection*{Solution}
\subsection{Generation of a measure from an outer measure (proof of completeness)}
Let us see that such a measure as the one described in the previous question is complete. Let $\mu^*$ be an outer measure on $X$ and $\Leb$ the $\salg$ of all sets fulfilling the Caratheodory condition. Let $\mu$ be the measure induced by $\mu^*$ on $\Leb$ ($\mu=\mu^* |_{\Leb}$).
\begin{proof}
    Let $N\in\Leb$ such that $\mu(N)=\mu^*(N)=0$ and let $E\subseteq N$.\\
    By monotonicty of $\mu^*$ (\ref{outer:mono}):
    \[
        0\leq \mu^*(E)\leq \mu^*(N)=0 \implies \mu^*(E)=0
    \]
    thus by the lemma seen in \ref{outer:zeromeas} we get that $E\in\Leb$ and so $\Leb$ is complete.
\end{proof}

%==== Question 7 ====%

\question
Describe the construction of the Lebesgue measure in $\R$ and in $\R^n$.

\subsection*{Solution}

\subsection{Construction of the Lebesgue measure on \texorpdfstring{$\R$}{R}}
Let $I$ be a family of open, bounded intervals in $\R$:
\[
    I \coloneqq \{ (a,b) : a,b\in\R, a\leq b \}
\]
Let us note that $\emptyset\in I$.\\
Now let us consider a function $\lambda_0$:
\begin{align*}
     & \lambda_0 : I \to \R_+   \\
     & \lambda_0 (\emptyset) =0 \\
     & \lambda_0 ((a,b)) = b-a
\end{align*}
Here we take $X=\R$, $(K,\nu)=(I,\lambda_0)$ and construct the outer Lebesgue measure $\lambda^*$ as seen above (\ref{outer:gen}):
\[
    \lambda^*(E) \coloneqq \left\{ \begin{array}{l}
        \inf \left\{ \sum_{n=1}^{\infty} \lambda_0(I_n) \, : \quad E\subseteq \bigcup_{n=1}^{\infty} I_n ,\; \seq{I}\subseteq I \right\}, \quad \forall E\subseteq\R \text{ s.t. } E \text{ has a countable covering }\seq{I}\subseteq I \\
        +\infty, \text{ otherwise}
    \end{array}\right.
\]
The corresponding $\salg$ is the Lebesgue $\salg$ $\Leb(\R)$ and now we define the Lebesgue measure $\lambda$ as the measure generated by the outer Lebesgue measure (as seen in \ref{meas:gen}):
\[
    \lambda \coloneqq \lambda^*|_{\Leb(\R)}
\]

\subsection{Construction of the Lebesgue measure on \texorpdfstring{$\R^n$}{Rn}}
Analogously to what we have seen above we first define an outer measure and then a (complete) measure but we take:
\[
    I^n = \left\{ \bigtimes_{k=1}^{n} (a_k,b_k): \; a_k,b_k\in\R, \; a_k\leq b_k  \right\}
\]
and accordingly we define:
\begin{align*}
     & \lambda_0^n : I^n \to \R_+                                                         \\
     & \lambda_0^n (\emptyset) = 0                                                        \\
     & \lambda_0^n \left( \bigtimes_{k=1}^{n} (a_k,b_k) \right) = \prod_{k=1}^n (b_k-a_k)
\end{align*}
and therefore we take $X=\R^n$ and $(K,\nu)=(I^n,\lambda_0^n)$, we define the outer Lebesgue measure $\lambda^{*,n}$ on $\R^n$ and the Lebesgue $\salg$ $\Leb(\R^n)$ and finally we construct the n-dimensional Lebesgue measure as:
\[
    \lambda^n \coloneqq \lambda^{*,n} |_{\Leb(\R^n)}
\]

%==== Question 8 ====%

\question
Prove that any countable subset $E\subset\R$ is Lebesgue measurable and $\lambda(E) = 0$.

\subsection*{Solution}

\subsection{All countable sets are \texorpdfstring{$\Leb$}{L}-measurable and \texorpdfstring{$\lambda(E)=0$}{l(E)=0}}
Any countable subset $E\subset\R$ is $\Leb$-measurable and $\lambda(E)=0$
\begin{proof}
    Let $a\in\R$, clearly $\{ a \} \subseteq (a-\epsilon, a]$ $\forall \epsilon >0$, thus by the definition of $\lambda^*$:
    \[
        \lambda^*(\{ a \}) \leq \lambda^*( (a-\epsilon, a] )  = \epsilon \to 0  \implies \{ a \}\in\Leb
    \]
    Now if E is countable we may write as follows:
    \[
        E = \bigcup_{n=1}^{\infty} \{ a_n \} \quad a_n\in\R, \; n\in\N
    \]
    and so by monotonicty (\ref{outer:mono}):
    \[
        0 \leq \lambda^*(E) = \lambda^*\left( \bigcup_{n=1}^{\infty} \{ a_n \} \right) \leq \sum_{n=1}^{\infty} \lambda^*(a_n) = 0
    \]
    thus $\lambda^*(E)=0 \implies E\in \Leb$ by the lemma seen above (\ref{outer:zeromeas})
\end{proof}

%==== Question 9 ====%

\question
Show that $\B[\R] \subseteq \Leb(\R)$. Is the inclusion strict? Which is the relation between $(\R,\Leb(\R), \lambda)$ and
$(\R, \B[\R], \lambda)$?

\subsection*{Solution}

\subsection{\texorpdfstring{$\B[\R] \subseteq \Leb(\R)$}{B(R) is included in L(R)}}
\begin{proof}
    Since $\B[\R]=\sigma_0((a,+\infty))$ it is enough to show that $(a,+\infty)\in\Leb(\R)$. We already know from above that all bounded intervals belong to $\Leb(\R)$. \\
    Now, let $A\subseteq\R$ be any set. We assume $a\notin A$, otherwise we would replace $A$ with $A\setminus\{a\}$ and this would leave the Lebesgue outer measure unchanged. Furthermore $(a,+\infty)\in\Leb(\R) \iff (a,+\infty)$ satisfies the Caratheodory Condition (\ref{CarIneq}):
    \[
        \lambda^*(A_1)+\lambda^*(A_2)\leq\lambda^*(A) \label{a}
    \]
    where $A_1 = A\cap (-\infty,a)$ and $A_2 = A \cap (a,+\infty)$.\\
    Since $\lambda^*(A)$ is defined as an $\inf$, to verify the above, it is necessary and sufficient to show that for \textbf{any countable collection} $\seq{I}$ of \textbf{open bounded} intervals that \textbf{covers} $A$ we have that:
    \[
        \lambda^*(A_1)+\lambda^*(A_2)\leq \sum_{k=1}^{\infty} \lambda_0(I_k)
    \]
    For every $k\in\N$ we define:
    \begin{align*}
        I_k' \coloneqq I_k \cap (-\infty, a) \\
        I_k'' \coloneqq I_k \cap (a,+\infty)
    \end{align*}
    then:
    \[
        I_k' \cap I_k'' = \emptyset (\text{disjoint}) \implies \lambda_o(I_k) = \lambda_0(I_k') + \lambda_0 (I_k'')
    \]
    Let us note that $\seq{I'}$ is a countable cover for $A_1$ and $\seq{I''}$ is a countable cover for $A_2$.
    Hence:
    \begin{align*}
        \lambda^*(A_1)=\sum_{k=1}^{\infty} \lambda_0(I_k') \\
        \lambda^*(A_2)=\sum_{k=1}^{\infty} \lambda_0(I_k'')
    \end{align*}
    therefore:
    \[
        \lambda^*(A_1)+\lambda^*(A_2)\leq \sum_{k=1}^{\infty} \lambda_0(I_k') + \sum_{k=1}^{\infty} \lambda_0(I_k'') = \sum_{k=1}^{\infty} \lambda_0(I_k)
    \]
    which equivalento to the condition above.
\end{proof}

\subsection{\texorpdfstring{$\B[\R] \subsetneqq \Leb(\R)$}{B(R) is strictly included in L(R)}}
The inclusion demonstrated above can be shown to be strict. A counterexample can be produced (see \href{https://math.stackexchange.com/questions/141017/lebesgue-measurable-set-that-is-not-a-borel-measurable-set}{\color{blue}{here}}) but it is quite pathological.

\subsection{Relation between \texorpdfstring{$(\R,\Leb(\R),\lambda)$}{(R,L(R),l)} and \texorpdfstring{$(\R,\B[\R],\lambda)$}{(R,B(R),l)}}
$(\R,\Leb(\R),\lambda)$ is the completion of $(\R,\B[\R],\lambda|_{\B[\R]})$. Indeed as we have shown above $\B[\R]$ is not a complete $\salg$ while $\Leb(\R)$ is.

%==== Question 10 ====%

\question
Is the translate of a measurable set measurable?

\subsection*{Solution}

\subsection{The translate of a measurable set is measurable}
The translate of a measurable set is also measurable. \\
Let us see a simple example: let $(a,b)$ be an interval and $(a+h,b+h)$ its translate.
\begin{align*}
     & \lambda((a,b)) = b-a                   \\
     & \lambda((a+h,b+h)) = (b+h)-(a+h) = b-a
\end{align*}

%==== Question 11 ====%

\question
Write the excision property and prove it. Write and prove (partially) the theorem concerning
the regularity of the Lebesgue measure on $\R$.

\subsection*{Solution}
\subsection{Excision property}\label{ExcProp}
If $A\in\Leb(\R)$, $\lambda^*(A)\leq +\infty$ and $A\subseteq B$, then:
\[
    \lambda^*(B\setminus A) = \lambda^* (B) - \lambda^*(A)
\]
\begin{proof}
    Since $A\in\Leb(\R)$ we can use the Caratheodory equality (\ref{CarEq}) using $Z=B$, $E=A$:
    \[
        \lambda^*(B) = \lambda^*(\underbrace{B\cap A}_{=A \; (A\subseteq B)}) + \lambda^* (B\setminus A)
    \]
    so, since $\lambda^*(A)\leq +\infty$ we may write:
    \[
        \lambda^*(B\setminus A) = \lambda^*(B)-\lambda^*(A)
    \]
\end{proof}

\subsection{Regularity of the Lebesgue Measure}
Let $E\subseteq\R$, \tfae
\begin{enumerate}[i)]
    \item\label{LebReg:1} $E\in\Leb(\R)$
    \item\label{LebReg:2} $\forall \epsilon >0$ $\exists A \subseteq \R$ open s.t.
          \[
              E\subseteq A \quad \lambda^*(A\setminus E) < \epsilon
          \]
    \item\label{LebReg:3} $\exists G \subseteq \R$ in the class $G_{\delta}$ (countable intersections of open sets) s.t.
          \[
              E\subseteq G \quad \lambda^*(G\setminus E)=0
          \]
    \item\label{LebReg:4} $\forall \epsilon >0$ $\exists C \subseteq \R$ closed s.t.
          \[
              C\subseteq E \quad \lambda^*(E\setminus C) < \epsilon
          \]
    \item\label{LebReg:5} $\exists F \subseteq \R$ in the class $F_{\delta}$ (countable unions of closed sets) s.t.
          \[
              F\subseteq E \quad \lambda^*(E\setminus F)=0
          \]
\end{enumerate}
\begin{proof}
    Let us give a (partial) proof:\\
    \begin{itemize}
        \item
              $(\ref{LebReg:1})\implies(\ref{LebReg:2})$: if $E\in\Leb(\R)$, $\lambda(E)<+\infty$ then by definition of outer measure (\ref{outer:def}):
              \[
                  \forall \epsilon >0 \; \exists\seq{I}\text{ that covers } E \text{ and } \sum_{k=1}^{\infty} \lambda_0(I_k) < \lambda^*(E)+\epsilon
              \]
              Let us now define the set $O$:
              \[
                  O\coloneqq \bigcup_{k=1}^{\infty} I_k, \; O \text{ is open}, \; E\subseteq O
              \]
              and so we may write:
              \begin{align*}
                   & \lambda^*(O) \overset{sub-add \; (\ref{outer:sub})}{\leq} \sum_{k=1}^{\infty} \lambda_0(I_k) < \lambda^*(E)+\epsilon \\
                   & \implies \lambda^*(O)-\lambda^*(E) < \epsilon
              \end{align*}
              and by the Excision property (\ref{ExcProp}) ($E\in\Leb(\R), \; \lambda^*(E)<+\infty$):
              \[
                  \lambda^*(O\setminus E) = \lambda^*(O)-\lambda^*(E) < \epsilon
              \]
              and so we have obtained the second statement (\ref{LebReg:2}).
        \item
              $(\ref{LebReg:2}) \implies (\ref{LebReg:3})$, $\forall k\in\N$ we choose $O_k \supseteq E$ open for which:
              \[
                  \lambda^*(O_k\setminus E) < \frac{1}{k}
              \]
              and then define:
              \[
                  G = \bigcap_{k=1}^{\infty} O_k \implies G\in G_{\delta}, \; G\supseteq E
              \]
              Moreover $\forall k\in \N$:
              \[
                  G\setminus E \subseteq O_k\setminus E
              \]
              so by monotonicty (\ref{outer:mono}):
              \[
                  \lambda^*(G\setminus E) \leq \lambda^*(O_k\setminus E) < \frac{1}{k}
              \]
              let us apply a limit $k\to\infty$ to both sides:
              \[
                  \lambda^*(G\setminus E) = 0
              \]
        \item
              $(\ref{LebReg:3}) \implies (\ref{LebReg:1})$, let us note that $G\setminus E\in\Leb(\R)$ since $\lambda^*(G\setminus E)=0 $ by lemma \ref{zerosetsaremeas} and:
              \begin{align*}
                   & G\in\Leb(\R) \text{ since } G\in G_{\delta} \subseteq \B[\R] \subseteq \Leb(\R)                  \\
                   & \implies E = \underset{\in\Leb}{G} \cap (\underset{\in\Leb}{G \setminus E})^\complement \in \Leb
              \end{align*}

    \end{itemize}
\end{proof}

%==== Question 12 ====%

\question
Is it true that any subset $E\subseteq\R$ is $\Leb$-measurable? Is it possibile to find two disjoint sets
$A,B\subset\R$ for which $\lambda^*(A \cup B) < \lambda^*(A) + \lambda^*(B)$? Why?

\subsection*{Solution}

\subsection{Vitali's non-measurable sets}
Any measurable set $E\subseteq \R$ with $\lambda(E)>0$ contains a subset that fails to be measurable. \\
Therefore there exist subsets of $\R$ that are not $\Leb$-measurable.

\subsection{Disjoints sets for which \texorpdfstring{$\lambda^*(A\cup B) < \lambda^*(A)+\lambda^*(B)$}{the measure of the union is less than the sum of the measure}}
There are disjoint sets $A,B\subseteq\R$ for which:
\[
    \lambda^*(A\cup B) < \lambda^*(A)+\lambda^*(B)
\]

\begin{proof}
    Assume by contradiction that:
    \[
        \lambda^*(A\cup B) = \lambda^*(A)+\lambda^*(B) \quad \forall A,B \subseteq\R, \; A\cap B = \emptyset
    \]
    Now $\forall E,Z \subseteq \R$ we write:
    \[
        \lambda^*(\underbrace{Z\cap E}_{=A}) + \lambda^*(\underbrace{Z\cap E^{\complement}}_{=B}) = \lambda^*(\underbrace{Z}_{=A\cup B})
    \]
    thus any set $E$ would satisfy the Caratheodory condition (\ref{CarEq}) and be $\Leb$-measurable which is absurd since we know that Vitali's sets exist.
\end{proof}
