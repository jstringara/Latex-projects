\sheet

%>=====< Question 1 >=====<%

\question
What is the norm on $\mathcal{L}(X,Y)$? It satisfies two important equalities. Write and show them.

\subsection*{Solution}

\subsection{Linear continuous operators}
Let $X,Y$ be two normed spaces.
\[\mathcal{L}(X,Y)\coloneqq\{T:X\rightarrow Y \mbox{ s.t. $T$ is linear and continuous}\}\]

\subsection{Property of the norm on \texorpdfstring{$\mathcal{L}(X,Y)$}{the space of linear continuous operators}}
The norm on $\mathcal{L}(X,Y)$ satisfies two important equalities:
\begin{enumerate}
    \item If $T\in\mathcal{L}(X,Y)$, there $\exists\,M>0:\|T(x)\|_Y\leq M\quad\forall\,x\in X,\,\|x\|_X\leq 1$. \\[4pt]
    We have that $\displaystyle \sup_{x\in X,\|x\|_X\leq1}\|T(x)\|_Y$ is a norm on $\mathcal{L}(X,Y)$ and $(\mathcal{L}(X,Y),\|\cdot\|_{\mathcal{L}})$ is a normed space, 
    \[\|T\|_\mathcal{L}\coloneqq\sup_{\|x\|_X\leq1}\|T(x)\|_Y\]
    \item $\displaystyle\|T\|_\mathcal{L}=\sup_{\|x\|_X =1}\|T(x)\|_Y=\sup_{x\in X\setminus\{0\}}\frac{\|T(x)\|_Y}{\|x\|_X}$
    \begin{proof}
        \[\sup_{\|x\|_X\leq1}\|T(x)\|_Y\geq\sup_{\|x\|_X=1}\|T(x)\|_Y\qquad (*)\]
        If $\|x\|_X\leq1$, $x\ne0$ then
        \[\|T(x)\|_Y=\underbrace{\|x\|_X}_{\leq1}\bigg\|T\bigg(\frac{x}{\|x\|_X}\bigg)\bigg\|_Y\leq\bigg\|T\bigg(\frac{x}{\|x\|_X}\bigg)\bigg\|_Y\implies \sup_{\|x\|_X\leq1}\|T(x)\|_Y\leq\sup_{\|\xi\|=1}\|T(\xi)\|_Y\]
        \[\begin{aligned}
        &\implies \|T\|_\mathcal{L}\leq\sup_{\|x\|_X=1}\|T(x)\|_Y\\
        &\overset{(*)}{\implies} \|T\|_\mathcal{L}\geq\sup_{\|x\|_X=1}\|T(x)\|_Y
        \end{aligned}\quad\Bigg\rbrace \implies\|T\|_\mathcal{L}=\sup_{\|x\|_X=1}\|T(x)\|_Y\]
        Furthermore, 
        \[\frac{\|T(x)\|_Y}{\|x\|_X}=\bigg\|T\bigg(\frac{x}{\|x\|_X}\bigg)\bigg\|_Y\implies\sup_{x\in X\setminus\{0\}}\frac{\|T(x)\|_Y}{\|x\|_X}=\sup_{\|\xi\|=1}\|T(\xi)\|_Y=\| T\|_\mathcal{L}\]
    \end{proof}
\end{enumerate}

%>=====< Question 2 >=====<%

\question
Under which hypotheses on $X$ and $Y$ is $\mathcal{L}(X,Y)$ a Banach space?

\subsection*{Solution}

\subsection{Hypotesis for which \texorpdfstring{$\Leb(X,Y)$}{the space of linear operators} is a Banach space}
Let $X$ be a normed space and $Y$ be a Banach space. Then $\mathcal{L}(X,Y)$ is a Banach space.

%>=====< Question 3 >=====<%

\question
Let $\mathfrak{F}\subset\mathcal{L}(X,Y)$. For $\mathfrak{F}$ write the definition of pointwise and uniform boundedness. State and prove the UBP (or BS theorem).

\subsection*{Solution}

\subsection{Pointwise and uniform boundedness}
Let $X,Y$ be two Banach spaces and let $\mathfrak{F}\subset\mathcal{L}(X,Y)$.
\begin{itemize}
    \item $\mathfrak{F}$ is pointwise bounded $\iff$ $\forall\,x\in X\,$ $\,\exists\,M_x>0\,:\,$ $\,\displaystyle\sup_{T\in\mathfrak{F}}\|T(x)\|_Y\leq M_x$
    \item $\mathfrak{F}$ is uniformly bounded $\iff$ $\exists\,M>0\,:\,$ $\,\displaystyle\sup_{T\in\mathfrak{F}}\|T\|_\mathcal{L}\leq M$ $\iff$ $\,\displaystyle\sup_{T\in\mathfrak{F}}\sup_{\|x\|\leq1}\|T(x)\|_Y\leq M$ 
\end{itemize}
 
\subsection{Uniform Boundedness Principle (or Banach-Seinhaus Theorem)}
Let $X,Y$ be Banach spaces, $\mathfrak{F}\subset\mathcal{L}(X,Y)$. If $\mathfrak{F}$ is pointwise bounded, then $\mathfrak{F}$ is also uniformly bounded.
\begin{proof}
    $\forall\,n\in\N$ let $C_n\coloneqq\{x\in X: \|T(x)\|_Y\leq n\quad\forall\,T\in\mathfrak{F}\}$
    \begin{itemize}
        \item $C_n$ is closed.\\
        In fact, let $\{x_m\}\subset C_n$, $x_m\xrightarrow{m\rightarrow\infty}x_0\in X$. Therefore: \[\begin{aligned}
            &T(x_m)\xrightarrow[m\rightarrow\infty]{}T(x_0)\quad\forall\,T\in\mathfrak{F} \quad\implies\quad \|T(x_m)\|_Y\xrightarrow[m\rightarrow\infty]{}\|T(x_0)\|_Y\quad\implies\quad\|T(x_0)\|_Y \le n\\
            &\implies\quad x_0\in C_n \quad\implies\quad C_n\mbox{ is closed}
        \end{aligned}\]
        \item $\mathfrak{F}$ pointwise bounded $\implies$ $\displaystyle\bigcup_{n=1}^\infty C_n=X$\\
        Due to Baire theorem, there exists $n_0\in\N$ s.t. Int$C_0\ne\emptyset$ $\implies$ $\exists\,\overline{B}_\epsilon(x_0)\subset C_{n_0}$\\
        If $\|z\|_X\leq\epsilon\quad\implies\quad z+x_0\in\overline{B}_\epsilon(x_0)\subset C_{n_0}$,
        \[\begin{aligned}
                \|T(z)\|_Y&=\|T(z)+T(x_0)-T(x_0)\|_Y\leq\|T(z)+T(x_0)\|_Y+\|T(x_0)\|_Y\\
                &=\|T(z+x_0)\|_Y+\|T(x_0)\|_Y\leq2n_0,\quad\forall\,T\in\mathfrak{F}
        \end{aligned}\]
        $\forall\,x\in X\setminus\{0\},\quad\forall\,T\in\mathfrak{F}$
         \[\|T(x)\|_Y=\frac{\|x\|_X}{\epsilon}\bigg\|T\bigg(\frac{\epsilon x}{\|x\|_X}\bigg)\bigg\|_Y\leq\frac{2n_0}{\epsilon}\|x\|_X\implies\|T\|_\mathcal{L}\leq\frac{2n_0}{\epsilon}\eqqcolon M \implies\sup_{I\in\mathfrak{F}}\|T\|_\mathcal{L}\leq M\]
    \end{itemize}
\end{proof}

%>=====< Question 4 >=====<%

\question
From the UBP is it possible to infer an important property of operators defined by means of a pointwise limit. What is that? Justify your answer.

\subsection*{Solution}

\subsection{Corollary of UBP}
Let $X,Y$ be Banach spaces, $\{T_n\}_{n\in\N}\subset\mathcal{L}$. Assume that $\forall x\in X$ $\displaystyle\exists\lim_{\toi}T_n(x)$. Let $\displaystyle T(x)\coloneqq\lim_{\toi} T_n(x)$ then $\displaystyle T\in\mathcal{L}(X,Y)$.
\begin{proof}
    $T:X\rightarrow Y$ is linear. $\{T_n(x)\}_{n\in\N} \quad $is bounded $\forall x\in X$  
    \[\begin{aligned}
        &\iff \forall x\in X\quad\exists\,M_x>0:\quad\|T_n(x)\|_Y\leq M_x\quad\forall\,n\in\N \quad(\mbox{or equivalently } \sup_{n\in\N}\|T_n(x)\|_Y\leq Mx)\\
        &\iff \mathfrak{F}=\{T_n\}_{n\in\N} \mbox{ is pointwise bounded} \overset{{\mbox{UBP}}}{\implies}\mathfrak{F}=\{T_n\}_{n\in\N} \mbox{ is uniformly bounded}\\
        &\iff\exists\,M>0:\quad\sup_{n\in\N}\|T_n\|_\mathcal{L}\leq M\implies\|T_n(x)\|_Y\leq M\|x\|_X\quad\forall\,x\in X,\,\forall\,n\in\N
    \end{aligned}
    \]
    \[\|T(x)\|_Y=\lim_{\toi}\|T_n(x)\|_Y\leq M\|x\|_X\quad\forall\,x\in X\implies T\mbox{ is bounded }\implies T\in \mathcal{L}(X,Y)\]
\end{proof}

%>=====< Question 5 >=====<%

\question
Write the definition of open mapping. Let $f:\R^n\rightarrow\R^m$ be continuous; under which extra hypotheses is $f$ open? Let $T:\R^n\rightarrow\R^n$ be linear and onto; why is $T$ open? State the OMT.

\subsection*{Solution}

\subsection{Open mapping}
 Let $X,Y$ be normed space, $T:X\rightarrow Y$ is said to be an open mapping if for any open set $G\subset X$, $T(G)\subset Y$ is open.
 
\subsection{Open mapping from \texorpdfstring{$\R^n\rightarrow\R^m$}{Rn to Rm}}
Given a continuous $f$: $\R^n\rightarrow\R^m$,  with the further hypotesis $f$ injective, we can prove that $f$ is open.

\subsection{Open mapping from \texorpdfstring{$\R^n\rightarrow\R^n$}{Rn to Rn}}
Let $T: R^n\rightarrow\R^n$ be linear and onto, then $T$ is open.
\begin{proof}
    $T$ onto from $\R^n$ to itself implies that $T$  is injective.
    Moreover $\R^n$ has finite dimension hence linear operators from $\R^n$ are also continuous and thus the hypotesis of the previous theorem are satisfied , yielding  that $T$ is open.
\end{proof}

\subsection{Open Mapping Theorem}
Let $X;Y$ be Banach spaces, $T\in\mathcal{L}(X,Y)$ onto. Then $T$ is an open mapping.

%>=====< Question 6 >=====<%

\question
State and prove the IBM (or ICM) theorem. What is the analogous result in the finite dimensional case?

\subsection*{Solution}

\subsection{Inverse Bounded Mapping}
Let $T\in\mathcal{L}(X,Y)$, $T$ bijective. Then $T^{-1}\in\mathcal{L}(X,Y)$.
\begin{proof}
    \[\begin{array}{l}
        T\in\mathcal{L}(X,Y)\\
        T \mbox{ is bijective}
    \end{array}\quad\implies\quad \exists\, T^{-1}:Y\rightarrow X\mbox{ and } T^{-1} \mbox{ is linear}\]
    Claim: $T^{-1}$ is continuous $\overset{{\mbox{def}}}{\iff}$ $(T^{-1})^{-1}(E)\subset Y$ is open $\forall\,E\subset X$ open \\
    $(T^{-1})^{-1}(E)=T(E)$ is open, in view of the open mapping theorem.
\end{proof}

\subsection{Finite-dimensional case}
Let $T:V\to W$ with $\dim V < \infty$ and  $\dim W < \infty$ .
If $T$ is bijective, $T $ is linear and hence continuous.

%>=====< Question 7 >=====<%

\question
By the IBM theorem we can infer an important property about equivalent norms on Banach spaces. What is that? Justify your answer.

\subsection*{Solution}

\subsection{IBM Corollary} 
Let $(X,\|\cdot\|)$, $(X,\|\cdot\|_\#)$ be Banach spaces and s uppose that $\exists\,M>0:$ $\|x\|_\#\leq M\|x\|\quad\forall\,x\in X$. \newline 
Then $\|\cdot\|$ and $\|\cdot\|_\#$ are equivalent, i.e. 
\[\exists\,m>0:\quad \|x\|_\#\geq m\|x\|\quad\forall\,x\in X\]
\begin{proof}
    \[I:X\rightarrow X,\,\, I(x)=x \; \forall\,x\in X \implies I \text{ is bijective and continuous}\]
    \[I \text{ continuous} \iff I \text{ bounded } \iff \|x\|_\#=\|I(x)\|_\#\leq M\|x\|\quad\forall\,x\in X\]
    By IBM, $I^{-1}:X\rightarrow X$ is linear and bounded $\Rightarrow \exists\,m'>0:$
    \[\|x\|_\#=\|I^{-1}(x)\|\leq m'\|x\|_\#\quad \forall\,x\in X,\quad m=\frac{1}{m'}\]
\end{proof}

%>=====< Question 8 >=====<%

\question
Write the definitions of: closed operator; graph of an operator. Show that an operator is linear and closed iff its graph is closed.

\subsection*{Solution}

\subsection{Closed operator}
A linear operator $T:X\rightarrow Y$ is called \textbf{closed} if
\[
    \begin{array}{l}
        x_n\xrightarrow[\toi]{}x\quad\mbox{ in } X \\
        T(x_n)\xrightarrow[\toi]{}y\quad\mbox{ in } Y
    \end{array}\bigg\}\implies T(x)=y
\]

\subsection{Graph of an operator}
The graph of a linear operator $T$ is defined as
\[\mbox{graph }T\coloneqq\{(x,T(x)):x\in X\}\subseteq X\times Y\]

\subsection{Relation between close operator and graph}$T:X\rightarrow Y$ linear and closed $\iff$ graph $T\,\,(\subseteq X\times Y)$ is closed.
\begin{proof}
    Let $\{(x_n,T(x_n)\}\subset\mbox{graph }T$ be s.t.
    \[(x_n,T(x_n))\xrightarrow{\toi}(x,y)\in X\times Y\]
    \[\implies x_n\xrightarrow{\toi}x\text{ and } T(x_n)\xrightarrow[]{\toi}y\]
    \[ \text{graph $T$ is closed} \iff (x,y)\in \text{graph } T\iff y=T(x)\]
\end{proof}

%>=====< Question 9 >=====<%

\question
State and prove the closed graph theorem.

\subsection*{Solution}

\subsection{Closed Graph Theorem} 
Let $T:X\rightarrow Y$ be a linear closed operator with $X,Y$ Banach spaces, then $T\in\mathcal{L}(X,Y)$.
\begin{proof}
Let:
    \[\|x\|_\#\coloneqq\|x\|_X+\|T(x)\|_Y \implies \|x\|_X\leq\|x\|_\#\]
    By the previous corollary $\exists\,M\geq1 \text{ s.t.}$:\[
    \|x\|_\#=\|x\|_X+\|T(x)\|_Y\leq M \|x\|_X\quad\forall\,x\in X
    \]
    \[\implies \|T(x)\|_Y\leq (M-1)\|x\|_X\quad\forall\,x\in X\implies T \text{ is bounded } \implies T\in\mathcal{L}(X,Y)\]
\end{proof}

%>=====< Question 10 >=====<%

\question
Write the definition of dual space.

\subsection*{Solution}

\subsection{Dual space}
Let $X$ be a normed space, then its dual space $X^*\coloneqq \Leb(X,\R)$ is a Banach space with the norm $\|L\|=\displaystyle\sup_{\|x\|=1}|L(x)|$

%>=====< Question 11 >=====<%

\question
Exhibit an example of $T \in (L^p)^{*}$. Compute $\|T\|_{*}$.

\subsection*{Solution}

\subsection{Example of \texorpdfstring{$T\in (L^p)^{*}$}{a linear operator in the dual space of Lp}} 
$X=L^p(X,\mathscr{A},\mu)$, $\frac{1}{p}+\frac{1}{q}=1$, $g\in L^q$, $L_g:L^p\rightarrow \R$, $L_g(f)\coloneqq\int_Xfgd\mu$

\begin{itemize}
    \item $L_g$ is linear: $\forall f_1,f_2\in L^p, \quad\alpha_1,\alpha_2\in\R$
    \[L_g(\alpha_1f_1+\alpha_2f_2)=\int_X(\alpha_1f_1+\alpha_2f_2)gd\mu=\alpha_1\int_Xf_1gd\mu+\alpha_2\int_Xf_2gd\mu=\alpha_1L_g(f_1)+\alpha_2L_g(f_2)\]
    \item $L_g$ is bounded:
    \[|L_g(f)|=|\int_Xfgd\mu|\leq\int_X|fg|d\mu\leq \|f\|_p\underbrace{\|g\|_q}_{=M}\qquad\forall f\in L^p\]
\end{itemize}

From the previous inequality, we deduce that $\|L_g\|_*\leq\|g\|_q$. Consider $\displaystyle \varphi\coloneqq\frac{|g|^{q-2}g}{\|g\|_q^{q-1}}$,
\[L_g(\phi)=\int_X\phi gd\mu =\frac{1}{\|g\|_{q}^{q-1}}\int_X|g|^qd\mu=\frac{\|g\|_q^q}{\|g\|_q^{q-1}}=\|g\|_q\implies\|L_g\|_{X^*}=\|g\|_{L^q}\]

%>=====< Question 12 >=====<%

\question
Let $(V, \langle \cdot, \cdot \rangle)$ be a finite dimensional vector space endowed with an inner product. Characterize $V^{*}$.

\subsection*{Solution}

\subsection{Characterization of \texorpdfstring{$V^*$}{V*}}
Let $(V,\langle \cdot,\cdot \rangle)$ be a finite-dimensional vector space and $L: V \to \R $ be a linear operator. Then:
\[
\exists ! \,\,y \in V: L(x) = \langle x,y \rangle ,\quad \forall x \in V
\]
\begin{proof}

Let $B = \{v_1,v_2,\dots,v_n\}$ be an orthonormal basis of $V: \\$
\[
\forall x \in V, \quad x = \displaystyle\sum_{i=1}^n \alpha_i v_i
\quad \alpha_i = \langle v,v_i \rangle \quad \forall i=1,\dots,n
\]
\begin{align*}
    L(x) & = \alpha_1 L(x_1) + \dots + \alpha_n L(x_n) = \\
         & = \langle x,v_1 \rangle L(x_1) + \dots + \langle x,v_n \rangle L(x_n) = \\
         & = \big\langle x,\underbrace{v_1 L(x_1) + \dots  +  x,v_n L(x_n) }_{\coloneqq y}\big\rangle 
\end{align*}
Now we must show that such a $y$ is unique: suppose by contradiction that $\exists ! \,y' \in V: L(x) = \langle x,y' \rangle \quad \forall x \in V $
\[
\implies 0 = L(x) - L(x) = \langle x,y \rangle - \langle x,y' \rangle = \langle x,y-y \rangle \quad \forall x \in  V
\]
\[
 \implies y=y' \text{, contradiction}
\]
    
\end{proof}


%>=====< Question 13 >=====<%

\question
Let $Y$ be a vector subspace of $X = \R^2$. Given $f \in Y^{*}$ is it possibile to find $F \in X^{*}$ such that $F = f$ in $Y$ and $\|F\|_{X^{*}} = \|f\|_{Y^{*}}$ ?

\subsection*{Solution}

\subsection{Example}

Take $X=\R^2$ and $Y$ v.s.s. of $X$.
\[
f \in Y^* \iff f:Y\to \R \text{ is linear and continuous}
\]
We want $F \in X^* $ s.t. $F=f$ in $Y \subset X$ and $\|F\|_{X^*}=\|f\|_{Y^*}$ and we already know:
\[
f \in Y^* \implies \exists ! \eta \in Y  \text{ s.t. } f(x) = \langle \eta, x \rangle \quad \forall x \in Y
\]
Now let $F: X\to \R $, $F(x)= \langle \eta, x \rangle\quad \forall x \in X\\$
$F$ is clearly bounded and $\|F\|_{X^*}=\|f\|_{Y^*}$

%>=====< Question 14 >=====<%

\question
State the Hahn-Banach theorem in the continuous extension form.

\subsection*{Solution}

\subsection{Hahn-Baanch theorem, continuous form}
Let $X$ be a normed space and $Y$ v.s.s. of $X$, $f \in Y^*$, then $\exists\,f \in\, X^* $ s.t. $F(y)=f(y)\quad \forall y\in Y$, $\,\|F\|_{X^*}=\|f\|_{Y^*}$. 

%>=====< Question 15 >=====<%

\question
Let $X$ be a normed space, let $H$ be a hyperplane of $X$. When do we say that $H$ separates (or strictly separates) $A$ and $B$? If $X = \R^2$ , under which hypotheses on $A$ and $B$ is it possibile to find a line $H$ which separates them?

\subsection*{Solution}

\subsection{Separating hyperplane}
Let $X$ be an hyperplane, we say that $H$ separates $A \subseteq X$ and $B\subseteq X$ if:
\[
f(a) \le \alpha \le f(b) \quad\forall a\in A,\,\forall b \in B
\]
Moreover, we say  that $H$ strictly separates $A \subseteq X$ and $B\subseteq X$ if:
\[
\exists \epsilon>0 \text{ s.t. } f(a) \le \alpha - \epsilon,\quad f(b) \ge \alpha + \epsilon \quad \forall a \in A, \forall b \in B
\]
\subsection{Weaker hypotesis on \texorpdfstring{$\R^2$}{R2}}
If $X= \R^2$, we can say that it is possible to find a line (hyperplane with $n=2$) that separates $A $ and $B$ if $A,B $ are convex and disjoint.

%>=====< Question 16 >=====<%

\question
State the Hahn-Banach theorem in the separation form (first and second version).

\subsection*{Solution}

\subsection{Hahn-Banach theorem, separation form}
Let $X $ be a normed space. If $A,B$ are disjoint convex sets $\subseteq X$, $A$ is open and, then there $\exists $ a closed hyperplane $H$ which separates $A$ and $B$ ($\exists F \in X^*$ s.t. $F(a) \le \alpha \le F(b) \quad \forall a \in A, \forall b \in B$).
\\
With the further hypotesis that $B$ be compact, we can say that there exists a closed hyperplane $H$ that \emph{strictly} separates $A$ and $B$.
