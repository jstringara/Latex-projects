\sheet

%==== Question 1 ====%

\question
Write the definitions of: sequence of sets $\seq{E}$; increasing and decreasing sequence of sets $\seq{E}$;
$\limsup_{n\to\infty} E_n$, $\liminf_{n\to\infty} E_n$, $\lim_{n\to\infty} E_n$.

\subsection*{Solution}

\provdefs
\begin{itemize}
    \item \subsection{Sequence of sets} A family (or collection) of sets $\{E_i\}_{i\in I}$ is called a sequence of sets if $I=\N$ (i.e. it is indexed by the set of natural numbers $\N$)
    \item \subsection{Increasing sequence of sets} a sequence of sets $\seq{E}$ is said to be increasing (or ascending) if:
    \[
        E_n \subseteq E_{n+1} \quad \forall n\in\N    
    \]
    \item \subsection{Decreasing sequence of sets}
    A sequence of sets $\seq{E}$ is said to be decreasing (or descending) if:
    \[
        E_n \supseteq E_{n+1} \quad \forall n\in\N    
    \]
    \item \subsection{Limsup for a sequence of sets} for a sequence of sets $\seq{E}$ we define:
    \[
        \limsup_{n\to\infty} E_n \coloneqq \bigcap_{k=1}^{\infty} \bigcup_{n=k}^\infty E_n
    \]
    \item \subsection{Liminf for a sequence of sets} analogously:
    \[
        \limsup_{n\to\infty} E_n \coloneqq \bigcup_{k=1}^{\infty} \bigcap_{n=k}^\infty E_n
    \]
    \item \subsection{Limit for a sequence of sets} as for a sequence of real numbers if the limsup and liminf coincide we may define:
    \[
        \lim_{n\to\infty} E_n \coloneqq \liminf_{n\to\infty} E_n = \limsup_{n\to\infty} E_n    
    \]
\end{itemize}

%==== Question 2 ====%

\question
Write the definitions of: cover (or covering) of a set; subcover.

\subsection*{Solution}
\provdefs
\begin{itemize}
    \item \subsection{Cover of a set} a family of sets $\{E_i\}_{i\in I}$ is called a cover (or covering) of $X$ if:
    \[
        X \subseteq \bigcup_{i\in I} E_i    
    \]
    \item \subsection{Subcover} a sub-family of a cover $\{E_i\}_{i\in J}$ ($J\subseteq I$) which forms a cover is called a subcover.  
\end{itemize}

%==== Question 3 ====%

\question
Write the definitions of: equivalence relation, equivalence class, quotient set.

\subsection*{Solution}
\provdefs
\begin{itemize}
    \item \subsection{Equivalence relation} \label{equivrel} a relation $R$ in $X$ (i.e. a subset $R\subseteq X\times X$) is an equivalence relation if:
    \begin{enumerate}[i)]
        \item $(x,x) \in R$ $\forall x\in X$ (\textbf{reflexivity})
        \item $(x,y) \in R \implies (y,x)\in R$ (\textbf{simmetry})
        \item $(x,y) \in R, \, (y,z)\in R \implies (x,z)\in R$ (\textbf{transitivity})
    \end{enumerate}
    \item \subsection{Equivalence class} we define an equivalence class for $x$ w.r.t. $R$ as:
    \[
        E_x \coloneqq \{y\in X : yRx\}
    \]
    i.e. the set of all elements equivalent to $x$ for $R$
    \item \subsection{Quotient set} we define the quotient set of $X$ over $R$ as:
    \[
        X / R \coloneqq \{E_x: x\in X \}    
    \]
    i.e. it is the set of all equivalence classes.
\end{itemize}

%==== Question 4 ====%

\question
Write the definition of equipotent sets. Write the definition of cardinality of a set.

\subsection*{Solution}
\provdefs
\begin{itemize}
    \item \subsection{Equipotent sets} Two sets $X$ and $Y$ are called equipotent if there exists a bijections, that is, a function:
    \[
        f:X\to Y    
    \]
    that is both injective and surjective.
    \item \subsection{Cardinality of a set} the cardinality of a set $X$ is the collection of all sets equipotent to X.
\end{itemize}


%==== Question 5 ====%

\question
Write the definitions of: infinite set, finite set, countable set, uncountable set. Provide examples.

\subsection*{Solution}
\provdefs
\begin{itemize}
    \item \subsection{Finite sets} a set $X$ is finite if $\exists n\in\N$ such that there is a bijection:
    \[
        f:X\to {1,\dots,n}    
    \]
    \textbf{Example:} $\{\frac{1}{1}, \dots, \frac{1}{n}\}$
    \item \subsection{Infinite sets} $X$ is infinite if it is not finite.\\
    \textbf{Example:} $\N$ is clearly infinite
    \item \subsection{Countable sets} $X$ is countable if $X$ is equipotent to $\N$\\
    \textbf{Example:} $\Q$ can be put in bijection with $\N$
    \item \subsection{Uncountable sets} $X$ is uncountable if it is infinite and not countable.\\
    \textbf{Example:} $\R$ is clearly infinite and not countable since it has the cardinality of continuum.
\end{itemize}

%==== Question 6 ====%

\question
Write the definitions of: algebra, $\salg$, measurable space, measurable set. Show that if $\A$
is a $\salg$ and $\{E_k\}\subset \A$, then $\bigcap^{+\infty}_{k=1} E_k \in \A $.

\subsection*{Solution}
\provdefs
\begin{itemize}
    \item \subsection{Algebra} A family $\A\subseteq\Parts{X}$ is an algebra if:
    \begin{enumerate}[i)]
        \item $\emptyset \in \A$
        \item $E\in\A \implies E^\complement \in\A$
        \item $A,B\in\A \implies A\cup B\in\A$
    \end{enumerate}
    \item \subsection{\texorpdfstring{$\salg$}{Sigma-algebra}} A family $\A\subseteq\Parts{X}$ is a $\salg$ if:
    \begin{enumerate}[i)]
        \item $\emptyset \in \A$
        \item $E\in\A \implies E^\complement \in\A$
        \item $\seq{E}_{n\in\N} \subseteq\A \implies \bigcup_{n=1}^\infty E_n \in\A$
    \end{enumerate}
    \item \subsection{Measurable space} The couplet $(X,\A)$ where $\A$ is a $\salg$ is called a measurable space.
    \item \subsection{Measurable set} the elements of the $\salg$ of a measurable space are called measurable sets.
\end{itemize}

%==== Question 7 ====%

\question
State the theorem concerning the existence of the $\salg$ generated by a given set. Give an
idea of the proof.

\subsection*{Solution}

\subsection{Minimal \texorpdfstring{$\salg$}{sigma-algebra}}
Let $S\subseteq\Parts{X}$, then there exists a $\salg$ $\sigma_0(S)$ such that:\\
\begin{enumerate}
    \item $S\subseteq \sigma_0(S)$
    \item $\forall \salg \A\subseteq\Parts{X}$ such that $S\subseteq\A$ we have $\sigma_0(S) \subseteq\A$
\end{enumerate}
thus $\sigma_0(S)$ is the minimal $\salg$ generated by $S$.\\

\subsection*{Sketch of Proof}
We construct the set:
\[
    \mathcal{V} \coloneqq \{\A \subseteq \Parts{X} \| \A \supseteq S, \; \A \quad \salg \}    
\]
we may define:
\[
    \sigma_0(S)\coloneqq \bigcap \{\A: \: \A\in\mathcal{V}\}    
\]
%==== Question 8 ====%

\question
Write the definition of the Borel $\salg$ in a metric space. Provide classes of Borel sets.
Characterize $\B[\R], \B[\Rcomp]$ and $\B[\R^N]$.

\subsection*{Solution}

\subsection{Borel \texorpdfstring{$\salg$}{sigma-algebra}}
Let $(X,d)$ be a metric space and let $\mathcal{G}$ be the family of open sets of $X$, then we define the Borel $\salg$ as:
\[
    \B[X] \coloneqq \sigma_0(\mathcal{G})    
\]
The elements of $\mathcal{G}$ are called Borel sets,   let us enumerate some classes of them:

\subsection{Classes of Borel sets} 
\begin{enumerate}[i)]
    \item open sets
    \item closed sets (they are the complementary of open sets and this is a $\salg$)
    \item countable intersections of open sets, known as the family $G_\delta$
    \item countable union of closed sets, known as the family $F_\delta$.
\end{enumerate} 
Lastly, let us characterize the Borel $\salg$s $\B[\R],\B[\Rcomp]$ and $\B[\R^N]$:
\subsection{Characterization of \texorpdfstring{$\B[\R],\B[\Rcomp]$}{B(R), B(Rcomp)} and \texorpdfstring{$\B[\R^N]$}{B(Rn)}}
\begin{enumerate}
    \item $\B[\R]=\sigma_0(I)=\sigma_0(I_1)=\sigma_0(I_2)=\sigma_0(I_0)=\sigma_0(\hat{I})$\\
    where:
    \begin{align*}
        I&=\{ (a,b): a,b \in \R, a\leq b \} \\
        I_1&=\{ [a,b]: a,b \in \R, a\leq b \} \\
        I_2&=\{ (a,b]: a,b \in \R, a\leq b \} \\
        I_0&=\{ (a,b): -\infty\leq a < b <\infty \} \cup \{ (a,\infty): a\in\R \} \\
        \hat{I}&=\{ (a,\infty): a\in\R \}
    \end{align*}
    \item $\B[\Rcomp]=\sigma_0(\tilde{I})=\sigma_0(\tilde{I}_1)$\\
    where:
    \begin{align*}
        \tilde{I} &= \{ (a,b): a,b \in \R, a < b \} \cup \{ [-\infty,b): b\in\R \} \cup \{ (a,+\infty] : a\in\R \} \\
        \tilde{I}_1 &= \{ (a,+\infty] : a\in\R \}
    \end{align*}
    \item $\B[\R^N] = \sigma_0(K_1)=\sigma_0(K_2)$\\
    where:
    \begin{align*}
        K_1 &= \{ \text{n-dimensional closed rectangles}\}\\
        K_2 &= \{ \text{n-dimensional open rectangles }\}
    \end{align*}
\end{enumerate}
%==== Question 9 ====%

\question
Write the definitions of: measure, finite measure, $\s{finite measure}$, measure space, probability
space. Provide some examples of measures.

\subsection*{Solution}
\provdefs
\begin{itemize}
    \item \subsection{Measure} Let $X$ be a set and $\mathcal{C}\subseteq\Parts{X}$, then a function $\mu$:
    \[
        \mu:\mathcal{C}\to\Rcomppos    
    \]
    is a measure if:
    \begin{enumerate}
        \item $\mu(\emptyset)=0$
        \item \textbf{$\s{additivity}$:}\\
        $\forall \seq{E}\subseteq\mathcal{C}$ disjoint ($E_i\cap E_j\quad \forall i\neq j$) such that $\bigcup_{k=1}^{\infty} E_k \in \mathcal{C}$ we have that:
        \[
            \mu \left( \bigcup_{k=1}^{\infty} E_k \right) = \sum_{k=1}^{\infty} \mu(E_k) 
        \]
    \end{enumerate}
    \item \subsection{Finite measure} a measure $\mu$ defined as above is said to bw finite if:
    \[
        \mu(X) < + \infty    
    \]
    \item \subsection{\texorpdfstring{$\s{finite measure}$}{Sigma-finite measure}} a measure $\mu$ is said to be $\s{finite}$ if there exists a sequence $\seq{E}$ such that:
    \[
        X = \bigcup_{k=1}^\infty E_k, \quad \mu(E_k) <+\infty
    \]
    \item \subsection{Measure space} Let $\A\subseteq\Parts{X}$ be a $\salg$ and $\mu:\A\to\Rcomppos$ a measure, then the triplet $(X,\A,\mu)$ is called  a measure space.
    \item \subsection{Probability space} if $\mu(X)=1$ then we say that $(X,\A,\mu)$ is a probability space.
\end{itemize}
%====  Question 10 ====%

\question
State and prove the theorem regarding properties of measures. Why the two continuity properties
are called in this way? For what concerns continuity w.r.t. a descending sequence ${E_k}$, show that
the hypothesis $\mu(E_1) < +\infty$ is essential.

\subsection*{Solution}
\subsection{Properties of measures}
Let us state and prove the properties of a measure $\mu$ on a set $X$ and $\salg$ $\A$:
\begin{enumerate}[i)]
    \item \label{meas:add} \textbf{Additivity:} \\
    $\forall \{ E_1, \dots, E_n \} \subseteq \A$ disjoint we have:
    \[
        \mu\left( \bigcup_{k=1}^{\infty} E_k \right) = \sum_{k=1}^{\infty} \mu(E_k)
    \]
    \begin{proof}
    indeed if we define a sequence such that:
    \[
        \seq{E} = \left\{ \begin{array}{l}
            B_k = E_k \quad \forall k \leq n\\
            B_k = \emptyset \quad \forall k > n
        \end{array} \right.
    \]
    this sequence is also disjoint ($\A\cap\emptyset = \emptyset$ $\forall \A\in X$), thus we may write:
    \[
        \mu \underbrace{ \left( \bigcup_{k=1}^{\infty} E_k \right) }_{=\bigcup_{k=1}^{n} E_k \cup \emptyset}
         = \sum_{k=1}^{\infty} \mu(E_k) = \sum_{k=1}^{n} \mu(E_k) + \sum_{k=n+1}^{\infty} \underbrace{\cancelnum{}{\mu(E_k)}}_{=0}
    \]
    \end{proof}

    \item \label{meas:mono} \textbf{Monotonicity:} \\
    $\forall E,F \in \A$ we have:
    \[
        E\subseteq F \implies \mu(E) \leq \mu(F)    
    \]
    \begin{proof}
    We may write $F$ in the following way:
    \[
        F=E\cup (F \setminus E) 
    \]
    and since these two sets are obviously disjoint we may use (\ref{meas:add}) to write:
    \[
        \mu(F) = \mu(E)+\underbrace{\mu(E \setminus F)}_{\geq 0} > \mu(E)
    \]
    \end{proof}
    
    \item \label{meas:sub} \textbf{$\s{subadditivity}$:} \\
    $\forall \seq{E}\subseteq\A$ (\textbf{not} disjoint) we have:
    \[
        \mu\left( \bigcup_{k=1}^{\infty} E_k \right) \leq \sum_{k=1}^{\infty} \mu(E_k)
    \]
    \begin{proof}
    \provdef{}
    \[
        \left\{ \begin{array}{l}
            F_1 \coloneqq E_1 \\
            F_n \coloneqq E_n \setminus \bigcup_{k=1}^{n-1} E_k \quad \forall n > 1
        \end{array} \right.  
    \]
    Clearly $\seq{F} \subseteq \A$ and $\seq{F}$ is a disjoint sequence and:
    \[
        \begin{array}{l}
            F_k \subseteq E_k \quad \forall k\in\N \implies \mu(F_k) \leq \mu(E_k) \text{ by (\ref{meas:mono})} \\
            \bigcup_{k=1}^{\infty} F_k = \bigcup_{k=1}^{\infty} E_k    
        \end{array}
    \]
    thus we may write:
    \[
        \mu\left( \bigcup_{k=1}^{\infty} E_k \right) = \mu\left( \bigcup_{k=1}^{\infty} F_k \right) = \sum_{k=1}^{\infty} \mu(F_k) \leq \sum_{k=1}^{\infty} \mu(E_k)   
    \]
    \end{proof}

    \item \label{meas:contbel} \textbf{Continuity from below:} \\
    $\forall \seq{E}\subseteq\A, \; E_k \nearrow$ we have:
    \[
        \mu \left( \lim_{k\to\infty}E_k \right)  = \lim_{k\to\infty} \mu(E_k)    
    \]
    \begin{proof}
    \provdef[a new sequence $\seq{F}$]
    \[
        \left\{ \begin{array}{l}
            F_k \coloneqq E_k \setminus E_{k-1} \quad \forall k \in\N \text{ and } E_0 \coloneqq \emptyset \\
            \implies \bigcup_{k=1}^n F_k = E_n ,\; \bigcup_{k=1}^{\infty} F_k = \bigcup_{k=1}^{\infty} E_k
        \end{array} \right.
    \]
    and since $\seq{F}$ is a disjoint sequence (we may visually think of it as a set of ever increasing rings) we may use (\ref{meas:add}) to write:
    \begin{align*}
        \mu\left( \lim_{n\to\infty} E_n \right) &\tikzmarknode{eq1}{=} \mu\left( \bigcup_{k=1}^{\infty} E_k \right) = \mu\left( \bigcup_{k=1}^{\infty} F_k \right) \\
        & \tikzmarknode{eq2}{=} \lim_{n\to\infty}\sum_{k=1}^{n} \mu(F_k) = \lim_{n\to\infty} \mu \left( \bigcup_{k=1}^{\infty} F_k \right) \\
        & \tikzmarknode{eq3}{=} \lim_{n\to\infty} \mu(E_n)
    \end{align*} \tikz[overlay,remember picture]{\draw[shorten >=1pt,shorten <=1pt] (eq1) -- (eq2) -- (eq3);}
    \end{proof}

    \item \label{meas:contab} \textbf{Continuity from above:} \\
    $\forall \seq{E} \subseteq \A$, $E_k\searrow$, $\mu(E_1) < +\infty$ we have:
    \[
        \mu\left( \lim_{n\to\infty} E_n \right) = \lim_{n\to\infty} \mu(E_n)
    \]
    \begin{proof}
    Like we did above \provdef{a new sequence $\seq{F}$}
    \[
            F_k \coloneqq E_1 \setminus E_k \quad \forall k \in\N
    \]
    let us note that $\seq{F}$ is an increasing sequence thus by (\ref{meas:contbel}) we can write:
    \[
        \mu\left( \bigcup_{k=1}^{\infty} F_k \right) = \lim_{k\to\infty} \mu(F_k) = \mu(E_1) - \lim_{k\to\infty}  (E_k)   
    \]
    because by (\ref*{meas:mono}) $\\mu(F \setminus E) = \mu(F)-\mu(E)$, moreover:
    \begin{align*}
        \bigcup_{k=1}^{\infty} F_k & = \bigcup_{k=1}^{\infty} (E_1\cap E_k^\complement) = E_1\cap \left( \bigcup_{k=1}^{\infty} E_k^\complement \right) = E_1 \setminus \left( \bigcap_{k=1}^{\infty} E_k \right) \\
        &\implies \mu\left( \bigcup_{k=1}^{\infty} F_k \right) = \mu(E_1) - \mu \left( \bigcap_{k=1}^{\infty} E_k \right)
    \end{align*}
    thus combining these two and canceling the $\mu(E_1)$ on both sides we obtain:
    \[
        \lim_{k\to\infty} \mu(E_k) = \mu \left( \bigcap_{k=1}^{\infty} E_k \right)
    \]
    \end{proof}
    let us note that for this last, crucial, step $\mu(E_1)$ must be finite, otherwise we would not be able to cancel it out from both sides.
\end{enumerate}

%====  Question 11 ====%

\question
Write the definitions of: sets of zero measure; negligible sets. What is meant by saying that a property holds a.e.? Provide typical properties that can be true a.e. .

\subsection*{Solution}

\provdefs
\begin{itemize}
    \item \subsection{Sets of zero measure} 
    Given a measure space $(X,\A,\mu)$, we say that a set $E\subseteq X$ has zero measure if $E\in\A$ and $\mu(E)=0$. We denote the set of all sets of zero measure by $\mathcal{N}_{\mu}$
    \item \subsection{Negligible sets} a set $E\subseteq X$ is negligible if:
    \[
        \exists N\in\A \text{ s.t. } E \subseteq N, \; \mu(N)=0
    \]
    So any subset of a set of zero measure is negligible, we denote the collection of all negligible sets by $\tau_{\mu}$. Moreover let us note that $E$ doesn't need to be an element of $\A$ ($E\notin\A$)
    \item \subsection{Almost Everywhere} a property $P$ on $X$ is said to hold almost everywhere if:
    \[
       \mu( \{ x\in X: P(x) \text{ is false } \} ) = 0    
    \]
    We may also say that $\{ x\in X: P(x) \text{ is false } \} \in \mathcal{N}_{\mu}$\\
    \subsection*{Examples} typical properties that can be true a.e. are: equality, continuity, monotonicity, etc. etc.
\end{itemize}

%====  Question 12 ====%

\question
Write the definition of complete measure space. Exhibit an example of a measure space which is not complete.

\subsection*{Solution}
\subsection{Complete measure space}
A measure space $(X,\A,\mu)$ is said to be complete if $\tau_{\mu}\subseteq\A$
\subsection*{Counterexample}
Let $X=\{a,b,c\}$, $\A = \sigma (\{\emptyset, \{a\}, \{b,c\}, X \})$ and $\mu\equiv 0$, clearly here we have:
\[
    \tau_{\mu} \setminus \mathcal{N}_{\mu}= \{ \{b\}, \{c\} \}    
\]
and clearly $\{b\}, \{c\} \notin \A$. So this measure space is not complete.


