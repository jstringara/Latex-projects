\documentclass[a4paper]{report} %document class and format


%>=====< Language  and bibliography packages >=====<%

\usepackage[T1]{fontenc} % font encoding
\usepackage[utf8]{inputenc} % input encoding
\usepackage[style=numeric-comp,useprefix,hyperref,backend=bibtex]{biblatex}

%>=====< Math packages >=====<%

\usepackage{amsmath, amssymb, amsthm, mathrsfs, mathtools}
\usepackage{centernot} %for centered negation
\usepackage[makeroom]{cancel} %to cancel out stuff

%>=====< Other packages >=====<%

\usepackage{framed} % for framing text
\usepackage{titling} % for customizing title
\usepackage{titlesec} %for customizing sections title
\usepackage[ top=2cm, bottom=2cm, left=2cm, right=2cm]{geometry} % for smaller margins
\usepackage{xifthen} %for conditional statements
\usepackage{enumerate} %to modify lists
\usepackage{hyperref} %to insert hyperlinks
%\usepackage[open]{bookmark} %to have bookmarks open by default
\usepackage{titletoc} %to modify the toc
\usepackage{tikz}
\usetikzlibrary{tikzmark}

%>=====< Chapter >=====<% 

%redefine chapters
\titleformat{\chapter}[display]{\normalfont\Huge\bfseries} % format
{Sheet n. \thechapter} % label
{0ex} % sep
{ \vspace{1ex} \centering } % before-code
[ \vspace{-2.5ex} ] % after-code

%modify chapter in toc
\titlecontents{chapter}[0em]{\smallskip\bfseries}%\vspace{1cm}%
{}%
{\itshape\bfseries}%numberless%
{\hfill\contentspage}[\medskip]%

%new sheet command for toc
\newcommand{\sheet}{\chapter[Sheet n.\thechapter]{}}

%>=====< Section >=====<%

%redefine section
\titleformat{\section}[display]{\normalfont\huge\bfseries} % format
{Question \thesection} % label
{0ex} % sep
{ \vspace{0ex} } % before-code
[ \vspace{-1.5ex} ] % after-code

%modify chapter in toc
\titlecontents{section}[1.5em]{\smallskip\bfseries}%\vspace{1cm}%
{}%
{\itshape\bfseries}%numberless%
{\hfill\contentspage}[\medskip]%

%new question command for toc
\newcommand{\question}{\section[Question \thesection]{}}

%>=====< Subsection >=====<%

%redefine subsection
\titleformat{\subsection}[hang]{\normalfont\large\bfseries} % format
{} % label
{0pt} % sep
{} % before-code
[ \vspace{0.5ex} ] % after-code

%redefine subsection*
\titleformat{name=\subsection, numberless}[hang]{\normalfont\large\bfseries} % format
{} % label
{0pt} % sep
{} % before-code
[ \vspace{0.5ex} ] % after-code

%>=====< Subsubsection >=====<%

%redefine subsubsection
\titleformat{\subsubsection}[hang]{\normalfont\large\bfseries} % format
{} % label
{0pt} % sep
{} % before-code
[ \vspace{0.5ex} ] % after-code

%make it appear in toc
\setcounter{tocdepth}{4}
\setcounter{secnumdepth}{4}

%>=====< Math sets >=====<%

\newcommand{\N}{\mathbb{N}} %natural
\newcommand{\R}{\mathbb{R}} %real
\newcommand{\Rpos}{\R_{+}} %positive real
\newcommand{\Rcomp}{\overline{\R}} %complete real
\newcommand{\Rcomppos}{\Rcomp_{+}} %complete positive real
\newcommand{\Q}{\mathbb{Q}} %rational
\newcommand{\A}{\mathcal{A}} %calligraphic A
\newcommand{\B}[1][]{
  \ifthenelse{\isempty{#1}}{\mathcal{B}}{\mathcal{B}\left(#1\right)}
}%Borel sigma-algebra
\newcommand{\Parts}[1]{\mathcal{P}\left(#1\right)} %set of all parts
\renewcommand{\complement}{\mathsf{c}} %complement
\newcommand{\Leb}{\mathcal{L}} %Lebesgue sigma-algebra
\newcommand{\Mes}{\mathcal{M}} %measurable functions
\newcommand{\Smes}{\mathcal{S}} %measurable simple functions

%>=====< Math symbols >=====<%

\renewcommand{\epsilon}{\varepsilon} %epsilon
\renewcommand{\theta}{\vartheta} %theta
\renewcommand{\phi}{\varphi} %phi
\DeclareMathOperator*{\esssup}{ess\,sup} %esssup
\DeclareMathOperator*{\essinf}{ess\,inf} %essinf
\DeclareMathOperator*{\interior}{int} %interior 
\DeclareMathOperator*{\dist}{dist} %distance 
\DeclareMathOperator{\spn}{span} %span
\DeclareMathOperator{\supp}{supp} %support

%>=====< Math jargon >=====<%

\newcommand{\salg}{\sigma-\text{algebra}} %sigma-algebra
\newcommand{\s}[1]{\sigma-\text{#1}}%sigma-smth
\newcommand{\provdef}[1][]{
  \ifthenelse{\isempty{#1}}{Let us define:}{Let us define #1 as:}
} %smart definition
%define #1
\newcommand{\provdefs}{Let us define the following:} %multiple definitions
\newcommand{\seq}[1]{ \{ #1_n \} } %sequence
\newcommand{\subseq}[1]{ \{ #1_{n_k} \} } %subsequence
\newcommand{\tfae}{the following are equal:}  %the following are equal
\newcommand{\toin}[1]{\xrightarrow[n\to\infty]{#1} } %to in 
\newcommand*{\cvrule}{\smash{\rule{0.4pt}{2ex}}}

%>=====< Math commands added for compatibility >=====<%

%\newcommand{\proj}{\mbox{Proj_{H}}} this was the original, I changed it for consistency and good practice
\DeclareMathOperator*{\proj}{Proj} %projection
\newcommand{\wc}{\rightharpoonup}
\newcommand{\wsc}{\overset{\ast}{\rightharpoonup}}
\newcommand{\toi}{n\to\infty}

%>=====< Custom commands >=====<%

\newcommand{\cancelnum}[2]{\overset{\text{\normalsize#1}}{\xcancel{#2}}}

%>=====< Document >=====<%

\begin{document}
%==== Title ====% see https://www.overleaf.com/learn/latex/How_to_Write_a_Thesis_in_LaTeX_(Part_5)%3A_Customising_Your_Title_Page_and_Abstract for reference
\begin{titlepage}
    \begin{center}
        \vspace*{\fill}
        \Huge
        Answers to the Theory Questions\\
        \vspace{0.5em}
        \Large
        of the course of Real and Functional Analysis\\
        taught by prof. Fabio Punzo \\
        \vspace{0.5em}
        by Jacopo Stringara \\
        \vspace{0.5em}
        \href{mailto:jstringara@gmail.com}{jstringara@gmail.com} \\
        \vspace{0.5em}
        \today
        \vfill
        The source code for this document (and many more) can be found at:\\
        \href{https://github.com/jstringara/Latex-projects/tree/master/ARF}{https://github.com/jstringara/Latex-projects/tree/master/ARF}
    \end{center}
\end{titlepage}


%>=====< Table of contents >=====<%

\tableofcontents
\newpage

%>=====< Sheets >=====<%

\sheet








\sheet


%>=====< Question 1 >=====<%

\question
Write the definition of complete measure space. State the theorem concerning the existence of the completion of a measure space. Give just an idea of the proof.

\subsection*{Solution}

\subsection{Complete measure space}
A measure space $(X,\A,\mu)$ is said to be complete if $\tau_{\mu}\subseteq\A$

\subsection{Existence of the completion}
Let $(X,\A,\mu)$ be a measure space. \provdef{$\bar{\A}, \bar{\mu}$}
\begin{align*}
    \bar{\A}  & =\{ E\subseteq X: \exists F,G \in \A \text{ s.t. } F\subseteq E \subseteq G \; \mu(G \setminus F) =0 \} \\
    \bar{\mu} & : \bar{\A}\to\Rcomppos,\quad \bar{\mu}(E) \coloneqq \mu(F)
\end{align*}
then:
\begin{enumerate}
    \item $\bar{\A}$ is a $\salg$ , $\bar{\A} \supseteq \A$
    \item $\bar{\mu}$ is a complete measure, $\bar{\mu}|_{\A}=\mu$
\end{enumerate}
and the triplet $(X,\bar{A}, \bar{\mu})$ is a complete measure space and is called the completion of $(X,\A,\mu)$, i.e. it the smallest (w.r. to inclusion) complete measure space that cointains $(X,\A,\mu)$
\subsection*{Sketch of proof} \footnote{
    This is a partial proof of my own making. It has been review by the TA and professor Punzo and stated to be correct.
}
We  must prove two things:
\begin{itemize}
    \item \textbf{First:} that $\bar{\A}$ is a $\salg$ and that it contains $\A$, the latter is trivial since $\forall A\in\A \quad A\subseteq A\subseteq A \implies A\in\bar{\A}$ while the former is quite hardous so we shall just assume it to be true.
    \item \textbf{Second:} that $\bar{\mu}$ is a complete measure and $\bar{\mu}|_{\A}=\mu$.\\
          The latter is trivial (see above). We can also easily prove that it is a measure:
          \begin{enumerate}[i)]
              \item $\bar{\mu}(\emptyset)=\mu(\emptyset)=0$ since the only set contained inside $\emptyset$ is $\emptyset$ itself, as the container set we may take any zero set measure inside $\A$.
              \item that $\s{additivity}$ holds is clear since for any disjoint sequence $\seq{E}\subseteq\bar{\A}$ we may construct two sequences:
                    \[
                        \left\{ \begin{array}{l}
                            \seq{F}, \; F_k \subseteq E_k \\
                            \seq{G}, \; G_k \supseteq E_k
                        \end{array} \right. \forall k\in\N \text{ s.t. } \mu(G_k\setminus F_k) = 0
                    \]
                    Let us note the following:
                    \begin{itemize}
                        \item $\seq{F}$ is also disjoint because $\seq{E}$ is disjoint.
                        \item Moreover:
                              \begin{align*}
                                   & \bigcup_{k=1}^{\infty} F_k \subseteq \bigcup_{k=1}^{\infty} E_k \subseteq \bigcup_{k=1}^{\infty} G_k                                                                                                 \\
                                   & \bigcup_{k=1}^{\infty} G_k \setminus \bigcup_{k=1}^{\infty} F_k \subseteq \bigcup_{k=1}^{\infty} (G_k \setminus F_k )                                                                                \\
                                   & \mu\left(\bigcup_{k=1}^{\infty} G_k \setminus \bigcup_{k=1}^{\infty} F_k \right) \leq \mu\left(\bigcup_{k=1}^{\infty} (G_k \setminus F_k )\right) \leq \sum_{k=1}^{\infty} \mu(G_k\setminus F_k) = 0
                              \end{align*}
                              The last inequality is true thanks to the $\s{subadditivity}$ and monotonicty of $\mu$.
                    \end{itemize}
                    Thus we can say that:
                    \[
                        \bar{\mu}\left( \bigcup_{k=1}^{\infty} E_k \right) = \mu \left( \bigcup_{k=1}^{\infty} F_k \right) = \sum_{k=1}^{\infty} \mu(F_k) = \sum_{k=1}^{\infty} \bar{\mu}(E_k)
                    \]
          \end{enumerate}
          thus $\bar{\mu}$ is a measure.\\
          Let us prove that $\bar{\mu}$ is complete.
          Let $E_1 \in X$ and $E_2 \in \bar{\A}$ such that $\bar{\mu}(E_2)=\mu(F_2)=0$ and $E_1 \subseteq E_2$, let us note that:
          \[
              \left\{ \begin{array}{l}
                  \mu(G_2) = \cancelnum{0}{\mu(G_2\setminus F_2)} + \cancelnum{0}{\mu(F_2)}\\
                  \mu(G_2 \setminus \emptyset) = \mu(G_2) - 0                       \\
                  \emptyset \subseteq E_1 \subseteq G_2
              \end{array} \right. \implies E_1\in \bar{\A}, \; \bar{\mu}(E_1)=\mu(\emptyset) = 0
          \]
          thus any negligible set is also a zero measure set and $\bar{\mu}$ is complete.
\end{itemize}

%>=====< Question 2 >=====<%

\question
Write the definition of outer measure. State and prove the theorem concerning generation of
outer measure on a general set $X$, starting from a set $K \in\Parts{X}$, containing $\emptyset$, and a function
$\nu : K \to \Rcomppos, \; \nu(\emptyset) = 0$. Intuitively, which is the meaning of $(K, \nu)$?

\subsection*{Solution}

\subsection{Outer measure}\label{outer:def}
We say that a function: $\mu^*:\Parts{X}\to\Rcomppos$ (where $X$ is any set) is an outer measure if:
\begin{enumerate}[i)]
    \item $\mu^*(\emptyset)=0$
    \item \label{outer:mono}$E_1\subseteq E_2 \implies \mu^*(E_2) \leq \mu^*(E_2)$
    \item \label{outer:sub}$\mu^*\left( \bigcup_{k=1}^{\infty} E_k \right) \leq \sum_{k=1}^{\infty} \mu^*(E_k)$
\end{enumerate}

\subsection{Generation of an outer measure} \label{outer:gen}
Let $K\subseteq\Parts{X}, \, \emptyset\in K, \: \nu:K\to\Rcomppos, \; \nu(\emptyset)=0$, then we can generate an outer measure $\mu^*$ on $X$ defined as:
\[
    \left\{ \begin{array}{l}
        \mu^*(E) \coloneqq \inf \left\{ \sum_{k=1}^{\infty} \nu(I_k) : E\subseteq \bigcup_{k=1}^{\infty} I_k,\; \seq{I}\subseteq K \right\} , \text{ if } E \text{ can be covered by a countable union of sets } I_n\in K. \\
        \mu^*(E) \coloneqq +\infty, \text{ otherwise.}
    \end{array} \right.
\]

\begin{proof}
    Let us verify that such a $\mu^*$ meets the definition of outer measure (\ref{outer:def}):
    \begin{enumerate}[i)]
        \item $\emptyset\in K$, $0\leq\mu^*(\emptyset)\leq\nu(\emptyset)=0$ by the definition of $\mu^*$.
        \item $E_1\subseteq E_2$, we have two possible cases
              \begin{itemize}
                  \item if there exists a countable covering of $E_2$ then it is also a covering of $E_1$ and from the definitio of $\mu^*$ it follows that:
                        \[
                            \mu^*(E_1) \leq \mu^*(E_2)
                        \]
                  \item if there is no countable covering of $E_2$ then:
                        \[
                            \mu^*(E_1) \leq \mu^*(E_2) = +\infty
                        \]
              \end{itemize}
        \item this condition is obviously met if:
              \[
                  \sum_{k=1}^{\infty} \mu^*(E_k) = +\infty
              \]
              otherwise if we suppose that:
              \[
                  \sum_{k=1}^{\infty} \mu^*(E_k) < +\infty
              \]
              thus $\mu^*(E_k)<+\infty$ $\forall k\in\N$, by the definition of $\mu^*$ and $\inf$:
              \[
                  \forall \epsilon>0, \; \forall n\in\N \quad \exists \{ I_{n,k} \} \subseteq K
              \]
              such that:
              \[
                  E_n \subseteq \bigcup_{k=1}^{\infty} I_{n,k} \quad \text{ and } \quad \mu^*(E_n)+\frac{\epsilon}{2^n} > \sum_{k=1}^{\infty} \nu(I_{n,k})
              \]
              Now, since:
              \[
                  \bigcup_{n=1}^{\infty} E_n \subseteq \bigcup_{n,k=1}^{\infty} I_{n,k}, \quad \{ I_{n,k} \} \subseteq K
              \]
              it clearly follows that:
              \[
                  \mu^*(\bigcup_{n=1}^{\infty} E_n) \leq \sum_{n=1}^{\infty} \sum_{k=1}^{\infty} \nu(I_{n,k}) < \sum_{n=1}^{\infty}\mu^*(E_n) + \epsilon\cdot\cancelnum{1}{\sum_{n=1}^{\infty} \frac{1}{2^n}}
              \]
              because $\epsilon$ is arbitrary, we have the cocnlusion.
    \end{enumerate}
\end{proof}

The intuitive meaning $(K,\nu)$ is that $K$ is a special class of sets in $X$ and $\nu$ is a function that assigns a value to each set in $K$. On the other hand $\nu$ can be any real valued positive function, thus it is not necessary to be a measure.

%>=====< Question 3 >=====<%

\question
What is the Caratheodory condition? How can it be stated in an equivalent way? Prove it.

\subsection*{Solution}

\subsection{Caratheodory condition} \label{CarEq}
Let $\mu^*$ be an outer measure on a set $X$, then we say that $E\subset X$ is $\mu^*$-measurable if:
\[
    \mu^*(Z) = \mu^*(Z\cap E) + \mu^*(Z\setminus E) \quad \forall Z\in X
\]

\subsection{Equivalent statement}\label{CarIneq}
Let $\mu^*$ be an outer measure on a set $X$, then we say that $E\subset X$ is $\mu^*$-measurable if:
\[
    \mu^*(Z) \geq \mu^*(Z\cap E) + \mu^*(Z\setminus E) \quad \forall Z\in X
\]
\begin{proof}
    It is enough to note that $\forall E\subseteq X$ we have:
    \[
        Z = (Z\cap E) \cup (Z \cap E^\complement) \quad \forall Z\in X
    \]
    and thus by the subadditivity of $\mu^*$ (\ref{outer:sub}) we get:
    \[
        \mu^*(Z) \leq \mu^*(Z\cap E) + \mu^*(Z\setminus E) \quad \forall Z\in X
    \]
    and we may combine this inequality with the other to yield an equality.
\end{proof}

%>=====< Question 4 >=====<%

\question
Can it exist a set of zero outer measure, which does not fulfill the Caratheodory condition? Prove it.

\subsection*{Solution}

\subsection{All zero measure sets are in \texorpdfstring{$\Leb$}{L}} \label{zerosetsaremeas}
There cannot exist such a set $E$ because all sets of zero aouter measure meet the Caratheodory Inequality (\ref{CarIneq}).

\begin{proof}\label{outer:zeromeas}
    Indeed $\forall Z \subseteq X$ by the monotonicty of $\mu^*$ (\ref{outer:mono}) we have:
    \[
        \mu^*(\underbrace{Z\cap E}_{\subseteq E}) + \mu^*(\underbrace{Z\setminus E}_{\subseteq Z}) \leq \cancelnum{0}{\mu^*(E)} + \mu^*(Z)
    \]
\end{proof}


%>=====< Question 5 >=====<%

\question
State the theorem concerning generation of a measure as a restriction of an outer measure.

\subsection*{Solution}
\subsection{Generation of a measure from an outer measure}\label{meas:gen}
\provdef[$\mathcal{L}$]
\[
    \mathcal{L} \coloneqq \{ E\subseteq X : \; E \text{ is } \mu^*-\text{measurable } \}
\]
where $\mu^*$ is an outer measure on $X$, then:
\begin{enumerate}[i)]
    \item the collection $\Leb$ is a $\salg$
    \item $\mu^* |_{\Leb}$ is a complete measure on $\Leb$
\end{enumerate}

%>=====< Question 6 >=====<%

\question
Show that the measure induced by an outer measure on the $\salg$ of all sets fulfilling the
Caratheodory condition is complete.

\subsection*{Solution}
\subsection{Generation of a measure from an outer measure (proof of completeness)}
Let us see that such a measure as the one described in the previous question is complete. Let $\mu^*$ be an outer measure on $X$ and $\Leb$ the $\salg$ of all sets fulfilling the Caratheodory condition. Let $\mu$ be the measure induced by $\mu^*$ on $\Leb$ ($\mu=\mu^* |_{\Leb}$).
\begin{proof}
    Let $N\in\Leb$ such that $\mu(N)=\mu^*(N)=0$ and let $E\subseteq N$.\\
    By monotonicty of $\mu^*$ (\ref{outer:mono}):
    \[
        0\leq \mu^*(E)\leq \mu^*(N)=0 \implies \mu^*(E)=0
    \]
    thus by the lemma seen in \ref{outer:zeromeas} we get that $E\in\Leb$ and so $\Leb$ is complete.
\end{proof}

%>=====< Question 7 >=====<%

\question
Describe the construction of the Lebesgue measure in $\R$ and in $\R^n$.

\subsection*{Solution}

\subsection{Construction of the Lebesgue measure on \texorpdfstring{$\R$}{R}}
Let $I$ be a family of open, bounded intervals in $\R$:
\[
    I \coloneqq \{ (a,b) : a,b\in\R, a\leq b \}
\]
Let us note that $\emptyset\in I$.\\
Now let us consider a function $\lambda_0$:
\begin{align*}
     & \lambda_0 : I \to \R_+   \\
     & \lambda_0 (\emptyset) =0 \\
     & \lambda_0 ((a,b)) = b-a
\end{align*}
Here we take $X=\R$, $(K,\nu)=(I,\lambda_0)$ and construct the outer Lebesgue measure $\lambda^*$ as seen above (\ref{outer:gen}):
\[
    \lambda^*(E) \coloneqq \left\{ \begin{array}{l}
        \inf \left\{ \sum_{n=1}^{\infty} \lambda_0(I_n) \, : \quad E\subseteq \bigcup_{n=1}^{\infty} I_n ,\; \seq{I}\subseteq I \right\}, \quad \forall E\subseteq\R \text{ s.t. } E \text{ has a countable covering }\seq{I}\subseteq I \\
        +\infty, \text{ otherwise}
    \end{array}\right.
\]
The corresponding $\salg$ is the Lebesgue $\salg$ $\Leb(\R)$ and now we define the Lebesgue measure $\lambda$ as the measure generated by the outer Lebesgue measure (as seen in \ref{meas:gen}):
\[
    \lambda \coloneqq \lambda^*|_{\Leb(\R)}
\]

\subsection{Construction of the Lebesgue measure on \texorpdfstring{$\R^n$}{Rn}}
Analogously to what we have seen above we first define an outer measure and then a (complete) measure but we take:
\[
    I^n = \left\{ \bigtimes_{k=1}^{n} (a_k,b_k): \; a_k,b_k\in\R, \; a_k\leq b_k  \right\}
\]
and accordingly we define:
\begin{align*}
     & \lambda_0^n : I^n \to \R_+                                                         \\
     & \lambda_0^n (\emptyset) = 0                                                        \\
     & \lambda_0^n \left( \bigtimes_{k=1}^{n} (a_k,b_k) \right) = \prod_{k=1}^n (b_k-a_k)
\end{align*}
and therefore we take $X=\R^n$ and $(K,\nu)=(I^n,\lambda_0^n)$, we define the outer Lebesgue measure $\lambda^{*,n}$ on $\R^n$ and the Lebesgue $\salg$ $\Leb(\R^n)$ and finally we construct the n-dimensional Lebesgue measure as:
\[
    \lambda^n \coloneqq \lambda^{*,n} |_{\Leb(\R^n)}
\]

%>=====< Question 8 >=====<%

\question
Prove that any countable subset $E\subset\R$ is Lebesgue measurable and $\lambda(E) = 0$.

\subsection*{Solution}

\subsection{All countable sets are \texorpdfstring{$\Leb$}{L}-measurable and \texorpdfstring{$\lambda(E)=0$}{l(E)=0}}
Any countable subset $E\subset\R$ is $\Leb$-measurable and $\lambda(E)=0$
\begin{proof}
    Let $a\in\R$, clearly $\{ a \} \subseteq (a-\epsilon, a]$ $\forall \epsilon >0$, thus by the definition of $\lambda^*$:
    \[
        \lambda^*(\{ a \}) \leq \lambda^*( (a-\epsilon, a] )  = \epsilon \to 0  \implies \{ a \}\in\Leb
    \]
    Now if E is countable we may write as follows:
    \[
        E = \bigcup_{n=1}^{\infty} \{ a_n \} \quad a_n\in\R, \; n\in\N
    \]
    and so by monotonicty (\ref{outer:mono}):
    \[
        0 \leq \lambda^*(E) = \lambda^*\left( \bigcup_{n=1}^{\infty} \{ a_n \} \right) \leq \sum_{n=1}^{\infty} \lambda^*(a_n) = 0
    \]
    thus $\lambda^*(E)=0 \implies E\in \Leb$ by the lemma seen above (\ref{outer:zeromeas})
\end{proof}

%>=====< Question 9 >=====<%

\question
Show that $\B[\R] \subseteq \Leb(\R)$. Is the inclusion strict? Which is the relation between $(\R,\Leb(\R), \lambda)$ and
$(\R, \B[\R], \lambda)$?

\subsection*{Solution}

\subsection{\texorpdfstring{$\B[\R] \subseteq \Leb(\R)$}{B(R) is included in L(R)}}
\begin{proof}
    Since $\B[\R]=\sigma_0((a,+\infty))$ it is enough to show that $(a,+\infty)\in\Leb(\R)$. We already know from above that all bounded intervals belong to $\Leb(\R)$. \\
    Now, let $A\subseteq\R$ be any set. We assume $a\notin A$, otherwise we would replace $A$ with $A\setminus\{a\}$ and this would leave the Lebesgue outer measure unchanged. Furthermore $(a,+\infty)\in\Leb(\R) \iff (a,+\infty)$ satisfies the Caratheodory Condition (\ref{CarIneq}):
    \[
        \lambda^*(A_1)+\lambda^*(A_2)\leq\lambda^*(A) \label{a}
    \]
    where $A_1 = A\cap (-\infty,a)$ and $A_2 = A \cap (a,+\infty)$.\\
    Since $\lambda^*(A)$ is defined as an $\inf$, to verify the above, it is necessary and sufficient to show that for \textbf{any countable collection} $\seq{I}$ of \textbf{open bounded} intervals that \textbf{covers} $A$ we have that:
    \[
        \lambda^*(A_1)+\lambda^*(A_2)\leq \sum_{k=1}^{\infty} \lambda_0(I_k)
    \]
    For every $k\in\N$ we define:
    \begin{align*}
        I_k' \coloneqq I_k \cap (-\infty, a) \\
        I_k'' \coloneqq I_k \cap (a,+\infty)
    \end{align*}
    then:
    \[
        I_k' \cap I_k'' = \emptyset (\text{disjoint}) \implies \lambda_o(I_k) = \lambda_0(I_k') + \lambda_0 (I_k'')
    \]
    Let us note that $\seq{I'}$ is a countable cover for $A_1$ and $\seq{I''}$ is a countable cover for $A_2$.
    Hence:
    \begin{align*}
        \lambda^*(A_1)=\sum_{k=1}^{\infty} \lambda_0(I_k') \\
        \lambda^*(A_2)=\sum_{k=1}^{\infty} \lambda_0(I_k'')
    \end{align*}
    therefore:
    \[
        \lambda^*(A_1)+\lambda^*(A_2)\leq \sum_{k=1}^{\infty} \lambda_0(I_k') + \sum_{k=1}^{\infty} \lambda_0(I_k'') = \sum_{k=1}^{\infty} \lambda_0(I_k)
    \]
    which equivalento to the condition above.
\end{proof}

\subsection{\texorpdfstring{$\B[\R] \subsetneqq \Leb(\R)$}{B(R) is strictly included in L(R)}}
The inclusion demonstrated above can be shown to be strict. A counterexample can be produced (see \href{https://math.stackexchange.com/questions/141017/lebesgue-measurable-set-that-is-not-a-borel-measurable-set}{\color{blue}{here}}) but it is quite pathological.

\subsection{Relation between \texorpdfstring{$(\R,\Leb(\R),\lambda)$}{(R,L(R),l)} and \texorpdfstring{$(\R,\B[\R],\lambda)$}{(R,B(R),l)}}
$(\R,\Leb(\R),\lambda)$ is the completion of $(\R,\B[\R],\lambda|_{\B[\R]})$. Indeed as we have shown above $\B[\R]$ is not a complete $\salg$ while $\Leb(\R)$ is.

%>=====< Question 10 >=====<%

\question
Is the translate of a measurable set measurable?

\subsection*{Solution}

\subsection{The translate of a measurable set is measurable}
The translate of a measurable set is also measurable. \\
Let us see a simple example: let $(a,b)$ be an interval and $(a+h,b+h)$ its translate.
\begin{align*}
     & \lambda((a,b)) = b-a                   \\
     & \lambda((a+h,b+h)) = (b+h)-(a+h) = b-a
\end{align*}

%>=====< Question 11 >=====<%

\question
Write the excision property and prove it. Write and prove (partially) the theorem concerning
the regularity of the Lebesgue measure on $\R$.

\subsection*{Solution}
\subsection{Excision property}\label{ExcProp}
If $A\in\Leb(\R)$, $\lambda^*(A)\leq +\infty$ and $A\subseteq B$, then:
\[
    \lambda^*(B\setminus A) = \lambda^* (B) - \lambda^*(A)
\]
\begin{proof}
    Since $A\in\Leb(\R)$ we can use the Caratheodory equality (\ref{CarEq}) using $Z=B$, $E=A$:
    \[
        \lambda^*(B) = \lambda^*(\underbrace{B\cap A}_{=A \; (A\subseteq B)}) + \lambda^* (B\setminus A)
    \]
    so, since $\lambda^*(A)\leq +\infty$ we may write:
    \[
        \lambda^*(B\setminus A) = \lambda^*(B)-\lambda^*(A)
    \]
\end{proof}

\subsection{Regularity of the Lebesgue Measure}
Let $E\subseteq\R$, \tfae
\begin{enumerate}[i)]
    \item\label{LebReg:1} $E\in\Leb(\R)$
    \item\label{LebReg:2} $\forall \epsilon >0$ $\exists A \subseteq \R$ open s.t.
          \[
              E\subseteq A \quad \lambda^*(A\setminus E) < \epsilon
          \]
    \item\label{LebReg:3} $\exists G \subseteq \R$ in the class $G_{\delta}$ (countable intersections of open sets) s.t.
          \[
              E\subseteq G \quad \lambda^*(G\setminus E)=0
          \]
    \item\label{LebReg:4} $\forall \epsilon >0$ $\exists C \subseteq \R$ closed s.t.
          \[
              C\subseteq E \quad \lambda^*(E\setminus C) < \epsilon
          \]
    \item\label{LebReg:5} $\exists F \subseteq \R$ in the class $F_{\delta}$ (countable unions of closed sets) s.t.
          \[
              F\subseteq E \quad \lambda^*(E\setminus F)=0
          \]
\end{enumerate}
\begin{proof}
    Let us give a (partial) proof:\\
    \begin{itemize}
        \item
              $(\ref{LebReg:1})\implies(\ref{LebReg:2})$: if $E\in\Leb(\R)$, $\lambda(E)<+\infty$ then by definition of outer measure (\ref{outer:def}):
              \[
                  \forall \epsilon >0 \; \exists\seq{I}\text{ that covers } E \text{ and } \sum_{k=1}^{\infty} \lambda_0(I_k) < \lambda^*(E)+\epsilon
              \]
              Let us now define the set $O$:
              \[
                  O\coloneqq \bigcup_{k=1}^{\infty} I_k, \; O \text{ is open}, \; E\subseteq O
              \]
              and so we may write:
              \begin{align*}
                   & \lambda^*(O) \overset{sub-add \; (\ref{outer:sub})}{\leq} \sum_{k=1}^{\infty} \lambda_0(I_k) < \lambda^*(E)+\epsilon \\
                   & \implies \lambda^*(O)-\lambda^*(E) < \epsilon
              \end{align*}
              and by the Excision property (\ref{ExcProp}) ($E\in\Leb(\R), \; \lambda^*(E)<+\infty$):
              \[
                  \lambda^*(O\setminus E) = \lambda^*(O)-\lambda^*(E) < \epsilon
              \]
              and so we have obtained the second statement (\ref{LebReg:2}).
        \item
              $(\ref{LebReg:2}) \implies (\ref{LebReg:3})$, $\forall k\in\N$ we choose $O_k \supseteq E$ open for which:
              \[
                  \lambda^*(O_k\setminus E) < \frac{1}{k}
              \]
              and then define:
              \[
                  G = \bigcap_{k=1}^{\infty} O_k \implies G\in G_{\delta}, \; G\supseteq E
              \]
              Moreover $\forall k\in \N$:
              \[
                  G\setminus E \subseteq O_k\setminus E
              \]
              so by monotonicty (\ref{outer:mono}):
              \[
                  \lambda^*(G\setminus E) \leq \lambda^*(O_k\setminus E) < \frac{1}{k}
              \]
              let us apply a limit $k\to\infty$ to both sides:
              \[
                  \lambda^*(G\setminus E) = 0
              \]
        \item
              $(\ref{LebReg:3}) \implies (\ref{LebReg:1})$, let us note that $G\setminus E\in\Leb(\R)$ since $\lambda^*(G\setminus E)=0 $ by lemma \ref{zerosetsaremeas} and:
              \begin{align*}
                   & G\in\Leb(\R) \text{ since } G\in G_{\delta} \subseteq \B[\R] \subseteq \Leb(\R)                  \\
                   & \implies E = \underset{\in\Leb}{G} \cap (\underset{\in\Leb}{G \setminus E})^\complement \in \Leb
              \end{align*}

    \end{itemize}
\end{proof}

%>=====< Question 12 >=====<%

\question
Is it true that any subset $E\subseteq\R$ is $\Leb$-measurable? Is it possibile to find two disjoint sets
$A,B\subset\R$ for which $\lambda^*(A \cup B) < \lambda^*(A) + \lambda^*(B)$? Why?

\subsection*{Solution}

\subsection{Vitali's non-measurable sets}
Any measurable set $E\subseteq \R$ with $\lambda(E)>0$ contains a subset that fails to be measurable. \\
Therefore there exist subsets of $\R$ that are not $\Leb$-measurable.

\subsection{Disjoints sets for which \texorpdfstring{$\lambda^*(A\cup B) < \lambda^*(A)+\lambda^*(B)$}{the measure of the union is less than the sum of the measure}}
There are disjoint sets $A,B\subseteq\R$ for which:
\[
    \lambda^*(A\cup B) < \lambda^*(A)+\lambda^*(B)
\]

\begin{proof}
    Assume by contradiction that:
    \[
        \lambda^*(A\cup B) = \lambda^*(A)+\lambda^*(B) \quad \forall A,B \subseteq\R, \; A\cap B = \emptyset
    \]
    Now $\forall E,Z \subseteq \R$ we write:
    \[
        \lambda^*(\underbrace{Z\cap E}_{=A}) + \lambda^*(\underbrace{Z\cap E^{\complement}}_{=B}) = \lambda^*(\underbrace{Z}_{=A\cup B})
    \]
    thus any set $E$ would satisfy the Caratheodory condition (\ref{CarEq}) and be $\Leb$-measurable which is absurd since we know that Vitali's sets exist.
\end{proof}

\sheet

%>=====< Question 1 >=====<%

\question

Write the definition of measurable function. Show the measurability of the composite function.

\subsection*{Solution}

\subsection{Measurable function}

Let $(X,\A)$ and $(X',\A')$ be two measurable spaces and $f$ a function:
\[
    f:X\to X'
\]
$f$ is said to be measurable if:
\[
    f^{-1}(A)\in\A\quad\forall A\in\A'
\]

\subsection{Measurability of the composite function} \label{meas:comp}
Let $(X,\A)$, $(X',\A')$ and $(X'',\A'')$ be three measurable spaces and $f:X\to X'$ and $g:X'\to X''$ two measurable functions. Then the composite function $g\circ f:X\to X''$ is measurable.

\begin{proof}
    \begin{align*}
         & \forall E \in \A' \quad f^{-1}(E)\in\A   \\
         & \forall F \in \A'' \quad g^{-1}(F)\in\A' \\
    \end{align*}
    thus:
    \[
        \forall F\in \A'' \quad (g\circ f)^{-1}(F) = f^{-1} \left[ \underbrace{g^{-1}(F)}_{\coloneqq E \in \A'} \right] \in\A
    \]
\end{proof}

%>=====< Question 2 >=====<%

\question

Characterize measurability of functions and prove it.

\subsection*{Solution}

\subsection{Characterization of Measurability}\label{CharMeas}

Let $(X,\A)$ and $(X',\A')$ be two measurable spaces and $\mathcal{C} ' \subseteq \Parts{X'}$ such that $\sigma_0(\mathcal{C}')=\A'$ then:
\[
    f:X\to X' \text{ measurable } \iff f^{-1}(E)\in\A \quad \forall E\in\mathcal{C}'
\]

\begin{proof}
    Let us prove both sides of the implication:
    \begin{itemize}
        \item \textbf{$(\implies)$:} Suppose $f$ be measurable $\implies$ $\mathcal{C}'\subseteq \A'$ and so we get the thesis.
        \item \textbf{$(\impliedby)$:} Let us define the following:
              \[
                  \Sigma \coloneqq \{ E\subseteq X': \; f^{-1}(E)\in\A \}
              \]
              We can easily see that $\Sigma$ is a $\salg$ so $\mathcal{C}'\subseteq \Sigma$ and thus:
              \[
                  \A'=\sigma_0(\mathcal{C}')\subseteq\Sigma
              \]
              and we get the thesis.
    \end{itemize}
\end{proof}

%>=====< Question 3 >=====<%

\question

Write the definitions of:
\begin{enumerate}[a)]
    \item \label{Bmeas} Borel measurable functions;
    \item \label{Lmeas} Lebesgue measurable functions.
\end{enumerate}

\subsection*{Solution}

\subsection{\ref{Bmeas}) Borel measurable functions}
Let $(X,d), (X,\B)$ and $(X',d'), (X', \B')$ be couples of metric spaces and measurable spaces. A function $f$:
\[
    f:X\to X' \text{ measurable}
\]
is called Borel-measurable or $\B$-meaurable.

\subsection{\ref{Lmeas}) Lebesgue measurable functions}
Let $(X,\Leb)$ be a measurable space and $(X',d')$ a metric space, $(X',\B')$ a measurable space, then:
\[
    f:X\to X' \text{ measurable}
\]
is called Lebesgue-measurable or $\Leb$-measurable.

%>=====< Question 4 >=====<%

\question

Prove that continuous functions are both Borel and Lebesgue measurable.

\subsection*{Solution}

\subsection{Continuous functions are \texorpdfstring{$\B$}{Borel}-measurable}
A continuous function $f:X\to X'$ is $\B$-measurable.

\begin{proof}
    Let $\mathcal{C}'$ be the class of open sets of $X'$ and $\mathcal{C}$ the class of open sets of $X$. We have:
    \[
        \forall E \in \mathcal{C}' \quad f^{-1}(E)\in\mathcal{C} \subseteq \B \; (\text{ by definition of continuity })
    \]
    and $\B'=\sigma_0(\mathcal{C}')$ so we get the thesis.
\end{proof}

\subsection{Continuous functions are \texorpdfstring{$\Leb$}{Lebesgue}-measurable}
A continuous function $f:X\to X'$ is $\Leb$-measurable.

\begin{proof}
    Since $\B \subset \Leb$ and the previous statement has been proven true, the thesis follows trivially.
\end{proof}

%>=====< Question 5 >=====<%

\question

Characterize Lebesgue measurability of functions and prove it.

\subsection*{Solution}

\subsection{Characterization of Lebesgue measurability}
All we must do is apply the Characterization of Measurability (\ref{CharMeas}) taking $(X,\A=\Leb)$, $(X', \A'=\B')$ and $\mathcal{C}'$ the class of open sets of $X'$, since $\B'=\sigma_0(\mathcal{C}')$. We then can write:
\[
    f:X\to X' \text{ Lebesgue measurable } \iff f^{-1}(E)\in\Leb \quad \forall E\in\mathcal{C}'
\]

\begin{proof}
    Let us prove both sides of the implication:
    \begin{itemize}
        \item \textbf{$(\implies)$:} Suppose $f$ be Lebesgue measurable $\implies$ $\mathcal{C}'\subseteq \B'$ and so we get the thesis.
        \item \textbf{$(\impliedby)$:} Let us define the following:
              \[
                  \Sigma \coloneqq \{ E\subseteq X': \; f^{-1}(E)\in\Leb \}
              \]
              We can easily see that $\Sigma$ is a $\salg$ so $\mathcal{C}'\subseteq \Sigma$ and thus:
              \[
                  \B'=\sigma_0(\mathcal{C}')\subseteq\Sigma
              \]
              and we get the thesis.
    \end{itemize}
\end{proof}

%>=====< Question 6 >=====<%

\question

Establish and show all equivalent statements to the fact that $f : X \to \Rcomp$ is measurable.

\subsection*{Solution}

\subsection{Equivalent statements of measurability}
Let $(X,\A)$ be a measurable space and $f:X\to\Rcomp$ a function, \tfae

\begin{enumerate}[i)]
    \item \label{statomeas:1} $f$ is measurable;
    \item \label{statomeas:2} $\{ f>\alpha \}\in\A$ $\forall\alpha\in\R$;
    \item \label{statomeas:3} $\{ f\geq\alpha \}\in\A$ $\forall\alpha\in\R$;
    \item \label{statomeas:4} $\{ f<\alpha \}\in\A$ $\forall\alpha\in\R$;
    \item \label{statomeas:5} $\{ f\leq\alpha \}\in\A$ $\forall\alpha\in\R$.
\end{enumerate}

\begin{proof}
    Let us prove all the coimplications:\\
    \textbf{(\ref{statomeas:1}) $\iff$ (\ref{statomeas:3}):}
    \begin{align*}
         & \A'=\B[\Rcomp]=\sigma_0( \overbrace{ \{(\alpha,+\infty]:\; \alpha\in\R\} }^{\mathcal{C}'} )                   \\
         & f \text{ is measurable } \iff f^{-1} ( \underbrace{(\alpha,+\infty]}_{E} ) \in \A \quad \forall \alpha \in \R
    \end{align*}
    \textbf{(\ref{statomeas:2}) $\implies$ (\ref{statomeas:3}):}
    \[
        \{ f\geq \alpha \} = \cap_{n=1}^\infty \overbrace{\{ f > \alpha-\frac{1}{n} \}}^{\in\A} \in\A
    \]
    \textbf{(\ref{statomeas:3}) $\implies$ (\ref{statomeas:4}):}
    \[
        \{ f<\alpha \} = \{ f\geq\alpha \}^\complement \in\A
    \]
    \textbf{(\ref{statomeas:4}) $\implies$ (\ref{statomeas:5}):}
    \[
        \{ f \leq \alpha \} = \cap_{n=1}^\infty \overbrace{\{ f < \alpha+\frac{1}{n} \}}^{\in\A} \in\A
    \]
    \textbf{(\ref{statomeas:5}) $\implies$ (\ref{statomeas:2}):}
    \[
        \{ f>\alpha \} = \{ f\leq\alpha \}^\complement \in\A
    \]
\end{proof}

%>=====< Question 7 >=====<%

\question

Let $f, g \in \Mes(X, \A)$. What can we say about measurability of $\{f < g\},\; \{f \leq g\},\; \{f = g\}$? Justify the answer.

\subsection*{Solution}

\subsection{Measurability of \texorpdfstring{$\{f < g\},\; \{f \leq g\},\; \{f = g\}$}{ \{f less than g\}, \{f less or equal to g\}, \{f equal to g\}}}
Let $f, g \in \Mes(X, \A)$, we have:
\begin{enumerate}[i)]
    \item $\{f < g\}\in\A$
    \item $\{f \leq g\}\in\A$
    \item $\{f = g\}\in\A$
\end{enumerate}

\begin{proof}
    \hspace*{\fill} %leave a blank line
    \begin{enumerate} [i)]
        \item $\{ f < g \} = \bigcup_{r\in\Q} \left[ \underbrace{\overbrace{\{ f < r \}}^{\in\A} \cap \overbrace{\{ r < g\}}^{\in\A}}_{\in\A} \right]$
        \item $ \{ f\leq g \} = \{ f > g \}^\complement\in\A$ by the previous point.
        \item $\{f=g\} = \underset{\in\A}{\{ f\leq g \}} \cap \underset{\in\A}{\{ f\geq g \}} \in \A$
    \end{enumerate}
\end{proof}

%>=====< Question 8 >=====<%

\question

Let $\{f_n\} \subset \Mes(X, \A)$. Show that $\sup_n f_n, \inf_n f_n, \limsup_n f_n, \liminf_n f_n \in \Mes(X, \A)$. Can there exist two functions $f, g \in \Mes(X, \A)$ such that $\max\{f, g\} \notin \Mes(X, \A)$? Why?

\subsection*{Solution}

\subsection{Measurability of \texorpdfstring{$\sup_n f_n, \inf_n f_n, \limsup_n f_n, \liminf_n f_n$}{sup fn, inf fn, limsup fn, liminf fn}} \label{meas:extremes}
Let $\{f_n\} \subset \Mes(X, \A)$, we have:
\begin{enumerate}[i)]
    \item $\sup_n f_n \in \Mes(X, \A)$
    \item $\inf_n f_n \in \Mes(X, \A)$
    \item $\limsup_n f_n \in \Mes(X, \A)$
    \item $\liminf_n f_n \in \Mes(X, \A)$
\end{enumerate}

\begin{proof}
    \hspace*{\fill} %leave a blank line
    \begin{enumerate} [i)]
        \item $\forall \alpha \in \R \quad \{ \sup_{n\in\N} f_n > \alpha \} = \bigcup_{n=1}^\infty \{ f_n > \alpha \} \in \A \implies \sup_{n\in\N} f_n \in \Mes$
        \item $\inf_n f_n = - \sup_{n\in\N} (-f_n) \in \Mes(X,\A)$
        \item $\limsup_n f_n = \inf_{k\geq 1} \sup_{n \geq k} f_n \in \Mes(X, \A)$
        \item $\liminf_n f_n = \sup_{k\geq 1} \inf_{n\geq k} f_n \in \Mes(X, \A)$
    \end{enumerate}
\end{proof}

\subsection{Corollary, \texorpdfstring{$\max\{f,g\}\in\Mes$}{max(f,g) is measurable}}
There cannot exist two functions $f, g \in \Mes(X, \A)$ such that $\max\{f, g\} \notin \Mes(X, \A)$.
Indeed we can write:
\[ \max(f,g) = f\cdot \chi_{\{f\geq g\}} + g \cdot \chi_{\{g>f\}} \]
In other words the max can be written as the sum and product of measurable functions, hence it is measurable itself.

%>=====< Question 9 >=====<%

\question

Let $f, g \in \Mes(X, \A)$. Show that $f + g, f\cdot g \in \Mes(X, \A)$.

\subsection*{Solution}

\subsection{Measurability of \texorpdfstring{$f + g, f\cdot g$}{sum f and g, product f and g}} \label{meas:sumprod}
Let $f,g:X\to\R$ and $f,g\in\Mes(X,\A)$, we have that $f+g, f\cdot g \in \Mes(X, \A)$.

\begin{proof}
    Let us define a few new functions $\phi, \psi$ and $\chi$:
    \[
        \left\{ \begin{array}{l}
            \phi(x) = X\to\R^2 \quad \phi(x) \coloneqq \left( f(x), g(x) \right) \\
            \psi(x) = \R^2\to\R \quad \psi(s,t) \coloneqq s+t                    \\
            \chi(x) = \R^2\to\R \quad \chi(s,t) \coloneqq s\cdot t
        \end{array} \right.
        \implies
        \left\{ \begin{array}{l}
            \psi \circ \phi = f + g \\
            \chi \circ \phi = f \cdot g
        \end{array} \right.
    \]
    Now, clearly $\psi,\chi \in C^0(\R^2)$ (hence measurable), let us prove that $\phi $ is also measurable. We use the Characterization of Measurability (\ref{CharMeas}):
    \[
        \phi: (X,\A)\to (\R^2,\B[\R^2]) \text{ is measurable } \iff \forall E \subseteq \R^2 \text{ open } \phi^{-1}(E) \in \A
    \]
    We take:
    \begin{align*}
        E            & = R \coloneqq (a,b) \times (c,d)                                                         \\
        \phi^{-1}(R) & \tikzmarknode{eq1}{=} \{ x\in X: \; (f(x),g(x))\in R \}                                  \\
                     & \tikzmarknode{eq2}{=} \{ x\in X: \; f(x)\in (a,b) \} \cap \{ x\in X: \; g(x)\in (c,d) \} \\
                     & \tikzmarknode{eq3}{=} f^{-1}(a,b) \cap g^{-1}(c,d) \in \A
    \end{align*} \tikz[overlay,remember picture]{\draw[shorten >=1pt,shorten <=1pt] (eq1) -- (eq2) -- (eq3);}
    Thus $\forall E \subseteq \R^2$ open, we may write:
    \begin{align*}
        E         & = \bigcup_{k=1}^\infty R_k, \quad R_k = (a_k,b_k) \times (c_k,d_k)                                       \\
        \phi^{-1} & = \bigcup_{k=1}^\infty \phi^{-1}(R_k) = \bigcup_{k=1}^\infty f^{-1}(a_k,b_k) \cap g^{-1}(c_k,d_k) \in \A
    \end{align*}
    Hence $\phi \in \Mes(X,\A)$, and we have:
    \[
        \psi \circ \phi, \; \chi \circ \phi \in \Mes(X,\A)
    \]
\end{proof}

%>=====< Question 10 >=====<%

\question

Prove that A is measurable if and only if $\chi_A$ is a measurable function.

\subsection*{Solution}

\subsection{A is measurable if and only if \texorpdfstring{$\chi_A$}{the indicator function of A} is a measurable function} \label{AinA:chi}

Let $A\subseteq X$ and $\chi_A$ be the indicator function of $A$. We have:
\[
    \chi_A \in \Mes(X, \A) \iff \quad A \in \A
\]

\begin{proof}
    \[
        \{ \chi_A > \alpha \}  = \left\{ \begin{array}{l}
            X \quad \alpha <0         \\
            A \quad 1 > \alpha \geq 0 \\
            \emptyset \quad \alpha \geq 1
        \end{array} \right.
    \]
    Now, $X,\emptyset\in\A$ by definition, so:
    \[
        A\in\A \iff \chi_A \in \Mes
    \]
\end{proof}

%>=====< Question 11 >=====<%

\question

Prove or disprove the following statements:
\begin{enumerate}[a)]
    \item\label{fmeas:fpmmeas} $f \in \Mes(X, \A) \iff f_{\pm} \in \Mes_+(X, \A)$;
    \item\label{fmeas:fabsmeas} $f \in \Mes(X, \A) \iff |f| \in \Mes(X, \A)$.
\end{enumerate}

\subsection*{Solution}

\subsection{Measurability of \texorpdfstring{$f_{\pm}$}{f positive, f negative} and \texorpdfstring{$|f|$}{absolute value of f}}

Let $f:X\to\R$, we have:
\begin{enumerate}[i)]
    \item $f \in \Mes(X, \A) \iff f_{\pm} \in \Mes_+(X, \A)$
    \item $f \in \Mes(X, \A) \iff |f| \in \Mes(X, \A)$
\end{enumerate}

\begin{proof}
    \hspace*{\fill} %leave a blank line
    \begin{enumerate} [i)]
        \item \begin{itemize}
                  \item \textbf{($\implies$):} if $f \in \Mes(X, \A)$, then we define $f_+$ as:
                        \[
                            f_+(x) = \max\{f(x),0\} \geq 0 \quad \forall x\in X
                        \]
                        and since $f,0 \in \Mes(X,\A)$ and $\max$ is a measurable function we have that $f_+ = \max \circ (f,0) \in \Mes_+(X,\A)$ by (\ref{meas:comp}). We may analogously prove the same for $f_-$.
                  \item \textbf{($\impliedby$):} if $f_+ \in \Mes_+(X, \A)$, then we define $f = f_+ - f_-$, and since $f_+,f_-,f \in \Mes(X,\A)$ we have that $f \in \Mes(X,\A)$ by (\ref{meas:sumprod}).
              \end{itemize}
        \item $f\in\Mes \implies f_{+}, f_{-} \in \Mes$ by the previous point $\implies |f|=f_+ + f_- \in \Mes$ by (\ref{meas:sumprod}).
    \end{enumerate}
\end{proof}

%>=====< Question 12 >=====<%

\question

Write the definition of simple function. What is its canonical form? How can we characterize measurability of a simple function? Write the definition of step function.

\subsection*{Solution}

\subsection{Definition of simple function}
Let $X$ be a set and $s:X\to\R$ a function. We say that $s$ is a simple function if $s(X)$ is a finite set. \\
Furthermore we define the two sets:
\begin{align*}
     & \Smes(X, \A) \coloneqq \{ \text{ measurable simple functions}  \}                           \\
     & \Smes_+(X, \A) \coloneqq \{ \text{ measurable simple functions with non-negative values} \}
\end{align*}

\subsection{Canonical form of simple function} \label{simple:canon}
The canonical form of a simple function is:
\[
    s(x) = \sum_{i=1}^n c_i \chi_{E_i}(x)
\]
where:
\begin{align*}
     & c_i \in \R \; \forall i=1,\dots,n                                      \\
     & E_i = \{ x\in X : \; s(x)=c_i \} \; \forall i=1,\dots,n                \\
     & X = \bigcup_{i=1}^n E_i, \; E_k\cap E_l = \emptyset \; \forall k\neq l
\end{align*}
i.e. $E_i$ is a partition of $X$.

\subsection{Measurability of simple function}
A simple function is measurable if and only if we have the following:
\[
    E_i \in \A \; \forall i=1,\dots,n
\]
i.e. :
\[
    s(x)\in \Mes(X,\A) \iff E_i \in \A \; \forall i=1,\dots,n
\]
this is because $s(x)$ is a linear combination of indicator functions.

\subsection{Step Functions}
Let $I=[a_0,a_1)$ be an interval and $P=\{ a_o \equiv x_0 < x_1 < \dots < x_n \equiv a_1 \}$ a partition of $I$. A function $f:I\to\R$ is a step function if:
\[
    f \coloneqq \sum_{k=0}^{n-1} c_k \chi_{[x_k,x_{k+1})} (x)
\]

%>=====< Question 13 >=====<%

\question

State and give a sketch of the proof of the Simple Approximation Theorem.

\subsection*{Solution}

\subsection{Simple Approximation Theorem}\label{SAT}
Let $(X,\A)$ be a measurable space and $f:X\to\Rcomp$. Then there exists a sequence of simple functions $\seq{s}$ such that:
\[
    s_n \xrightarrow{n\to\infty} f \text{ in } X \text{ (pointwise)}
\]
\textbf{Furthermore:}
\begin{enumerate}[i)]
    \item if $f\in\Mes(X,\A)$, then $\seq{s}\subseteq\Smes(X,\A)$;
    \item if $f\geq 0 \implies \seq{s} \uparrow$, $0\leq s_n \leq f$;
    \item $f$ bounded $\implies s_n \xrightarrow{n\to\infty} f$ uniformly in $X$.
\end{enumerate}

\subsection*{Sketch of proof}
Let $f\geq0$, bounded and $0 \leq f \leq 1$ $\forall x\in X$.
\[
    f : X \to [0,1]
\]
Let us divide $[0,1]$ in $2^n$ intervals of equal length $\forall n \in \N$, then we define:
\begin{align*}
     & E_k^{(n)} \coloneqq \left\{ x\in X: \; \frac{k}{2^n} \leq f(x) \leq \frac{k+1}{2^n} \right\} \quad k = 0,\dots,2^n-1 \\
     & s_n \coloneqq \sum_{k=0}^{2^n-1} \frac{k}{2^n} \chi_{E_k^{(n)}}(x) \quad \forall n\in\N
\end{align*}
Clearly $\seq{s}$ has the desired properties.

%>=====< Question 14 >=====<%

\question

Write the definitions of $\esssup_X f$ and $\essinf_X f$. State their properties and prove some of them.

\subsection*{Solution}

\subsection{Definition of \texorpdfstring{$\esssup_X f$}{essup f}}
Let $(X, \A, \mu)$ be a measure space and $f$ a function on $X$. We define:
\[
    \esssup_X f(x) \coloneqq \inf \left\{ \sup_{x\in N^\complement} f(x) : \; N \in \mathcal{N}_{\mu} \right\}
\]

\subsection{Definition of \texorpdfstring{$\essinf_X f$}{essinf f}}
Let $(X, \A, \mu)$ be a measurable space and $f$ a function on $X$. We define:
\[
    \essinf_X f(x) \coloneqq \sup \left\{ \inf_{x\in N^\complement} f(x) : \; N \in \mathcal{N}_{\mu} \right\}
\]

\subsection{Properties of \texorpdfstring{$\esssup_X f$}{essup f} and \texorpdfstring{$\essinf_X f$}{essinf f}}
Let $(X, \A, \mu)$ be a measure space and $f,g\in\Mes(x,\A)$ two functions on $X$. We have that:
\begin{enumerate}[i)]
    \item $\exists N \in \mathcal{N}_\mu$ such that $\esssup_X f = \sup_{x\in N^\complement} f$ and $f\leq \esssup_X f$ almost surely $x\in X$;
    \item $\esssup_X f = -\essinf_X -f$;
    \item $\esssup_X k\cdot f  = k \cdot \esssup_X f$;
    \item $f\leq g \text{ a.e. in } X \implies \esssup_X f \leq \esssup_X g$;
    \item $\esssup_X (f+g) \leq \esssup_X f + \esssup_X g$;
    \item $f = g$ almost everywhere in $X$ $\implies \esssup_X f = \esssup_X g$;
    \item $g \geq 0$ almost everywhere in $X$ $\implies f\cdot g \leq (\esssup_X f)\cdot g$ almost everywhere in $X$.
\end{enumerate}

\begin{proof}
    Let us give a partial proof:
    \begin{enumerate}[i)]
        \item Suppose $\esssup_X f < +\infty$, $\forall k\in\N$ $\exists N_k \in \mathcal{N}_\mu$ such that:
              \[
                  \sup_{x\in N_k} f < \esssup_X f + \frac{1}{k}
              \]
              We define $N \coloneqq \bigcup_{k=1}^\infty N_k$. Then $N \in \mathcal{N}_\mu$ and:
              \begin{align*}
                   & N^\complement = \bigcap_{k=1}^\infty N^\complement_k \subseteq N^\complement_k \quad \forall k \in \N                              \\
                   & \implies  \esssup_X f \leq \sup_{N^\complement} f \leq \sup_{N^\complement_k} f < \esssup_X f + \frac{1}{k} \quad \forall k \in \N
              \end{align*}
              Now we pass apply a limit $k\to+\infty$ and we get:
              \begin{align*}
                   & \sup_{N^\complement} f = \esssup_X f                                                             \\
                   & N \supseteq \bar{N} \coloneqq \{ x\in X: \; f(x) > \esssup_X f(x) \} \in\A                       \\
                   & \implies \bar{N} \in \mathcal{N}_\mu \implies f \leq \esssup_X f \text{ almost everywhere in } X
              \end{align*}
    \end{enumerate}
\end{proof}

%>=====< Question 15 >=====<%

\question

What is $\Leb^\infty$? Which is the relation between functions finite a.e. and essentially bounded functions? Justify the answer.

\subsection*{Solution}

\subsection{Definition of \texorpdfstring{$\Leb^\infty$}{the set of essentially bounded functions}}
Let $(X, \A, \mu)$ be a measure space. A function $f\in\Mes(X,\A)$ is said to be essentially bounded if:
\[
    \esssup_X f < +\infty
\]
and we define the set of essentially bounded functions as:
\[
    \Leb^\infty (X,\A,\mu) \coloneqq \{ f:X\to\Rcomp : \; \text{ f is essentially bounded }\}
\]

\subsection{Relation between functions finite a.e. and essentially bounded functions}
We have that:
\begin{enumerate}
    \item $f\in\Leb^\infty \implies f$ is finite a.e. in $X$;
    \item in general if $f$ is finite a.e. in $X$ $\centernot\implies f\in\Leb^\infty$.
\end{enumerate}

\begin{proof}
    \begin{enumerate}
        \item We can easily see that:
              \[
                  |f| \leq \esssup |f| < + \infty \text{ almost everywhere in } X
              \]
              thus $f$ is finite almost everywhere in $X$;
        \item Let us assume that:
              \[
                  f \text{ is finite a.e. in }  X \implies f\in\Leb^\infty
              \]
              and let us see a clear counterexample of this, take:
              \[
                  f(x): \R \to \Rcomp \coloneqq \begin{cases}
                      \frac{1}{|x|} & x \neq 0 \\
                      + \infty      & x = 0
                  \end{cases}
              \]
              Clearly $f$ is finite in $E=\R\setminus\{0\}$, i.e. $f$ is finite a.e. in $\R$. Let us note that $\lambda(\{0\}) = 0 $. Thus:
              \[
                  \esssup_X |f| = +\infty \implies f \notin \Leb^\infty
              \]
    \end{enumerate}
\end{proof}

\sheet 




%>=====< Question 8 >=====<%

\question

State and prove the vanishing lemma for functions $f \in \Mes_+(X, \A)$.

\subsection*{Solution}

\subsection{Vanishing lemma for \texorpdfstring{$f \in \Mes_+(X, \A)$}{nonnegative measurable functions}} \label{VanLem}

Let $f \in \Mes_+(X, \A)$ be such that $\int_X f \, d\mu = 0 $, then we have that $f=0$ a.e. in $X$.

\begin{proof}
    Let us note that the thesis is equivalent to $\mu(\{ f > 0\}) = 0$. Let us define:
    \[
        \{ f > 0 \} = \bigcup_{n=1}^\infty \left\{ f > \frac{1}{n} \right\}    
    \]
    Clearly we have that:
    \begin{enumerate}[a)]
        \item $\left\{ f > \frac{1}{n} \right\} \uparrow \{ f > 0 \}$
        \item $\frac{1}{n} \cdot \chi_{\left\{ f > \frac{1}{n} \right\}} \leq f \cdot \chi_{\left\{ f > \frac{1}{n} \right\}}$
    \end{enumerate}
    by Chebychev inequality (\ref{ChebIneq}) we have:
    \[
        \mu \left( \left\{ f > \frac{1}{n} \right\} \right) \leq \frac{1}{1/n} \cdot \int_X f \, d\mu = 0 \quad \forall n\in\N
    \]
    thus by the continuity from below of $\mu$ (\ref{meas:contbel}) we have:
    \[
        \mu(\{ f > 0 \}) = \mu \left( \bigcup_{n=1}^\infty \left\{ f > \frac{1}{n} \right\} \right) = \lim_{n\to\infty} \mu \left( \left\{ f > \frac{1}{n} \right\} \right) = = 0
    \]
\end{proof}

%>=====< Question 9 >=====<%

\question

State and prove the Monotone Convergence Theorem (or Beppo Levi Theorem).

\subsection*{Solution}

\subsection{Monotone Convergence Theorem}\label{MCT}
Let $\seq{f} \subseteq \Mes_+(X,\A)$ and $f:X\to\Rcomppos$ be such that:
\begin{enumerate}[i)]
    \item $f_n \leq f_{n+1}$ in $X$ $\forall n\in\N$
    \item $f_n \xrightarrow{n\to\infty} f$ pointwise in $X$
\end{enumerate}
then:
\[
    \lim_{n\to\infty} \int_X f_n \, d\mu = \int_X \lim_{n\to\infty} f_n \, d\mu =\int_X f \, d\mu    
\]

\begin{proof}
    $f\in\Mes_+(X,\A)$ \\
    by monotonicity of the integral for functions (\ref{LebInt:monofunc}) we have:
    \[
        \alpha \coloneqq \int_X f_n \, d\mu \leq \int_X f_{n+1} \, d\mu \leq \int_X f \, d\mu  \longrightarrow \alpha \leq \int_X f \, d\mu
    \]
    now, we have to prove that $\alpha \geq \int_X f \, d\mu$. Indeed $\forall \epsilon \in (0,1)$, $\forall s \in \Smes_f$ let:
    \[
        E_n \coloneqq \{ (1-\epsilon) s \leq f_n \} \quad n\in\N    
    \]
    Let us note that:
    \begin{enumerate}[a)]
        \item $\seq{E} \subseteq \A$;
        \item $\seq{E} \uparrow$, since $\seq{f} \uparrow$;
        \item $X=\bigcup_{n=1}^\infty E_n$.
    \end{enumerate}
    Clearly $\bigcup_{n=1}^\infty E_n \subseteq X$, we have to show that $X \subseteq \bigcup_{n=1}^\infty E_n$. Now, let us fix $x\in X$, we have two possibilities:
    \begin{itemize}
        \item \textbf{$f(x)=+\infty$:} then $\exists \bar{n}\in\N$ such that $\forall n> \bar{n}$:
            \[
                (1-\epsilon) s(x) < f_n(x) \implies x\in E_n \; \forall n > \bar{n} \implies x \in \bigcup_{n=1}^\infty E_n
            \]
        \item \textbf{$f(x)<+\infty$:} then $\exists \bar{n}\in\N$ such that $\forall n> \bar{n}$:
            \[
                (1-\epsilon) s(x) \leq (1-\epsilon) f(x) < f_n(x) \implies x\in E_n \; \forall n > \bar{n} \implies x \in \bigcup_{n=1}^\infty E_n
            \]
    \end{itemize}
    Thus we have that $X \subseteq \bigcup_{n=1}^\infty E_n$ and $\bigcup_{n=1}^\infty E_n \subseteq X$ $\implies X = \bigcup_{n=1}^\infty E_n$ . \\
    It clearly follows that:
    \[
        (1-\epsilon)\cdot \int_{E_n} s \, d\mu \leq \int_{E_n} f_n \, d\mu \leq \int_X f \, d\mu    
    \]
    now let $n\to\infty$ ($E_n \xrightarrow{n\to\infty} X$):
    \[
        (1-\epsilon)\cdot \int_{X} s \, d\mu \leq  \lim_{n\to\infty} \int_{X} f_n \, d\mu = \alpha 
    \]
    but since $\epsilon \in (0,1)$ can be arbitrarily small we have:
    \[
        \int_X s \, d\mu \leq \alpha \implies \sup_{s\in\Smes_f} \int_X s \, d\mu  = \int_X f \, d\mu \leq \alpha    
    \]
    thus we have proved that $\int_X f \, d\mu = \alpha $.
\end{proof}

%>=====< Question 10 >=====<%

\question

State and prove Fatou's Lemma.

\subsection*{Solution}

\subsection{Fatou's lemma}\label{Fatlem}

Let $\seq{f} \subseteq \Mes_+(X,\A)$, then:
\[
    \liminf_{n\to\infty} \int_X f_n \, d\mu \geq \int_X \left( \liminf_{n\to\infty} f_n \right) \, d\mu
\]

\begin{proof}
    We already know that $\liminf_{n\to\infty} f_n \in \Mes_+(X,\A)$ by (\ref{meas:extremes}).\\
    Let us a define a new sequence $\seq{g}$ such that:
    \[
        g_k: X \to \Rcomppos \quad g_k \coloneqq \inf_{n\geq k} f_n    
    \]
    We can clearly see that:
    \begin{enumerate}[a)]
        \item $\seq{g} \subseteq \Mes_+(X,\A)$, $\seq{g} \uparrow$;
        \item $g_k \leq f_k$ for all $k\in\N$;
        \item $\lim_{k\to\infty} g_k = \sup_{k\geq 1} g_k = \sup_{k\geq 1} \inf_{n\geq k} f_n = \liminf_{n\to\infty} f_n$.
    \end{enumerate}
    thus by monotonicty of the integral for functions (\ref{LebInt:monofunc}) and (b) we have:
    \[
        \int_X g_k \, d\mu \leq \int_X  f_k \, d\mu \quad \forall k\in\N   
    \]
    Now, since $\seq{g}$ is an increasing sequence so is $\int_X g_k \, d\mu$ and thus it admits a limit (which coincides with its $\liminf$), thus, if we apply the $\liminf$ to both sides, we have:
    \[
        \liminf_{k\to\infty} \int_X g_k \, d\mu = \lim_{k\to\infty} \int_X  g_k \, d\mu \leq \liminf_{k\to\infty} \int_X f_k \, d\mu 
    \]
    Now let us apply the Monotone Convergence Theorem (\ref{MCT}) to the right hand side:
    \[
        \int_X \lim_{k\to\infty} g_k \, d\mu \overset{(c)}{=} \int_X \left( \liminf_{n\to\infty} f_n \right) \, d\mu \leq \liminf_{k\to\infty} \int_X f_k \, d\mu
    \]
    and so we have obtained our thesis.
\end{proof}

\sheet 

%>=====< Question 1 >=====<%

\question
State and prove the theorem concerning integration of series with general terms $f_n \in \Mes_+(X, \A)$.

\subsection*{Solution}

\subsection{Integral of a series with positive terms} \label{IntoS:pos}
Let $\seq{f}\subseteq \Mes_+(X,\A)$, $f_n :X\to\Rcomppos$ $\forall n \in \N$ then:
\[
    \int_X \left( \sum_{n=1}^\infty f_n \right) \, d\mu = \sum_{n=1}^\infty  \left( \int_X f_n \, d\mu \right)   
\]

\begin{proof} Let us provide a proof of our own making. \footnote{
        This proof has been reviewed by professor Punzo and stated to be correct.
    }\\
    Cleary $\sum_{n=1}^\infty f_n \in \Mes_+(X,\A)$, since each addendum is a non-negative measurable function. Let us now note that:
    \[
        \sum_{k=1}^n f_k \underset{n\to\infty}{\uparrow} \sum_{k=1}^\infty f_k
    \]
    indeed:
    \begin{align*}
        & \sum_{k=1}^n f_k \xrightarrow{n\to\infty} \sum_{k=1}^\infty f_k \text{ pointwise in } X\\
        & \sum_{k=1}^n f_k \leq \sum_{k=1}^n f_k + f_{n+1} = \sum_{k=1}^{n+1}f_k \text{ in } X \; \forall n\in\N
    \end{align*}
    so we may apply the Monotone Convergence Theorem (\ref{MCT}) to conclude:
    \[
        \int_X \left( \sum_{n=1}^\infty f_n \right) \, d\mu =  \int_X \left( \lim_{n\to\infty} \sum_{k=1}^n f_k \right) \, d\mu \overset{MCT}{=} \lim_{n\to\infty}\int_X \left(  \sum_{k=1}^n f_k \right) \, d\mu
    \]
    Now, for our last step, let us apply the linearity of the integral:
    \[
        \lim_{n\to\infty}\int_X \left(  \sum_{k=1}^n f_k \right) \, d\mu = \lim_{n\to\infty} \sum_{k=1}^n \left( \int_X  f_k  \right) = \sum_{k=1}^\infty  \left( \int_X f_k \, d\mu \right)
    \]
\end{proof}

%>=====< Question 2 >=====<%

\question
Let $f \in\Mes_+(X, \A)$. Show that $\nu(E) \coloneqq \int_E f \, d\mu$ is a measure; state and prove its properties.

\subsection*{Solution}

\subsection{Measure induced by a function}
Let $f \in\Mes_+(X, \A)$, then $\nu(E) \coloneqq \int_E f \, d\mu$, $\nu: \A \to \Rcomppos$ is a measure.

\begin{proof} Let us show that $\nu$ meets the definition of a measure:
    \begin{enumerate}[i)]
        \item $\nu(\emptyset) = 0$ thanks to the properties of the integral (see \ref{LebInt:null}), since $\mu(\emptyset)=0$.
        \item Let $\seq{E}\subseteq \A$ disjoint such that $\bigcup_{k=1}^\infty E_k = X$, then:
            \begin{align*}
                \nu(E) & \tikzmarknode{eq1}{=} \int_X f \cdot \chi_E \, d\mu = \int_X \left( f \cdot \sum_{k=1}^\infty \chi_{E_k} \right) \, d\mu \text{ thanks to the disjointedness of } \seq{E} \\
                & \tikzmarknode{eq2}{=} \sum_{k=1}^\infty \left( \int_X f\cdot \chi_{E_k} \, d\mu \right) = \sum_{k=1}^\infty \nu(E_k)
            \end{align*} \tikz[overlay,remember picture]{\draw[shorten >=1pt,shorten <=1pt] (eq1) -- (eq2);}
            The penultimate passage was achieved thanks to (\ref{IntoS:pos})
    \end{enumerate}
\end{proof}

\subsection{Properties of the induced measure}
\begin{enumerate}[i)]
    \item Let $g\in\Mes_+(X,\A)$, then: 
        \[
            \int_X g \, d\nu = \int_X g\cdot f \, d\mu    
        \]
    \item $\forall E \in \A$  $\mu(E)=0\implies \nu(E)=0$
    \item $\forall f\in \Mes_+$  $\nu(E) = 0 \implies \mu(E)=0$
\end{enumerate}

\begin{proof}
    \hspace*{\fill} %leave a blank line
    \begin{enumerate}[i)]
        \item Let us show this equality with $g\equiv s \in \Smes_+(X,\A)$, with canonical form:
            \[
                s = \sum_{k=1}^n c_k\cdot \chi_{F_k} \;, \quad \{F_k\}\subseteq\A \;, \quad X = \bigcup_{k=1}^n F_k   
            \]
        then:
            \begin{align*}
                \int_X s \, d\nu & \tikzmarknode{eq1}{=} \sum_{k=1}^n c_k \cdot \nu(F_k) = \sum_{k=1}^n c_k \cdot \left( \int_X f \, d\mu \right) \\
                & \tikzmarknode{eq2}{=} \int_X \left( \sum_{k=1}^n c_k \cdot f \cdot \chi_{F_k} \right)
            \end{align*} \tikz[overlay,remember picture]{\draw[shorten >=1pt,shorten <=1pt] (eq1) -- (eq2);}
            If $g\in\Mes_+(X,\A)$ then we can get the thesis by approximation (\ref{SAT}).
        \item $\nu(E)=0$ thanks to the properties of the integral (\ref{LebInt:null}) since $\mu(E)=0$.
        \item Let us take the function $\chi_E\in\Mes_+(X,\A)$ (see \ref{AinA:chi}). Then, since the hypothesis is true $\forall f \in \Mes_+$, we may write:
            \[
                \nu(E) = \int_E \chi_E \, d\mu = 1 \cdot \mu(E) = 0 \implies \mu(E)=0
            \]
    \end{enumerate}
\end{proof}

%>=====< Question 3 >=====<%

\question
Let $f, g \in \Mes_+(X, \A)$. Show that if $f = g$ a.e. in $X$ then $\int_X f \, d\mu = \int_X g\, d\mu$.

\subsection*{Solution}

\subsection{\texorpdfstring{$f = g$ a.e. $\implies \int_X f \, d\mu = \int_X g\, d\mu$}{if f and g are equal almost everywhere then their integrals are equal}} \label{fgpos:SfSg}
Let $f,g\in\Mes_+(X,\A)$ such that $f=g$ a.e. in $X$, then: 
\[
    \int_X f \, d\mu = \int_X g\, d\mu
\]

\begin{proof}
    Let us define the following set:
    \[
        N \coloneqq \left\{ x \in X : \; f(x) \neq g(x) \right\} \in\A  
    \]
    Clearly we have that:
    \begin{align*}
        & \mu(N) =  0 \\
        & \int_{N^\complement} f \, d\mu = \int_{N^\complement} g \, d\mu
    \end{align*}
    Both results are a consequence of the definition of almost everywhere. Thus we may write:
    \begin{align*}
        \int_X f \, d\mu & \tikzmarknode{eq1}{=} \cancelnum{0}{\int_N f \, d\mu} + \int_{N^\complement} f \, d\mu = \int_{N^\complement} g \, d\mu\\
        & \tikzmarknode{eq2}{=} \underbrace{\int_N g \, d\mu}_{=0} + \int_{N^\complement} g \, d\mu = \int_X g \, d\mu\\
    \end{align*} \tikz[overlay,remember picture]{\draw[shorten >=1pt,shorten <=1pt] (eq1) -- (eq2);}
    let us note that we have partitioned $X$ with $N$ and $N^\complement$.
\end{proof}

%>=====< Question 4 >=====<%

\question
Write the definition of: integrable functions; Lebesgue integral; $\Leb^1 (X, \A, \mu)$.

\subsection*{Solution}

\subsection{Integrable function}
Let $f:X\to\Rcomppos$ be a function, we say that $f$ is integrable on X if $f\in\Mes(X,A)$ and:
\[
    \int_X f_+ \, d\mu < \infty \quad \int_X f_- \, d\mu < \infty
\]
let us note that both $f_+$ and $f_-$ are non-negative measurable functions ($f_\pm\in\Mes_+(X,\A)$).

\subsection{\texorpdfstring{$\Leb^1(X,\A,\mu)$}{L1}}
We define $\Leb^1(X,\A,\mu)$ as the set of all integrable functions $f:X\to\Rcomp$:
\[
    \Leb^1(X,\A,\mu) \coloneqq \left\{ f:X\to\Rcomp: \; f\in\Mes(X,\A), \; \int_X f_\pm \, dx < +\infty  \right\}
\]

\subsection{Lebesgue integral}
Let $f\in\Leb^1$, then we define its integral as:
\[
    \int_X f \, d\mu \coloneqq \int_X f_+ \, d\mu - \int_X f_- \, d\mu
\]
this is called the Lebesgue integral of $f$. Moreover we define:
\[
    \int_E f \, d\mu \coloneqq \int_X f\cdot \chi_E \, d\mu = \int_E f_+\cdot \chi_E \, d\mu - \int_E f_-\cdot \chi_E \, d\mu
\]

%>=====< Question 5 >=====<%

\question
Let $f : X \to \Rcomp$. How is the integrability of $f$  related to that of $f_\pm$ and of $|f|$? Justify the answer. Show that if $f \in \Leb^1$, then $|\int_X f \, d\mu | \leq \int_X |f| \, d\mu$. Give an alternative definition of $\Leb^1 (X, \A, \mu)$.

\subsection*{Solution}

\subsection{Properties of the Lebesgue integral}
Let $f:X\to\Rcomp$ be a function, then: 
\begin{enumerate}[i)]
    \item $f\in\Leb^1 \iff f_\pm \in \Leb^1$
    \item $f\in\Leb^1 \iff f \in \Mes$ and $|f|\in\Leb^1$
    \item $f\in\Leb^1 \implies |\int_X f \, d\mu| \leq \int_X |f| \, d\mu$
\end{enumerate}

\begin{proof}
    \hspace*{\fill} %leave a blank line
    \begin{enumerate}[i)]
        \item if $f\in\Leb^1$ then by definition ($\iff$) we have that:
            \[
                f\in\Mes \quad \text{and} \quad \int f_{\pm} \, d\mu < \infty    
            \]
            Now, $f\in\Mes \iff f_{\pm}\in\Mes_+$ by (\ref{fmeas:fpmmeas}) thus we have that:
            \[
                f_{\pm}\in \Mes, \;\int f_{\pm} \, d\mu < \infty  \iff f_{\pm}\in\Leb^1
            \]
            and the two conditions are equivalent.
        \item \footnote{
            We may also use the previous point and the fact that $\Leb^1$ is a vector space, but we'll see this later.
        } as above, if $f\in\Leb^1$ then by definition ($\iff$) we have that:
            \[
                f\in\Mes \quad \text{and} \quad \int f_{\pm} \, d\mu < \infty    
            \]
            Now, $f\in\Mes \iff |f|\in\Mes_+$ by (\ref{fmeas:fabsmeas}) thus we have that:
            \[
                |f|\in \Mes, \;\int |f| \, d\mu < \infty  \iff |f|\in\Leb^1
            \]
            this is true by virtue of the previous point and the fact that $|f| = f_+ + f_-$, indeed:
            \[
                \int |f| \, d\mu = \int f_+ \, d\mu + \int f_- \, d\mu < \infty
            \]
            since each addendum is finite. Thus the two conditions are equivalent.
        \item Let $f\in\Leb^1$, then thanks to the triangular inequality:
            \begin{align*}
                \left| \int_X f \, d\mu \right| & \tikzmarknode{eq1}{=} \left| \int_X f_+ \, d\mu - \int_X f_- \, d\mu \right| \\
                & \tikzmarknode{eq2}{\leq} \left| \int_X f_+ \, d\mu \right| + \left| \int_X f_- \, d\mu \right| \\
                & \tikzmarknode{eq3}{=} \int_X f_+ \, d\mu + \int_X f_- \, d\mu\\
                & \tikzmarknode{eq4}{=} \int_X |f| \, d\mu
            \end{align*} \tikz[overlay,remember picture]{\draw[shorten >=1pt,shorten <=1pt] (eq1) -- (eq2) -- (eq3) -- (eq4);}
    \end{enumerate}    
\end{proof}

\subsection{ALternative definition of \texorpdfstring{$\Leb^1$}{L1}} \footnote{This definition has not been directly provided by prof. Punzo. I found no reference to this definition neither in my personal notes nor his, rather I found it in a student's notes from last year's course. Nevertheless, it is clearly attested in the literature and it is equivalent and alternative to the previous definition.}
We can also more compactely define $\Leb^1(X,\A,\mu)$ through the absolute value:
\[
    \Leb^1(X,\A,\mu) \coloneqq \left\{ f:X\to\Rcomp: \; f\in\Mes(X,\A), \; \int_X |f| \, dx < +\infty  \right\}
\]


%>=====< Question 6 >=====<%

\question
Prove that $\Leb^1$ is a vector space.

\subsection*{Solution}

\subsection{ \texorpdfstring{$\Leb^1$}{L1} is a vector space}
$\Leb^1(X,\A,\mu)$ is a vector space.

\begin{proof}
    Let $f,g\in\Leb^1$ and $\lambda\in\R$, then:
    \begin{align*}
        & \implies f_{\pm}, g_{\pm} \text{ finite a.e. in } X \text{ by (\ref{intfin:ffin})} \\
        & \implies f,g \text{ finite a.e. in } X
    \end{align*}
    so we can define:
    \[
        h \coloneqq f+ \lambda g \text{ defined a.e. in } X
    \]
    clearly $h\in\Mes$ by the properties of measurable functions (\ref{meas:sumprod}) and:
    \[
        \int_X |h| \, d\mu = \int_X |f| + |\lambda| \int_X |g| \, d\mu < \infty 
    \]
    since both addenda are finite. Thus $h\in\Leb^1$ and 
    $\Leb^1$ is a vector space.
\end{proof}

%>=====< Question 7>=====<%

\question
State and prove the vanishing lemma for $\Leb^1$ functions.

\subsection*{Solution}

\subsection{Vanishing lemma for \texorpdfstring{$f\in\Leb^1$}{integrable functions}}
Let $f\in\Leb^1(X,\A,\mu)$ be such that:
\[
    \int_E f \, d\mu = 0 \quad \forall E \in \A
\]
then $f=0$ a.e. in $X$.

\begin{proof}
    Let us define two sets:
    \begin{align*}
        E_+ & \coloneqq \{x\in X : f(x) \geq 0\} \\
        E_- & \coloneqq \{x\in X : f(x) \leq 0\}
    \end{align*}
    they are both in $\A$ since $f\in\Mes$ (see \ref{statomeas:3}), so we have that:
    \begin{align*}
        & \int_{E_+} f \, d\mu = 0 \implies f=0 \text{ a.e. in } E_+\\
        & \int_{E_-} f \, d\mu = 0 \implies f=0 \text{ a.e. in } E_-
    \end{align*}
    so we have that $f=0$ a.e. in $X=E_+\cup E_-$.
\end{proof}

%>=====< Question 8>=====<%

\question
Let $f \in \Leb^1$, $g \in \Mes$, $f = g$ a.e. in $X$. Show that $g \in \Leb^1$ and $\int_X g\, d\mu = \int_X f \, d\mu$.

\subsection*{Solution}

\subsection{f = g a.e. \texorpdfstring{$\implies g \in \Leb^1$}{then g is integrable} and \texorpdfstring{$\int_X g\, d\mu = \int_X f \, d\mu$}{its integral is the same as f}}
Let $f\in\Leb^1$, $g\in\Mes$ and $f=g$ a.e. in $X$. Then:
\[
    g \in \Leb^1 \text{ and } \int_X f \, d\mu = \int_X g \, d\mu
\]

\begin{proof}
    We have that:
    \begin{align*}
        & f_+=g_+ \text{ a.e. in } X  \text{ and } f_-=g_- \text{ a.e. in } X \\
        & \implies \int_X f_+ \, d\mu = \int_X g_+ \, d\mu \text{ and } \int_X f_- \, d\mu = \int_X g_- \, d\mu
    \end{align*}
    thanks to (\ref{fgpos:SfSg}), thus we get:
    \[
        \int_X f \, d\mu = \int_X f_+ \, d\mu + \int_X f_- \, d\mu = \int_X g_+ \, d\mu + \int_X g_- \, d\mu = \int_X g \, d\mu
    \]
\end{proof}

%>=====< Question 9 >=====<%

\question
State and prove the Lebesgue theorem. In which case it is simple to find a dominating function?


\subsection*{Solution}

\subsection{Lebesgue theorem (or Dominated convergence theorem)}\label{DCT}
Let $\seq{f}\subseteq \Mes(X,\A)$ be a sequence of measurable functions and $f\in\Mes(X,\A)$ be a function such that:
\[
    f_n \xrightarrow{n\to\infty} f \text{ a.e. in } X
\]
if $\exists g \in \Leb^1$ such that:
\[
    |f_n| \leq g \quad \text{ a.e. in } X \; \forall n \in \N
\]
then:
\[
    f_n,f \in \Leb^1 \text{ and } \int_X |f_n-f| \, d\mu \xrightarrow{n\to\infty} 0    
\]
in particular:
\[
    \int_X f_n \, d\mu \xrightarrow{n\to\infty} \int_X f \, d\mu
\]

\begin{proof}
    We shall prove this theorem by applying  Fatou's lemma (\ref{Fatlem}). \\
    Now since $|f_n| \leq g$ we can pass the limit and get:
    \[
        |f| \leq g \text{ a.e. in } X
    \]
    So we have:
    \begin{align*}
        & \int_X |f_n| \, d\mu \leq \int_X g \, d\mu \quad \forall n \in \N \\
        & \int_X |f| \, d\mu \leq \int_X g \, d\mu
    \end{align*}
    so since $f_n,f\in\Mes$ and $g\in\Leb^1$ we can deduce that:
    \[
        \implies f_n,f\in\Leb^1    
    \]
    So they are also finite a.e., let us now define  a new sequence $\seq{g}$:
    \[
        g_n \coloneqq 2g - |f_n -f | \quad \forall n \in \N  
    \]
    by the previous inequalities we get:
    \[
        |f_n -f | \leq |f_n| + |f|\leq 2g \text{ a.e. in } X \; \forall n \in \N    
    \]
    therefore:
    \[
        g_n \geq 0 \text{ a.e. in } X \; \forall n \in \N \implies g_n \in Mes_+    
    \]
    \newpage
    thus we can write:
    \begin{align*}
        2 \int_X g \, d\mu & \tikzmarknode{eq1}{=} \int_X \left( \lim_{n\to\infty} g_n \right) \, d\mu \\
        & \tikzmarknode{eq2}{\leq} \liminf_{n\to\infty} \int_X g_n \, d\mu  \text{ by Fatou's lemma \ref{Fatlem}}\\
        & \tikzmarknode{eq3}{=} \liminf_{n\to\infty} \int_X \left[ 2g - |f_n -f | \; \right] \, d\mu \\
        & \tikzmarknode{eq4}{=} 2 \int_X g \, d\mu - \limsup_{n\to\infty}  \left( \int_X |f_n-f| \, d\mu \right) \\
    \end{align*} \tikz[overlay,remember picture]{\draw[shorten >=1pt,shorten <=1pt] (eq1) -- (eq2) -- (eq3) -- (eq4);}
    thus we may simplify $2\int_X g \, d\mu$ on both sides and invert the inequality sign we get:
    \[
        \limsup_{n\to\infty}  \left( \int_X |f_n-f| \, d\mu \right) \leq 0
    \]
    so since $\int_X |f_n-f| \, d\mu \geq 0$ it admits a limit and we have:
    \[
        \lim_{n\to\infty} \int_X |f_n-f| \, d\mu = 0
    \]
    Moreover:
    \[
        \left| \int_X f_n \, d\mu - \int_X f \, d\mu \right| \leq \lim_{n\to\infty} \int_X |f_n-f| \, d\mu  \xrightarrow{n\to\infty} 0
    \]
\end{proof}

\subsection{Simple case for the Lebesgue Theorem}
If we have the following situation:
\begin{enumerate}
    \item $\mu(X) < + \infty$ 
    \item $\exists M > 0$:  $| f_n | \leq M$ a.e. in $X$ $\forall n \in \N$
\end{enumerate}
Then we can choose $g\coloneqq M$ and we get:
\[
    \int_X |g| \, d\mu = \int_X M \, d\mu = M \cdot \mu(X) < +\infty \iff g\in\Leb^1    
\]
thus we have easily met the thesis of the Lebesgue Theorem (\ref{DCT}).

%>=====< Question 10 >=====<%

\question
Describe the relations between Peano-Jordan and Lebesgue measures, and between the Riemann (also in the generalized sense) and the Lebesgue integral.

\subsection*{Solution}

\subsection{Every Peano-Jordan-measurable set is Lebesgue measurable}
Let $E\subseteq\R^n$, if $E$ is Peano-Jordan-measurable then it is also Lebesgue-measurable ($E\in\Leb(\R^n)$) and the its measures coincide:
\[
    m_{PJ} (E) = \lambda(E)    
\]
Thus the set of Peano-Jordan-measurable sets is strictly included in the set of Lebesgue-measurable sets. Indeed the set $[0,1]\cap\Q$ is Lebesgue-measurable (with measure zero) but is not Peano-Jordan-measurable (this is due to the fact that Peano-Jordan-measurable sets do not form a $\salg$).

\subsection{The Riemann integral and the Lebesgue integral}

\subsubsection*{Proper integrals}
Let $I=[a,b]$ be a closed interval and $R(I)$ the set of Riemann-integrable functions over $I$. For any function $f\in R(I)$ we have $f\in\Leb^1(I,\Leb(I),\lambda)$ and:
\[
    \int_I f \, d\lambda = \int_a^b f(x) \, dx    
\]
To state it plainly, we can say that the set of R-integrable functions and the set of $\Leb$-integrable functions coincide on closed intervals and the integrals of such functions also coincide.

\subsubsection*{Improper integrals}
Let $I=(\alpha,\beta)$ and let $R^i(I)$ be the set of functions $f:I\to\R$ integrable in the generalized (improper) sense. Then we have:
\begin{enumerate}[i)]
    \item $f\in R^i(I) \implies f\in\Mes(I,\Leb(I))$
    \item $|f|\in R^i(I) \implies f\in \Leb^1(I,\Leb(I),\lambda)$ and moreover:
        \[
            \int_I f \, d\lambda = \int_\alpha^\beta f(x) \, dx    
        \]
\end{enumerate}
Let us note here the crucial fact that this statement does not imply (as is instead the case for sets) that all $R^i$-integrable functions are also $\Leb$-integrable. This is due to the requirement that the absolute value of $f$ be in $\Leb^1$.
\subsubsection{Counter-Example}
take the function:
\[
    f(x) \coloneqq \begin{cases}
        \frac{sin(x)}{x}  & x\neq 0\\
        1   & x=0
    \end{cases} \quad f: I=[0,+\infty)\to \R   
\]
and on one hand we have that:
\[
    f\in R^i(I) \quad \int_0^\infty \frac{sin(x)}{x} \, dx = \frac{\pi}{2}     
\]
while on the other we have:
\[
    \int_{\R_+} \left| \frac{sin(x)}{x} \right| \, d\lambda = \int_0^\infty \left| \frac{sin(x)}{x} \right| \, dx = +\infty \implies f\not\in \Leb^1
\]
Therefore we may conclude that not all $R^i$-integrable functions are also L-Integrable.

%>=====< Question 11 >=====<%

\question
State the theorem for integration of series (without sign restriction on the general term $f_n$).

\subsection*{Solution}

\subsection{Integration of series with general terms}
Let $\seq{f}\subseteq \Leb^1$ be such that:
\[
    \sum_{n=1}^\infty \left( \int_X | f_n | \right) \, d\mu < +\infty    
\]
then the series $\sum_{n=1}^\infty f_n$ converges a.e. in $X$ and we have that:
\[
    \int_X \left( \sum_{n=1}^\infty f_n \right) \, d\mu = \sum_{n=1}^\infty \left( \int_X f_n \, d\mu \right)   
\] 

%>=====< Question 12 >=====<%

\question
Write the definitions of $L^1$ and of $L^\infty$. Show that they are metric spaces. Are $\Leb^1$ and $\Leb^\infty$ metric spaces?

\subsection*{Solution}

\subsection{Definition of \texorpdfstring{$L^1$}{L1}}
Let $(X,\A,\mu)$ be ameasure space and let $R$ be the equivalence relation (see \ref{equivrel}) such that:
\[
    fRg \iff f=g \text{ a.e. in } X
\]
then we define $L^1$ as the quotient set of $\Leb^1$ with respect to this relation:
\[
    L^1 (X,\A,\mu) \coloneqq \Leb^1(X,\A,\mu) / R    
\]
and we denote the classes of equivalence inside of it as:
\[
    [f] \coloneqq \{ g\in\Leb^1 : \; fRg\}    
\]

\subsection{Definition of \texorpdfstring{$L^\infty$}{Linf}}
As done above we define $L^\infty$ as:
\[
    L^\infty(X,\A,\mu) \coloneqq \Leb^\infty(X,\A,\mu)/R    
\]

\subsection{\texorpdfstring{$L^1$}{L1} and \texorpdfstring{$L^\infty$}{Linf} are metric spaces}
Both $L^1$ and $L^\infty$ are metric spaces with the following distance functions:
\[
    d_1(f,g) \coloneqq \int_X |f-g| \, d\mu \quad \quad d_\infty(f,g) \coloneqq \esssup_X |f-g| 
\]

\begin{proof}
    Let us prove this for $L^1$ only, as the proof for $L^\infty$ is analogous. \\
    Now, let us show that $d_1$ meets the definition of a distance, $d_1: L^1\times L^1 \to \R$. Indeed, since $f,g\in L^1$ ($\int_X f \, d\mu, \int_X g \, d\mu < + \infty$), we have:
    \[
        \int_X |f-g| \, d\mu \leq \int_X |f| \, d\mu + \int_X |g| \, d\mu < + \infty
    \]
    moreover:
    \begin{enumerate}[i)]
        \item $d(f,g)\geq 0 $ $\forall f,g \in L^1$;
        \item $d(f,f)=0$ $\forall f \in L^1$;
        \item $d(f,g)=0 \iff \int_X |f-g| \, d\mu =0$ by the vanishing lemma (\ref*{VanLem}) we get $|f-g|=0$ a.e. in $X$, thus $f=g$ a.e. in $X$;
        \item $d(f,g)=d(g,f)$ $\forall f,g \in L^1$;
        \item $d(f,g)\leq d(f,h)+d(h,g)$ $\forall f,g,h \in L^1$ by the triangular inequality and monotonicity of the integral.
    \end{enumerate}
    Let us note that the equality almost everywhere is an exact match under the equivalence relation $R$. In other words $f,g \in [f]$ and they are the same element with respect to $L^1$ and we can say that $d_1(f,g)=0\implies f \overset{L^1}{=}g$. Therefore we can say that $L^1$ is a metric space equipped with the distance $d_1$.\\
    Let us also note that this isn't true for $\Leb^1$ since it isn't quotiented by the equivalence relation $R$, thus it isn't a metric space. The same argument can be applied to $L^\infty$ and $\Leb^\infty$.
\end{proof}

%>=====< Question 13 >=====<%

\question
For a sequence of functions $\seq{f} \subset \Mes$, write the definitions of: pointwise convergence; uniform convergence; almost everywhere convergence; convergence in $L^1$; convergence in $L^\infty$; convergence in measure.

\subsection*{Solution}

Let $\seq{f}\subseteq \Mes(X,\A)$, $f_n:X\to\R$, $f:X\to\Rcomp$. We can define the following:

\subsection{Pointwise convergence}
We say that $f_n \xrightarrow{n\to\infty} f$ pointwise if:
\[
    f_n(x) \xrightarrow{n\to\infty} f(x) \quad \forall x \in X
\]
in the sense of a sequence of real numbers ($f_n(x)\in\R$).

\subsection{Uniform convergence}
We say that $f_n \xrightarrow{n\to\infty} f$ uniformly if:
\[
    \sup_{x\in X} |f_n(x)-f(x)| \xrightarrow{n\to\infty} 0
\]

\subsection{Almost everywhere convergence}
We say that $f_n \xrightarrow{n\to\infty} f$ almost everywhere if:
\[
    \{ x\in X : \; f_n(x) \xrightarrow{n\\to\infty} f(x) \}^\complement \in \mathcal{N}_\mu
\]
that is to say the set where $f_n$ doesn't converge to $f$ is measurable and has measure zero.

\subsection{Convergence in \texorpdfstring{$L^1$}{L1}}
Let $\seq{f}\subseteq L^1$ and assume (for now) $f\in L^1$. We say that $f_n \xrightarrow{n\to\infty} f$ in $L^1$ if:
\[
    d_1(f_n,f) = \int_X |f_n-f| \, d\mu \xrightarrow{n\to\infty} 0    
\]

\subsection{Convergence in \texorpdfstring{$L^\infty$}{Linf}}
Let $\seq{f}\subseteq L^\infty$ and assume (for now) $f\in L^\infty$. We say that $f_n \xrightarrow{n\to\infty} f$ in $L^\infty$ if:
\[
    d_\infty(f_n,f) = \esssup_X |f_n-f| \xrightarrow{n\to\infty} 0
\]

\subsection{Convergence in measure}\label{conv:meas}
We say that $f_n \xrightarrow{n\to\infty} f$ in measure if:
\[
    \mu (\{ |f_n-f| \geq \epsilon \}) \xrightarrow{n\to\infty} 0 \quad \forall \epsilon > 0   
\]

\sheet

%>=====< Question 1 >=====<%

\question
Is it true that if $f_n \to f$ in measure, then $f_n \to f$ a.e.? Justify the answer.

\subsection*{Solution}

\subsection{Convergence in measure does not imply convergence a.e.}
In general, convergence in measure does not imply convergence a.e.. This can be clearly shown by way of Rademacher's sequence (a.k.a. the typewriter sequence):

\subsection{Rademacher sequence}
Let us define the Rademacher sequence iteratively:
\begin{align*}
    & f_1(x) = \mathbb{I}_{[0,1]}(x) \\
    & f_2(x) = \mathbb{I}_{[0,1/2]}(x) \\
    & f_3(x) = \mathbb{I}_{[1/2,1]} (x)\\
    & \quad \vdots \\
    & f_n(x) = \mathbb{I}_{\left[\frac{n-2^k}{2^k}, \frac{n-2^{k}+1}{2^k}\right]}(x) \quad 2^k \leq n \leq 2^{k+1} \; k \in \N
\end{align*}
In other words for each $k\in\N$ we divide $[0,1]$ into $2^k$ intervals and "hover" over them.
This way we have a function whose $L^1$-limit (and thus by extension its limit in measure) is $0$, indeed we have:
\begin{align*}
    & \int_{[0,1]} f_1 \, d\mu = 1 \\
    & \int_{[0,1]} f_2 \, d\mu = \int_{[0,1]} f_3 \, d\mu = \frac{1}{2} \\
    & \quad \vdots \\
    & \int_{[0,1]} f_n \, d\mu \xrightarrow{n\to\infty} 0
\end{align*}
but, on the other hand, if we fix $x\in [0,1]$ the sequence $\seq{f}$ will oscillate between the value $0$ and $1$ infinitely many times as $n\to\infty$. Thus $f_n$ cannot be said to converge a.e. in $[0,1]$. 

%>=====< Question 2 >=====<%

\question
What is the relation between convergence in measure and convergence a.e. up to subsequences?

\subsection*{Solution}

\subsection{Convergence in measure implies convergence a.e. up to subsequences} \label{measure->aesubs}
Let $f_n,f \in \Mes(X,\A)$ be finite a.e. in $X$. If $f_n \to f$ in measure, then there exists a subsequence $\subseq{f}$ such that:
\[
    f_{n_k} \xrightarrow{k\to\infty} f \text{ a.e. in } X.
\]

%>=====< Question 3 >=====<%

\question
Under which hypothesis on $X$, does convergence a.e. imply convergence in measure? What happens if one omits the key assumption on $X$?

\subsection*{Solution}

\subsection{Convergence a.e. implies convergence in measure when \texorpdfstring{$\mu(X)<+\infty$}{ the measure of X is finite}}
Let $\mu(X)<+\infty$ and $f_n,f \in \Mes(X,\A)$ be finite a.e. in $X$. If $f_n \to f$ a.e. in $X$, then:
\[
    f_n \xrightarrow{n\to\infty} f \text{ in measure}    
\]
The assumption that $\mu(X)<+\infty$ is necessary, as the following counterexample shows:

\subsubsection{Counterexample}
Let us take $f_n \coloneqq \chi_{[n,+\infty)}$, clearly $f_n \to 0$ pointwise (and thus a.e.) in $\R$ but we have that $\lambda(\R)=+\infty$ and thus $f_n \centernot\to 0$ in measure. Indeed we have:
\[
    \mu \left( \left\{ f_n \geq \frac{1}{2} \right\} \right) = +\infty \quad \forall n \in \N 
\]

%>=====< Question 4 ====%

\question
Show that convergence in $L^1$ implies convergence in measure.

\subsection*{Solution}

\subsection{Convergence in \texorpdfstring{$L^1$}{L1} implies convergence in measure} \label{L1->measure}
Let $f_n,f \in L^1(X,\A,\mu)$. If $f_n \toin{L^1} f$, then:
\[
    f_n \to f \text{ in measure}    
\]

\begin{proof}
    Suppose by contradiction that:
    \[
        f_n \centernot \to f \text{ in measure}    
    \]
    then, by definition of convergence in measure (\ref{conv:meas}), $\exists \epsilon, \sigma > 0$ such that:
    \[
         \mu \left( \left\{ |f_n-f| \geq \epsilon \right\} \right) \geq \sigma
    \]
    for infinitely many $n\in\N$. Thus we may write:
    \begin{align*}
        \int_X |f_n-f| \, d\mu & \tikzmarknode{eq1}{\geq} \int_{ \{ |f_n-f| \geq \epsilon \} } |f_n-f| \, d\mu \geq \int_{ \{ |f_n-f| \geq \epsilon \} } \epsilon \, d\mu  \\
        & \tikzmarknode{eq2}{=} \epsilon \cdot \mu \left( \left\{ |f_n-f| \geq \epsilon \right\} \right) \geq \epsilon \cdot \sigma
    \end{align*} \tikz[overlay,remember picture]{\draw[shorten >=1pt,shorten <=1pt] (eq1) -- (eq2);}
    for infinitely many $n\in\N$, thus:
    \[
        \implies f_n \centernot{\toin{L^1}} f    
    \]
    which is absurd.
\end{proof}

%>=====< Question 5 >=====<%

\question
Show that convergence in $L^1$ implies convergence a.e. up to subsequences.

\subsection*{Solution}

\subsection{Convergence in \texorpdfstring{$L^1$}{L1} implies convergence a.e. up to subsequences}
If $f_n \toin{L^1} f$, then:
\[
    \exists \subseq{f} \text{ such that } f_{n_k} \xrightarrow{k\to\infty} f \text{ a.e. in } X
\]

\begin{proof}
    This can be proven by trivially applying the fact that convergence in $L^1$ implies convergence in measure (\ref{L1->measure}) and that, in turn, convergence in measure implies convergence a.e. up to subsequences (\ref{measure->aesubs}).
\end{proof}

%>=====< Question 6 >=====<%

\question
Does convergence in measure imply convergence in $L^1$? Does convergence a.e. imply convergence in $L^1$? Justify the answer.

\subsection*{Solution}

\subsection{Convergence in measure or convergence a.e. do not imply convergence in \texorpdfstring{$L^1$}{L1}}
Neither convergence in measure nor convergence a.e. imply convergence in $L^1$. This can be shown by way of the following counterexample:

\subsubsection{Counterexample}
Let $(X=[0,1], \Leb(X), \lambda|_X)$ and $f_n(x) = n \cdot \chi_{[0,1/n]}(x)$ clearly we have that:
\[
    f_n \toin{a.e.} 0 \text{ in } [0,1]    
\]
and, thus, since $\lambda(X)=1$, we have that:
\[
    f_n \toin{\lambda} 0 \text{ in } [0,1]
\]
but on the other hand, we have that:
\[
    \int_0^1 |f_n-0| \, d\lambda = \int_0^1 f_n \, d\lambda = \int_0^{\frac{1}{n}} n \, d\lambda = n \cdot \frac{1}{n} = 1 \quad \forall n \in \N    
\]
so $f_n \toin{L^1} 1$ and it cannot be that $f_n \toin{L^1} 0$. So convergence a.e. and in measure do not imply convergence in $L^1$.

%>=====< Question 7 >=====<%

\question
Write the definitions of: product measurable space, section of a measurable set. What is the product measure? Why is the definition well-posed?

\subsection*{Solution}

\subsection{Product measurable space}
Let $(X_1, \A_1)$, $(X_2, \A_2)$ be two measurable spaces. Consider the set $R\subseteq \Parts{X_1 \times X_2}$ defined as follows:
\[
    R \coloneqq \{ E_1 \times E_2: \; E_1 \in \A_1, \, E_2 \in \A_2 \}    
\]
let us defined the \textbf{product $\salg$} as:
\[
    \sigma_0(R) \equiv \A_1 \times \A_2   
\]
then the measurable space $(X_1 \times X_2, \A_1 \times \A_2)$ is called the \textbf{product measurable space} of $(X_1, \A_1)$ and $(X_2, \A_2)$.

\subsection{Section of a measurable set}
Let $E\subseteq X_1 \times X_2$, then we define the following two sections:
\begin{align*}
    E_{x_1} & \coloneqq \{ x_2 \in X_2: \; (x_1, x_2) \in E \} \quad x_1 \in X_1 \\
    E_{x_2} & \coloneqq \{ x_1 \in X_1: \; (x_1, x_2) \in E \} \quad x_2 \in X_2
\end{align*}
We have that $E_{x_1} \in \A_2$ $\forall x_1\in X_1$ and $E_{x_2} \in \A_1$ $\forall x_2\in X_2$.

\subsection{Product measure}\label{prodmeas}
Let $(X_1, \A_1, \mu_1)$, $(X_2, \A_2, \mu_2)$ be two $\s{finite}$ measure spaces with measure $\mu_1$ and  $\mu_2$ respectively and let $E\in \A_1 \times \A_2$, then we define the following:
\begin{align*}
    \phi_1: X_1 \to \Rcomppos \quad & \phi_1 (x_1) \coloneqq \mu_2(E_{x_1}) \quad \forall x_1 \in X_1 \\
    \phi_2: X_2 \to \Rcomppos \quad & \phi_2 (x_2) \coloneqq \mu_1(E_{x_2}) \quad \forall x_2 \in X_2
\end{align*}
these are well defined thanks to the fact that $E_{x_1} \in \A_2$ and $E_{x_2} \in \A_1$.\\
Moreover we have that:
\begin{enumerate}[i)]
    \item $\phi_i \in \Mes_+(X_i,\A_i)$ $i=1,2$
    \item \begin{flalign*}
            & \int_{X_1} \phi_1(x_1) \, d\mu_1 = \int_{X_2} \phi_2(x_2) \, d\mu_2 &   
        \end{flalign*}
\end{enumerate}
We thus define the \textbf{product measure} as the function:
\[
    \mu_1 \times \mu_2 : \A_1 \times \A_2 \to \Rcomppos \quad (\mu_1 \times \mu_2)(E) \coloneqq \int_{X_1} \phi_1(x_1) \, d\mu_1 = \int_{X_2} \phi_2(x_2) \, d\mu_2    
\]
let us note that this is a $\s{finite}$ measure and it is well-posed since for both $\phi_1$ and $\phi_2$ the Lebesgue integral is well defined because $\phi_1\in \Mes_+(X_1,\A_1)$ and $\phi_2 \in \Mes_+(X_2,\A_2)$. Lastly, let us note that to have this condition it is essential for $\mu_1$ and $\mu_2$ to be $\s{finite}$.

%>=====< Question 8 >=====<%

\question
Is the product measure space complete? Justify the answer. Which is the relation between $(\R^{m+n}, \Leb(\R^{m+n}), \lambda_{m+n})$ and $(\R^{m+n}, \Leb(\R^m) \times \Leb(\R^n), \lambda_m \times \lambda_n)$?

\subsection*{Solution}

\subsection{The product space is incomplete}
In general the product measure space is incomplete. Let us show this trough a counterexample:

\subsubsection{Counterexample}
Let us consider these two spaces:
\[
    (\R^m \times \R^n, \Leb(\R^m) \times \Leb(\R^n), \lambda_m \times \lambda_n ) \text{ and } (\R^{m+n}, \Leb(\R^{m+n}), \lambda_{m+n})
\]
for simplicity's sake here we take $m=n=1$. As we already know, the space $(\R^2, \Leb(\R^2), \lambda_2)$ is a complete space. Now, let us consider Vitali'set: $V\subseteq[0,1]$, $V\not\in\Leb(\R)$ and let us take the set:
\[
    E \coloneqq \{ x_0 \} \times V \quad (x_0\in\R)    
\]
Clearly, if we take the section $E_{x_0}$, we have that:
\[
    E_{x_0} = V \not\in \Leb(\R) \implies E \not\in \Leb(\R) \times \Leb(\R)    
\]
but we have that:
\[
    E \subseteq F \coloneqq \{ x_0 \} \times [0,1]    
\]
and that $F\in \Leb(\R) \times \Leb(\R)$, furthermore, by the definition of product mesaure (\ref{prodmeas}), we observe that:
\[
    (\lambda \times \lambda)(F) = \int_{[0,1]} \cancelnum{0}{\lambda(\{ x_0 \})} \, d\lambda = 0    
\]
therefore we have proved that there exists a set $E$ that is contained within a set $F$ of zero measure but isn't measurable itself. In other words we have proved that $(\R^2, \Leb(\R)\times\Leb(\R), \lambda \times \lambda)$ is not a complete measure space.\\
Futhermore we can observe quite easily that this means that $(\R^2, \Leb(\R^2), \lambda_2)$ is the completion of $(\R^2, \Leb(\R)\times\Leb(\R), \lambda \times \lambda)$. This argument can be extended to all pairs of $m$ and $n$.

%>=====< Question 9 >=====<%

\question
State the Tonelli theorem.

\subsection*{Solution}

\subsection{Tonelli's theorem} \label{Tonelli}
Let $(X_1, \A_1, \mu_1)$, $(X_2, \A_2, \mu_2)$ be two $\s{finite}$ measure spaces and $f\in\Mes_+(X_1 \times X_2, \A_1 \times \A_2)$. let us define the following:
\begin{align*}
    \psi_1: X_1 \to \Rcomppos \quad & \psi_1 (x_1) \coloneqq \int_{X_2} f(x_1, x_2) \, d\mu_2 \quad \forall x_1 \in X_1 \\
    \psi_2: X_2 \to \Rcomppos \quad & \psi_2 (x_2) \coloneqq \int_{X_1} f(x_1, x_2) \, d\mu_1 \quad \forall x_2 \in X_2
\end{align*}
\begin{enumerate}[i)]
    \item $\psi_i (x_i) \in \Mes_+(X_i,\A_i)$ $i=1,2$
    \item \begin{flalign*}
        & \int_{X_1 \times X_2} f(x_1,x_2) \, d(\mu_1\times\mu_2) = \int_{X_1} \biggl[ \underbrace{ \int_{X_2} f(x_1,x_2) \, d\mu_2 }_{\psi_1(x_1)} \biggr] \, d\mu_1 = \int_{X_2}  \biggl[ \underbrace{\int_{X_1} f(x_1,x_2) \, d\mu_1 }_{\psi_2(x_2)} \biggr] \, d\mu_2 &
    \end{flalign*}
\end{enumerate}

%>=====< Question 10 >=====<%

\question
State the Fubini theorem. By means of a counterexample, show that it is not possible to omit the hypothesis $f \in L^1$.

\subsection*{Solution}

\subsection{Fubini's theorem} \label{Fubini}
Let $(X_1, \A_1, \mu_1)$, $(X_2, \A_2, \mu_2)$ be two $\s{finite}$ measure spaces and $f \in L^1(X_1 \times X_2, \A_1 \times \A_2, \mu_1 \times \mu_2)$, then:
\begin{enumerate}[i)]
    \item \begin{flalign*}
            & f(x_1, \cdot) \in L^1(X_1,\A_1,\mu_1) \text{ a.e. for } x_1 \in X_1& \\
            & f(\cdot, x_2) \in L^1(X_2,\A_2,\mu_2) \text{ a.e. for } x_2 \in X_2&
        \end{flalign*}
    \item \begin{flalign*}
            & \psi_1 (x_1) \coloneqq \int_{X_2} f(x_1, x_2) \, d\mu_2, \quad \psi_1\in L^1(X_1,\A_1,\mu_1) & \\
            & \psi_2 (x_2) \coloneqq \int_{X_1} f(x_1, x_2) \, d\mu_1, \quad \psi_2\in L^1(X_2,\A_2,\mu_2) & \\
        \end{flalign*}
    \item\label{Fubini3} \begin{flalign*} 
            & \int_{X_1 \times X_2} f(x_1,x_2) \, d(\mu_1\times\mu_2) = \int_{X_1} \biggl[ \underbrace{ \int_{X_2} f(x_1,x_2) \, d\mu_2 }_{\psi_1(x_1)} \biggr] \, d\mu_1 = \int_{X_2}  \biggl[ \underbrace{\int_{X_1} f(x_1,x_2) \, d\mu_1 }_{\psi_2(x_2)} \biggr] \, d\mu_2 &
        \end{flalign*}
\end{enumerate}

\subsubsection{Counterexample}
The hypothesis that $f \in L^1$ is necessary, let us consider the following example: \\
\begin{align*}
    (X_i, \A_i, \mu_i) = ( (0,1), \Leb((0,1)), \lambda) \quad i=1,2 \\
    f(x_1, x_2) = \frac{x_1^2 - x_2^2}{(x_1^2+x_2^2)^2} \quad (x_1,x_2) \in (0,1)^2
\end{align*}
We have that $f \in C\left( (0,1)^2 \right) \implies f \in \Mes$ so let us consider the integral of its positive part:
\[
    \int_{X_1 \times X_2} f_+ (x_1, x_2) \, d(\lambda \times \lambda)    
\]
\newpage %otherwise tikz makes a whole mess
and apply Tonelli's theorem (\ref{Tonelli}) since ($f_+\geq 0$):
\begin{align*}
    \int_{X_1 \times X_2} f_+ (x_1, x_2) \, d(\lambda \times \lambda) & \tikzmarknode{eq1}{=} \int_{X_1} \left[ \int_{X_2} f_+(x_1,x_2) \, d\lambda \right] \, d\lambda \\
    & \tikzmarknode{eq2}{=} \int_0^1 \int_0^{x_1} \frac{x_1^2 - x_2^2}{(x_1^2+x_2^2)^2} \, dx_2 \, dx_1 \\
    & \tikzmarknode{eq3}{=} \frac{1}{2} \int_0^1 \frac{1}{x_1} \, dx_1 = + \infty
\end{align*} \tikz[overlay,remember picture]{\draw[shorten >=1pt,shorten <=1pt] (eq1) -- (eq2) -- (eq3);}
thus $f\notin L^1$ and indeed the equality in (\ref{Fubini3}) does not hold:
\begin{align*}
    & \int_{X_1} \biggl[ \int_{X_2} f(x_1,x_2) \, d\mu_2 \biggr] \, d\mu_1 = \dots = \frac{\pi}{4} \\
    & \int_{X_2}  \biggl[ \int_{X_1} f(x_1,x_2) \, d\mu_1 \biggr] \, d\mu_2 = \dots = -\frac{\pi}{4}
\end{align*}

%>=====< Question 11 >=====<%

\question
Write the definition of Lebesgue point. What is about the measure of the set of points that are not Lebesgue points for a function $f \in L^1$?

\subsection*{Solution}

\subsection{Lebesgue point}
A point $x_0 \in [a,b]$ is a \textbf{Lebesgue point} of a function $f$ if:
\[
    \lim_{h\to0} \frac{1}{h} \int_{x_0}^{x_0+h} |f(t)-f(x_0)| \, dt = 0
\]

\subsection{Integrable functions and Lebesgue points}\label{intfunc:lebpoi}
If $f\in L^1((a,b))$ then almost every $x_0 \in X$ is a Lebesgue point of $f$. Therefore the set of points that are not Lebesgue points for $f$ has measure zero.

%>=====< Question 12 >=====<%

\question
State and prove the First Fundamental Theorem of Calculus for $f \in L^1$.

\subsection*{Solution}

\subsection{First Fundamental Theorem of Calculus for \texorpdfstring{$L^1$}{L1}} \label{FTC:1}
Let $X=[a,b]$ and $f \in L^1([a,b])$, we define the integral function $F$ of $f$ as follows:
\[
    F(x) \coloneqq \int_a^x f(t) \, dt \quad x \in [a,b]
\]
then $F$ is differentiable almost everywhere in $(a,b)$ and:
\[
    F'(x) = f(x)    
\]

\begin{proof}
    Let $x_0 \in [a,b]$ be a Lebesgue point for $f$ and $h\neq 0$ be such that $x_0+h \in [a,b]$. Let us write the incremental ratio for F:
    \[
        \frac{F(x_0+h)-F(x_0)}{h} - f(x) = \frac{1}{h} \int_{x_0}^{x_0+h} [f(t) - f(x)] \, dt    
    \]
    we can do this since $f(x)$ is independent of $t$, we may thus write:
    \[
        \left| \frac{F(x_0+h)-F(x_0)}{h} - f(x) \right| \leq \frac{1}{|h|} \int_{x_0}^{x_0+h} |f(t) - f(x)| \, dt \xrightarrow{h\to0} 0
    \]
    thanks to the definition of Lebesgue point, hence:
    \[
        F'(x) = f(x)    
    \]
    but in view of the previous point (\ref{intfunc:lebpoi}) we write:
    \[
        F'(x) = f(x) \text{ a.e. in } (a,b)
    \]
    since almost every $x$ is a Lebesgue point for $f$.
\end{proof}


%>=====< Question 13 >=====<%

\question
Let $f : [a, b] \to \R$. Write the definitions of: variation of $f$ relative to a partition of $[a, b]$; total variation of f over $[a, b]$; function of bounded variation.

\subsection*{Solution}

\subsection{Variation of \texorpdfstring{$f$}{f} relative to a partition of \texorpdfstring{$[a, b]$}{[a,b]}}
Let $f : [a, b] \to \R$ and $P$ be a partition of $[a, b]$:
\[
    P \coloneqq \{ a \equiv x_0 < x_1 < \dots < x_n \equiv b \}    
\]
we define the variation of $f$ with respect to the partition $P$ as:
\[
    v_a^b(f,P) \coloneqq \sum_{k=1}^n |f(x_k) - f(x_{k-1})|
\]

\subsection{Total variation of \texorpdfstring{$f$}{f} over \texorpdfstring{$[a, b]$}{[a,b]}}
Let $\mathcal{P}$ be the collection of all partitions $P$ of $[a, b]$. We define the total variation of $f$ over $[a, b]$ as:
\[
    V_a^b(f) \coloneqq \sup_{P \in \mathcal{P}} v_a^b(f,P)    
\]

\subsection{Function of bounded variation}
A function $f : [a, b] \to \R$ is said to be of \textbf{bounded variation} if $V_a^b(f) < +\infty$. We define the set of functions of bounded variation as:
\[
    BV([a,b]) \coloneqq \{ f:[a,b]\to\R: \, V_a^b(f)<+\infty \}    
\]

%>=====< Question 14 >=====<%

\question
Let $f : [a, b] \to \R$ be monotone. Why $f \in BV ([a, b])$? Show that if $f \in BV ([a, b])$, then $f$ is bounded.

\subsection*{Solution}

\subsection{Monotone functions are of bounded variation}
If $f:[a,b]\to\R$ is a monotone function (either decreasing or increasing), then we have that:
\[
    V_a^b(F) = |f(b)-f(a)| < +\infty \implies f \in BV([a,b])    
\]

\subsection{All functions of bounded variation are bounded}
If $f\in BV([a,b]) \implies f$ is bounded, in fact we have that:
\[
    \sup_{x\in[a,b]} |f(x)| \leq |f(a)| + V_a^b(f)    
\]
thus $f$ must be bounded if $f\in BV$.

%>=====< QUestion 15 >=====<%

\question
What is the Jordan decomposition of a BV function?

\subsection*{Solution}

\subsection{Jordan decomposition of a BV function}
Let $f:[a,b]\to\R$, then \tfae
\begin{enumerate}[i)]
    \item $f \in BV([a,b])$
    \item $\exists \, \phi, \psi : [a,b]\to\R$ both increasing such that:
        \[
            f = \phi - \psi    
        \]
        this is called the \textbf{Jordan decomposition} of $f$.
\end{enumerate}

%>=====< QUestion 16 >=====<%

\question
Why a function of bounded variation is differentiable a.e.?

\subsection*{Solution}

\subsection{Monotonicity implies a.e. differentiability}
Let $f:I\to\R$ be a monotone function, then $f$ is differentiable a.e. in $I$.

\subsection{All BV functions are differentiable a.e.}
For all functions $f \in BV([a,b])$ we can write its Jordan decomposition as $f = \phi - \psi$. Both $\phi$ and $\psi$ are increasing and thus, by the previous point, are a.e. differentiable in $I$ and so is $f$ since it's the difference of the two and the derivative is a linear operator.

%>=====< QUestion 17 >=====<%

\question
Let $f : [a, b] \to \R$ be an increasing function. What can we cay about $f'$ and $\int_{[a,b]}f' \, d\lambda$? Justify the answer.

\subsection*{Solution}

\subsection[]{Derivative and integral of the derivative of an increasing function}
If $f:I\to\R$ is increasing, then:
\begin{align*}
    & f' \text{ exists a.e. in } I \\
    & \int_I f' \, d\lambda \leq f(b)-f(a)
\end{align*}

%>=====< QUestion 18 >=====<%

\question
Can there exists a function $f \in BV ([a, b])$ with $f'\not\in L^1([a, b])$? Justify the answer.

\subsection*{Solution}

\subsection{All BV functions have a Lebesgue-integrable derivative}
There cannot exist a BV function with an unintegrable derivative. In other words:
\[
    f \in BV([a,b]) \implies f' \in L^1([a,b])
\]
indeed for any function $f\in BV([a,b])$ we may write it through its Jordan decomposition $f=\phi-\psi$. Now, since both $\phi$ and $\psi$ are increasing we may apply the previous point and say that both $\phi'$ and $\psi'$ are in $L^1$. Thus since $L^1$ is a vector space $\phi,\psi\in L^1 \implies f\in L^1$.

\sheet

%>=====< Question 1 >=====<%

\question
Write the definition of absolutely continuous function. Show that an absolutely continuous function is also uniformly continuous, but the viceversa is not true; furthermore, a Lipschitz function is absolutely continuous, but the viceversa is not true.

\subsection*{Solution}

\subsection{Absolutely continuous function}
Let $J=[a,b]$,  $f:J\to\R$, we denote by $\mathcal{F}(J)$ the set of finite collections of closed sub-intervals of $J$ without interior points in common.
We say that the function $f$ is absolutely continuous in $J$, if $\forall \epsilon > 0$, $\exists \delta > 0$ such that:
\[
    \forall \{ [a_k, b_k]\} \in \mathcal{F}(J) \quad (k=1,\dots,n)    
\]
for which:
\[
    \sum_{k=1}^n (b_k-a_k) < \delta   
\]
one has:
\[
    \sum_{k=1}^n |f(b_k)-f(a_k)| < \epsilon    
\]
and we denote by $AC([a,b])$ the set of all absolutely continuous functions in $J=[a,b]$.

\subsection{AC functions are also uniformly continuous}
Let $J=[a,b]$, $f:J\to\R$, $f \in AC([a,b])$. If we take:
\[
    \{ [a_k, b_k] \} = \begin{cases}
        \{ [x,y] \} & y \geq x \\
        \{ [y,x] \} & x > y
    \end{cases}
\]
by the definition of absolute continuity we have that:
\[
    |x-y| < \delta \implies |f(y) - f(x)| < \epsilon    
\]
which is the definition of uniform continuity.
The converse isn't true, let us see a poignant counterexample:

\subsubsection{Counterexample}
Let us take the following $f$:
\[
    f(x) = \begin{cases}
       x \sin(\frac{1}{x}) & x\in[-1,1]\setminus\{0\} \\
       0 & x=0
    \end{cases}    
\]
this function is not absolutely continuous in $[-1,1]$ but is  is uniformly continuous in $[-1,1]$.

\subsection{Lipschitz functions are absolutely continuous}
Let $f$ be a Lipschitz function in $J=[a,b]$, we have that $f\in AC$.\\
\begin{proof} Indeed we have that:
    \[
        \sum_{k=1}^n |f(b_k)-f(a_k)| \leq L \cdot \sum_{k=1}^n (b_k-a_k) =L \cdot \delta = \epsilon
    \]
    thus we can choose $\delta = \epsilon/L$.
\end{proof}
The converse is not true, let us show this trough a counterexample:

\subsubsection{Counterexample}
Let $J=[0,1]$ and $f(x)=\sqrt{x}$. Let us write $f$ as:
\[
    f(x) = \int_0^x \frac{1}{2\sqrt{x}} \, dt  \quad x\in[0,1]    
\]
and we have that $\frac{1}{2\sqrt{x}}\in\Leb^1(J)\implies f \in AC(J)$ but $f$ is clearly not a Lipschitz function.

%>=====< Question 2 >=====<%

\question
Let $f \in \Mes_+(X, \A)$ be such that $\int_X f \, d\mu < +\infty$. Show that for any $\epsilon > 0$ there exists $\delta > 0$ such that for any $E \in \A$ with $\mu(E) < \delta$ there holds $\int_E f d\mu < \epsilon$.

\subsection*{Solution}

\subsection{If \texorpdfstring{$f\in\Mes_+$}{f is measurable} and integrable, the integral is continuous in the measure}\label{int:cont}
Let $f \in \Mes_+(X, \A)$ be such that $\int_X f \, d\mu < +\infty$, then $\epsilon>0$, $\exists \delta > 0$ such that:
\[
    \forall E \in \A \text{ with } \mu(E) < \delta \quad \int_E f \, d\mu < \epsilon    
\]
\begin{proof}
    Let $F_n \coloneqq \{ f<n \}$ with $n\in\N$. Clealry we have that:
    \begin{align*}
        & F_n \in \A \quad F_n \uparrow X \\%\implies \lim_{n\to\infty} F_n = X 
        & X = \{ f = + \infty \} \cup \left( \bigcup_{n=1}^\infty F_n \right)   
    \end{align*}
    So we have that:
    \[
        \int_X f \, d\mu < + \infty \implies f \text{ is finite a.e. } \implies \mu\left( \{ f = +\infty\}\right) = 0
    \]
    and thus we can write:
    \[
        \int_X f \, d\mu = \lim_{n\to\infty} \int_{F_n} f \, d\mu
    \]
    and so, by the definition of limit, $\forall \epsilon>0$ $\exists \bar{n}\in\N$ such that $\forall n > \bar{n}$:
    \[
        \left| \int_X f \, d\mu - \int_{F_n} f \, d\mu \right| = \left| \int_{F_n^\complement} f \, d\mu \right| < \frac{\epsilon}{2}
    \]
    therefore, for a fixed $n>\bar{n}$, we get:
    \begin{align*}
        \int_E f \, d\mu & \tikzmarknode{eq1}{=} \int_{ E\cap F_n} f\, d\mu + \int_{E \cap F_n^\complement} f \, d\mu \\
        & \tikzmarknode{eq2}{<} n\cdot \mu (E) + \frac{\epsilon}{2} \quad \text{ by the above and the fact that } E \supset E\cap F_n \\
        & \tikzmarknode{eq3}{<} n \cdot \delta + \frac{\epsilon}{2} = \epsilon \quad \text{  if we choose } \delta = \frac{\epsilon}{2n}
    \end{align*}\tikz[overlay,remember picture]{\draw[shorten >=1pt,shorten <=1pt] (eq1) -- (eq2) -- (eq3);}
    In short, we have used the fact that $f$ is finite a.e. to get the limit and, in turn, from this we derived the above inequality which yields us the thesis.
\end{proof}

%>=====< Question 3 >=====<%

\question
Show that if $f \in L^1([a, b])$, then $F(x) \coloneqq \int_{[a,x]} f \, d\lambda$ is absolutely continuous in $[a, b]$.

\subsection*{Solution}

\subsection{The integral function if AC} \label{intfunc:AC}
Let $I=[a,b]$ and $f\in L^1([a,b])$, then:
\[
    F(x) \coloneqq \int_{[a,x]} f \, d\lambda \in AC(I)    
\]

\begin{proof}
    Consider the following set $E$:
    \[
        E \coloneqq \bigcup_{k=1}^n [a_k, b_k] \text{ with } \{[a_k , b_k]\} \in \mathscr{F}(I)
    \]
    then, since such intervals are disjoint, we have that:
    \[
        \lambda(E) = \sum_{k=1}^n \lambda([a_k, b_k]) = \sum_{k=1}^n (b_k-a_k)
    \]
    thus we may write:
    \begin{align*}
        \sum_{k=1}^n |F(b_k) - F(a_k)| & \tikzmarknode{eq1}{=} \sum_{k=1}^n \left| \int_{[a_k, b_k]} f \, d\lambda \right| \\
        & \tikzmarknode{eq2}{\leq} \sum_{k=1}^n \int_{[a_k, b_k]} |f| \, d\lambda \\
        & \tikzmarknode{eq3}{=} \int_E |f| \, d\lambda \quad \text{ by the definition of E}
    \end{align*}\tikz[overlay,remember picture]{\draw[shorten >=1pt,shorten <=1pt] (eq1) -- (eq2) -- (eq3);}
    and so by the previous point (\ref{int:cont}) we get the thesis.
\end{proof}

%>=====< Question 4 >=====<%

\question
Which is the relation between the spaces $BV ([a, b])$ and $AC([a, b])$?

\subsection*{Solution}

\subsection{All AC functions are BV}\label{ac:bv}
Let $f \in AC([a, b])$, then $f\in BV([a, b])$.

%>=====< Question 5 >=====<%

\question
State and prove the Second Fundamental Theorem of Calculus.

\subsection*{Solution}

\subsection{The Second Fundamental Theorem of Calculus}\label{FTC:2}
Let $F:[a,b]\to\R$, \tfae
\begin{enumerate}[i)]
    \item $F\in AC([a,b])$
    \item $F$ is differentiable a.e. in $[a,b]$ with $F'\in L^1([a,b])$ and we have:
        \[
           F(x) = \int_a^x F' \, d\lambda + F(a) \quad \forall x \in [a,b] 
        \]
\end{enumerate}

\begin{proof}
    \hspace*{\fill} %leave a blank line
    \begin{itemize}
        \item \textbf{$(i)\implies(ii)$:} 
            By the previous point we have $F\in AC([a,b])\implies F \in BV([a,b])\implies F$ is differentiable a.e. in $[a,b]$ and $F'\in L^1([a,b])$. Let us also suppose that $F$ is increasing. \\
            We define the following:
            \[
                G(x) \coloneqq \int_a^x F' \, d\lambda \quad x\in [a,b]    
            \]
            hence $G$ is differentiabe a.e. in $[a,b]$ by the First Fundamental Theorem of Calculus (\ref{FTC:1}) and we have:
            \[
                (F-G)'(x) = F'(x) - G'(x) = F'(x) - F'(x) = 0 \quad \text{ a.e. in } [a,b]
            \]
            furthermore $G\in AC$ by (\ref{intfunc:AC}) which implies $F-G\in AC$ and so $\forall a\leq x_1 \leq x_2 \leq b$ we have:
            \[
                [F(x_2) - G(x_2)] - [F(x_1) - G(x_1)] = F(x_2) - F(x_1) - \int_{[x_1, x_2]} F' \, d\lambda  \geq 0   
            \]
            thanks to the definition of $G$ and the trivial fact that $\int_{[x_1, x_2]} F' \, d\lambda \leq F(x_2)-F(x_1)$. Since this is true for any pair of  $x_1,x_2$ such that $x_1\leq x_2$, it makes $(F-G)$ increasing. However, we have also proved that $(F-G)'=0$ a.e. in $[a,b]$. We must thus conclude that:
            \begin{align*}
                & \exists c\in\R \quad F-G\equiv c \text{ in } [a,b] \\
                & \implies F(x) - G(x) = F(a) - \cancelnum{0}{G(a)} 
            \end{align*}
            thus, if we substitute $G$ for its definition and bring to left hand side, we, at last, attain the thesis:
            \[
                F(x) = \int_a^x F' \, d\lambda + F(a) \quad \forall x \in [a,b]
            \]
        \item \textbf{$(ii)\implies(i)$:} 
            we already know that the integral function is $AC$ (see \ref{intfunc:AC}), thus its translation by $F(a)$ is also $AC$.
    \end{itemize}
\end{proof}

%>=====< Question 6 >=====<%

\question
Write the definitions of: dense set, separable metric space, nowhere dense set, set of first category, set of second category. Provide an example of a nowhere dense and one of a set of first category.

\subsection*{Solution}

Let $X$ be a metric space equipped with a metric $d$.

\subsection{Dense set}
A set $A \subset X$ is dense in $X$ if $\bar{A} = X$, where:    
\[
    \bar{A} = \{ y \in X : \; \exists \seq{x} \subset A, \, x_n \to y \}    
\]

\subsection{Separable metric space}
$X$ is a separable metric space if there exists a subset $A$ which is countable and dense in $X$.

\subsection{Nowhere dense set}
A set $E\subseteq X$ is said to be nowhere dense if:
\[
    \interior( \bar{E}) = \emptyset
\]

\subsubsection{Example}
We take $E=\mathbb{Z} \subset X=\R$, since $\mathbb{Z}$ is the countable union of all integers and thus its interior is empty. We have:
\[
    E = \bar{E} \implies \interior( \bar{E}) = \interior(E)=\emptyset      
\]

\subsection{Set of first category}
A set $E\subseteq X$ is said to be of first category (or meagre) in $X$ if $E$ is the union of countably many nowhere dense sets. 

\subsubsection{Example}
We take $E=\Q$ and $X=\R$, since $\Q$ is the countable union of all rational numbers. Thus $\Q$ is a set of first category in $\R$.

\subsection{Set of second category}
A set $E \subseteq X$ which is not of first category, is said to be of second category in $X$.

%>=====< Question 7 >=====<%

\question
State the Baire category theorem and its corollary.

\subsection*{Solution}

\subsection{Baire's theorem} \label{Baire}
Let $(X,d)$ be a complete metric space, then $X$ is of second category in itself.

\subsection{Corollary to Baire's theorem}
The intersection of a countable family of opens sets dense in $X$ is a set dense in $X$.

\question
Write the definitions of: compact metric space; sequentially compact metric space, totally bounded metric space. Explain how these properties are related.

\subsection*{Solution}
Let $(X,d)$ be a metric space.

\subsection{Compact metric space}
$X$ is said to be compact if from any open cover of $X$ we can extract a finite open subcover.

\subsection{Sequentially compact metric space}
$X$ is said to be sequentially compact if from any sequence $\seq{x} \subset X$ we can extract a subsequence which converges to some $x_0\in X$.

\subsection{Totally bounded metric space}
$X$ is said to be totally bounded if $\forall \epsilon>0$ $\exists A \subset X$ finite such that:
\[
    \dist(x,A)<\epsilon \quad \forall x \in X     
\]
where:
\[
   \dist(x,A) = \inf_{ y \in A } d(x,y) 
\]

\subsection{Relation between compactness, sequential compactness and total boundedness} \label{compact:altdef}
\tfae
\begin{enumerate}[i)]
    \item $X$ is compact
    \item $X$ is sequentially compact
    \item $X$ is totally bounded and complete.
\end{enumerate}

%>=====< Question 9 >=====<%

\question
Write the $\epsilon-\delta$ definition of equicontinuous subset $F$ of $C^0(X)$, where $X$ is a compact metric space. Explain from which parameters $\delta$ depends. In particular, write the definition when $F = \{f_n\}_{n\in\N}$.

\subsection*{Solution}

\subsection{Equicontinuous set}
A subset $A\subset C^0(X)$ is said to be equicontinuous if $\forall\epsilon>0$ there exists $\delta_\epsilon>0$ such that:
\[
    \forall f \in A, \, x,y \in X, \, d(x,y)<\delta_\epsilon \implies |f(x) - f(y)| < \epsilon
\]
Let us note that here $\delta_\epsilon$ depends on only $\epsilon$ and nothing else.
Moreover if we take a set $F=\{f_n\}_{n\in\N}$, then  asking that $F$ is equicontinuous is equivalent to asking that every $f_n$ is uniformly continuous with respect to a shared $\delta_\epsilon$.

%>=====< Question 10 >=====<%

\question
State the Ascoli-Arzelà theorem.

\subsection*{Solution}

\subsection{Ascoli-Arzelà theorem}
$F\subset C^0(X)$ is bounded and equicontinuous if and only if it is relatively compact (i.e. its closure is compact). More succintly:
\[
    F \text{ bounded and equicontinuous } \iff \bar{F} \text{ compact }    
\]

%>=====< Question 11 >=====<%

\question
Write the statement of the Ascoli-Arzelà theorem when the subset of $C^0(X)$ is a sequence $\seq{f}$.

\subsection*{Solution}

\subsection{Ascoli-Arzelà theorem for sequences}
Let $F=\{f_n\}_{n\in\N} \subset C^0(X)$ be a sequence of functions, then $F$ is bounded and equicontinuous:
\begin{itemize}
    \item $d(x,y)<\delta_\epsilon \implies |f_n(x) - f_n(y)| < \epsilon $ $\forall n \in \N$
    \item $\exists k > 0$ such that $\forall x \in X$ $f_n(x)<k$ $\forall n \in \N$
\end{itemize}
if and only if its closure is compact (or alternatively, by \ref{compact:altdef}, sequentially compact).\\

\sheet

%>=====< Question 1 >=====<%

\question
Show that $C^0([a, b])$ is separable.

\subsection*{Solution}

\subsection{Stone-Weierstrass theorem}
The set of polynomials is dense in $C^0([a,b])$

\subsection{\texorpdfstring{$C^0([a,b])$}{C0} is separable}
$C^0([a,b])$ is separable.

\begin{proof}
    By the Stone-Weierstrass theorem we have that, for any $f\in C^0([a,b])$, given any $\epsilon > 0$, there exists a polynomial $p$ such that:
    \[ \dist(f,p)=\sup_{x\in [a,b]} |f-p| < \frac{\epsilon}{2} \]
    So we can find a polynomial $r$ with \textbf{rational} coefficients such that:
    \[ \dist(p,r)<\frac{\epsilon}{2}\]  
    hence by the triangular inequality:
    \[ \dist(f,r) \leq \dist(f,p) + \dist(p,r) < \epsilon \]
    therefore the set of polynomials with rational coefficients is dense in $C^0([a,b])$. So, since such a set is countable, $C^0([a,b])$ is separable and we have the thesis.
\end{proof}

%>=====< Question 2 >=====<%

\question
Write the definition of normed space and provide examples. What is the metric space induced by a given normed space?

\subsection*{Solution}

\subsection{Normed space}
Let $X$ be a vector space, a norm on $X$ is a function such that:
\[ \|x\| : X \to [0, \infty) \]
and:
\begin{enumerate}[i)]
    \item $\|x\|=0 \iff x=0$
    \item $\forall \alpha \in \R$, $x \in X$: $\|\alpha x\| = |\alpha| \cdot \|x\|$
    \item $\forall x,y \in X$: $\|x+y\| \leq \|x\| + \|y\|$ 
\end{enumerate} 
and we say that the pair $(X, \| \cdot \|)$ is a normed space.

\subsection{Examples of normed spaces}
\begin{enumerate}[i)]
    \item $\R^n$ with a norm of the family:
        \begin{align*}
            & \|x\|_p \coloneqq \left( \sum_{i=1}^n |x_i|^p \right)^{1/p} \quad p \in [1,\infty) \\
            & \|x\|_\infty \coloneqq \max_{i=1,\dots,n} |x_i|
        \end{align*}
    \item $C^0([a,b])$ with the norm:
        \begin{equation*}
            \|f\|_{C^0} \coloneqq \sup_{x \in [a,b]} |f(x)| = \max_{x \in [a,b]} |f(x)|
        \end{equation*}
    \item $L^1(X,\A,\mu)$ with the norm:
        \begin{equation*}
            \|f\|_1 \coloneqq \int_X |f(x)| \, dx
        \end{equation*}
    \item $L^\infty(X,\A,\mu)$ with the norm:
        \begin{equation*}
            \|f\|_\infty \coloneqq \esssup_X |f(x)|
        \end{equation*}
    \item $C^k([a,b])$ with the norm:
        \begin{equation*}
            \|f\|_{C^k} \coloneqq \sum_{i=0}^k \|f^{(i)}\|_{\infty}
        \end{equation*}
    \item $BV([a,b])$, with two possible norms:
        \begin{equation*}
            \|f\|_{BV} \coloneqq \begin{cases}
                & |f(a)| + V_a^b(f) \\
                & \|f\|_1 + V_a^b(f)
            \end{cases}
        \end{equation*}
    \item $AC([a,b])$ with two possible norms:
        \begin{equation*}
            \|f\|_{AC} \coloneqq \begin{cases}
                & |f(a)| + \|f'\|_1 \\
                & \|f\|_1 + \|f'\|_1
            \end{cases}
        \end{equation*}
    \item $\ell^p, \ell^\infty$, we take a sequence of real numbers of the form:
        \[ x = \{ x^{(k)} \}_{k\in\N} = (x^{(1)}, x^{(2)}, \dots) \]
        and we define the norms:
        \begin{align*}
            & \|x\|_p \coloneqq \left( \sum_{k=1}^\infty |x^{(k)}|^p \right)^{1/p} \quad p \in [1,\infty) \\
            & \|x\|_\infty \coloneqq \sup_{k=1,\dots,\infty} |x^{(k)}|
        \end{align*}
        we can define two normed spaces as follows:
        \begin{align*}
            \ell^p & \coloneqq \{x \text{ sequence of real numbers }: \|x\|_p < \infty\} \\
            \ell^\infty & \coloneqq \{x \text{ sequence of real numbers }: \|x\|_\infty < \infty\}
        \end{align*}
\end{enumerate}

\subsection{Metric space induced by a normed space}
Let $(X, \| \cdot \|)$ be a normed space. The metric space induced by $(X, \| \cdot \|)$ is the pair $(X, d)$ where $d$ is the distance function defined by:
\[ \dist(x,y) \coloneqq \|x-y\| \]

%>=====< Question 3 >=====<%

\question
In a normed space, write the definitions of: convergent sequence; Cauchy sequence; bounded sequence. Which are the relations among these notions? Show that if $x_n \to x$, then $\| x_n \| \to \|x\|$ as $n \to +\infty$.

\subsection*{Solution}

\subsection{Convergent sequence}
Let $(X, \| \cdot \|)$ be a normed space. A sequence $\seq{x} \subset X$ is said to be convergent to $x \in X$ if:
\[ x_n \xrightarrow{n \to +\infty} x \iff d(x_n, x) \xrightarrow{n \to +\infty} 0 \iff \|x_n - x\| \xrightarrow{n \to +\infty} 0 \]
Furthermore:
\[ x_n \xrightarrow{n \to +\infty} x \implies \|x_n\| \xrightarrow{n \to +\infty} \|x\| \]
Since:
\[ | \|x_n\| - \|x\| | \leq \|x_n - x\| \quad \forall n \in \N \]

\subsection{Cauchy sequence}
Let $(X, \| \cdot \|)$ be a normed space. A sequence $\seq{x} \subset X$ is said to be Cauchy if:
\[ \forall \epsilon > 0 \; \forall \bar{n} \in \N \quad \| x_m - x_n \| < \epsilon \quad \forall m,n \geq \bar{n} \]

\subsection{Bounded sequence}
A sequence $\seq{x} \subset X$ is said to be bounded if:
\[ \exists M>0 \quad \|x_n\| < M \quad \forall n \in \N \]

\subsection{Relations among convergent, Cauchy and bounded sequences}
\begin{enumerate}[i)]
    \item $\seq{x}$ is convergent $\implies$ $\seq{x}$ is Cauchy.
    \item $\seq{x}$ is Cauchy $\implies$ $\seq{x}$ is bounded.
\end{enumerate}

%>=====< Question 4 >=====<%

\question
Write the definition of series in a normed space. Is it true that if $\sum^{+\infty}_{n=0} \|x_n\|$ is convergent, then $\sum^{+\infty}_{n=0} x_n$ is convergent.

\subsection*{Solution}

\subsection{Series in a normed space}
Let $(X, \| \cdot \|)$ be a normed space and $\seq{x} \subset X$ be a sequence. Let us define the sequence of partial sums (series) as the following:
\[ s_n \coloneqq x_0 + \cdots + x_n = \sum_{k=0}^n x_k \]
It is said to be convergent if:
\[ \exists x \in X : \; s_n \xrightarrow{n \to +\infty} x \iff \| s_n - x \| \xrightarrow{n \to +\infty} 0 \]
and we say that:
\[ \sum^{+\infty}_{n=0} x_n \text{ is the sum of the series} \]
Moreover, we have that:
\[ \sum^{+\infty}_{n=0} \|x_n\| \text{ is convergent } \centernot\implies \sum^{+\infty}_{n=0} x_n \text{ is convergent} \]

%>=====< Question 5 >=====<%

\question
What is a complete normed space? Write the definition of Banach space, provide examples.

\subsection*{Solution}

\subsection{Complete normed space}
Let $(X, \| \cdot \|)$ be a normed space. The space $(X, \| \cdot \|)$ is said to be complete if the metric space induced by $(X, \| \cdot \|)$ is complete.
\[ (X, \|\cdot\|) \text{ is complete } \iff (X, d) \text{ is complete } \iff \text{ every Cauchy sequence in } X \text{ is convergent} \]

\subsection{Banach space}
A complete normed \textbf{vector} space is called a Banach space. Examples of Banach spaces are the same as those given above for normed spaces.

%>=====< Question 6 >=====<%

\question
State the criterion, involving convergence of series, for completeness of a normed space.

\subsection*{Solution}

\subsection{Criterion for completeness of a normed space} \label{Series CriterionForComplete}
\begin{enumerate}[i)]
    \item Let $X$ be a Banach space and $\seq{x} \subset X$. If $\sum^{+\infty}_{n=1} \|x_n\|$ is convergent, then $\sum^{+\infty}_{n=1} x_n$ is convergent.
    \item Let X be a normed space. If for any $\seq{x} \subset X$ such that the series $\sum^{+\infty}_{n=1} \| x_n \|$ is convergent, we also have that $\sum_{n=1}^{+\infty} x_n$ is convergent, then X is a Banach space.
\end{enumerate}

%>=====< Question 7 >=====<%

\question
State and prove the Riesz's Lemma.

\subsection*{Solution}

\subsection{Riesz's Lemma}
Let $X$ be a normed space, $E \subsetneq X$ a closed subspace, then $\forall \epsilon > 0$ $\exists x \in X$ such that:
\[ \| x \| =1 \text{ and } \dist(x,E) \geq 1 - \epsilon \] 
where $\dist(x,E) \coloneqq \inf_{\xi \in E} \|x - \xi \|$ and $x$ is called the "almost orthogonal element".

\begin{proof}
    \hspace*{\fill}\\ %leave a blank line
    Let $y \in X \setminus E$, then:
    \[ d \coloneqq \dist(y,E) > 0 \text{ since } E \text{ is closed} \]
    Now, let $\epsilon \in (0,1)$, then, in view of the definition of $\dist(x,E)$, we have:
    \[ d = \dist(y,E) = \inf_{\xi \in E} \|y - \xi \| \]
    and thus we can find $\zeta \in E$ such that:
    \[ d \leq \| y - \zeta \| \leq \frac{d}{1- \epsilon} \]
    Now, let us define the following:
    \[ x \coloneqq \frac{y-\zeta}{ \| y-\zeta \| } \]
    So by definition $x$ has $\|x\|=1$ and now, thanks to the homogeneity of the norm and the closedness of $E$, $\forall \xi \in E$ we also have that:
    \begin{align*}
        \| x - \xi\| & \tikzmarknode{eq1}{=} \left\| \frac{y-\zeta}{ \| y-\zeta \| } - \xi \right\| = \frac{1}{ \| y-\zeta \| } \left\| y-\zeta - \xi \cdot \| y-\zeta \| \right\| \\
        & \tikzmarknode{eq2}{=} \frac{1}{ \| y-\zeta \| } \left\| y - \underbrace{(\zeta + \xi \cdot \| y-\zeta \| )}_{\in E} \right\| \geq \frac{d}{\| y-\zeta \|} \geq 1 - \epsilon
    \end{align*}\tikz[overlay,remember picture]{\draw[shorten >=1pt,shorten <=1pt] (eq1) -- (eq2);}
    therefore we have that $\dist(x,E) \geq 1 - \epsilon$ and we have the thesis.
\end{proof}

%>=====< Question 8 >=====<%

\question
State and prove the Riesz's Theorem.

\subsection*{Solution}

\subsection{Riesz's Theorem}
Let $X$ be a normed space, if the closed ball $\bar{B}_1(0)$ is compact, then $\dim(X)<\infty$.

\begin{proof}
    \hspace*{\fill}\\ %leave a blank line
    Let $x_1 \in \bar{B}_1(0)$ and $Y_1 \coloneqq \spn\{x_1 \}$. Clearly $Y_1$ is a vector subspace of $X$ and $\dim(Y_1)=1<+\infty$ $\iff$ $Y_1$ is closed.
    \begin{itemize}
        \item If $X=Y_1$, then $\dim(X) < \infty$ and we have the thesis.
        \item If $X \neq Y$, then we can use Riesz's Lemma with $\epsilon = 1/2$ to find $x-2 \in \bar{B}_1(0)$ such that:
            \[ \| x_1 - x_2 \| \geq 1/2 \]
            and we define the following set:
            \[ Y_2 \coloneqq \spn\{x_1, x_2 \} \]
        and we repeat the argument above:
        \begin{itemize}
            \item If $X=Y_2$, then $\dim(X) < \infty$ and we have the thesis.
            \item If $X \neq Y_2$, then we can use again Riesz's Lemma $x_3 \in \bar{B}_1(0)$ such that:
                \[ \| x_3 - x_i \| \geq 1/2 \text{ for } i=1,2 \]
                and we define the following set:
                \[ Y_3 \coloneqq \spn\{x_1, x_2, x_3 \} \]
                \dots
        \end{itemize}
    \end{itemize}
    If $X$ is not finite dimensional this argument can be iterated to construct a sequence:
    \[ \seq{x} \subseteq \bar{B}_1(0) \text{ such that } \| x_i - x_j \| \geq 1/2 \text{ } i \neq j, \; \forall i,j \in \N \]
    hence $\seq{x}$ is a bounded sequence ($\|x_n\| \leq 1$ $\forall n \in \N$) but $\seq{x}$ has no convergent subsequence. Thus $\bar{B}_1(0)$ is not sequentially compact and so $\bar{B}_1(0)$ is not compact. 
\end{proof}

%>=====< Question 9 >=====<%

\question
Write the definition of equivalent norms. In which type of vector spaces all norms are equivalent? Exhibit an example of a vector space that can be endowed with two norms that are not equivalent.

\subsection*{Solution}

\subsection{Equivalent norms}
Let $(X,\|\cdot\|)$ and $(X,\|\cdot\|')$ be two normed spaces. We say that $\|\cdot\|$ and $\|\cdot\|'$ are equivalent if there exist two constants $m,M>0$ such that:
\[ m \cdot \|x\| \leq \|x\|' \leq M \cdot \|x\| \text{ for all } x \in X \]

\subsection{All norms are equivalent in finite dimensional normed spaces}
If $X$ is a normed space and $\dim(X)<\infty$, then all norms are equivalent.

\subsection{Example of two non equivalent norms}
Both $(C^0([a,b]),\| \cdot\|_\infty)$ and $((C^0([a,b]),\| \cdot\|_1)$ are normed spaces, but since $\dim(C^0([a,b]))=\infty$ we have that $\| \cdot\|_\infty$ and $\| \cdot\|_1$ are not equivalent.

%>=====< Question 10 >=====<%

\question
Is it true in general that any vector subspace of a given normed space is closed?

\subsection*{Solution}

\subsection{Closedness of vector subspaces}
Let $X$ be a normed space and $Y$ a vector subspace of $X$. We have the following:
\begin{itemize}
    \item $\dim(Y) < \infty \implies Y$ is closed.
    \item $\dim(Y) = \infty \centernot\implies Y$ is closed.
\end{itemize}
Therefore we can say that in general not all vector subspaces of a normed space are closed.

%>=====< Question 11 >=====<%

\question
Write the definitions of $\Leb^p$ and $L^p$ . Show that $L^p$ is a vector space (and its preliminary lemma).

\subsection*{Solution}

\subsection{Definition of \texorpdfstring{$\Leb^p$}{Lp}}
Let $(X,\A, \mu)$ be a measure space, $p\in[1,+\infty]$. We define the space $\Leb^p$ as:
\[ \Leb^p(X,\A, \mu) \coloneqq \left\{ f:X\to\Rcomp \text{ measurable}, \; \int_X |f|^p \, d\mu < +\infty \right\} \]

\subsection{Definition of \texorpdfstring{$L^p$}{Lp}}
On the space $\Leb^p(X,\A, \mu)$ we define the following equivalence relation $R$:
\[ f,g\in\Leb^p \quad f R g \iff f=g \text{ a.e. in } X \]
We define the space $L^p$ as the quotient space $\Leb^p/R$:
\[ L^p(X,\A, \mu) \coloneqq \Leb^p(X,\A, \mu)/R \]

%>=====< Question 12 >=====<%

\question
Write the definition of conjugate numbers. Show Young's inequality.

\subsection*{Solution}

\subsection{Definition of conjugate numbers}
Let $p,q\in[1,+\infty]$. We say that $p$ and $q$ are conjugate if:
\begin{itemize}
    \item $p,q \in (1,+\infty)$ and $\frac{1}{p}+\frac{1}{q}=1$.
    \item $p=1$ and $q=+\infty$ or viceversa.
\end{itemize}

\subsection{Young's inequality}
Let $p,q \in (1,+\infty)$ be conjugate numbers and $a,b > 0$, then:
\[ ab \leq \frac{a^p}{p} + \frac{b^q}{q} \]

\begin{proof}
    \hspace*{\fill}\\ %leave a blank line
    Let us define the following convex function:
    \[ \phi(x) \coloneqq e^x \quad \phi(tx + (1-t)y) \leq t\phi(x) + (1-t)\phi(y) \; \forall x,y \in \R, \, t \in [0,1] \]
    Now, we choose $t=1/p$, $1-t=1/q$, $x=\log(a^p)$ and $y=\log(b^q)$, thus we have:
    \begin{align*}
        ab & \tikzmarknode{eq1}{=} e^{\log(a)} \cdot e^{\log(b)} =  e^{\frac{1}{p}\log(a^p)} \cdot e^{\frac{1}{q}\log(b^q)} \\
        & \tikzmarknode{eq2}{\leq}  \frac{1}{p}e^{\log(a^p)} + \frac{1}{q}e^{\log(b^q)} \\
        & \tikzmarknode{eq3}{=} \frac{1}{p} a^p + \frac{1}{q} b^q
    \end{align*}\tikz[overlay,remember picture]{\draw[shorten >=1pt,shorten <=1pt] (eq1) -- (eq2) -- (eq3);}

\end{proof}

%>=====< Question 13 >=====<%

\question
Show Hölder's inequality.

\subsection*{Solution}

\subsection{Hölder's inequality}
Let $f,g \in \Mes(X,\A)$ and $p,q \in [1,+\infty]$ be two conjugate numbers, then:
\[ \| f \cdot g \|_1 \leq \| f \|_p \cdot \| g \|_q \]
where we have:
\begin{align*}
    & \|f\|_p \coloneqq \left( \int_X |f|^p \, d\mu \right)^{1/p} \quad p \in [1,+\infty) \\
    & \|f\|_\infty \coloneqq \esssup_X |f|
\end{align*}

\begin{proof}
    \hspace*{\fill}\\ %leave a blank line
    Let us divide the proof into two possible cases:
    \begin{enumerate}[i)]
        \item $p,q \in (1,+\infty)$:
            \begin{itemize}
                \item If $\|f\|_p\cdot\|g\|_q=+\infty$, the inequality is trivial.
                \item If $\|f\|_p\cdot\|g\|_q=0$, we have that $f=0$ a.e. $\vee$ $g=0$ a.e. $\implies$ $f\cdot g=0$ a.e. $\implies$ $\|f\cdot g\|_1=0$ and the inequality holds.
                \item If$\|f\|_p$ and $\|g\|_q$ exist finite and non-zero, we fix $x\in X$ and define the two following quantities:
                    \[ a \coloneqq \frac{|f|^p}{\|f\|^p_p} \quad b \coloneqq \frac{|g|^q}{\|g\|^q_q} \]
                    Now we apply Young's inequality to these two quantities:
                    \begin{align*}
                        a^{1/p} b ^{1/q} & \tikzmarknode{eq1}{=} \frac{|f|}{\|f\|_p} \cdot \frac{|g|}{\|g\|_q} \\
                        & \tikzmarknode{eq2}{\leq} \frac{1}{p}\frac{|f|^p}{\|f\|^p_p} + \frac{1}{q}\frac{|g|^q}{\|g\|^q_q}
                    \end{align*}\tikz[overlay,remember picture]{\draw[shorten >=1pt,shorten <=1pt] (eq1) -- (eq2);}
                    Let us integrate both sides of the inequality:
                    \begin{align*}
                        \frac{1}{\|f\|_p\|g\|_q} \cdot \int_X |f \cdot g| \, d\mu & \tikzmarknode{eq1}{=} \frac{1}{p}\cancelnum{1}{\frac{\int_X |f|^p \, d\mu}{\|f\|^p_p}} + \frac{1}{q}\cancelnum{1}{\frac{\int_X |g|^q \, d\mu}{\|g\|^q_q}} \\
                        & \tikzmarknode{eq2}{=} \frac{1}{p} + \frac{1}{q} = 1
                    \end{align*}\tikz[overlay,remember picture]{\draw[shorten >=1pt,shorten <=1pt] (eq1) -- (eq2);}
            \end{itemize}
        \item $p=1$, $q=+\infty$ (or viceversa): \\
            Let us recall:
            \[ |g| \leq \|g\|_\infty = \esssup_X |g| \text{ a.e. in X} \implies |fg| \leq |f| \|g\|_\infty \]
            let us integrate both sides of the inequality:
            \[ \int_X |fg| \, d\mu = \|fg\|_1 \leq \|g\|_\infty \int_X |f| \, d\mu = \|f\|_1 \|g\|_\infty \]
            and so the inequality holds.
    \end{enumerate}

\end{proof}

%>=====< Question 14 >=====<%

\question
Show Minkowski's inequality.

\subsection*{Solution}

\subsection{Minkowski's inequality} \label{minkowski}
Let $f,g \in \Mes(X,\A)$ and $p\in(1,+\infty)$, then:
\[ \| f + g \|_p \leq \| f \|_p + \| g \|_p \]

\begin{proof}
    \hspace*{\fill}\\ %leave a blank line
    Let us divide the proof into three possible cases:
    \begin{itemize}
        \item $p\in(1,+\infty)$:
            \begin{align*}
                \|f+g\|_p^p & \tikzmarknode{eq1}{=} \int_X |f+g|^p \, d\mu = \int_X \underbrace{|f+g|}_{\leq |f| + |g|} |f+g|^{p-1} \, d\mu \\
                & \tikzmarknode{eq2}{\leq} \int_X |f| |f+g|^{p-1} \, d\mu + \int_X |g| |f+g|^{p-1} \, d\mu
            \end{align*}\tikz[overlay,remember picture]{\draw[shorten >=1pt,shorten <=1pt] (eq1) -- (eq2);}
            We now apply Hölder's inequality to the two integrals, recall $q=p/(p-1)$:
            \begin{align*}
                \int_X |f| |f+g|^{p-1} \, d\mu &\leq \|f\|_p \left\| |f+g|^{p-1} \right\|_q \\
                \int_X |g| |f+g|^{p-1} \, d\mu &\leq \|g\|_p \left\| |f+g|^{p-1} \right\|_q \\
                \left\| |f+g|^{p-1} \right\|_q & \tikzmarknode{eq1}{=} \left( \int_X |f+g|^{(p-1)q} \, d\mu \right)^{1/q} \\
                & \tikzmarknode{eq2}{=} \left( \int_X |f+g|^p \, d\mu \right)^{1/q} = \|f+g\|_p^{p/q}
            \end{align*}\tikz[overlay,remember picture]{\draw[shorten >=1pt,shorten <=1pt] (eq1) -- (eq2);}
            It thus follows that:
            \begin{align*}
                & \|f+g\|_p^p \leq (\|f\|_p\|+\|g\|_p) \cdot \| |f+g| \|_p^{p/q} \\
                & \implies \|f+g\|_p^{p-p/q=1} \leq \|f\|_p + \|g\|_p\\
                & \implies \|f+g\|_p \leq \|f\|_p + \|g\|_p
            \end{align*}
        \item $p=1$, thanks to the triangular inequality we have:
            \[ \|f+g\|_1 = \int_X |f+g| \, d\mu \leq \int_X |f| \, d\mu + \int_X |g| \, d\mu\]
        \item $p=+\infty$, thanks to the triangular inequality we have:
            \[ \|f+g\|_\infty = \esssup_X |f+g| \leq \esssup_X (|f|+|g|) \leq \esssup_X |f| + \esssup_X |g| \]
    \end{itemize}
\end{proof}

%>=====< Question 15 >=====<%

\question
Show that $L^p$ is a normed space.

\subsection{\texorpdfstring{$L^p$}{Lp} is a normed space}
$L^p$ is a normed space with norm:
\begin{align*}
    & \|f\|_p \coloneqq \left(\int_X |f|^p \, d\mu \right)^{1/p} \quad p\in[1,+\infty) \\
    & \|f\|_\infty \coloneqq \esssup_X |f| 
\end{align*}

\begin{proof}
    Clearly, we have that:
    \begin{itemize}
        \item $\|\cdot\|:L^p\to [0,+\infty)$
        \item $\|f\|_p = 0 \iff f=0$ a.e. in X $\iff f=0$ in $L^p$ thank to its quotientiation with respect to equality a.e.
        \item $\forall \alpha \in \R$ we have:
            \[ \|\alpha f\|_p = |\alpha| \|f\|_p \]
        \item We have the triangular inequality thanks to Minkowski's inequality:
            \[ \|f+g\|_p \leq \|f\|_p + \|g\|_p \]
    \end{itemize}
\end{proof}

%TODO list
% - [x] Question 1
% - [x] Question 2
% - [x] Question 3
% - [x] Question 4
% - [x] Question 5
% - [x] Question 6
% - [x] Question 7
% - [x] Question 8
% - [x] Question 9
% - [x] Question 10
% - [x] Question 11
% - [x] Question 12
% - [x] Question 13
% - [x] Question 14
% - [x] Question 15

\sheet

%>=====< Question 1 >=====<%

\question
Show the inclusion of $L^p$ spaces. Which hypothesis is essential? Justify the answer.

\subsection*{Solution}

\subsection{Inclusion of \texorpdfstring{$L^p$}{Lp} spaces}
Suppose that $\mu(X)<+\infty$, then we have:
\[ 1\leq p \leq q \leq +\infty \implies L^q(X,\A,\mu) \subseteq L^p(X,\A,\mu) \]

\begin{proof}
    The thesis follow if we can show that there exists a constant $C=C(X,p,q)$ such that:
    \[ \|f\|_p = C \|f\|_q \quad \forall f  \in L^q\]
    Let us divide the proof into two cases:
    \begin{itemize}
        \item $q=+\infty$:
            \[ \|f\|_p^p = \int_X |f|^p \, d\mu \leq \|f\|_\infty^p \cdot \mu(X) \]
            So we can make the following trivial choice:
            \[ C = ( \mu(X) )^{1/p} \]
        \item $q\in[1,+\infty)$:
            By applying Hölder's inequality  with $1/r + 1/s = 1$ we can write:
            \begin{align*}
                \|f\|_p^p &\tikzmarknode{eq1}{=} \int_X |f|^p \, d\mu = \int_X 1\cdot|f|^p \, d\mu \\
                & \tikzmarknode{eq2}{\leq} \left( \int_X 1^r \, d\mu \right)^{1/r} \cdot \left( \int_X |f|^{p\cdot s} \, d\mu \right)^{1/s} \\
            \end{align*}\tikz[overlay,remember picture]{\draw[shorten >=1pt,shorten <=1pt] (eq1) -- (eq2);}
            Now, if we take $ps=q$ we have:
            \[ ps=q \implies \frac{1}{s} = \frac{p}{q} \implies \frac{1}{r} = 1 - \frac{1}{s} = \frac{q-p}{q}\]
            hence we write:
            \[ \|f\|_p^p \leq [ \, \mu(X) \, ]^{\frac{q-p}{q}} \cdot \left( \int_X |f|^q \, d\mu \right)^{\frac{p}{q}} \iff \|f\|_p \leq [ \mu(X) ]^{\frac{q-p}{pq}} \cdot \|f\|_q \]
            thus if we choose:
            \[ C = [ \mu(X) ]^{\frac{q-p}{pq}} \]
            the thesis is proved.
    \end{itemize}
\end{proof}
We can observe that the hypothesis that $\mu(X)<+\infty$ is essential, otherwise the two choices of $C$ lose all sense since $\infty$ doesn't obey normal algebraic rules.

%>=====< Question 2 >=====<%

\question
State and prove the interpolation inequality.

\subsection*{Solution}

\subsection{Interpolation inequality}
Let $(X,\A,\mu)$ be a measure space and $1\leq p \leq q \leq +\infty$. If $f\in L^p\cap L^q$, then:
\[ f\in L^r \quad \forall r \in (p,q)\]
Moreover:
\[ \|f\|_r \leq \|f\|_p^\alpha \cdot \|f\|_q^\alpha \]
where $\alpha \in (0,1)$ such that:
\[ \frac{1}{r} = \frac{\alpha}{p} + \frac{1-\alpha}{q}\]
\begin{proof}
    \[\|f\|^r_r = \int_X |f|^r \, d\mu = \int_X \underbrace{ |f|^{\alpha r}}_{\phi} \cdot \underbrace{ |f|^{(1-\alpha)r}}_{\psi} \]
    Now, since $f\in L^p$, we have that:
    \[ \phi\in L^{\frac{p}{\alpha r}} \iff \|\phi\|_{\frac{p}{\alpha r}} = \left( \int_X |f|^{\xcancel{\alpha r} \cdot \frac{p}{\xcancel{\alpha r}}} \, d\mu \right)^{\frac{\alpha r}{p}} < +\infty \]
    and analogously for $\psi$, since $f\in L^q$, we have that:
    \[ \psi\in L^{\frac{q}{(1-\alpha)r}} \iff \|\psi\|_{\frac{q}{(1-\alpha)r}} = \left( \int_X |f|^{\xcancel{(1-\alpha)r} \cdot \frac{q}{\xcancel{(1-\alpha)r}}} \, d\mu \right)^{\frac{(1-\alpha)r}{q}} < +\infty \]
    Now, we take the follwing two constants:
    \[ P \coloneqq \frac{p}{\alpha r} \quad Q \coloneqq \frac{q}{(1-\alpha)r}\]
    We can immediately see that these two are conjugate numbers and we can apply Hölder's inequality:
    \begin{align*} %check this
        & \underbrace{ \int_X |\phi\psi| \, d\mu }_{ \int_X |f|^r \, d\mu } \leq \underbrace{ \left( \int_X |\phi|^P \, d\mu \right)^{1/P} }_{ \left( \int_X |f|^p \, d\mu \right)^{\frac{\alpha r }{p}} } \cdot \underbrace{ \left( \int_X |\psi|^Q \, d\mu \right)^{1/Q} }_{ \left( \int_X |f|^q \, d\mu \right)^{\frac{(1-\alpha)r}{q}} } \\
        & \iff \|f\|_r \leq \|f\|_p^\alpha \cdot \|f\|_q^{(1-\alpha)}
    \end{align*}
    let us note that to arrive at the last coimplication we have elevated both sides to the power of $1/r$.
\end{proof}

%>=====< Question 3 >=====<%

\question
Show the completeness of $L^p$ spaces.

\subsection*{Solution}

\subsection{\texorpdfstring{$L^p$}{Lp} is a Banach space}
$L^p(X,\A,\mu)$ is a Banach space $\forall p\in[1,+\infty]$.

\begin{proof}
    Let $p\in[1,+\infty)$, to prove the thesis it is enough to show that, given $\seq{f}\subset L^p$, if $\sum_{n=1}^\infty \|f_n\|_p$ converges then $\sum_{n=1}^\infty f_n$ converges in $L^p$. This is due to (\ref{Series CriterionForComplete}).\\
    Let us see that this is indeed true, if we define:
    \[ g_k \coloneqq \sum_{n=1}^k |f_n| \]
    thanks to Minkowski's inequality (\ref{minkowski}) we have that:
    \[ \|g_k\|_p \leq \|f_1\|_p + \cdots + \|f_k\|_p \leq M \coloneqq \sum_{n=1}^\infty \|f_n\|_p \]
    Let us now define:
    \[ g \coloneqq \sum_{n=1}^\infty |f_n|\]
    And we have that $\seq{g}$ is an increasing sequence and $g_k$ is measurable $\forall k\in\N$, so $\{|g_n|^p\}$ is also an increasing sequence and $|g_k|^p$ is measurable $\forall k \in \N$.\\
    Therefore we can apply the MCT (\ref{MCT}):
    \begin{align*}
        \lim_{k\to +\infty} \int_X |g_k|^p \, d\mu &\tikzmarknode{eq1}{=} \int_X \lim_{k\to +\infty} |g_k|^p \, d\mu \\
        &\tikzmarknode{eq2}{=} \int_X |g|^p \, d\mu \leq M^p \\
        & \implies g \in L^p \implies g \text{ finite a.e. in } X \\
        & \implies \sum_{n=1}^\infty f_n \text{ conveges (absolutely) a.e. in } X
    \end{align*}\tikz[overlay,remember picture]{\draw[shorten >=1pt,shorten <=1pt] (eq1) -- (eq2);}
    Now let us define the following:
    \begin{align*}
        & s(x) \coloneqq \sum_{n=1}^\infty f_n(x) \\
        & s_k(x) \coloneqq \sum_{n=1}^k f_n(x)
    \end{align*}   
    We know that:
    \begin{align*}
        & s_k(x) \xrightarrow{k\to +\infty} s(x) \quad \text{a.e. in } X \\
        & | s_k - s|^p \leq \left| \sum_{k+1}^\infty f_n \right|^p \leq \left| \sum_{k+1}^\infty |f_n| \right|^p \leq g^p \text{ a.e. in } X \; \forall k \in \N
    \end{align*}
    So we have that:
    \[ s_k\xrightarrow{k\to+\infty} s \iff |s_k-s|^p\xrightarrow{k\to+\infty}0 \]
    and:
    \[ g^p \in L^1 \iff g \in L^p \]
    And by applying the DCT (\ref{DCT}) we have that:
    \[ \lim_{k\to+\infty} \int_X |s_k-s|^p \, d\mu = \int_X \lim_{k\to+\infty} |s_k-s|^p    \, d\mu =0 \iff \sum_{n=1}^\infty f_n \text{ converges in } L^p \]
\end{proof}

%>=====< Question 4 >=====<%

\question
State the Lusin theorem.

\subsection*{Solution}

\subsection{Lusin's theorem}
Let $\Omega \in \Leb(\R)$, $\lambda(\Omega)<+\infty$, $f:\R\to\R$ measurable, such that $f=0$ in $\Omega^c$. Then:
\begin{align*}
    &\forall \epsilon >0 \; \exists g \in C^0_c(\R) \text{ such that }\\
    & \lambda(\{ x\in \R: \; f(x)\neq g(x)\}) < \epsilon \text{ and } \sup_\R |g| \leq \esssup_\R |f|
\end{align*} %check this

%>=====< Question 5 >=====<%

\question
Show that the set of simple functions with support of finite measure is dense in $L^p$ $(p \in [1, +\infty))$.

\subsection*{Solution}

\subsection{Simple functions with support of finite measure}
Let us define the set of simple functions with support of finite measure:
\[ \tilde{\Smes}(\R) \coloneqq \{ s\in\Smes(\R): \; \lambda(\supp(s))<+\infty \} \]

\subsection{\texorpdfstring{$\tilde{\Smes}(\R)$}{The set of simple functions with support of finite measure} is dense in \texorpdfstring{$L^p$}{Lp}}
$\tilde{\Smes}(\R)$ is dense in $L^p$ $\forall p\in[1,+\infty)$.

\begin{proof}
    We have that:
    \[ s\in\tilde{\Smes}(\R) \iff s\in\Smes(\R), \; s\in L^p \quad \forall p \in [1,+\infty) \]
    since: %check punzo's notes
    \[ \|s\|_p^p = \sum_{k=1}^n c_k \cdot \mu(E_k) < +\infty \iff \mu(E_k) < +\infty \; \forall i +1, \dots, n \]
    So we have that $\tilde{\Smes(\R)} \subset L^p$. Let $f\in L^p(\R)$ and suppose that $f\geq0$ a.e. in $\R$, thanks to the SAT (\ref{SAT}) we have that $\exists \seq{s} \subset \Smes(\R)$ such that:
    \begin{align*}
        \seq{s} \uparrow, \; 0 \leq s_n \leq f, \; s_n\to f \text{ a.e. in } \R \\
        \implies \seq{s} \subset L^p \implies \seq{s} \subset \tilde{\Smes}(\R)
    \end{align*}
    thanks to what we have shown above.
    We claim that:
    \[ s_n \to f \text{ in } L^p \iff \|s_n-f\|_p^p \to 0 \iff \lim_{n\to+\infty} \int_\R |s_n-f| \, d\lambda = 0 \]
    Indeed we have that:
    \begin{align*}
        |s_n-f| & \to 0 \text{ a.e in } \R \\
        | f- s_n|^p & \tikzmarknode{eq1}{\leq} (|f| + |s_n|)^p \leq (|f|+|f|)^p \\
        & \tikzmarknode{eq2}{=} 2^p |f|^p = g\in L^1(\R)
    \end{align*}\tikz[overlay,remember picture]{\draw[shorten >=1pt,shorten <=1pt] (eq1) -- (eq2);} %check this
    So the hypothesis of the DCT (\ref{DCT}) are satisfied and we may apply it:
    \[ \implies \lim_{n\to\infty} \int_\R |s_n-f|^p \, d\lambda = \int_\R \cancelnum{0}{\lim_{n\to\infty} |s_n-f|^p} \, d\lambda = 0 \]
    And we get the thesis.
\end{proof}
Let us note that if $f$ is sign-changing the same argument can be applied to its positive and negative parts.

%>=====< Question 6 >=====<%

\question
Show that $C^0_c(\R)$ is dense in $L^p(\R)$ $(p \in [1, +\infty))$.

\subsection*{Solution}

\subsection{\texorpdfstring{$C^0_c$}{The set of continuous functions with compact support} is dense in \texorpdfstring{$L^p$}{Lp}}
$C^0_c(\R)$ is dense in $L^p(\R)$ $(p \in [1, +\infty))$.

\begin{proof}
    Let $f\in L^p(\R)$ and $\epsilon>0$. We can find $s\in\tilde{\Smes}(\R)$ such that:
    \[ \|f-s\| < \epsilon \]
    Now, we apply Lusin's theorem:
    \[ \exists g \in C^0_c(\R) \text{ s.t. } \lambda(\{ g\neq s\}) < \epsilon, \; \|g\|_\infty \leq \|s\|_\infty\]
    which implies:
    \begin{align*}
        \|f-g\|_p & \tikzmarknode{eq1}{\leq} \|f-s\|_p + \|s-g\|_p \\
        & \tikzmarknode{eq2}{=} \epsilon + \left( \int_\R |s-g|^p \, d\lambda \right)^{1/p} \\
        & \tikzmarknode{eq3}{=} \epsilon + \left( \int_{ \{g\neq s\} } |s-g|^p \, d\lambda \right)^{1/p} \\
        & \tikzmarknode{eq4}{\leq} \epsilon + 2 \|s\|_\infty \left( \int_{\{g\neq s\}} \, d\lambda \right)^{1/p} \\
        & \tikzmarknode{eq5}{<} \epsilon + 2 \|s\|_\infty \left( \lambda(\{ g\neq s\}) \right)^{1/p} \\
        & \tikzmarknode{eq6}{\leq} \epsilon + 2 \|s\|_\infty \epsilon^{1/p}
    \end{align*}\tikz[overlay,remember picture]{\draw[shorten >=1pt,shorten <=1pt] (eq1) -- (eq2) -- (eq3) -- (eq4) -- (eq5) -- (eq6);}
\end{proof}

%>=====< Question 7 >=====<%

\question
Show that $L^p(\R)$ is separable $(p \in [1, +\infty))$.

\subsection*{Solution}

\subsection{\texorpdfstring{$L^p$}{Lp} is separable}
Let $\Omega \subseteq \R^n$ be an open set, the set:
\[ L^p(\Omega, \Leb(\Omega), \lambda)\]
is separable $\forall p \in [1,+\infty)$.

\begin{proof}
    Let us assume for simplicity that $\Omega = \R$. Let $f\in L^p(\R)$ and $\epsilon>0$. We know that:
    \[ \exists g \in C^0_c \text{ s.t. } \|f-g\|_p < \epsilon\]
    thanks to the preceding theorem. Furthermore:
    \[ \exists n_0 \in \N \text{ s.t. } \supp(g)\subset[-n_0,n_0]\]
    Since $C^0([-n_0,n_0])$ is separable, there exists a polynomial $\xi$ with rational coefficients such that:
    \[ \|g-\xi\|_{L^\infty([-n_0,n_0])} < \epsilon \]
    Therefore we write:
    \begin{align*}
        \| f - \xi \cdot \chi_{[-n_0,n_0]} \|_p & \tikzmarknode{eq1}{\leq} \| f - g \|_p + \| g - \xi \cdot \chi_{[-n_0,n_0]} \|_p \\
        & \tikzmarknode{eq2}{<} \epsilon + \left( \int_{[-n_0,n_0]} |g-\xi|^p \, d\lambda \right)^{1/p} \\
        & \tikzmarknode{eq3}{<} \epsilon + \|g-\xi\|_\infty \left( \int_{[-n_0,n_0]} \, d\lambda \right)^{1/p} \\
        & \tikzmarknode{eq4}{<} \epsilon + \|g-\xi\|_\infty \cdot (2n_0)^{1/p} \\
        & \tikzmarknode{eq5}{<} \epsilon + \epsilon \cdot (2n_0)^{1/p}
    \end{align*}\tikz[overlay,remember picture]{\draw[shorten >=1pt,shorten <=1pt] (eq1) -- (eq2) -- (eq3) -- (eq4) -- (eq5);}
    And the set of all such polynomials is countable since they have rational coefficients.
\end{proof}

%>=====< Question 8 >=====<%

\question
Show that $L^\infty(\R)$ is not separable.

\subsection*{Solution}

\subsection{Lemma}
Let $X$ be a Banach space. Assume that there exists a family $\{A_i\}_{i\in I} \subseteq X$ such that:
\begin{enumerate}[i)]
    \item $\forall i \in I$ $A_i$ is open
    \item $A_i \cap A_j = \emptyset$ $\forall i \neq j$
    \item $I$ is uncountable
\end{enumerate}
then $X$ is not separable.

%proof (?)

\subsection{\texorpdfstring{$L^\infty$}{L infinity} is not separable}
$L_\infty(\R, \Leb(\R), \lambda)$ is not separable.

\begin{proof}
    Consider the following  uncountable sequence of functions:
    \[ \{\chi_{[-\alpha,\alpha]}\}_{\alpha>0} \subset L^\infty(\R) \]
    If $\alpha\neq\alpha'$, then we have:
    \[ \| \chi_{[-\alpha,\alpha]} - \chi_{[-\alpha',\alpha']} \|_\infty = 1 \]
    Let us now consider the ball centered in $\chi_{[-\alpha,\alpha]}$ of radius $1/2$:
    \[ A_\alpha \coloneqq B_{\chi_{[-\alpha,\alpha]}}\left(\frac{1}{2}\right) \coloneqq \left\{ f\in L^\infty(\R): \; \|\chi_{[-\alpha,\alpha]}-f\|<\frac{1}{2} \right\}\] %check this 
    And we have:
    \[ A_\alpha \cap A_{\alpha'} = \emptyset \quad \alpha \neq \alpha' \]
    and so by the previous lemma $L^\infty(\R)$ is not separable.
\end{proof}

%>=====< Question 9 >=====<%

\question
How $\ell^p$ and $L^p$ are related? What is the inclusion of $\ell^p$ spaces?

\subsection*{Solution}

\subsection{Relation between \texorpdfstring{$\ell^p$}{lp} and \texorpdfstring{$L^p$}{Lp}}
$\ell^p$ is a Banach space $\forall p \in [1,+\infty]$, with the followig norms:
\begin{align*}
    \|x\|_{\ell^p} & \coloneqq \left( \sum_{n=1}^\infty |x^{(k)}|^p \right)^{1/p} \\
    \|x\|_{\ell^p} & \coloneqq \sup_{n\in\N} |x^{(k)}| 
\end{align*}
We can observe that: 
\[ \ell^p = L^p (\N, \Parts{\N}, \mu^\#)\]
where $\mu^\#$ is the counting measure.
The elements of $\ell^p$ are functions of the form:
\[ f:\N\to\R, \; f=\{ x^{(n)} \}_{n\in\N} \]
in other words $\ell^p$ is the space of all sequences of real numbers whose series converges (for $p$ finite) and of all bounded sequences (for $p=\infty$).
Therefore it's quite trivial to observe that the usual norm on $L^p$ (the integral norm) here becomes an infinite sum.
Moreover, just like for $L^p$ we have that:
\begin{itemize}
    \item $\ell^p$ is separable $p\in[1,\infty)$
    \item $\ell^\infty$ is not separable
\end{itemize}

\subsection{Inclusion of \texorpdfstring{$\ell^p$}{lp} spaces}
Since $\mu^\#(\N)=+\infty$ we don't have:
\[ 1 \leq p \leq q \leq \infty  \implies \ell^p \subseteq \ell^p \]
in actuality we have:
\[ q \leq p \implies \ell^q \subseteq \ell^p \]
therefore for $\ell^p$ spaces (contrary to what we have for $L^p$ spaces) we have that $\ell^\infty$ is the largest space (w.r. to inclusion).

%>=====< Question 10 >=====<%

\question
Write the definitions of: linear operator; bounded operator; functional; continuous operator; Lipschitz operator.

\subsection*{Solution}

\subsection{Linear operator}
We say that an operator $T:X\to Y$ is linear if:
\[ T(\alpha v_1 + \beta v_2) = \alpha T(v_1) + \beta T(v_2) \quad \forall v_1, v_2 \in X, \; \forall \alpha, \beta \in \R \]

\subsection{Bounded operator}
We say that an operator $T:X\to Y$ is bounded if:
\[ \exists M > 0: \; \|T(x)\|_Y \leq M \|x\|_X \quad \forall x \in X \]

\subsection{Functional}
We say that an operator $T:X\to \R$ is a functional.

\subsection{Continuous operator}
Let $T:X\to Y$, we say that $T$ is continuous in $x_0 \in X$ if and only if:
\[ \forall \{x_n\} \subset X,\; x_n \xrightarrow{n\to\infty} x_0 \]
we have that:
\[ T(x_n) \xrightarrow{n\to\infty} T(x_0) \]

\subsection{Lipschitz operator}
Let $T:X\to Y$, we say that $T$ is Lipschitz if and only if:
\[ \exists L>0 : \; \|T(x) - T(y)\|_Y \leq L\|x-y\|_X \quad \forall x,y \in X \]

%>=====< Question 11 >=====<%

\question
State and prove the theorem about the characterization of linear, bounded operators.

\subsection*{Solution}

\subsection{Characterization of linear, bounded operators}
Let $T:X\to Y$ be a linear operator, \tfae
\begin{enumerate}
    \item $T$ is bounded
    \item $T$ is continuous at $x_0=0$
    \item $T$ is Lipschitz
\end{enumerate} 

\begin{proof}
    \hspace*{\fill}\\ %leave a blank line
    \begin{itemize}
        \item $(i) \implies (ii)$
            \[ \|T(x)-T(y)\|_Y = \|T(x-y)\|_Y \leq M \|x-y\|_X \]       
        \item $(ii) \implies (iii)$, since we have that:
            \[ \{x_n\}\subset X, \; x_n\xrightarrow{n\to\infty}0 \iff \|x_n\|_X \to 0 \]
            we also get that:
            \[ \|T(x_n)-\cancelnum{0}{T(0)}\|_Y = \|T(x_n)\|_Y \leq M\|x_n\|_X \to 0\]
            see below as to why $T(0)=0$.
        \item $(iii) \implies (i)$, suppose by contradiction that T is not bounded, then there exists $\seq{x}\subset X$, $x_n\neq0$ such that:
            \[ \|T(x_n)\|_Y \geq n\|x_n\|_X \]
            we can define the following sequence:
            \[ \zeta_n \coloneqq \frac{x_n}{n\|x_n\|_X }\xrightarrow{n\to\infty}0\]
            in fact:
            \[ \|\zeta_n\|_X = \frac{1}{n} \xrightarrow{n\to\infty}0 \]
            but:
            \begin{align*}
                T(\zeta_n) = \frac{1}{n\|x_n\|_X} T(x_n) \implies \|T(\zeta_n)\|_Y & \tikzmarknode{eq1}{=} \frac{1}{n\|x_n\|_X} \|T(x_n)\|_Y \\
                & \tikzmarknode{eq2}{\geq} \frac{1}{n\|x_n\|_X} \cdot n \|x_n\|_X = 1\\
            \end{align*}\tikz[overlay,remember picture]{\draw[shorten >=1pt,shorten <=1pt] (eq1) -- (eq2);}
            \newpage %to make tikz work nicely
            thus we can write:
            \[ T(\zeta) \xrightarrow{n\to\infty} T(0)=0 \]
            which means that $T$ is not continuous at $x_0=0$ and this is clearly a contradiction.
    \end{itemize}
\end{proof}

%>=====< Question 12 >=====<%

\question
Let $X, Y$ be normed spaces, $T : X \to Y$ be a linear operator. Prove or disprove the following statement: $T$ is continuous in $X$ if and only if it is continuous at $x_0 = 0$.

\subsection*{Solution}

\subsection{An operator is continuous if and only if it is continuous at \texorpdfstring{$x_0=0$}{x0=0}}
Since the operator is linear, we have that:
\[ T(0)=T(0\cdot x)=0\cdot T(x)=\underline{0} \]
thus thanks to linearity if $T$ is continuous at $x_0=0$ it is continuous for every $x$, indeed if we have:
\[ \|T(0)-T(x_n)\| \xrightarrow{x_n\to 0} 0 \]
we can define:
\[ y = y+0, \; y_n = y + x_n \to y \]
from which we get:
\begin{align*}
    \implies \|T(y)-T(y_n)\| & = \| T(0+y) - T(y+x_n) \| \\
    & = \| T(0)+T(y)-T(y)-T(x_n) \| = \| T(0)-T(x_n) \| \xrightarrow{x_n\to 0} 0
\end{align*}
and, since we have equal signs all the way through, we can navigate through the proof in the opposite sense. So we have the coimplication and the equivalence: $T$ continuous in $0$ if and only if $T$ is continuous in $X$.

%TODO list
% - [x] Question 1
% - [x] Question 2
% - [x] Question 3
% - [x] Question 4
% - [x] Question 5
% - [x] Question 6
% - [x] Question 7
% - [ ] Question 8: I could add the proof of the lemma although it isn't strictly required
% - [x] Question 9
% - [x] Question 10
% - [x] Question 11
% - [x] Question 12

\sheet

%>=====< Question 1 >=====<%

\question
What is the norm on $\Leb(X,Y)$? It satisfies two important equalities. Write and show them.

\subsection*{Solution}

\subsection{Norm on \texorpdfstring{$\Leb(X,Y)$}{the space of continuous operators}}
We can define the norm on $\Leb(X,Y)$ as follows:
\[ \| T \|_{\Leb} = \sup_{x\in X, \; \|x\|_X\leq1} \|T(x)\|_Y \]

%what are the conditions he is talking about

%>=====< Question 2 >=====<%

\question
Under which hypotheses on $X$ and $Y$ is $\Leb(X,Y)$ a Banach space?

\subsection*{Solution}

\subsection{When \texorpdfstring{$\Leb$}{the space of continuous operators} is a Banach space}
If $X$ is a normed space and $Y$ a Banach space then $\Leb(X,Y)$ is a Banach space.

%>=====< Question 3 >=====<%

\question
Let $\mathfrak{F} \subset \Leb(X,Y)$. For $\mathfrak{F}$ write the definition of pointwise and uniform boundedness. State and prove the UBP (or BS theorem).

\subsection*{Solution}

\subsection{Pointwise boundedness}
A family of operators $\mathfrak{F} \subset \Leb$ is pointwise bounded if and only if:
\[ \forall x \in X \; \exists M_x > 0: \quad \sup_{T\in\mathfrak{F}} \|T(x)\|_Y < M_x \]

\subsection{Uniform boundedness}
A family of operators $\mathfrak{F} \subset \Leb$ is uniformly bounded if and only if:
\[ \forall x \in X \; \exists M > 0: \quad \sup_{T\in\mathfrak{F}} \|T\|_{\Leb} < M \]
which is equal to writing:
\[ \forall x \in X \; \exists M > 0: \quad \sup_{T\in\mathfrak{F}} \sup_{\|x\|\leq1} < M \]

\subsection{Uniform boundedness Principle}
Let $X$,$Y$ be two Banach spaces, $\mathfrak{F}\subset\Leb(X,Y)$. If $\mathfrak{F}$ is pointwise bounded, then $\mathfrak{F}$ is also uniformly bounded.

\begin{proof}
    For all $n\in\N$ we define:
    \[ C_n \coloneqq \{ x\in X: \; \|T(x)\|_Y\leq n \; \forall T \in \mathfrak{F} \} \]
    Let us observe that $C_n$ is closed for all $n$. In fact, let $\seq{x}\subset C_n$ such that $x_n \xrightarrow{n\to\infty} x_0 \in X$, by continuity we have:
    \[ T(x_n) \xrightarrow{n\to\infty} T(x_0) \; \forall T \in \mathfrak{F} \implies \|T(x_n)\|_Y \xrightarrow{n\to\infty} \|T(x_0)\|_Y \]
    but since each $x_n \in C_n$ we also have:
    \begin{align*}
        & 0 \leq \|T(x_n)\|_Y \leq n \; \forall n \in \N \implies \|T(x_0)\|_Y \leq n \\
        & \implies x_0 \in C_n \implies X = \bigcup_{n=1}^\infty C_n
    \end{align*}
    Now we apply Baire's Theorem (\ref{Baire}) to write:
    \[ \exists n_0 \in \N \text{ such that } \int(C_{n_0}) \neq \emptyset \implies \exists \bar{B}_\epsilon(x_0) \subset C_{n_0} \]
    And so we have that:
    \[ \text{If } \|z\|_X \leq \epsilon \implies z+x_0 \in \bar{B}_\epsilon(x_0) \subset C_{n_0} \]
    \begin{align*}
        \implies \|T(z)\|_Y & \tikzmarknode{eq1}{=} \|T(z)+T(x_0)-T(x_0)\|_Y \\
        & \tikzmarknode{eq2}{\leq} \|T(z)+T(x_0)\|_Y + \|T(x_0)\|_Y \\
        & \tikzmarknode{eq3}{=} \|T(\underbrace{z+x_0}_{\in C_{n_0}})\|_Y + \|T(\underbrace{x_0}_{\in C_{n_0}})\|_Y \leq 2 n_0 \; \forall T \in \mathfrak{F}
    \end{align*}
\end{proof}\tikz[overlay,remember picture]{\draw[shorten >=1pt,shorten <=1pt] (eq1) -- (eq2) -- (eq3);}

%>=====< Question 4 >=====<%

\question
From the UBP it is possibile to infer an important property of operators defined by means of a point-wise limit. What is that? Justify your answer.

\subsection*{Solution}

%>=====< Question 5 >=====<%

\question
Write the definition of open mapping. Let $f: \R^n \to \R^m$ be continuous; under which extra hypotheses is $f$ open? Let $T: \R^n \to \R^n$ be linear and onto; why is $T$ open? State the OMT.

\subsection*{Solution}

%>=====< Question 6 >=====<%

\question
State and prove the IBM (or ICM) theorem. What is the analogous result in the finite dimensional case?

\subsection*{Solution}

%>=====< Question 7 >=====<%

\question
By the IBM theorem we can infer an important property about equivalent norms on Banach spaces. What is that? Justify your answer.

\subsection*{Solution}

%>=====< Question 8 >=====<%

\question
Write the definitions of: closed operator; graph of an operator. Show that an operator is linear and closed iff its graph is closed.

\subsection*{Solution}

%>=====< Question 9 >=====<%

\question
State and prove the closed graph theorem.

\subsection*{Solution}

%>=====< Question 10 >=====<%

\question
Write the definition of dual space.

\subsection*{Solution}

%>=====< Question 11 >=====<%

\question
Exhibit an example of $T \in (L^p)^{*}$. Compute $\|T\|_{*}$.

\subsection*{Solution}

%>=====< Question 12 >=====<%

\question
Let $(V, \langle \cdot, \cdot \rangle)$ be a finite dimensional vector space endowed with an inner product. Characterize $V^{*}$.

\subsection*{Solution}

%>=====< Question 13 >=====<%

\question
Let $Y$ be a vector subspace of $X = \R^2$. Given $f \in Y^{*}$ is it possibile to find $F \in X^{*}$ such that $F = f$ in $Y$ and $\|F\|_{X^{*}} = \|f\|_{Y^{*}}$ ?

\subsection*{Solution}

%>=====< Question 14 >=====<%

\question
State the Hahn-Banach theorem in the continuous extension form.

\subsection*{Solution}

%>=====< Question 15 >=====<%

\question
Let $X$ be a normed space, let $H$ be a hyperplane of $X$. When do we say that $H$ separates (or strictly separates) $A$ and $B$? If $X = \R^2$ , under which hypotheses on $A$ and $B$ is it possibile to find a line $H$ which separates them?

\subsection*{Solution}

%>=====< Question 16 >=====<%

\question
State the Hahn-Banach theorem in the separation form (first and second version).

\subsection*{Solution}

\sheet

%>=====< Question 1 >=====<%

\question
State and prove the three corollaries of the Hanh-Banach Theorem.

\subsection*{Solution}

\subsection{First corollary of the Hahn-Banach theorem}
Let $x_0 \in X \backslash \{0\}$, $X$ normed space, then $\exists L_{x_0} \in X^* \mbox{ s.t.: }$ \[
\|L_{x_0}\|_{X^*} =1,\quad L_{x_0}(x_0) = \|x_0 \|
\]

\begin{proof}
Let $Y\coloneqq \{\lambda x_0 : \lambda \in \R\}=\text{Span}(x_0)$, $Y$ is a v.s.s. of $X$.
\[
L_0:Y\to \R ,\quad L(\lambda x_0) = \lambda \|x_0\|,\quad L_0 \in \Leb
\]
By H-B theorem: \[
\exists \Tilde{L}_0: X \to \R , \quad \Tilde{L}_0 \in X^*, \quad \|\Tilde{L}_0\|_{X^*}= \sup_S |\underbrace{\lambda \|x_0\|}_{=L_0(\lambda x_0)}| = 1 
\]
where $S = \{\lambda x_0 : \lambda x_0 \in Y , \| \lambda x_0\| = 1\}$, moreover:
\[
\Tilde{L}_0 = L_0(x_0) = \|x_0\| \implies L_{x_0} \coloneqq \Tilde{L}_0 \text{ satisfies the required properties }
\]
\end{proof}

\subsection{Second corollary of the Hahn-Banach theorem}
Let $y,z \in X$, if $L(y)=L(z),\forall L \in X^*$ then $ y=z$.

\begin{proof}
Suppose by contradiction that $\exists y,z \in X, y\neq z$ s.t. $L(y)=L(z)\quad\forall L \in X^*$, then:
\[
x \coloneqq y-z \neq 0 \implies L(x)=L(y-z)=L(y)-L(z)=0\quad\forall L \in X^*
\]
and by the preceding corollary:
\[
\exists L_x \in X^* \mbox{ s.t. } L(x) = \|x\| \neq 0, \mbox{ contradiction}
\]
\end{proof}

\subsection{Third corollary of the Hahn-Banach theorem}
Let $Y \subseteq X$ be a v.s.s. with $\Bar{Y}\neq X$, $x_0 \in X\backslash \Bar{Y}.$ Then:
\[
\exists L \in X^*: L(x_0)\neq 0,\quad L\arrowvert_Y=0
\]

\begin{proof}
Let $Z \coloneqq \{\lambda x_0 + y:\,y\in Y,\,\lambda \in R\} \subset X$ and define:
\[
L_0: Z\to\R,\quad L_0(\lambda x_0 + y) \coloneqq \lambda
\]
Notice that $L_0(x_0) = L_0(1\cdot x_0+0\cdot y) = 1$ and:
\[
\ker(L_0)= \{\lambda x_0 + y \in Z :\,\underbrace{L_0(\lambda x_0 + y)=0}_{\lambda=0}=Y\} \implies L\arrowvert_Y=0
\]
By H.-B. theorem: 
\[
\exists \Tilde{L}_0 \in X^*: \Tilde{L}_0 = L_0\text{ in }Z \supseteq Y \implies L \coloneqq \Tilde{L}_0 \text{ satisfies } L(x_0)  \neq 0,\;  L\arrowvert_Y=0
\]
\end{proof}

%>=====< Question 2 >=====<%

\question
Give a sufficient condition for separability of $X$.

\subsection*{Solution}

\subsection{Sufficient condition for separability}
Let $X$ be a normed space:\[
X^* \text{ separable } \implies X \text{ separable }
\]

%>=====< Question >=====<%

\question
Write the definition of uniformly convex normed space. What is the geometric interpretation?

\subsection*{Solution}

\subsection{Uniform convexity}
Let $X$ be a normed spcae.
$X$ is uniformly convex if $\forall \epsilon >0 \;\exists \,\delta$ s.t.:\[
\forall x,y\in X,\;\|x\|\le 1,\;\|y\|\le 1,\;\|x-y\|\ge \epsilon \implies \|\tfrac{x-y}{2}\|\le 1-\delta \\[0pt]
\]We can think of it as two different points on the unit sphere imply that the center of the line segment that connects them has to lie inside the unit sphere.

%>=====< Question >=====<%

\question
Write the Clarkson's inequalities in $L^p$. Show that $L^p$ is uniformly convex for any $ p \in(1, \infty)$.

\subsection*{Solution}

\subsection{Clarkson's inequality}
Given $f,g \in L^p(\Omega)$, $\Omega \in \Leb(R^N)$, we have:
\newline\newline
$\begin{array}{rl}
p \ge 2: &     \qquad\|\frac{f+g}{2}\|_p^p + \|\frac{f-g}{2}\|_p^p  \le \frac12 \big (\|f\|_p^p + \|g\|_p^p \big)\\[4pt]
p \in (1,2): & \qquad\|\tfrac{f+g}{2}\|_p^q + \|\tfrac{f-g}{2}\|_p^q  \le \big (\frac12\|f\|_p^p + \frac12\|g\|_p^p \big)^{\frac1{p-1}}
\end{array}$
\subsection{\texorpdfstring{$L^p(\Omega)$}{Lp on Omega} is uniformly convex}
\begin{proof}
Take $\epsilon>0,\;f,g \in L^p(\Omega),\;\|f\|\le 1,\;\|g\|\le 1,\;\|f-g\|\ge \epsilon$, we have:$\\[4pt]$
$p\ge 2$:
\[
\big\|\frac{f+g}{2}\big\|_p^p < 1- \big(\frac{\epsilon}{2}\big)^p
\]
\[
\implies  \big\|\frac{f+g}{2}\big\|_p < 1 - \delta,\quad \delta \coloneqq 1-\big[1-\big( \tfrac{\epsilon}{2}\big)^p\big]^{\tfrac1p}
\]
$1<p<2$:
\[
\big\|\frac{f+g}{2}\big\|_p^q = 1- \big(\frac{\epsilon}{2}\big)^q
\]
\[
\implies \big\|\frac{f+g}{2}\big\|_p < 1 - \delta,\quad \delta \coloneqq 1-\big[1-\big( \tfrac{\epsilon}{2}\big)^q\big]^{\tfrac1q}
\]
\end{proof}

%>=====< Question >=====<%

\question
Write the definition of bidual space. Introduce the canonical (or evaluation) map. Show that it is linear and that it preserves the norm.

\subsection*{Solution}

\subsection{Bidual space}
Let $X $ be a normed space and $X^* $ its dual. We defie the bidual of $X$ as:
\[
X^{**} \coloneqq (X^*)^*
\]

\subsection{Canonical map}
$\forall x \in X$ we define $\Lambda _x: X^* \to \R$ as: 
\[\Lambda _x(L) \coloneqq L(x),\quad \forall L \in X^*\]
\[
\Lambda_x \in \Leb :\quad |\Lambda _x(L)| = |L(x)| \le \|L\|_{X^*} \underbrace{\|x\|_X}_{\le M}
\]
\[
\implies \Lambda _x \in X^{**},\quad \|\Lambda _x\|_{X^{**}} \le \|x\|_X
\]
Finally we can define the canonical map as
\[ \tau: X \to X^{**}, \quad \tau(x) = \Lambda _x,\quad \forall x \in X
\]

\subsection{Properties of the canonical map}

\begin{itemize}
    \item [i)] $\tau$ is linear
    \item [ii)] $\|\tau(x)\|_{X^{**}}=\|x\|_X$
    \item [iii)] $\tau(x)$ is injective
\end{itemize}

\begin{proof} $\\$
Linearity is trivial and the injectivity comes directly from ii) which is the only one we are proving, clearly: \[\|x\|_X \ge \|\tau(x)\|_{X^{**}},\;\forall x \in X\\ \]
By a corollary of the H-B theorem:
\[
\forall x \in X\backslash\{0\}\; \exists L \in X^*\mbox{ s.t. }\|L\|_{X^{**}} = \|x\|_X
\]
and thus:
\[
\|\tau(x)\|_{X^{**}} = \|\Lambda \|_{X^{**}}=\displaystyle\sup_{\|x\|_X=1} |\Lambda _x(L)| \ge \|x\|_X
\]
\end{proof}

%>=====< Question >=====<%

\question
Write the definition of reflexive space. Let $X$ be a reflexive space and $\phi\in X^{**}$. What about $\phi(L)$ for any $L\in X^*$?

\subsection*{Solution}

\subsection{Reflexive space}
$X$ is said to be reflexive if $\tau(X) = X^{**} .$

\subsection{Characterization of reflexive spaces}
\[X \text{ reflexive } \iff \forall \phi \in X^{**},\;\phi(L)=L(x),\;\forall L\in X^*
\] 

%>=====< Question >=====<%

\question
State the Milman-Pettis theorem. Show that $L^p$ is reflexive for any $p \in (1,+\infty)$. What about $L^1 $ and $L^\infty$?

\subsection*{Solution}

\subsection{Milman-Pettis theorem}
Let $X$ be an uniformly convex Banach space, then $X$ is reflexive.
\subsection{Reflexivity of \texorpdfstring{$L^p$}{Lp}}
$L^p$ is reflexive $\forall p \in (1,+\infty)$.

\begin{proof}
This follows immediately by Milman-Pettis theorem since for $ p \in (1,+\infty)$, $L^p$ is uniformly convex.\newline
Moreover, it can be proved that $L^1 $ and $L^\infty$ are not reflexive.
\end{proof}

%>=====< Question >=====<%

\question
State and prove the Riesz theorem in $L^p$ spaces.

\subsection*{Solution}

\subsection{Riesz theorem in \texorpdfstring{$L^p$}{Lp}}
Let $(X,\A,\mu) $ be a complete measure space, $p\in (1,+\infty)$. For any $\Lambda \in (L^p)^*,\,\exists! g \in L^q $ s.t.:
\[
\Lambda(f)  = \int_X fg\,d\mu \quad \forall f \in L^p
\]
Furthermore: 
\[
\|\Lambda \|_{(L^p)^*} = \|g\|_{L^q}
\]
\begin{proof}
Let $p \in (1,+\infty),$ we define: \[
T: L^q \to (L^p)^*\]
\[ u\mapsto T(u): L^p \to \R \]
\[ [T(u)](f) \coloneqq \displaystyle\int_{\Omega} fu\,d\lambda\]
Now note that our thesis is equivalent to showing that $T$ is surjective, and to prove the latter let:\[
E\coloneqq T(L^q)\subseteq (L^p)^*, \,E\text{ is closed}
\]
and let
\[
H \in (L^p)^{**} \quad(\iff H\in \Leb : (L^p)^* \to \R)
\]
be such that:   
\[
H\arrowvert_E = 0 \quad (\iff H(T(u))=0 \;\,\forall u \in L^q)
\]
\[
\iff H(T) = T(h) \text{ (since $L^p$ is reflexive)}, \,\,h\coloneqq \tau^{-1} (H) \in L^p
\]
\[
\implies \forall u \in L^q,\; \int_\Omega hu\,d\lambda = 0
\]
Let $u \coloneqq |h|^{p-2}h,\;u\in L^q$ since $h \in L^p$
\[
\implies \int_\Omega |h|^p \,d\lambda = 0 \implies h\equiv 0 \text{ in } L^p \underset{H=\tau(0)}{\Longrightarrow} H=0 \text{ in } (L^q)^* \implies \Bar{E} = E = (L^p)^*
\]


\end{proof}
We can also prove that the same thesis holds for $p=1, q=\infty$, provided that $\mu$ is $\sigma$-finite.

%>=====< Question >=====<%

\question
Show that the dual of $L^\infty$ is "strictly bigger" than $L^1$.

\subsection*{Solution}

\subsection{\texorpdfstring{$(L^\infty)^* \supsetneq L^1$}{The dual of L infinity is strictly bigger than L1}}
$\exists L \in (L^\infty)^* $ s.t. $L $ is not in the form $Lg$ with $g\in L^1$.
\begin{proof}
Indeed consider $L_0 \in (C_c^0 (\R^N))^*$, we have 
 that $\big(C_c^0 (\R^N),\|\cdot\|_\infty\big)$ is a subset of $L^\infty$.
 \[L_0(f)\coloneqq f(0),\quad \forall f \in C_c^0 (\R^N)\]
Clearly $L_0$ is bounded, moreover:
 \[
 |L_0(f)| = |f(0)|  \le \|f\|_\infty\quad \forall f \in C_c^0 (\R^N) \implies L_0 \text{ is bounded }
 \]
\[
\overset{\text{H.B.}}{\Longrightarrow} \exists L \in [L^\infty(\R^N)]^* \text{ extension of } L_0
\]
Now we claim that:
\[
\nexists g \in L^1 \mbox{ s.t. } L(f)=\int_{\R^N}fg\,d\lambda,\quad \forall f \in L^\infty
\]    
Suppose by contradiction that such a $g$ exists, then:
\[
L(f)=L_0(f) = \int_{\R^N}fg\,d\lambda = f(0) = 0 
\]
\[
f \in C_c^0,\,f(0)=0 \implies g =0 \text{ a.e. in } \R^N
\]
\[
\implies L(f) = \int_{\R^N}0\cdot f\,d\lambda = 0\quad \forall f \in L^\infty
\]
\[
 f \in C_c^0 (\R^N),\; f(0) \neq 0 \implies L(f)=f(0)\neq 0 \implies \text{ contradiction}
\]

\end{proof}

%>=====< Question >=====<%

\question
Write the definition of weak convergence. How can it be formulated in $L^p$ and in $\ell^p$?

\subsection*{Solution}

\subsection{Weak convergence}
Let $X $ be a Banach space, $\{x_n\} \subset X$, $x\in X$, we say that $x_n \wc x$ if $L(x_n) \to L(x) \quad \forall L\in X^*$

\subsection{Weak convergence in \texorpdfstring{$L^p$}{Lp}}
In $L^p(\Omega),\text{ with }p\in[1,+\infty) $ we have that:
\[
f_n \wc f \iff \int_\Omega f_n g\,d\lambda \xrightarrow{\toi} \int_\Omega f g\,d\lambda  \quad\forall g \in L^q \overset{p\neq 1}{\iff} \int_\Omega f_n \phi\,d\lambda \xrightarrow{\toi} \int_\Omega f \phi\,d\lambda \quad\forall \phi \in C^1_c(\Omega) 
\]

\subsection{Weak convergence in \texorpdfstring{$ell^p$}{lp}}
In $\ell^p(\Omega),\text{ with }p\in[1,+\infty) $ we have that:
\[x_n \wc x \iff \sum_{k=1}^\infty x_n^{(k)} y^{(k)} \xrightarrow{\toi} \sum_{k=1}^\infty x^{(k)} y^{(k)}  \quad\forall y \in \ell^q ) \]

%>=====< Question >=====<%

\question
Show that strong convergence implies weak convergence. Provide a counterexample for the
converse implication.

\subsection*{Solution}

\subsection{Strong convergence implies weak convergence}
$x_n \to x \implies x_n \wc x$
\begin{proof}
   $\forall L \in X^*:$
    \[
    |L(x_n)-L(x)| = |L(x_n-x)| \le \|L\|_{X^*}\overbrace{\|x_n-x\|}^{\to 0}
    \]
    \[
    \implies L(x_n) \xrightarrow{\toi} L(x)
    \]
\end{proof}\noindent
We claim that the converse is not true, for instance, let $X=\ell^2,\; \{e_n\}\subset\ell^2,\;e_n^{(k)}=\delta_{k,n}$
\[
e_n \wc 0 \iff T(e_n) \to T(0) = 0\quad \forall T \in (\ell^2)^*
\]
\[
\forall x \equiv x^{(j)} \in \ell^2,\; \sum e^{(j)} x^{(j) }= x^(n) \to 0 \implies e_n \wc 0
\]
but:
\[
\|e_n\|_2 = 1 \quad \forall n \in \N \implies e_n \nrightarrow 0
\]

\sheet

%>=====< Question 1 >=====<%

\question
Show that the weak limit is unique.

\subsection*{Solution}

\subsection{Weak limit uniqueness}
The weak limit is unique.

\begin{proof}
    Suppose by contradiction that $\exists x_1,x_2 \mbox{ s.t. }x_n\wc x_1,\,\,x_n\wc x_2$.
    \[
    \implies \forall L \in X^*:\, \lvert L(x_n)-L(x_1)\rvert \to 0 \mbox{ and } \lvert L(x_n)-L(x_1)\rvert\to 0
    \]
    \[
    \implies L(x_1)-L(x_2)\,\,\, \forall L\in X^* \implies x_1=x_2 \mbox{, contradiction}
    \]
\end{proof}

%>=====< Question 2 >=====<%

\question
If $\seq{x}$ weakly converges to $x$, can $\seq{x}$ be unbounded? State and prove lower semicontinuity w.r.t. weak convergence of $x\mapsto\|x\|$.

\subsection*{Solution}

\subsection{Boundedness of weak convergent series}
If $x_n \wc x$ then $\{x_n\}$ is bounded.
\subsection{Lower semicontinuity w.r.t. weak convergence}
$x_n \wc x \,\,\implies \displaystyle\liminf_{n\to \infty}\|x_n\|_X \ge \|x\|$
\begin{proof}
    Let $x\in X\backslash \{0\},\,\, L \in X^*:\|L\|_X=1,\,\,L(x)=\|x\|$, we have:
    \[
    0<\|x\|=L(x)=\displaystyle\lim_{\toi}L(x_n)=\displaystyle\lim_{\toi}\lvert L(x_n)\rvert
    \]
    On the other hand: \[
    \lvert L(x_n)\rvert \le \underbrace{\|L\|_X}_{=1} \|x_n\|_X = \|x_n\|_X
    \]
    \[
    \implies \|x\|_X = \lim_{\toi}\lvert L(x_n)\rvert \le \displaystyle\liminf_{n\to \infty}\|x_n\|_X 
    \]
\end{proof}

%>=====< Question 3 >=====<%

\question
Show that if $x_n \wc x$ and $L_n \to L$ in $X^*$, then $L_n(x_n) \to L(x)$ as $\toi$.

\subsection*{Solution}

\begin{proof}
    \[
    \begin{array}{rll}
       \lvert L_n(x_n)-L(x)\rvert =  & \lvert L_n(x_n)-L(x_n)+L(x_n)-L(x)\rvert &\le     \\
         \le&\lvert L_n(x_n)-L(x_n)\rvert +\lvert L(x_n)-L(x)\rvert &\le\\
        \le & \underbrace{\|L_n-L\|_X}_{\to 0}\underbrace{\|x_n\|_X}_{\le M}+\underbrace{\lvert L(x_n)-L(x)\rvert}_{\to 0} &\to 0
    \end{array}
    \]
\end{proof}

%>=====< Question 4 >=====<%

\question
Show that $T \in \Leb(X, Y )$ is weak-weak continuous.

\subsection*{Solution}

\subsection{Weak-weak continuity}
Let $X,Y$ be Banach spaces and $T \in L(X, Y)$. Then: $x_n \wc x \implies T(x_n) \wc T(x)$.
\begin{proof}
    Let $L\in Y^*$, $\Lambda : X\to \R \mbox{ s.t. } \Lambda(x)=L(T(x)),\,\,\forall x\in X$
    \[ \begin{array}{rrcl}
       \Lambda \in X^* \implies &\Lambda (x_n)  & \to & \Lambda(x)\\
       \implies& L(T(x_n)) & \to & L(T(x))\\
         \iff& T(x_n) & \wc & T(x)
   \end{array}\]
\end{proof}

%>=====< Question 5 >=====<%

\question
the definition of weak* convergence.

\subsection*{Solution}

\subsection{Weak* convergence}
We say that $\{L_n\}\subset X^*$ converges weakly* to $L\in X^*$ whenever $L_n(x)\to L(x) ,\,\, \forall x\in X$. We write: $L_n \wsc L $.

%>=====< Question 6 >=====<%

\question
Which is the relation between weak and weak* convergence? Justify your answer

\subsection*{Solution}

\subsection{Relation between weak and weak* convergence}
$L_n \wc L \implies L_n \wsc L$. If $X$ is reflexive the converse implication is also true.

\begin{proof}
\[
\begin{array}{rcccl}
L_n \wc L \in X^{ *}\iff & \Lambda (L_n)  & \to & \Lambda(L) & \forall \Lambda\in X^{**} \\
\implies & \Lambda (L_n) & \to & \Lambda(L) &  \forall \Lambda\in \tau(X) \subseteq X^{**}\\
\iff & L(x_n) & \wc & L(x) & \forall x \in X \\
\end{array}\]
    If $X$ is reflexive, $\tau(X)=X^{**}$ and the converse immediately follows from what we demonstrated above.
\end{proof}

%>=====< Question 7 >=====<%

\question
Write the properties of weak* convergence.

\subsection*{Solution}

\subsection{Properties of weak* convergence}
Let $X$ be a Banach space, then:
\begin{itemize}
    \item[i)] Weak* limit is unique.
    \item[ii)] $L_n \wsc L \implies \{L_n\}$ bounded in $X^*$.
    \item[iii)] $L\mapsto \|L\|_{X^*}$ is lower semicontinuous w.r.t. weak* convergence.
    \item[iv)] $L_n \wsc L,\,\,x_n \wsc x \,\implies L_n(x_n)\to L(x)$.
\end{itemize}

%>=====< Question 8 >=====<%

\question
State the Banach-Alaoglu theorem. Why can we say that from a bounded sequence in $\Leb^{\infty}$ we can extract a subsequence which weakly* converges in $\Leb^{\infty}$?

\subsection*{Solution}

\subsection{Banach-Alaoglu theorem}
Let $X$ be a separable Banach space. Then any bounded sequance $\{L_n\} \subset X^*$ admits a subsequence that weakly* converges to some $L \in X^*$.

\subsection{Bounded sequences in \texorpdfstring{$\Leb^{\infty}$}{Linf}}
Let $\{f_n\} \subset \Leb^{\infty}(\Omega)$ bounded, $L_n: \Leb^1(\Omega)\mapsto \R$.\\
\[
L_n \in \big[\Leb^1(\Omega)\big]^* \,\,\,\,\forall n\in \N, \,\,\,\,L_n(g)\coloneqq \displaystyle\int_{\Omega}f_ng d\lambda\,\,\,\forall g \in \Leb^1(\Omega) \implies \{L_n\}\mbox{ is bounded in } \big[\Leb^1(\Omega)\big]^*:
\]
\[\lvert L_n(g)\rvert \le \|f_n\|_{\infty} \|g\|_1 \le c\|g\|_1 \implies \|L_n\|_{(\Leb^1)^*}\le c\,\,\,\forall n\in \N\\
\]
By B.-A. theorem:
$\exists \{L_{n_h}\}\subset \{L_n\}\mbox{ s.t. }L_{n_h}\underset{h\to\infty}{\wsc}  L \mbox{ for some } L\in (\Leb^1)^* $
$\iff L_{n_h}(g)\underset{h\to\infty}{\to} L(g)\,\, \forall g \in \Leb^1 $
\[
 \implies \exists ! f \in \Leb^{\infty} \mbox{ s.t. } L(g) = \displaystyle\int_{\Omega}fg d\lambda\,\,\,\forall g \in \Leb^1(\Omega)
\]
Therefore, $\{f_n\}$ bounded in $\Leb^{\infty}$ possesses a subsequence $\{f_{n_h}\}$ which weakly* converges to some $f\in\Leb^{\infty} $.

%>=====< Question 9 >=====<%

\question
State and prove the corollary of the Banach-Alaoglu theorem in a separable and reflexive Banach space.

\subsection*{Solution}

\subsection{Corollary of the Banach-Alaoglu theorem}
Let $X$ be a separable and reflexive Banach space. Then every bounded sequence $\{x_n\} \subset X$ has a subsequence which weakly converges.
\begin{proof}
    $
    \\\\
    \begin{aligned}
        \begin{cases}
            X^* \text{ separable} \\
            \{\tau(x_n)\} \subset X^{**} \text{ bounded} 
        \end{cases}
    \end{aligned}$ $\implies $ by B.-A. theorem for $X^{**}$  we have: \[
    \begin{array}{lrclc}
        & \exists \{\tau(t_{n_h})\} \mbox{ s.t. } \tau(t_{n_h}) & \underset{h\to\infty}{\wsc} & \Lambda & \mbox{for some } \Lambda \in X^{**} \\
        \iff & \big[\tau(x_{n_h})\big](f) & \underset{h\to\infty}{\to} & \Lambda(f) & \forall f \in X^{**} \\
        \iff & f(x_{n_h}) & \underset{h\to\infty}{\to} & f(x) & x\coloneqq\tau^{-1}(\Lambda) \\
        \iff & x_{n_h} & \underset{h\to\infty}{\wc} & x
    \end{array}
    \]    
\end{proof}

%>=====< Question 10 >=====<%

\question
State the Eberlein-Smulyan theorem.

\subsection*{Solution}

\subsection{Eberlein-Smulyan theorem}
Let $X$ be a Banach space. If any bounded sequence contains a weakly convergent subsequence, $X$ is reflexive.

%>=====< Question  11 >=====<%

\question
the definition of compact operator. Write the definition of operator of finite rank.

\subsection*{Solution}

\subsection{Compact operator}
$K:X\to Y$ is said to be compact if $\forall E \subseteq X$ bounded, $\overline{K(E)}$ is compact (i.e. $K(E)$ is relatively compact).
\subsection{Finite rank operator}
$T \in \Leb(X,Y) $ has finite rank if $\dim(Im(T)) <\infty$.

%>=====< Question 12 >=====<%

\question
How is a compact operator related to operators of finite rank? Can a compact operator be surjective?

\subsection*{Solution}

\subsection{Relation between compact and finite rank operators}
$T \in \Leb(X,Y)$,\,\, $T$ has finite rank $\implies T$ is compact.

\subsection{Surjectivity of compact operator}
Let $K \in \Leb(X,Y)\mbox{, with }\dim Y=+\infty$. If $K$ is compact, it can not be surjective.

\begin{proof}
Suppose $K $ is not surjective. 
Notice that: \[\\ \begin{array}{lr}
    &  0 \in B_1(0) \subset X  \\
     \implies &K(B_1(0))\subset Y\mbox{ is open}  \\
  \implies   & \exists \,\delta>0 \mbox{ s.t. } B_{\delta}(0) \subseteq K(B_1(0))\\
  \implies&\overline{B_{\delta}(0)} \subseteq \overline{K(B_{1}(0))}\\
  \implies & \overline{B_{\delta}(0)}\mbox{ is compact}
\end{array}\]
\end{proof}

%>=====< Question 13 >=====<%

\question
the theorem about the characterization of compact operators.

\subsection*{Solution}

\subsection{Characterization of compact operators}
\begin{itemize}
    \item[i)] If $K\in K(X,Y)$ then $x_n \wc x \implies K(x_n)\to K(x)$ 
    \item[ii)] If the previous implication holds and $K\in \Leb(X,Y)$, then $K\in K(X,Y)$.
\end{itemize}

%>=====< Question 14 >=====<%

\question
Write the definition of pre-Hilbert and of Hilbert spaces. Write the parallelogram law.

\subsection*{Solution}

\subsection{Pre-Hilbert space}
$H$ vector space with a scalar product is called pre-Hilbert space.

\subsection{Hilbert space}
A Hilbert space H is a pre-Hilbert space that is also a complete metric space w.r.t. the distance induced by its scalar product.

\subsection{Parallelogram law}
Let $H$ be a pre-Hilbert space, then $\forall x,y\in H:$\[
\|x+y\|2+\|x-y\|^2 = 2\big(\|x \|^2+\|y\|^2\big)\]

%>=====< Question  15 >=====<%

\question
State and prove the projection theorem and its corollary.

\subsection*{Solution}

\subsection{Projection theorem}
Let $H$ be a Hilbert space. Let $K \subset H$ be a closed convex subset. Then $\forall f\in H\,\, \exists ! u\in K\mbox{ s.t. } $
\[ (*)\,\,\,\,\|f-u\|= \displaystyle\min_{v\in K}\|f-v\| =: dist(f,K) \]
Moreover, \[ u \mbox{ fulfills } (*) \iff (**)\,\,\,\, u\in K, \langle  f-u,v-u\rangle\,\le 0\,\,\,\forall v \in K \]

\begin{proof}
Let $\{v_n\} \subset K$ be a minimizing sequence for $\displaystyle\min_{v\in K}\|f-v\|$, i.e:
\[ d_n \coloneqq\|f-v_n\| \to \displaystyle \inf_{v\in K} \|f-v\| =: d \]
We claim that $\{v_n\}$ is Cauchy. \newline 
By the parallelogram law applied with $x= f-v_n,\,\, y=f-v_m$ we get:
\[ \big\|f-\tfrac{v_n+v_m}{2}\big\|^2+\big\|\tfrac{v_n-v_m}{2}\big\|^2 = \tfrac{1}{2}\big(d_n^2+d_m^2\big) \]
Since $K$ is convex, 
\[ \|\tfrac{v_n+v_m}{2}\|^2 \in K \implies\,\, \|f-\tfrac{v_n-v_m}{2}\|^2 \ge d \]
Therefore,
\[ \|\tfrac{v_n-v_m}{2}\|^2\le \tfrac{1}{2}\big(d_n^2+d_m^2\big)-d^2 \,\xrightarrow{n,m\to \infty}\, \tfrac{1}{2}\big(d^2+d^2\big)-d^2 = 0 \]
\[ \implies\displaystyle\lim_{n,m \to \infty} \|v_n-v_m\| = 0\implies \mbox{ $\{v_n\}$ is Cauchy} \]
Notice that:
\[
    \begin{array}{rcl}
        H \mbox{ complete} & \implies&\,\exists\, u\in H: v_n \xrightarrow{n\to \infty} u \\
        \{v_n\}\subset  K \mbox{ closed} & \implies& u \in K
    \end{array}
\]
Furthermore,
\[ d \le \underbrace{\|f-v\|}_{\in K} \le \underbrace{\|f-v_n\|}_{\to d} + \underbrace{\|v_n-u\|}_{\to 0} \xrightarrow{\toi} d \]
\[ \implies \|f-u\|=d \]
Now suppose that there exist $u,\tilde{u}\in K$ such that $u\neq\tilde{u}$ and $\|f-u\|=\|f-\tilde{u}\|=d$.
\newline
By the parallelogram law with $x=f-u,\,\, y=f-\tilde{u}$:
\[ \big\|f-\tfrac{u+\tilde{u}}{2}\big\|^2+\underbrace{\big\|\tfrac{u-\tilde{u}}{2}\big\|^2}_{=:\epsilon>0} = \tfrac{1}{2}\big(d^2+d^2\big)=d^2 \]
\[ \implies d^2 \le \big\|f-\tfrac{u+\tilde{u}}{2}\big\|^2 = d^2 - \epsilon < d^2,\,\,\,\,\mbox{contradiction} \]\newline
We're left to prove that $(*) \iff (**)$.
Assume at first that $u \mbox{ fulfills } (*)$ and let $w \in K$, then:
\[ v \coloneqq (1-t)u +tw \in K\,\,\forall t\in [0,1] \]
\[ \|f-u\| \le \|f-v\| = \|f-[(1-t)u +tw]\| = \|(f-u)-t(w-u)\| \]
\[ \implies \|f-u\|^2  \le \|f-v\|^2 = \|f-u\|^2+t^2\|w-u\|^2-2t\langle f-u,w-u\rangle \]
\[ \implies 2\langle f-u,w-u\rangle  \le t\|w-u\|\,\,\forall t \in [0,1] \] 
As $t \to 0^+$ we get $(**)$.\newline
Conversely, suppose $(**)$ holds. Then:
\[ \|u-f\|^2-\|v-f\|^2=2\underset{\le 0}{\langle f-u,v-u\rangle }-\|u-v\|^2\le 0 ,\,\,\forall v\in K \]
\end{proof}

\subsection{Corollary of the projection theorem}
Suppose $M \subset H$ is a closed v.s.s. and let $f\in H$, then:
\[ u = proj_M f \iff u\in M, \langle f-u,v\rangle = 0\,\,\forall v \in M\,\,\,(***) \]

\begin{proof}
By $(**)$:
\[ \langle f-u,v-u \rangle \le 0\,\, \forall v \in M \]
\[ \implies \langle f-u,tv-u \rangle \le 0\,\,\forall v \in M,\,\forall t \in \R \]
\[ \implies t\langle f-u,v \rangle \le\langle f-u,u \rangle \,\forall t \in \R \]
Assume by contradiction that $\langle f-u,v \rangle \neq 0$, wlog let $\langle f-u,v \rangle >0$, then for $t> \displaystyle\frac{\langle f-u,u \rangle}{\langle f-u,v \rangle}$ the above inequality does not hold.
\newline Conversely, if $u $ satisfies $ (***)$, then
\[ \langle f-u,\xi -u \rangle = 0\,\,\, \forall \xi \in M\,\,\,\,\,(v=\underset{\in M}{\xi - u}) \iff (**) \iff (*) \]
\end{proof}

%>=====< Question 16 >=====<%

\question
State and prove the Riesz theorem.

\subsection*{Solution}

\subsection{Riesz theorem}
Let $H$ be a Hilbert space, for any $\phi \in H^* \,\,\exists! f\in H\mbox{ so that:}\\$
\[ (*) \,\,\,\,\phi(u)=\langle f,u\rangle\,\,\,\forall u\in H \]
Moreover,
\[ \|\phi\|_{\Leb}=\|f\|_H \]

\begin{proof}
    Let $M\coloneqq \phi^{-1}(0)$, so $M$ is a closed v.s.s. of $H$, if $M\equiv H $ we simply take $f=0$ and we are done.
    \newline Consider the case $M \subsetneq H$, we claim that:
    \[ \exists g \in H\mbox{ s.t. } \|g\|=1,\,\,\, g\in M^{\perp} \]
    In fact, let $g_0\in H,\,g_0 \notin M.$ Let $g_1\coloneqq P_m g_0$
    \[ g\coloneqq\displaystyle\frac{g_0-g_1}{\|g_0-g_1\|} \]
    satisfies $\|g\|=1,\,\,\, g\in M^{\perp}$
    For any $u \in H$, set:
    $ v\coloneqq u-\lambda g,\,\,\mbox{where } \lambda = \displaystyle\frac{\phi(u)}{\phi(g)} \,\,\,\,\,\,(\phi(g)\neq 0) $
    \[ \phi(v) = \phi(u) -\lambda \phi(g) =  \phi(u) - \frac{\phi(u)}{\cancel{\phi(g)}}\cancel{\phi(g)} = \phi(u)-\phi(u) = 0 \implies v \in M \]
    Therefore,
    \[ \langle  g,v\rangle=0 \implies\,\, \langle  g,u-\lambda g\rangle =0 \implies\,\, \langle g,u\rangle = \lambda\underset{=1}{\cancel{\|g\|}} \]
    \[ \implies \phi(u) = \phi(g)\langle g,u\rangle = \langle \underbracket{\phi(g)g}_{\coloneqq f},u\rangle \]
    Now we can show that such $f$ is unique: suppose by contradiction that there exist $f,f_1 \in H$ that satisfy $(*)$:
    \[ \implies 0 = \phi(u) - \phi(u) =  \langle  f,u\rangle - \langle  f_1,u\rangle = \langle  f-f_1,u\rangle\,\, \forall u\in H \]
    \[ \implies f-f_1=0 \implies f=f_1 \mbox{, contradiction.  }    \]
\end{proof}

\sheet

%>=====< Question  1>=====<%

\question
Show that the projector operator is continuous.

\subsection*{Solution}

\subsection{Projector operator continuity}
Let H be a Hilbert space, $M \subset  H$ be a v.s.s. of $H$ and let $ f \in H$, then: 
\[ u = \proj_{H} f \iff u \in M \quad <f-u,v> = 0 \quad \forall v \in M \]
Moreover $\forall f \in H$ let 
\[ P: H \to M,\,\,\,\, P(f) \coloneqq u \]
\[ Q: H \to M^{\perp},\,\,\,\, Q(f) \coloneqq f-u \]
We have:

\begin{itemize}
    \item[i)] $P(f) + Q(f) = f, \,\, \forall f \in H$
    \item[ii)] $f \in M \,\,\,\,\implies P(f) = f,\, Q(f) = 0$\\
    $f \in M^{\perp} \implies P(f) = 0,\, Q(f) = f$
    \item[iii)] $\|f-P(f)\| = dist(f,M) \,\,\,\,\forall f\in H$
    \item[iv)] $\|f\| = \|P(f)\|+\|Q(f)\|$
    \item[v)] $P,Q$ are linear. Futhermore, by the equality iv) it follows that: 
\[ \|P(f)\| \leq \|f\|, \,\,\, |Q(f)\| \leq \|f\| \implies P,Q \in \Leb(H) \]
\end{itemize}

%>=====< Question 2 >=====<%

\question
Write the definition of orthonormal basis of a Hilbert space. Exhibit some examples.

\subsection*{Solution}

\subsection{Orthonormal basis}
A sequence $\{e_n\}_{n\in\N} \subset H$ Hilbet space, is said to be an orthonormal basis of $H$ if:\\

$<e_n,e_m> = 0\,\,\,\forall n\neq m,\,\,\|e_n\|=1\,\,\, \forall n\in\N$
Some examples of o.n.b. are:
\begin{itemize}
    \item[i)] $\B = \{\sqrt{\tfrac{2}{\pi}}\sin{(nx)}\}_{n\ge 1} \cup \{\sqrt{\tfrac{2}{\pi}}\cos{(nx)}\}_{n\ge 0}$  in $\Leb^2((0,\pi))$
    \item[ii)] $e_n^{(k)} = \delta_{nk}$  in $\ell^2$
\\
\end{itemize}

%>=====< Question 3 >=====<%

\question
What is it possible to say about convergence of an o.n.b.?

\subsection*{Solution}

\subsection{Orthonormal basis convergence}
Let $\{e_n\}_{n\in\N}$ be an o.n.b., then:

\begin{itemize}
    \item[a)] $e_n \wc 0$
    \item[b)] $e_n \nrightarrow 0$
\end{itemize}

\begin{proof}
    $\forall f \in H$, Parseval's identity implies:
    \[\|f\|^2 = \sum_{k=1}^n<f,e_n>^2 \,\,\implies\,\,<f,e_n>\to 0 \iff e_n \wc 0
    \]
On the other hand we have $\|e_n\|=1\,\,\forall n\in\N$ and thus $e_n \nrightarrow 0$.
\end{proof}

%>=====< Question 4 >=====<%

\question
Write the definitions of $\rho$(T), $\sigma$(T), EV(T), $\sigma\rho$(T). What is the relation between EV(T) and $\sigma$(T)?
    
\subsection*{Solution}

Let $E$ be a Banach space and $T\in \Leb(E)$.

\subsection{Resolvent set}
\[\rho(T)\coloneqq\{\lambda\in\R:T-\lambda I: E\to E \mbox{ is bijective}\}\]

\subsection{Spectrum}
\[\sigma(T)=\R\backslash\rho(T)\]

\subsection{Eigenvalues}
\[EV(T)\equiv\sigma_p(T)\coloneqq\big\{\lambda\in\R:Ker(T-\lambda I)\neq \{0\}\big\}\]

\subsection{Relation between \texorpdfstring{$EV(T)$}{the set of Eigenvalues} and \texorpdfstring{$\sigma$(T)}{the spectrum}}
In general $EV(T)\displaystyle\subseteq \sigma(T)$ but when $\dim(E)<\infty$, we have the equality $EV(T)= \sigma(T)$

\subsubsection{Structure of the spectrum}
Let $T\in K(E)$, with $\dim(E)=\infty$. Then:

\begin{itemize}
    \item[i)] $0 \in  \sigma(T)$
    \item[ii)] $\sigma(T)\backslash \{0\} = EV(T)\backslash \{0\} $
    \item[iii)] One of the following holds:
    \begin{itemize}
        \item[(a)] $\sigma(T)= \{0\}$    
        \item[(b)] $\sigma(T)\backslash \{0\}$ is a finite set
        \item[(c)] $\sigma(T)\backslash \{0\}$ is a convergent sequence with limit 0.
    \end{itemize}
\end{itemize}

%>=====< Question 5 >=====<%

\question
State and prove (only partially) the spectral theorem.

\subsection*{Solution}

\subsection{Spectral theorem}
Let $H$ be a separable Hilbert space and let $T:H\to H $ be a linear, compact, bounded and symmetric operator. Then there exist an o.n.b. of $H$ made of eigenvectors of $T$.

\begin{proof}
Let $\{\lambda_n\}$ be the set of all distinct non zero eigenvalues of $T$.
and set: \\
\[\lambda_0 = 0,\,\, E_0 \coloneqq Ker(T),\,\, E_n \coloneqq Ker(T-\lambda_n I),\, n\geq1 \]\\
Clearly $0<\dim(E_0)<\infty$ and we claim that:

\begin{itemize}
    \item[i)]$\dim(E_n)<\infty \,\,\,\forall n\geq 1$:\\\\
        In fact, suppose by contradiction that $\dim(E_n)=\infty$, we have that $E_n$ is itself an Hilbert space and thus we can construct an o.n.b. $\{v_k\}$ of $E_n$. \\
        From a previous deduction we know $v_k \nrightarrow 0$, but:\\\\
        $\begin{aligned}
            \begin{cases}
                T(v_k)=\lambda_nv_k\\
                T  \mbox{ compact} \implies T(v_k) \to 0
            \end{cases}
        \end{aligned}$
        $\implies v_k \rightarrow 0$,\,\, contradiction
    \item[ii)] $E_n,E_m\,(n\neq m)$ are orthogonal $\iff \forall u\in E_m, \forall v\in E_n \mbox{ we have:  } <u,v>=0$ \\\\
        To prove this, notice that $T(u)=\lambda_m u,\,\,\, T(v) = \lambda_n v $ and since $T$ is symmetric:\\\\
        $\begin{aligned}
            \begin{cases}
                <T(u),v>=\lambda_m<u,v>\\
                <u,T(v)>=\lambda_n<u,v>
            \end{cases}
        \end{aligned}\implies
        \begin{array}{rcl}
            \lambda_m<u,v>& =&<u,v>\\
            (\lambda_m-\lambda_n)<u,v>&=&0\\
            <u,v>&=&0
        \end{array}$      
    \item[iii)] $F=\text{Span}(\{E_n\}_{n\geq 1})$ is dense in $H$.\\\\
        Finally we choose in each subspace $E_n$ an o.n.b., the union of these is an o.n.b. of $H$ composed by eigenvectors of $T$ and we are done.
\end{itemize}
\end{proof}


%>=====< End Document >=====<%

\end{document}












