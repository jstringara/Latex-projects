\chapter{lecture 3: 19/09}

\section{Lèvy processes}

In quite simple words a Lèvy process is a stochastic process with independent
and stationary increments. We will see that this is a very strong property and
useful property that allows us to model many phenomena. In particular we will
try to model stocks prices of the form $S_t = S_0 e^{rt + X_t}$ where $X_t$ is a
Lèvy process. 

\begin{example}
    $ $ \\
    \begin{enumerate}
        \item Brownian motion $W_t$
        \item $-\frac{\sigma}{2}t + \sigma W_t$
    \end{enumerate}
\end{example}

\begin{definition}[Lèvy process]
    Let $X_t$ be a Cadlag stochastic process, such that $X_0=0$. We say that
    $X_t$ is a Lèvy process if it satisfies the following properties:
    \begin{enumerate}[i]
        \item $X_t$ has independent increments, i.e. for
            $0 \leq t_1 < t_2 < \dots < t_n$ we have that: 
            \[
                X_{t_1} - \cancelto{0}{X_0} \indep X_{t_2} - X_{t_1} \indep \dots \indep
                    X_{t_n} - X_{t_{n-1}}
            \]
        \item $X_t$ has stationary increments, i.e. $\forall t,s > 0$ we have
            that:
            \[
                X_{t+s} - X_t \overset{L}{=} X_s
            \]
        \item $X_t$ has stochastic continuity, i.e. $\forall \epsilon > 0$ we
            have that:
            \[
                \lim_{h \to 0} \P(|X_{t+h} - X_t| > \epsilon) = 0
            \]
    \end{enumerate}
\end{definition}

This last property ensures that the process is continuous in probability. In
other words that there are no jumps in the distribution of the jump times. No
single instant is more likely than all others to be a jump time. We have no way
of knowing with any probability when the processes' next jump will occur.

\begin{properties}
    $ $ \\
    \begin{itemize}
        \item \textbf{Infinite divisibility}: $X_t$ is infinitely divisible,
            i.e. $\forall n \in \N$, $\forall t>0$ we have that:
            \[
                X_t = \sum_{i=1}^n \hat{X}_i \quad \hat{X}_i = X_{i\cdot t/N} -
                X_{(i-1)\cdot t/N} \text{ iid}
            \]
            Thus, we can break the process into an infinite (but countable) 
            number of independent increments.
        \item \textbf{Characteristic function and exponent}: 
            \[
                \Phi_{X_t}(u) = \E[e^{iuX_t}] = e^{t\Psi(u)}
            \]
            where $\Psi(u)$ is the characteristic exponent of the process. The
            function $t\mapsto\Phi_{X_t}$ is the characteristic function of
            \[
                \Phi_{X_{t+s}}(u) = \Phi_{X_t}(u) \cdot \Phi_{X_s}(u)
            \]
            Indeed:
            \begin{align*}
                \Phi_{X_{t+s}}(u) & \tikzmarknode{eq1}{=} \E[e^{iuX_{t+s}}] =
                \E[e^{iu(X_t-X_s)} \cdot e^{iuX_s}] \\
                \text{thanks to independence } & \tikzmarknode{eq2}{=} 
                \E[e^{iu(X_t-X_s)}] \cdot \E[e^{iuX_s}] = \Phi_{X_{t+s}-X_s}(u)
                \cdot \Phi_{X_s}(u) \\
                \text{thanks to stationarity } & \tikzmarknode{eq3}{=}
                \Phi_{X_{t}}(u) \cdot \Phi_{X_s}(u)
            \end{align*} \tikz[overlay, remember picture]{
                \draw[shorten >= 1pt, shorten <= 1pt] (eq1) -- (eq2) -- (eq3);
            }
    \end{itemize}
\end{properties}

\begin{remark}
    The PP is the only Counting process which is also a Lèvy process. On the
    other hand the CPP is the only Lèvy process with piecewise constant
    trajectories. Indeed we may recall that $X_t = \sum_{i=1}^N Y_i$ where $Y_i$
    are iid random variables. Thus, if we make the choice $Y_i = 1$ we get the
    PP. Furthermore the characteristic function of $X_t$ depends entirely on the
    distribution of $Y_i$:
    \begin{align*}
        \Phi_{X_t}(u) & = \E[e^{iuX_t}] = e^{t\lambda \int_{\R^d} (e^{iuy} - 1) 
        \cdot f(dy)} \\
        \implies \Psi_{X_t}(u) & = \lambda \int_{\R^d} (e^{iuy} - 1) \cdot f(dy)
    \end{align*}
    where $f$ is the distribution of $Y_i$. Furthermore if $Y_i$ is Lebesgue
    integrable then $f(dy) = k(y)dy$.
\end{remark}

\begin{proof}
    \begin{align*}
        \E[e^{iuX_t}] & \tikzmarknode{eq1}{=} \E\left[ \E[e^{iuX_t} | N_t]
            \right] = \E\left[ \E\left[ e^{iu\sum_{i=1}^{N_t} Y_i} \right]
            \right] = \E\left[ \E\left[ e^{iuN_tY_i} | N_t \right] \right]\\
        & \tikzmarknode{eq2}{=} \E\left[ \left(\E\left[ e^{iuY_i} \right]
            \right)^{N_t} \right] = \E\left[ \Phi^{N_t}_{Y_i}(u) \right] =
            \sum_{n=0}^\infty \Phi^n_{Y_i}(u) \cdot \P(N_t = n) \\
        & \tikzmarknode{eq3}{=} \sum_{n=0}^\infty \Phi^n_{Y_i}(u) \cdot
            \frac{(\lambda t)^n}{n!} e^{-\lambda t} = e^{-\lambda t}\cdot e^{
            \Phi_{Y_i}(u) \cdot \lambda t} = e^{\lambda t \left( \Phi_{Y_i}(u)
            - 1 \right)}
    \end{align*} \tikz[overlay, remember picture]{
        \draw[shorten >= 1pt, shorten <= 1pt] (eq1) -- (eq2) -- (eq3);
    }
    Thanks to the indipendence of the $Y_i$ and the Taylor expansion of the
    exponential function. Now:
    \[
        \Phi_{Y_i}(u) = \E\left[ e^{iuY_i} \right] = \int_{\R^d} e^{iuy} \cdot
        f(dy) = \int_{\R^d} e^{iuy} \cdot k(y)dy
    \]
    Hence, by substituting in the previous equation we get:
    \[
        \Phi_{X_t}(u) = e^{\lambda t \left( \int_{\R^d} e^{iuy} \cdot k(y)dy -
        1 \right)} = e^{\lambda t \left( \int_{\R^d} e^{iuy} \cdot k(y)dy - 
        \int_{\R^d} 1 \cdot k(y)dy \right)}
    \]
    thus, thanks to the fact that $k(y)$ is a distribution function, we
    ultimately get:
    \[
        \Phi_{X_t}(u) = e^{\lambda t \int_{\R^d} (e^{iuy} - 1) \cdot k(y)dy}
    \]
\end{proof}

Sometimes we encounter the notation $\nu(dy) = \lambda k(y)dy = \lambda f(dy)$.
This is called the \textbf{Lèvy measure} of the process.

\begin{definition}[Lèvy measure of a CPP]
    Let $A$ be a Borel set in $\R^d$. We define the following quantities:
    \begin{align*}
        & \Delta X_t = X_t - X_{t^-} \\
        & X_{t^-} = \lim_{s \to t^-} X_s
    \end{align*}
    Where $\Delta X_t$ is the jump size of the process at time $t$. Then:
    \[
        \nu(A) = \E\left[ \#\left\{ t\in[0,1] :\; \Delta X_t \neq 0, \,
        \Delta X_t \in A \right\} \right]
    \]
    This is the average number of time in a unitary time interval in which a
    jump occurs in the set $A$.
\end{definition}

\begin{definition}[Random Measure]
    While $\nu$ is a deterministic measure on the set of Borel sets of $\R^d$,
    we can define the following random measure:
    \[
        J_X(B) = \#\left\{ (t,\Delta X_t) \in B \right\} \; \forall B \text{
            open set in } [0,+\infty) \times \R^d 
    \]
    This a random variable of the number of jumps times and jump sizes in the 
    set $B$.
\end{definition}

\begin{application}
    Let $X_t$ be a CPP.
    \[ 
        X_t = \sum_{i=1}^{N_t} Y_i = \sum_{s\in[0,t]} \Delta X_s = \int_{[0,t]
        \times \R^d} x \cdot J_X(ds \times dx)
    \]
    $J_X$ is a Poisson Random Measure with intensity $\nu(ds)dt = \lambda f(x)
    dx dt$.
\end{application}

\section{Exponential Lèvy processes}
In our framework we will model stocks and other processes as exponential Lèvy
processes of the form:
\[
    S_t = S_0 e^{rt + X_t}
\]
where $X_t$ is a Lèvy process. Indeed, different choices of $X_t$ will lead to
different models. For example:
\begin{itemize}
    \item $X_t = -\sigma^2/2 t + \sigma W_t$ is the Black-Scholes model under
        the risk-neutral measure $\Q$.
    \item $X_t = \sum_{i=1}^{N_t} Y_i$ yields us the Coumpound Poisson Process
        model. The problem is that since this model is piecewise constant, it
        cannot capture the continuous variation of the stock prices on the real
        market.
    \item $X_t = \delta t + \sigma W_t + \sum_{i=1}^{N_t} Y_i$ yields us with a
        jump-diffusion model (either Merton or Kou). We will see later how 
        $\delta$ shall be chosen in order to fit with the risk-neutral measure
        $\Q$.
\end{itemize}

\subsection*{Merton and Kou models}
\begin{itemize}
    \item \textbf{Merton model}: $Y_i \sim \mathcal{N}(\mu_j, \delta_j^2)$ hence
        we have a total of 4 parameters $\sigma, \lambda, \mu_j, \delta_j$.
        $\mu_j$ and $\delta_j$ are the mean and variance of the jumps size.
    \item \textbf{Kou model}:
        \[
            Y_i = \left\{ \begin{array}{l}
                    \text{Exp}(\lambda^+) \text{ with probability }
                    $p$ \\
                    -\text{Exp}(\lambda^-) \text{ with probability }
                    $1-p$
                \end{array} \right.
        \]
        Hence we have a total of 5 parameters $\sigma, \lambda, p, \lambda^+,
        \lambda^-$.
\end{itemize}

\section{Choice of \texorpdfstring{$\delta$}{delta}}
Let us recall that we are under the $\Q$ measure if we have that:
\begin{align*}
    & \E^\Q \left[ S_t \cdot e^{-rt} \right] = S_0 \\
    & \implies S_0 \E[e^{X_t}] = S_0 \cdot \Phi_{X_t}(-i)
\end{align*}
Thus we must choose $\delta$ such that:
\[ \Phi_{X_t}(-i) = 1 \iff \Psi_{X_t}(-i) = 0 \]
in order to satisfy the risk-neutral measure condition. Thus $\delta$ is not a
parameter of the model that we can freely choose but must calibrated from the
market in order to adhere to the risk-neutral measure.

\section{Lévy-Itô decomposition}
So far we have encountered three types of levy processes:
\begin{itemize}
    \item Continuous, the family of Brownian motions.
    \item Finite variation, the Compound Poisson Process, which is unique.
    \item Jump-diffusion processes.
\end{itemize}
But not all levy processes are these types. This dependes on whether $\nu$ can
be well defined.
TODO: add picture of how to divide up levy processes
For example:
\[ \nu((2,3)) = +\infty \implies \E [ \# \{ t\in[0,1]: \; \Delta X_t \neq 0, \,
    \Delta X_t \in (2,3) \} ] = +\infty \]
This is clearly not a Cadlag process. We have infinite activity in the interval
$(2,3)$. In other words we have an infinite number of terms in the summation
$\sum \Delta X_t$.

\begin{definition}[Lévy-Itô decomposition]
    Let $(X_t)_{t\geq0}\in\R^d$ be a Lévy process. Let $\nu$ be its Lévy measure.
    Then: 
    \begin{enumerate}[i)]
        \item $\nu$ is a Radon measure in $\R^d\setminus\{0\}$. In other words
            any compact set in $\R^d\setminus\{0\}$ is measurable.
        \item \[ \int_{x\in\R^d: \, |x|\geq1} \, \nu(dx) < +\infty \]
        \item \[ \int_{x\in\R^d: \, |x|<1} |x|^2 \, \nu(dx) < +\infty \] TODO: missed comment from professor
        \item We can define a triplet $(\gamma, A, \nu)$ such that:
            \[ X_t = \gamma t + B_t + X_t^l + \lim_{\epsilon\to0}
            \hat{X}_t^{\epsilon} \]
            where:
            \begin{itemize}
                \item $B_t$ is a Brownian motion with varianca-covariance matrix
                    $A$.
                \item \[ X_t^l = \sum_{\begin{subarray}{c}
                    s\in [0,t] \\
                    |\Delta X_s| \geq 1 
                \end{subarray}} = \int_{[0,t]\times \{x \geq 1\}} x \cdot
                J_X(ds \times dx) \]
                \item We have the Compound Poisson process with jumps in 
                $(\epsilon, 1)$:
                \[ X^\epsilon_t = \sum_{\begin{subarray}{c}
                    s\in[0,t] \\ \epsilon < |\Delta X_s| < 1
                \end{subarray}} \Delta X_s \]
                Thus we get the respective compensated process:
                \begin{align*}
                    \hat{X}^\epsilon_t & \tikzmarknode{eq1}{=} \sum_{
                        \begin{subarray}{c}
                           s\in[0,t] \\ \epsilon < |\Delta X_s| < 1 
                        \end{subarray}
                    } \Delta X_s - t \int_{\epsilon
                    < |x| < 1} x \, \nu(dx) \\
                    & \tikzmarknode{eq2}{=} \int_{[0,t]\times \{\epsilon < |x|
                    < 1\}} x \cdot (J_X(ds \times d) - \nu(dx) ds) 
                \end{align*} \tikz[overlay, remember picture]{
                    \draw[shorten >= 1pt, shorten <= 1pt] (eq1) -- (eq2);
                }
            \end{itemize}
    \end{enumerate}
Indeed we have that:
\begin{align*}
    \int_{x\in\R^d: \, |x|<1} |x|^2 \, \nu(dx) < +\infty & \implies \lim_{
        \epsilon \to 0^+} \hat{X}^\epsilon_t < +\infty \\
    \int_{x\in\R^d: \, |x|<1} |x|^2 \, \nu(dx) < +\infty &  \centernot\impliedby
        \lim_{ \epsilon \to 0^+} X^\epsilon_t < +\infty
\end{align*}
Thanks to Martingality and the Central Limiti Theorem.
\end{definition}

\begin{example}[One-dimensional case, from Jump-diffusion to Levy decomposition]
Since we are starting from a jump-diffusion process, we have that:
\[
    X_t = \delta_t + \sigma W_t + \sum_{s\in[0,t]} \Delta X_s \longrightarrow
   \sum_{s\in[0,t]} \Delta X_s = \left\{\begin{array}{l}\int_{|x|\geq1}\,\nu(dx)
    < +\infty \\ \int_{|x|<1} \, \nu(dx) < +\infty \end{array}\right.
\]
Clearly we have that $A = \sigma^2$. The Levy-Itô decomposition is:
\[
    X_t = \delta t + \sigma W_t + \sum_{\begin{subarray}{c}
        s\in[0,t] \\ |\Delta X_s| \geq 1
    \end{subarray}} \Delta X_s
    + \lim_{ \epsilon \to 0^+} \hat{X}^\epsilon_t
\]
Now, let $X_t$ be a Jump-Diffusion model: TODO (missed something)
\[ X_t = \gamma t + \sigma W_t + \sum_{\begin{subarray}{c}
    s\in[0,t]\\ |\Delta X_s| \geq 1\end{subarray}} \Delta
    X_s + \lim_{\epsilon \to 0^+} \left[ \sum_{\begin{subarray}{c}
        s\in[0,t]\\ \epsilon < |\Delta X_s| < 1
    \end{subarray}} \Delta X_s - t \int_{\epsilon < |x| < 1} x \, \nu(dx) \right]
\]
We can write since when $|\Delta X_s| < 1$ we have that:
\[ \int_{|x|<1} \,\nu(dx) < +\infty \implies \int_{|x|<1} x\,\nu(dx)<+\infty \]
Thus we can write:
\begin{align*}
    & \tikzmarknode{eq1}{=} \gamma t + \sigma W_t + \sum_{\begin{subarray}{c}
        s\in[0,t]\\ |\Delta X_s| \geq 1\end{subarray}}
        \Delta X_s + \sum_{\begin{subarray}{c} 
            s\in[0,t]\\ |\Delta X_s| < 1\end{subarray}} \Delta X_s
    - t \int_{|x|<1} x \,\nu(dx) \\
    & \tikzmarknode{eq2}{=} t \underbrace{\left( \gamma -\int_{|x|<1}x \,\nu(dx)
    \right)}_{\delta} + \sigma W_t + \sum_{s\in[0,t]} \Delta X_s
\end{align*} \tikz[overlay, remember picture]{
    \draw[shorten >= 1pt, shorten <= 1pt] (eq1) -- (eq2);
}
\end{example}

\begin{definition}[Levy-Khincin formula]
    Let $X_t$ be a Lèvy process with Lèvy-Itô decomposition $(\gamma, A, \nu)$.
    Then: 
    \[ \Phi_{X_t}(u) = \E[e^{iuX_t}] = e^{t\Psi_{X_t}(u)} \]
    where:
    \[
        \Psi_{X_t}(u) = iu\cdot\gamma-\frac{1}{2}u^TAu+\int_{\R^d}(e^{iux} - 1 -
        iu\cdot x\I_{|x|\leq1}) \,\nu(dx)
    \]
\end{definition}
\begin{proof}
Thanks to the Lèvy-Itô decomposition we have that:
\[
    X_t = \underbrace{\gamma t + B_t}_{X_t^c, \text{ continuous}} +
    \underbrace{X_t^l + \lim_{\epsilon \to 0^+} \hat{X}^\epsilon_t}_{X_t^j, 
    \text{ jump}}
\]
Thus we can split the characteristic function into two parts thanks to the fact
that the continuous and jumping parts are independent:
\[ \Phi_{X_t}(u) = \E[e^{iuX_t^c}] \cdot \E[e^{iuX_t^j}] \]
We shall now analyze them separately. For $X_t^c$ we have that:
\[
    \E[e^{iuX_t^c}] = \E[e^{iu\gamma t} \cdot e^{iuB_t}] =
    e^{iu\gamma t} \cdot \E[e^{iuB_t}]  = e^{iu\gamma t} \cdot e^{-\frac{1}{2}t
    u^T A u}
\]
Let us now analyze $X_t^j$, but first let us fix the value of $\epsilon$:
\begin{align*}
    X^{\epsilon, j}_t & = \sum_{|\Delta X_s| \geq 1} \Delta X_s + X^\epsilon_t\\
    X^j_t & = \lim_{\epsilon\to0^+} X^{\epsilon, j}_t
\end{align*}
With a fixed value for $\epsilon$ we can observe that $X^{\epsilon, j}_t$ is a
CPP. Thus:
\begin{align*}
    \E[e^{iuX^j_t}] &\tikzmarknode{eq1}{=} \lim_{\epsilon\to0^+}
    \E[e^{iuX^{\epsilon, j}_t}] = \lim_{\epsilon\to0^+} \E\left[
    e^{iu\cdot\left( \sum_{\epsilon<|\Delta X_s|} - t\int_{\epsilon<|x|<1} x \,
    \nu(dx) \right)} \right] \\
    \lim_{\epsilon\to0^+} &\tikzmarknode{eq2}{=} \E\left[e^{iu\sum_{|\Delta X_s|
    \geq 1}\Delta X_s}\right] \cdot \E\left[e^{iu\left( \sum_{\epsilon<|
    \Delta X_s|<1}\Delta X_s - t\int_{\epsilon<|x|<1} x \, \nu(dx) \right)}
    \right] \\
    \text{using the CPP properties }\lim_{\epsilon\to0^+} &\tikzmarknode{eq3}{=}
    e^{t\int_{|x|\geq1}(e^{iux}-1)\,\nu(dx)} \cdot e^{t \left( \int_{\epsilon<|
    x|<1}(e^{iux}-1)\,\nu(dx) - \int_{\epsilon<|x|<1} ux \, \nu(dx) \right)} \\
    \lim_{\epsilon\to0^+} &\tikzmarknode{eq4}{=} e^{t\int_{|x|\geq1}(e^{iux}-1)
    \, \nu(dx)} \cdot e^{t\int_{\epsilon<|x|<1}(e^{iux}-1-iux)\,\nu(dx)} \\
    \lim_{\epsilon\to0^+} &\tikzmarknode{eq5}{=} e^{t\int (e^{iux}-1-iux \I_{
        \epsilon<|x|<1})\,\nu(dx)}
\end{align*} \tikz[overlay, remember picture]{
    \draw[shorten >= 1pt, shorten <= 1pt] (eq1) -- (eq2) -- (eq3) -- (eq4) --
    (eq5);
}
Now, if we let $\epsilon \to 0^+$ we get:
\[ \E[e^{iuX^j_t}] = e^{t\int(e^{iux}-1-iux\I_{|x|<1})\,\nu(dx)} \]
\end{proof}

\begin{property}[Splitting the integral in the Levy-Khincin formula]
    \item \[ \lim_{\epsilon\to0^+} \hat{X}^\epsilon_t =
    \lim_{\epsilon\to0^+} \int_{\{\epsilon<|x|<1\} \times [0,t]} x \cdot
    \left( J_X(ds \times dx) - \nu(dx)ds \right) \]
    For a general Levy process this is finite $(+\infty)$. Under certain
    conditions we are able to split the integral into two parts:
    \[ \left. \begin{array}{l}
        \int_{|x|>1} \, \nu(dx) < +\infty \text{ finite activity} \\
        \int_{|x|\leq1} |x| \, \nu(dx) < +\infty \text{ finite variation}
    \end{array} \right\} \implies
        \lim_{\epsilon\to0^+} \int_{\epsilon<|x|<1} x \, \nu(dx) < +\infty
    \]
    TODO unsure about this, check with professor about the integrals before
    the brackets.
\end{property}

Let us formally define what variation actually consists of:

\begin{definition*}[Total variation]
    Let $f:[a,b]\to\R^d$ be a function. We define the total variation of $f$ as:
    \[ \sup_{a=t_0<\dots<t_n=b} \sum_{i=1}^n |f(t_i)-f(t_{i-1})| \quad n\in\N\]
    In plainer words, the total variation is the supremum of the sum of the
    "jumps" of the function over all possible partitions of its domain. 
    Furthermore a function is said to be of bounded variation if its total
    variation is finite.
\end{definition*}

\begin{theorem*}[Finite variation Levy process]
    Let $X_t$ be a Levy process with Levy-Itô decomposition $(\gamma, A, \nu)$.
    Then $X_t$ is of finite variation if and only if:
    \[ \int_{|x|\leq1} |x| \, \nu(dx) < +\infty \text{ and } A = 0 \]
\end{theorem*}
This yields us the following hierarchy of Levy processes (with respect to their
variation):
\[ \begin{array}{l}
    \int_{|x|\leq1} \, \nu(dx) < +\infty \rightarrow \text{ CPP} \\
    \int_{|x|\leq1} |x| \, \nu(dx) < +\infty \rightarrow \text{ Finite variation} \\
    \int_{|x|\leq1} |x|^2 \, \nu(dx) < +\infty \rightarrow \text{ General Levy process}
\end{array} \]
Let us also observe that only the first two are of bounded variation, and thus
can have their integral limit split into two parts.
