\chapter{Lecture 2: }

\section{Poisson Process}
In order to define the Poisson Processes we must first recall and explore the
notions of:
\begin{itemize}
    \item Exponential Random Variable
    \item Poisson Distribution
\end{itemize}

\subsection{Exponential Random Variable}
A random variable $Y$ is said to be exponentially distributed with parameter
$\lambda > 0$ if it has the following probability density function:
\[
    f_Y(y) = \lambda e^{-\lambda y} \I_{\Rpos}(y)
\]
or equivalently that it has the following cumulative distribution function:
\[
    F_Y(y) = 1 - e^{-\lambda y} \I_{\Rpos}(y)
\]
and we write $Y \sim \text{Exp}(\lambda)$.

\begin{theorem}[Absence of Memory]
Let $T\geq0$ be a random variable such that:
\[ \P(T>t+s|T>s) = \Pr(T>t) \quad \forall t,s > 0 \]
then $T$ can only be exponentially distributed.
\end{theorem}

In other words, the exponential distribution is the only continuous distribution
with absence of memory.

\begin{proof}
% leave a blank line in the pdf
$ $ \newline
\begin{itemize}
    \item $(\impliedby)$ We start with $T\sim\text{Exp}(\lambda)$, then its pdf
        is:
        \[
            f_T(y) = \lambda e^{-\lambda y} \I_{\Rpos}(y)
        \]
        Thus:
        \begin{align*}
            \P(T>t+s | T>s) & \tikzmarknode{eq1}{=}
                \frac{\P(T>t+s)}{\P(T>s)}
                = \frac{\int^\infty_{t+s} \lambda e^{-\lambda y} dy}
                {\int^\infty_s \lambda e^{-\lambda y} dy} 
                = \frac{e^{-\lambda(t+s)}}{e^{-\lambda t}}\\
                & \tikzmarknode{eq2}{=} e^{-\lambda s} =
                \int^\infty_s \lambda e^{-\lambda y} dy = \P(T>s)
        \end{align*} \tikz[overlay,remember picture]{
            \draw[shorten >= 1pt,shorten <= 1pt] (eq1) -- (eq2);
        }
    \item $(\implies)$ We define $g(t) = \P(T>t)$, then thanks to the hypothesis
        we get:
        \[
            g(t+s) = \P(T>t+s) = \P(T>t+s|T>t)\P(T>t) =
            \P(T>s) \P(T>t) = g(s)g(t)
        \]
        Now, $g$ is decreasing and right continuous thanks to the definition of
        a CDF.
        \begin{theorem*}
            The function $g(t)=e^{-\lambda t}$ is the only function that
            satisfies the two requirements above. 
        \end{theorem*}
        Thus:
        \[
            \P(T\leq t) = 1 - \P(T>t) = 1 - g(t) = 1 - e^{-\lambda t}
            \implies T\sim\text{Exp}(\lambda)
        \]
\end{itemize}
\end{proof}

\subsection{Poisson Distribution}
Let $N$ be a random variable with integer values, then $N$ is said to be
Poisson distributed with parameter $\lambda > 0$ if it has the following
probability mass function:
\[
    \P(N=n) = \frac{\lambda^n}{n!} e^{-\lambda} \I_{\N}(n)
\]
and we write $N\sim\text{Pois}(\lambda)$.
Its moment generating function is given by:
\[
    M_N(u) = \E[e^{uN}] = e^{\lambda(e^u-1)}
\]

We will now create a sort of "bridge" between the exponential distribution and
the Poisson distribution.

\subsection*{Counting Process}

\begin{theorem}[Counting Process]
    Let $\seq{\tau}$ be a sequence of iid random variables with exponential
    distribution with parameter $\lambda > 0$, then $\forall t > 0$, we define:
    \[
        N_t = \inf_t \left\{ n\geq 0 : \sum^{n+1}_{i=1} \tau_i > t \right\}
    \]
    Then we can say that $N_t$ is a Poisson distributed random variable with
    parameter $\lambda t$. Hence:
    \[
        \P(N_t = n) = \frac{(\lambda t)^n}{n!} e^{-\lambda t} \I_{\N}(n)
    \]
\end{theorem}

\begin{remark}
    For us the sequence $\seq{\tau}$ will be a sequence of inter-arrival times,
    i.e. the time between two consecutive jumps of the process.
    $N_t$ is called the \textbf{counting process}. It counts the number of jumps
    of the process up to time $t$.
\end{remark}

\begin{proof}
    Let us define the process $T_n = \sum^n_{i=1} \tau_i$. Now, since 
    $\tau_i\sim \text{Exp}(\lambda)$, then we have that the PDF of $T_n$ is:
    \[
        P_n(t) = \lambda e^{-\lambda t} \frac{(\lambda t)^{n-1}}{(n-1)!}
        \I_{\Rpos}(t)
    \]
    thus:
    \begin{align*}
        \P(T_{n+1}>t) & \tikzmarknode{eq1}{=} \int_t^\infty P_{n+1}(s) ds
            = 1 - \overbrace{\int_0^t P_{n+1}(s) ds}^{\P(T_{n+1}\leq t)} 
            = 1 - \int_0^t \underbrace{\lambda e^{-\lambda s}}_f
            \underbrace{\frac{(\lambda s)^n}{n!}}_{g'} ds \\
        \text{integrating by parts} & \tikzmarknode{eq2}{=} 1 + 
            \underbrace{e^{-\lambda t}}_f\underbrace{\frac{(\lambda s)^n}{n!}}_g
            - \int_0^t \underbrace{e^{-\lambda s}}_f
            \underbrace{\frac{(\lambda s)^n}{n!}n}_{g'} ds \\
        & \tikzmarknode{eq3}{=} 1 + e^{-\lambda t} \frac{(\lambda t)^n}{n!} -
            \int_0^t \lambda e^{-\lambda s} \frac{(\lambda s)^{n-1}}{(n-1)!} ds \\
        & \tikzmarknode{eq4}{=} 1 + e^{-\lambda t} \frac{(\lambda t)^n}{n!} -
            \P(T_n < t) = e^{-\lambda t} \frac{(\lambda t)^n}{n!} + 
            \P(T_n \geq t)
    \end{align*} \tikz[overlay,remember picture]{\draw[shorten >= 1pt,shorten <= 1pt] (eq1) -- (eq2) -- (eq3) -- (eq4);}
    Hence we can write:
    \[
        \P(N_t = t) = \P(T_{n+1}>t) - \P(T_n \geq t) = e^{-\lambda t}
        \frac{(\lambda t)^n}{n!} 
    \] 
    thus $N_t\sim\text{Pois}(\lambda t)$.
\end{proof}

\begin{remark}
    $ $ \newline % leave a blank line
    \begin{enumerate}
        \item If $Y_1 \sim \text{Pois}(\lambda_1)$ and $Y_2 \sim
            \text{Pois}(\lambda_2)$ are independent, then $Y_1 + Y_2 \sim
            \text{Pois}(\lambda_1 + \lambda_2)$.
        \item If $Y \sim \text{Pois}(\lambda)$, then we can divide it into:
            \[ 
                Y = \sum^n_{i=1} Y_i \quad \text{where } Y_i \sim \text{Pois}
                \left( \frac{\lambda}{n} \right)
            \]
    \end{enumerate}
\end{remark}

TODO: ask if the definition is done with $\geq$ or $>$ in the indicator function.
\begin{definition}[Poisson Process (PP)]
Let $\seq{T}$ be a sequence of iid random variables with exponential distribution
with parameter $\lambda > 0$, and $T_n = \sum^n_{i=1} \tau_i$, then:
\[ N_t = \sum^\infty_{n=1} \I_{\{t > T_n\}} \]
is called a Poisson Process with intensity $\lambda$.
\end{definition}

\begin{properties}
    $ $ \newline % leave a blank line
    \begin{itemize}
        \item $N_t\sim\text{Pois}(\lambda t)$ if $t>0$.
        \item $N_t$ is a Cadlag process. Hence it is right continuous and has
            left limits.
        \item its characteristic function can be computed as:
            \[
                \Phi_{N_t}(u) = \E[e^{iuN_t}] = e^{\lambda t(e^{iu}-1)}
            \]
            We also call $\psi_{N_t}(u) = \lambda t(e^{iu}-1)$ the characteristic
            exponent of $N_t$.
        \item $N_t$ has independent increments, i.e. $\forall t_1,\ldots,t_n$,
            we have that:
            \[ N_{t_n} - N_{t_{n-1}} \indep \ldots \indep N_{t_2} - N_{t_1} \indep
                N_{t_1} - \cancelto{0}{N_0}
            \]
    \end{itemize}
\end{properties}

We have so far defined the Weiner process, the Poisson process and we will at
this point move onto the compound Poisson process and the compensated Poisson
process. We will now see that the Poisson process is a special case of the
compound Poisson process. \\
Let us take a family of iid random variables $\seq{Y}$ and a sequence of iid
random variables $\seq{\tau}$ with exponential distribution of parameter 
$\lambda > 0$. Just like we did above we can define the process $T_n = 
\sum^n_{i=1} \tau_i$ and the counting process $N_t = \sum^\infty_{n=1}
\I_{\{t\geq T_n\}}$. Now we move on to define the compound Poisson process.

\begin{definition}[Compoind Poisson Process (CPP)]
    Let $\seq{Y}$ be a sequence of iid random variables and $\seq{\tau}$ be a
    sequence of iid random variables with exponential distribution of parameter
    $\lambda > 0$. Let also these two families be independent, such that
    $Y_i \indep N_t = \sum^\infty_{n=1} \I_{\{t\geq T_n\}}$. Then we define the
    compound Poisson process as:
    \[
        Y_t = \sum^\infty_{n=1} Y_n \I_{\{t\geq T_n\}}
    \]
\end{definition}

Given a compound Poisson process $Y_t$, we can represent graphically its path
as a step function, where the jump sizes are given by the values of $Y_n$ and
the time elapsed between two consecutive jumps is given by $\tau_n$.

TODO: add graph of a step function with the jumps and the time elapsed between
them.

One unsatisfactory aspect of the Poisson process is that it is not
a Martingale. Indeed, take any two times $s$ and $t$ such that $s<t$, then:
\[
    \E[N_t | N_s] = \E[N_t - N_s + N_s | N_s] = \E[N_t - N_s | N_s] + N_s
    = \E[N_t - N_s] + N_s = \lambda(t-s) + N_s \neq N_s
\]

\begin{definition}
    Let $N_t$ be a Poisson process with intensity $\lambda > 0$, then the
    compensated Poisson process is defined as:
    \[
        \tilde{N}_t = N_t - \lambda t
    \]
\end{definition}

This is no longer a counting process since (due to the $\lambda t$ term) it is
not strictly integer valued. However, it is a Martingale. Indeed by the same
reasoning as above we have that:
\[
    \E[\tilde{N}_t | \tilde{N}_s] = \E[\tilde{N}_t - \tilde{N}_s | \tilde{N}_s]
    + \tilde{N}_s = \E[\tilde{N}_t - \tilde{N}_s] + \tilde{N}_s = 0 + \tilde{N}_s
    = \tilde{N}_s
\]

\begin{theorem}[Uniqueness of the Poisson Process]
    Let $N_t$ be a counting process with intensity $\lambda > 0$, with integer
    values and independent and stationary increments then it can only be a
    Poisson process. 
\end{theorem}

TODO: check definition of stationarity.
\begin{definition}[Stationary]
    Let $X_t$ be a stochastic process, then we say that it is stationary if
    $\forall t_1,\ldots,t_n$ and $\forall h > 0$ we have that:
    \[
        (X_{t_1},\ldots,X_{t_n}) \sim (X_{t_1+h},\ldots,X_{t_n+h})
    \]
\end{definition}

\begin{recall}
For the Weiner process we have that $W_t \sim \mathcal{N}(0,t) \overset{L}{=}
\sqrt{t} \mathcal{N}(0,1)$. Thus simulating a Weiner process is rather simple as
it only requires to simulate a normal random variable of mean 0 and variance 1.
On the other hand, for the Poisson process we have:
\begin{itemize}
    \item $\seq{\tau}$ iid $\text{Exp}(\lambda)$
    \item $N_t \sim \text{Pois}(\lambda t)$
\end{itemize}
if we want to simulate a Poisson process we need to simulate either the
inter-arrival times themselves or directly the counting process.
\begin{itemize}
    \item The first method, that of simulating is called \textbf{Countdown
        Sampling}. And can be achieved in MATLAB with the following code:
        
\begin{minted}[linenos, breaklines]{matlab}
% Countdown Simulation
lambda = 2;
t = 1;
tau = icdf('Exponential', rand, 1/lambda);
while (sum(tau) < t) %this is just the condition that we are before T
tau_temp = icdf('Exponential', rand, 1/lambda) #?????????
tau = [tau, tau_temp];
end
Nt = length(tau)-1;
T = cumsum(tau);
T = T(1:end-1); %the last time is beyond T
\end{minted}

\end{itemize}
\end{recall}

TODO: finish the coding part
