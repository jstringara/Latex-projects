\sheet

%>=====< Question 1 >=====<%

\question

Write the definition of measurable function. Show the measurability of the composite function.

\subsection*{Solution}

\subsection{Measurable function}

Let $(X,\A)$ and $(X',\A')$ be two measurable spaces and $f$ a function:
\[
    f:X\to X'
\]
$f$ is said to be measurable if:
\[
    f^{-1}(A)\in\A\quad\forall A\in\A'
\]

\subsection{Measurability of the composite function} \label{meas:comp}
Let $(X,\A)$, $(X',\A')$ and $(X'',\A'')$ be three measurable spaces and $f:X\to X'$ and $g:X'\to X''$ two measurable functions. Then the composite function $g\circ f:X\to X''$ is measurable.

\begin{proof}
    \begin{align*}
         & \forall E \in \A' \quad f^{-1}(E)\in\A   \\
         & \forall F \in \A'' \quad g^{-1}(F)\in\A' \\
    \end{align*}
    thus:
    \[
        \forall F\in \A'' \quad (g\circ f)^{-1}(F) = f^{-1} \left[ \underbrace{g^{-1}(F)}_{\coloneqq E \in \A'} \right] \in\A
    \]
\end{proof}

%>=====< Question 2 >=====<%

\question

Characterize measurability of functions and prove it.

\subsection*{Solution}

\subsection{Characterization of Measurability}\label{CharMeas}

Let $(X,\A)$ and $(X',\A')$ be two measurable spaces and $\mathcal{C} ' \subseteq \Parts{X'}$ such that $\sigma_0(\mathcal{C}')=\A'$ then:
\[
    f:X\to X' \text{ measurable } \iff f^{-1}(E)\in\A \quad \forall E\in\mathcal{C}'
\]

\begin{proof}
    Let us prove both sides of the implication:
    \begin{itemize}
        \item \textbf{$(\implies)$:} Suppose $f$ be measurable $\implies$ $\mathcal{C}'\subseteq \A'$ and so we get the thesis.
        \item \textbf{$(\impliedby)$:} Let us define the following:
              \[
                  \Sigma \coloneqq \{ E\subseteq X': \; f^{-1}(E)\in\A \}
              \]
              We can easily see that $\Sigma$ is a $\salg$ so $\mathcal{C}'\subseteq \Sigma$ and thus:
              \[
                  \A'=\sigma_0(\mathcal{C}')\subseteq\Sigma
              \]
              and we get the thesis.
    \end{itemize}
\end{proof}

%>=====< Question 3 >=====<%

\question

Write the definitions of:
\begin{enumerate}[a)]
    \item \label{Bmeas} Borel measurable functions;
    \item \label{Lmeas} Lebesgue measurable functions.
\end{enumerate}

\subsection*{Solution}

\subsection{\ref{Bmeas}) Borel measurable functions}
Let $(X,d), (X,\B)$ and $(X',d'), (X', \B')$ be couples of metric spaces and measurable spaces. A function $f$:
\[
    f:X\to X' \text{ measurable}
\]
is called Borel-measurable or $\B$-meaurable.

\subsection{\ref{Lmeas}) Lebesgue measurable functions}
Let $(X,\Leb)$ be a measurable space and $(X',d')$ a metric space, $(X',\B')$ a measurable space, then:
\[
    f:X\to X' \text{ measurable}
\]
is called Lebesgue-measurable or $\Leb$-measurable.

%>=====< Question 4 >=====<%

\question

Prove that continuous functions are both Borel and Lebesgue measurable.

\subsection*{Solution}

\subsection{Continuous functions are \texorpdfstring{$\B$}{Borel}-measurable}
A continuous function $f:X\to X'$ is $\B$-measurable.

\begin{proof}
    Let $\mathcal{C}'$ be the class of open sets of $X'$ and $\mathcal{C}$ the class of open sets of $X$. We have:
    \[
        \forall E \in \mathcal{C}' \quad f^{-1}(E)\in\mathcal{C} \subseteq \B \; (\text{ by definition of continuity })
    \]
    and $\B'=\sigma_0(\mathcal{C}')$ so we get the thesis.
\end{proof}

\subsection{Continuous functions are \texorpdfstring{$\Leb$}{Lebesgue}-measurable}
A continuous function $f:X\to X'$ is $\Leb$-measurable.

\begin{proof}
    Since $\B \subset \Leb$ and the previous statement has been proven true, the thesis follows trivially.
\end{proof}

%>=====< Question 5 >=====<%

\question

Characterize Lebesgue measurability of functions and prove it.

\subsection*{Solution}

\subsection{Characterization of Lebesgue measurability}
All we must do is apply the Characterization of Measurability (\ref{CharMeas}) taking $(X,\A=\Leb)$, $(X', \A'=\B')$ and $\mathcal{C}'$ the class of open sets of $X'$, since $\B'=\sigma_0(\mathcal{C}')$. We then can write:
\[
    f:X\to X' \text{ Lebesgue measurable } \iff f^{-1}(E)\in\Leb \quad \forall E\in\mathcal{C}'
\]

\begin{proof}
    Let us prove both sides of the implication:
    \begin{itemize}
        \item \textbf{$(\implies)$:} Suppose $f$ be Lebesgue measurable $\implies$ $\mathcal{C}'\subseteq \B'$ and so we get the thesis.
        \item \textbf{$(\impliedby)$:} Let us define the following:
              \[
                  \Sigma \coloneqq \{ E\subseteq X': \; f^{-1}(E)\in\Leb \}
              \]
              We can easily see that $\Sigma$ is a $\salg$ so $\mathcal{C}'\subseteq \Sigma$ and thus:
              \[
                  \B'=\sigma_0(\mathcal{C}')\subseteq\Sigma
              \]
              and we get the thesis.
    \end{itemize}
\end{proof}

%>=====< Question 6 >=====<%

\question

Establish and show all equivalent statements to the fact that $f : X \to \Rcomp$ is measurable.

\subsection*{Solution}

\subsection{Equivalent statements of measurability}
Let $(X,\A)$ be a measurable space and $f:X\to\Rcomp$ a function, \tfae

\begin{enumerate}[i)]
    \item \label{statomeas:1} $f$ is measurable;
    \item \label{statomeas:2} $\{ f>\alpha \}\in\A$ $\forall\alpha\in\R$;
    \item \label{statomeas:3} $\{ f\geq\alpha \}\in\A$ $\forall\alpha\in\R$;
    \item \label{statomeas:4} $\{ f<\alpha \}\in\A$ $\forall\alpha\in\R$;
    \item \label{statomeas:5} $\{ f\leq\alpha \}\in\A$ $\forall\alpha\in\R$.
\end{enumerate}

\begin{proof}
    Let us prove all the coimplications:\\
    \textbf{(\ref{statomeas:1}) $\iff$ (\ref{statomeas:3}):}
    \begin{align*}
         & \A'=\B[\Rcomp]=\sigma_0( \overbrace{ \{(\alpha,+\infty]:\; \alpha\in\R\} }^{\mathcal{C}'} )                   \\
         & f \text{ is measurable } \iff f^{-1} ( \underbrace{(\alpha,+\infty]}_{E} ) \in \A \quad \forall \alpha \in \R
    \end{align*}
    \textbf{(\ref{statomeas:2}) $\implies$ (\ref{statomeas:3}):}
    \[
        \{ f\geq \alpha \} = \cap_{n=1}^\infty \overbrace{\{ f > \alpha-\frac{1}{n} \}}^{\in\A} \in\A
    \]
    \textbf{(\ref{statomeas:3}) $\implies$ (\ref{statomeas:4}):}
    \[
        \{ f<\alpha \} = \{ f\geq\alpha \}^\complement \in\A
    \]
    \textbf{(\ref{statomeas:4}) $\implies$ (\ref{statomeas:5}):}
    \[
        \{ f \leq \alpha \} = \cap_{n=1}^\infty \overbrace{\{ f < \alpha+\frac{1}{n} \}}^{\in\A} \in\A
    \]
    \textbf{(\ref{statomeas:5}) $\implies$ (\ref{statomeas:2}):}
    \[
        \{ f>\alpha \} = \{ f\leq\alpha \}^\complement \in\A
    \]
\end{proof}

%>=====< Question 7 >=====<%

\question

Let $f, g \in \Mes(X, \A)$. What can we say about measurability of $\{f < g\},\; \{f \leq g\},\; \{f = g\}$? Justify the answer.

\subsection*{Solution}

\subsection{Measurability of \texorpdfstring{$\{f < g\},\; \{f \leq g\},\; \{f = g\}$}{ \{f less than g\}, \{f less or equal to g\}, \{f equal to g\}}}
Let $f, g \in \Mes(X, \A)$, we have:
\begin{enumerate}[i)]
    \item $\{f < g\}\in\A$
    \item $\{f \leq g\}\in\A$
    \item $\{f = g\}\in\A$
\end{enumerate}

\begin{proof}
    \hspace*{\fill} %leave a blank line
    \begin{enumerate} [i)]
        \item $\{ f < g \} = \bigcup_{r\in\Q} \left[ \underbrace{\overbrace{\{ f < r \}}^{\in\A} \cap \overbrace{\{ r < g\}}^{\in\A}}_{\in\A} \right]$
        \item $ \{ f\leq g \} = \{ f > g \}^\complement\in\A$ by the previous point.
        \item $\{f=g\} = \underset{\in\A}{\{ f\leq g \}} \cap \underset{\in\A}{\{ f\geq g \}} \in \A$
    \end{enumerate}
\end{proof}

%>=====< Question 8 >=====<%

\question

Let $\{f_n\} \subset \Mes(X, \A)$. Show that $\sup_n f_n, \inf_n f_n, \limsup_n f_n, \liminf_n f_n \in \Mes(X, \A)$. Can there exist two functions $f, g \in \Mes(X, \A)$ such that $\max\{f, g\} \notin \Mes(X, \A)$? Why?

\subsection*{Solution}

\subsection{Measurability of \texorpdfstring{$\sup_n f_n, \inf_n f_n, \limsup_n f_n, \liminf_n f_n$}{sup fn, inf fn, limsup fn, liminf fn}} \label{meas:extremes}
Let $\{f_n\} \subset \Mes(X, \A)$, we have:
\begin{enumerate}[i)]
    \item $\sup_n f_n \in \Mes(X, \A)$
    \item $\inf_n f_n \in \Mes(X, \A)$
    \item $\limsup_n f_n \in \Mes(X, \A)$
    \item $\liminf_n f_n \in \Mes(X, \A)$
\end{enumerate}

\begin{proof}
    \hspace*{\fill} %leave a blank line
    \begin{enumerate} [i)]
        \item $\forall \alpha \in \R \quad \{ \sup_{n\in\N} f_n > \alpha \} = \bigcup_{n=1}^\infty \{ f_n > \alpha \} \in \A \implies \sup_{n\in\N} f_n \in \Mes$
        \item $\inf_n f_n = - \sup_{n\in\N} (-f_n) \in \Mes(X,\A)$
        \item $\limsup_n f_n = \inf_{k\geq 1} \sup_{n \geq k} f_n \in \Mes(X, \A)$
        \item $\liminf_n f_n = \sup_{k\geq 1} \inf_{n\geq k} f_n \in \Mes(X, \A)$
    \end{enumerate}
\end{proof}

\subsection{Corollary, \texorpdfstring{$\max\{f,g\}\in\Mes$}{max(f,g) is measurable}}
There cannot exist two functions $f, g \in \Mes(X, \A)$ such that $\max\{f, g\} \notin \Mes(X, \A)$.
Indeed we can write:
\[ \max(f,g) = f\cdot \chi_{\{f\geq g\}} + g \cdot \chi_{\{g>f\}} \]
In other words the max can be written as the sum and product of measurable functions, hence it is measurable itself.

%>=====< Question 9 >=====<%

\question

Let $f, g \in \Mes(X, \A)$. Show that $f + g, f\cdot g \in \Mes(X, \A)$.

\subsection*{Solution}

\subsection{Measurability of \texorpdfstring{$f + g, f\cdot g$}{sum f and g, product f and g}} \label{meas:sumprod}
Let $f,g:X\to\R$ and $f,g\in\Mes(X,\A)$, we have that $f+g, f\cdot g \in \Mes(X, \A)$.

\begin{proof}
    Let us define a few new functions $\phi, \psi$ and $\chi$:
    \[
        \left\{ \begin{array}{l}
            \phi(x) = X\to\R^2 \quad \phi(x) \coloneqq \left( f(x), g(x) \right) \\
            \psi(x) = \R^2\to\R \quad \psi(s,t) \coloneqq s+t                    \\
            \chi(x) = \R^2\to\R \quad \chi(s,t) \coloneqq s\cdot t
        \end{array} \right.
        \implies
        \left\{ \begin{array}{l}
            \psi \circ \phi = f + g \\
            \chi \circ \phi = f \cdot g
        \end{array} \right.
    \]
    Now, clearly $\psi,\chi \in C^0(\R^2)$ (hence measurable), let us prove that $\phi $ is also measurable. We use the Characterization of Measurability (\ref{CharMeas}):
    \[
        \phi: (X,\A)\to (\R^2,\B[\R^2]) \text{ is measurable } \iff \forall E \subseteq \R^2 \text{ open } \phi^{-1}(E) \in \A
    \]
    We take:
    \begin{align*}
        E            & = R \coloneqq (a,b) \times (c,d)                                                         \\
        \phi^{-1}(R) & \tikzmarknode{eq1}{=} \{ x\in X: \; (f(x),g(x))\in R \}                                  \\
                     & \tikzmarknode{eq2}{=} \{ x\in X: \; f(x)\in (a,b) \} \cap \{ x\in X: \; g(x)\in (c,d) \} \\
                     & \tikzmarknode{eq3}{=} f^{-1}(a,b) \cap g^{-1}(c,d) \in \A
    \end{align*} \tikz[overlay,remember picture]{\draw[shorten >=1pt,shorten <=1pt] (eq1) -- (eq2) -- (eq3);}
    Thus $\forall E \subseteq \R^2$ open, we may write:
    \begin{align*}
        E         & = \bigcup_{k=1}^\infty R_k, \quad R_k = (a_k,b_k) \times (c_k,d_k)                                       \\
        \phi^{-1} & = \bigcup_{k=1}^\infty \phi^{-1}(R_k) = \bigcup_{k=1}^\infty f^{-1}(a_k,b_k) \cap g^{-1}(c_k,d_k) \in \A
    \end{align*}
    Hence $\phi \in \Mes(X,\A)$, and we have:
    \[
        \psi \circ \phi, \; \chi \circ \phi \in \Mes(X,\A)
    \]
\end{proof}

%>=====< Question 10 >=====<%

\question

Prove that A is measurable if and only if $\chi_A$ is a measurable function.

\subsection*{Solution}

\subsection{A is measurable if and only if \texorpdfstring{$\chi_A$}{the indicator function of A} is a measurable function} \label{AinA:chi}

Let $A\subseteq X$ and $\chi_A$ be the indicator function of $A$. We have:
\[
    \chi_A \in \Mes(X, \A) \iff \quad A \in \A
\]

\begin{proof}
    \[
        \{ \chi_A > \alpha \}  = \left\{ \begin{array}{l}
            X \quad \alpha <0         \\
            A \quad 1 > \alpha \geq 0 \\
            \emptyset \quad \alpha \geq 1
        \end{array} \right.
    \]
    Now, $X,\emptyset\in\A$ by definition, so:
    \[
        A\in\A \iff \chi_A \in \Mes
    \]
\end{proof}

%>=====< Question 11 >=====<%

\question

Prove or disprove the following statements:
\begin{enumerate}[a)]
    \item\label{fmeas:fpmmeas} $f \in \Mes(X, \A) \iff f_{\pm} \in \Mes_+(X, \A)$;
    \item\label{fmeas:fabsmeas} $f \in \Mes(X, \A) \iff |f| \in \Mes(X, \A)$.
\end{enumerate}

\subsection*{Solution}

\subsection{Measurability of \texorpdfstring{$f_{\pm}$}{f positive, f negative} and \texorpdfstring{$|f|$}{absolute value of f}}

Let $f:X\to\R$, we have:
\begin{enumerate}[i)]
    \item $f \in \Mes(X, \A) \iff f_{\pm} \in \Mes_+(X, \A)$
    \item $f \in \Mes(X, \A) \iff |f| \in \Mes(X, \A)$
\end{enumerate}

\begin{proof}
    \hspace*{\fill} %leave a blank line
    \begin{enumerate} [i)]
        \item \begin{itemize}
                  \item \textbf{($\implies$):} if $f \in \Mes(X, \A)$, then we define $f_+$ as:
                        \[
                            f_+(x) = \max\{f(x),0\} \geq 0 \quad \forall x\in X
                        \]
                        and since $f,0 \in \Mes(X,\A)$ and $\max$ is a measurable function we have that $f_+ = \max \circ (f,0) \in \Mes_+(X,\A)$ by (\ref{meas:comp}). We may analogously prove the same for $f_-$.
                  \item \textbf{($\impliedby$):} if $f_+ \in \Mes_+(X, \A)$, then we define $f = f_+ - f_-$, and since $f_+,f_-,f \in \Mes(X,\A)$ we have that $f \in \Mes(X,\A)$ by (\ref{meas:sumprod}).
              \end{itemize}
        \item $f\in\Mes \implies f_{+}, f_{-} \in \Mes$ by the previous point $\implies |f|=f_+ + f_- \in \Mes$ by (\ref{meas:sumprod}).
    \end{enumerate}
\end{proof}

%>=====< Question 12 >=====<%

\question

Write the definition of simple function. What is its canonical form? How can we characterize measurability of a simple function? Write the definition of step function.

\subsection*{Solution}

\subsection{Definition of simple function}
Let $X$ be a set and $s:X\to\R$ a function. We say that $s$ is a simple function if $s(X)$ is a finite set. \\
Furthermore we define the two sets:
\begin{align*}
     & \Smes(X, \A) \coloneqq \{ \text{ measurable simple functions}  \}                           \\
     & \Smes_+(X, \A) \coloneqq \{ \text{ measurable simple functions with non-negative values} \}
\end{align*}

\subsection{Canonical form of simple function} \label{simple:canon}
The canonical form of a simple function is:
\[
    s(x) = \sum_{i=1}^n c_i \chi_{E_i}(x)
\]
where:
\begin{align*}
     & c_i \in \R \; \forall i=1,\dots,n                                      \\
     & E_i = \{ x\in X : \; s(x)=c_i \} \; \forall i=1,\dots,n                \\
     & X = \bigcup_{i=1}^n E_i, \; E_k\cap E_l = \emptyset \; \forall k\neq l
\end{align*}
i.e. $E_i$ is a partition of $X$.

\subsection{Measurability of simple function}
A simple function is measurable if and only if we have the following:
\[
    E_i \in \A \; \forall i=1,\dots,n
\]
i.e. :
\[
    s(x)\in \Mes(X,\A) \iff E_i \in \A \; \forall i=1,\dots,n
\]
this is because $s(x)$ is a linear combination of indicator functions.

\subsection{Step Functions}
Let $I=[a_0,a_1)$ be an interval and $P=\{ a_o \equiv x_0 < x_1 < \dots < x_n \equiv a_1 \}$ a partition of $I$. A function $f:I\to\R$ is a step function if:
\[
    f \coloneqq \sum_{k=0}^{n-1} c_k \chi_{[x_k,x_{k+1})} (x)
\]

%>=====< Question 13 >=====<%

\question

State and give a sketch of the proof of the Simple Approximation Theorem.

\subsection*{Solution}

\subsection{Simple Approximation Theorem}\label{SAT}
Let $(X,\A)$ be a measurable space and $f:X\to\Rcomp$. Then there exists a sequence of simple functions $\seq{s}$ such that:
\[
    s_n \xrightarrow{n\to\infty} f \text{ in } X \text{ (pointwise)}
\]
\textbf{Furthermore:}
\begin{enumerate}[i)]
    \item if $f\in\Mes(X,\A)$, then $\seq{s}\subseteq\Smes(X,\A)$;
    \item if $f\geq 0 \implies \seq{s} \uparrow$, $0\leq s_n \leq f$;
    \item $f$ bounded $\implies s_n \xrightarrow{n\to\infty} f$ uniformly in $X$.
\end{enumerate}

\subsection*{Sketch of proof}
Let $f\geq0$, bounded and $0 \leq f \leq 1$ $\forall x\in X$.
\[
    f : X \to [0,1]
\]
Let us divide $[0,1]$ in $2^n$ intervals of equal length $\forall n \in \N$, then we define:
\begin{align*}
     & E_k^{(n)} \coloneqq \left\{ x\in X: \; \frac{k}{2^n} \leq f(x) \leq \frac{k+1}{2^n} \right\} \quad k = 0,\dots,2^n-1 \\
     & s_n \coloneqq \sum_{k=0}^{2^n-1} \frac{k}{2^n} \chi_{E_k^{(n)}}(x) \quad \forall n\in\N
\end{align*}
Clearly $\seq{s}$ has the desired properties.

%>=====< Question 14 >=====<%

\question

Write the definitions of $\esssup_X f$ and $\essinf_X f$. State their properties and prove some of them.

\subsection*{Solution}

\subsection{Definition of \texorpdfstring{$\esssup_X f$}{essup f}}
Let $(X, \A, \mu)$ be a measure space and $f$ a function on $X$. We define:
\[
    \esssup_X f(x) \coloneqq \inf \left\{ \sup_{x\in N^\complement} f(x) : \; N \in \mathcal{N}_{\mu} \right\}
\]

\subsection{Definition of \texorpdfstring{$\essinf_X f$}{essinf f}}
Let $(X, \A, \mu)$ be a measurable space and $f$ a function on $X$. We define:
\[
    \essinf_X f(x) \coloneqq \sup \left\{ \inf_{x\in N^\complement} f(x) : \; N \in \mathcal{N}_{\mu} \right\}
\]

\subsection{Properties of \texorpdfstring{$\esssup_X f$}{essup f} and \texorpdfstring{$\essinf_X f$}{essinf f}}
Let $(X, \A, \mu)$ be a measure space and $f,g\in\Mes(x,\A)$ two functions on $X$. We have that:
\begin{enumerate}[i)]
    \item $\exists N \in \mathcal{N}_\mu$ such that $\esssup_X f = \sup_{x\in N^\complement} f$ and $f\leq \esssup_X f$ almost surely $x\in X$;
    \item $\esssup_X f = -\essinf_X -f$;
    \item $\esssup_X k\cdot f  = k \cdot \esssup_X f$;
    \item $f\leq g \text{ a.e. in } X \implies \esssup_X f \leq \esssup_X g$;
    \item $\esssup_X (f+g) \leq \esssup_X f + \esssup_X g$;
    \item $f = g$ almost everywhere in $X$ $\implies \esssup_X f = \esssup_X g$;
    \item $g \geq 0$ almost everywhere in $X$ $\implies f\cdot g \leq (\esssup_X f)\cdot g$ almost everywhere in $X$.
\end{enumerate}

\begin{proof}
    Let us give a partial proof:
    \begin{enumerate}[i)]
        \item Suppose $\esssup_X f < +\infty$, $\forall k\in\N$ $\exists N_k \in \mathcal{N}_\mu$ such that:
              \[
                  \sup_{x\in N_k} f < \esssup_X f + \frac{1}{k}
              \]
              We define $N \coloneqq \bigcup_{k=1}^\infty N_k$. Then $N \in \mathcal{N}_\mu$ and:
              \begin{align*}
                   & N^\complement = \bigcap_{k=1}^\infty N^\complement_k \subseteq N^\complement_k \quad \forall k \in \N                              \\
                   & \implies  \esssup_X f \leq \sup_{N^\complement} f \leq \sup_{N^\complement_k} f < \esssup_X f + \frac{1}{k} \quad \forall k \in \N
              \end{align*}
              Now we pass apply a limit $k\to+\infty$ and we get:
              \begin{align*}
                   & \sup_{N^\complement} f = \esssup_X f                                                             \\
                   & N \supseteq \bar{N} \coloneqq \{ x\in X: \; f(x) > \esssup_X f(x) \} \in\A                       \\
                   & \implies \bar{N} \in \mathcal{N}_\mu \implies f \leq \esssup_X f \text{ almost everywhere in } X
              \end{align*}
    \end{enumerate}
\end{proof}

%>=====< Question 15 >=====<%

\question

What is $\Leb^\infty$? Which is the relation between functions finite a.e. and essentially bounded functions? Justify the answer.

\subsection*{Solution}

\subsection{Definition of \texorpdfstring{$\Leb^\infty$}{the set of essentially bounded functions}}
Let $(X, \A, \mu)$ be a measure space. A function $f\in\Mes(X,\A)$ is said to be essentially bounded if:
\[
    \esssup_X f < +\infty
\]
and we define the set of essentially bounded functions as:
\[
    \Leb^\infty (X,\A,\mu) \coloneqq \{ f:X\to\Rcomp : \; \text{ f is essentially bounded }\}
\]

\subsection{Relation between functions finite a.e. and essentially bounded functions}
We have that:
\begin{enumerate}
    \item $f\in\Leb^\infty \implies f$ is finite a.e. in $X$;
    \item in general if $f$ is finite a.e. in $X$ $\centernot\implies f\in\Leb^\infty$.
\end{enumerate}

\begin{proof}
    \begin{enumerate}
        \item We can easily see that:
              \[
                  |f| \leq \esssup |f| < + \infty \text{ almost everywhere in } X
              \]
              thus $f$ is finite almost everywhere in $X$;
        \item Let us assume that:
              \[
                  f \text{ is finite a.e. in }  X \implies f\in\Leb^\infty
              \]
              and let us see a clear counterexample of this, take:
              \[
                  f(x): \R \to \Rcomp \coloneqq \begin{cases}
                      \frac{1}{|x|} & x \neq 0 \\
                      + \infty      & x = 0
                  \end{cases}
              \]
              Clearly $f$ is finite in $E=\R\setminus\{0\}$, i.e. $f$ is finite a.e. in $\R$. Let us note that $\lambda(\{0\}) = 0 $. Thus:
              \[
                  \esssup_X |f| = +\infty \implies f \notin \Leb^\infty
              \]
    \end{enumerate}
\end{proof}
