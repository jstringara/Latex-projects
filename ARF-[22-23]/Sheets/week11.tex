\sheet

%>=====< Question 1 >=====<%

\question
State and prove the three corollaries of the Hanh-Banach Theorem.

\subsection*{Solution}

\subsection{First corollary of the Hahn-Banach theorem}
Let $x_0 \in X \backslash \{0\}$, $X$ normed space, then $\exists L_{x_0} \in X^* \mbox{ s.t.: }$ \[
\|L_{x_0}\|_{X^*} =1,\quad L_{x_0}(x_0) = \|x_0 \|
\]

\begin{proof}
Let $Y\coloneqq \{\lambda x_0 : \lambda \in \R\}=\text{Span}(x_0)$, $Y$ is a v.s.s. of $X$.
\[
L_0:Y\to \R ,\quad L(\lambda x_0) = \lambda \|x_0\|,\quad L_0 \in \Leb
\]
By H-B theorem: \[
\exists \Tilde{L}_0: X \to \R , \quad \Tilde{L}_0 \in X^*, \quad \|\Tilde{L}_0\|_{X^*}= \sup_S |\underbrace{\lambda \|x_0\|}_{=L_0(\lambda x_0)}| = 1 
\]
where $S = \{\lambda x_0 : \lambda x_0 \in Y , \| \lambda x_0\| = 1\}$, moreover:
\[
\Tilde{L}_0 = L_0(x_0) = \|x_0\| \implies L_{x_0} \coloneqq \Tilde{L}_0 \text{ satisfies the required properties }
\]
\end{proof}

\subsection{Second corollary of the Hahn-Banach theorem}
Let $y,z \in X$, if $L(y)=L(z),\forall L \in X^*$ then $ y=z$.

\begin{proof}
Suppose by contradiction that $\exists y,z \in X, y\neq z$ s.t. $L(y)=L(z)\quad\forall L \in X^*$, then:
\[
x \coloneqq y-z \neq 0 \implies L(x)=L(y-z)=L(y)-L(z)=0\quad\forall L \in X^*
\]
and by the preceding corollary:
\[
\exists L_x \in X^* \mbox{ s.t. } L(x) = \|x\| \neq 0, \mbox{ contradiction}
\]
\end{proof}

\subsection{Third corollary of the Hahn-Banach theorem}
Let $Y \subseteq X$ be a v.s.s. with $\Bar{Y}\neq X$, $x_0 \in X\backslash \Bar{Y}.$ Then:
\[
\exists L \in X^*: L(x_0)\neq 0,\quad L\arrowvert_Y=0
\]

\begin{proof}
Let $Z \coloneqq \{\lambda x_0 + y:\,y\in Y,\,\lambda \in R\} \subset X$ and define:
\[
L_0: Z\to\R,\quad L_0(\lambda x_0 + y) \coloneqq \lambda
\]
Notice that $L_0(x_0) = L_0(1\cdot x_0+0\cdot y) = 1$ and:
\[
\ker(L_0)= \{\lambda x_0 + y \in Z :\,\underbrace{L_0(\lambda x_0 + y)=0}_{\lambda=0}=Y\} \implies L\arrowvert_Y=0
\]
By H.-B. theorem: 
\[
\exists \Tilde{L}_0 \in X^*: \Tilde{L}_0 = L_0\text{ in }Z \supseteq Y \implies L \coloneqq \Tilde{L}_0 \text{ satisfies } L(x_0)  \neq 0,\;  L\arrowvert_Y=0
\]
\end{proof}

%>=====< Question 2 >=====<%

\question
Give a sufficient condition for separability of $X$.

\subsection*{Solution}

\subsection{Sufficient condition for separability}
Let $X$ be a normed space:\[
X^* \text{ separable } \implies X \text{ separable }
\]

%>=====< Question >=====<%

\question
Write the definition of uniformly convex normed space. What is the geometric interpretation?

\subsection*{Solution}

\subsection{Uniform convexity}
Let $X$ be a normed spcae.
$X$ is uniformly convex if $\forall \epsilon >0 \;\exists \,\delta$ s.t.:\[
\forall x,y\in X,\;\|x\|\le 1,\;\|y\|\le 1,\;\|x-y\|\ge \epsilon \implies \|\tfrac{x-y}{2}\|\le 1-\delta \\[0pt]
\]We can think of it as two different points on the unit sphere imply that the center of the line segment that connects them has to lie inside the unit sphere.

%>=====< Question >=====<%

\question
Write the Clarkson's inequalities in $L^p$. Show that $L^p$ is uniformly convex for any $ p \in(1, \infty)$.

\subsection*{Solution}

\subsection{Clarkson's inequality}
Given $f,g \in L^p(\Omega)$, $\Omega \in \Leb(R^N)$, we have:
\newline\newline
$\begin{array}{rl}
p \ge 2: &     \qquad\|\frac{f+g}{2}\|_p^p + \|\frac{f-g}{2}\|_p^p  \le \frac12 \big (\|f\|_p^p + \|g\|_p^p \big)\\[4pt]
p \in (1,2): & \qquad\|\tfrac{f+g}{2}\|_p^q + \|\tfrac{f-g}{2}\|_p^q  \le \big (\frac12\|f\|_p^p + \frac12\|g\|_p^p \big)^{\frac1{p-1}}
\end{array}$
\subsection{\texorpdfstring{$L^p(\Omega)$}{Lp on Omega} is uniformly convex}
\begin{proof}
Take $\epsilon>0,\;f,g \in L^p(\Omega),\;\|f\|\le 1,\;\|g\|\le 1,\;\|f-g\|\ge \epsilon$, we have:$\\[4pt]$
$p\ge 2$:
\[
\big\|\frac{f+g}{2}\big\|_p^p < 1- \big(\frac{\epsilon}{2}\big)^p
\]
\[
\implies  \big\|\frac{f+g}{2}\big\|_p < 1 - \delta,\quad \delta \coloneqq 1-\big[1-\big( \tfrac{\epsilon}{2}\big)^p\big]^{\tfrac1p}
\]
$1<p<2$:
\[
\big\|\frac{f+g}{2}\big\|_p^q = 1- \big(\frac{\epsilon}{2}\big)^q
\]
\[
\implies \big\|\frac{f+g}{2}\big\|_p < 1 - \delta,\quad \delta \coloneqq 1-\big[1-\big( \tfrac{\epsilon}{2}\big)^q\big]^{\tfrac1q}
\]
\end{proof}

%>=====< Question >=====<%

\question
Write the definition of bidual space. Introduce the canonical (or evaluation) map. Show that it is linear and that it preserves the norm.

\subsection*{Solution}

\subsection{Bidual space}
Let $X $ be a normed space and $X^* $ its dual. We defie the bidual of $X$ as:
\[
X^{**} \coloneqq (X^*)^*
\]

\subsection{Canonical map}
$\forall x \in X$ we define $\Lambda _x: X^* \to \R$ as: 
\[\Lambda _x(L) \coloneqq L(x),\quad \forall L \in X^*\]
\[
\Lambda_x \in \Leb :\quad |\Lambda _x(L)| = |L(x)| \le \|L\|_{X^*} \underbrace{\|x\|_X}_{\le M}
\]
\[
\implies \Lambda _x \in X^{**},\quad \|\Lambda _x\|_{X^{**}} \le \|x\|_X
\]
Finally we can define the canonical map as
\[ \tau: X \to X^{**}, \quad \tau(x) = \Lambda _x,\quad \forall x \in X
\]

\subsection{Properties of the canonical map}

\begin{itemize}
    \item [i)] $\tau$ is linear
    \item [ii)] $\|\tau(x)\|_{X^{**}}=\|x\|_X$
    \item [iii)] $\tau(x)$ is injective
\end{itemize}

\begin{proof} $\\$
Linearity is trivial and the injectivity comes directly from ii) which is the only one we are proving, clearly: \[\|x\|_X \ge \|\tau(x)\|_{X^{**}},\;\forall x \in X\\ \]
By a corollary of the H-B theorem:
\[
\forall x \in X\backslash\{0\}\; \exists L \in X^*\mbox{ s.t. }\|L\|_{X^{**}} = \|x\|_X
\]
and thus:
\[
\|\tau(x)\|_{X^{**}} = \|\Lambda \|_{X^{**}}=\displaystyle\sup_{\|x\|_X=1} |\Lambda _x(L)| \ge \|x\|_X
\]
\end{proof}

%>=====< Question >=====<%

\question
Write the definition of reflexive space. Let $X$ be a reflexive space and $\phi\in X^{**}$. What about $\phi(L)$ for any $L\in X^*$?

\subsection*{Solution}

\subsection{Reflexive space}
$X$ is said to be reflexive if $\tau(X) = X^{**} .$

\subsection{Characterization of reflexive spaces}
\[X \text{ reflexive } \iff \forall \phi \in X^{**},\;\phi(L)=L(x),\;\forall L\in X^*
\] 

%>=====< Question >=====<%

\question
State the Milman-Pettis theorem. Show that $L^p$ is reflexive for any $p \in (1,+\infty)$. What about $L^1 $ and $L^\infty$?

\subsection*{Solution}

\subsection{Milman-Pettis theorem}
Let $X$ be an uniformly convex Banach space, then $X$ is reflexive.
\subsection{Reflexivity of \texorpdfstring{$L^p$}{Lp}}
$L^p$ is reflexive $\forall p \in (1,+\infty)$.

\begin{proof}
This follows immediately by Milman-Pettis theorem since for $ p \in (1,+\infty)$, $L^p$ is uniformly convex.\newline
Moreover, it can be proved that $L^1 $ and $L^\infty$ are not reflexive.
\end{proof}

%>=====< Question >=====<%

\question
State and prove the Riesz theorem in $L^p$ spaces.

\subsection*{Solution}

\subsection{Riesz theorem in \texorpdfstring{$L^p$}{Lp}}
Let $(X,\A,\mu) $ be a complete measure space, $p\in (1,+\infty)$. For any $\Lambda \in (L^p)^*,\,\exists! g \in L^q $ s.t.:
\[
\Lambda(f)  = \int_X fg\,d\mu \quad \forall f \in L^p
\]
Furthermore: 
\[
\|\Lambda \|_{(L^p)^*} = \|g\|_{L^q}
\]
\begin{proof}
Let $p \in (1,+\infty),$ we define: \[
T: L^q \to (L^p)^*\]
\[ u\mapsto T(u): L^p \to \R \]
\[ [T(u)](f) \coloneqq \displaystyle\int_{\Omega} fu\,d\lambda\]
Now note that our thesis is equivalent to showing that $T$ is surjective, and to prove the latter let:\[
E\coloneqq T(L^q)\subseteq (L^p)^*, \,E\text{ is closed}
\]
and let
\[
H \in (L^p)^{**} \quad(\iff H\in \Leb : (L^p)^* \to \R)
\]
be such that:   
\[
H\arrowvert_E = 0 \quad (\iff H(T(u))=0 \;\,\forall u \in L^q)
\]
\[
\iff H(T) = T(h) \text{ (since $L^p$ is reflexive)}, \,\,h\coloneqq \tau^{-1} (H) \in L^p
\]
\[
\implies \forall u \in L^q,\; \int_\Omega hu\,d\lambda = 0
\]
Let $u \coloneqq |h|^{p-2}h,\;u\in L^q$ since $h \in L^p$
\[
\implies \int_\Omega |h|^p \,d\lambda = 0 \implies h\equiv 0 \text{ in } L^p \underset{H=\tau(0)}{\Longrightarrow} H=0 \text{ in } (L^q)^* \implies \Bar{E} = E = (L^p)^*
\]


\end{proof}
We can also prove that the same thesis holds for $p=1, q=\infty$, provided that $\mu$ is $\sigma$-finite.

%>=====< Question >=====<%

\question
Show that the dual of $L^\infty$ is "strictly bigger" than $L^1$.

\subsection*{Solution}

\subsection{\texorpdfstring{$(L^\infty)^* \supsetneq L^1$}{The dual of L infinity is strictly bigger than L1}}
$\exists L \in (L^\infty)^* $ s.t. $L $ is not in the form $Lg$ with $g\in L^1$.
\begin{proof}
Indeed consider $L_0 \in (C_c^0 (\R^N))^*$, we have 
 that $\big(C_c^0 (\R^N),\|\cdot\|_\infty\big)$ is a subset of $L^\infty$.
 \[L_0(f)\coloneqq f(0),\quad \forall f \in C_c^0 (\R^N)\]
Clearly $L_0$ is bounded, moreover:
 \[
 |L_0(f)| = |f(0)|  \le \|f\|_\infty\quad \forall f \in C_c^0 (\R^N) \implies L_0 \text{ is bounded }
 \]
\[
\overset{\text{H.B.}}{\Longrightarrow} \exists L \in [L^\infty(\R^N)]^* \text{ extension of } L_0
\]
Now we claim that:
\[
\nexists g \in L^1 \mbox{ s.t. } L(f)=\int_{\R^N}fg\,d\lambda,\quad \forall f \in L^\infty
\]    
Suppose by contradiction that such a $g$ exists, then:
\[
L(f)=L_0(f) = \int_{\R^N}fg\,d\lambda = f(0) = 0 
\]
\[
f \in C_c^0,\,f(0)=0 \implies g =0 \text{ a.e. in } \R^N
\]
\[
\implies L(f) = \int_{\R^N}0\cdot f\,d\lambda = 0\quad \forall f \in L^\infty
\]
\[
 f \in C_c^0 (\R^N),\; f(0) \neq 0 \implies L(f)=f(0)\neq 0 \implies \text{ contradiction}
\]

\end{proof}

%>=====< Question >=====<%

\question
Write the definition of weak convergence. How can it be formulated in $L^p$ and in $\ell^p$?

\subsection*{Solution}

\subsection{Weak convergence}
Let $X $ be a Banach space, $\{x_n\} \subset X$, $x\in X$, we say that $x_n \wc x$ if $L(x_n) \to L(x) \quad \forall L\in X^*$

\subsection{Weak convergence in \texorpdfstring{$L^p$}{Lp}}
In $L^p(\Omega),\text{ with }p\in[1,+\infty) $ we have that:
\[
f_n \wc f \iff \int_\Omega f_n g\,d\lambda \xrightarrow{\toi} \int_\Omega f g\,d\lambda  \quad\forall g \in L^q \overset{p\neq 1}{\iff} \int_\Omega f_n \phi\,d\lambda \xrightarrow{\toi} \int_\Omega f \phi\,d\lambda \quad\forall \phi \in C^1_c(\Omega) 
\]

\subsection{Weak convergence in \texorpdfstring{$ell^p$}{lp}}
In $\ell^p(\Omega),\text{ with }p\in[1,+\infty) $ we have that:
\[x_n \wc x \iff \sum_{k=1}^\infty x_n^{(k)} y^{(k)} \xrightarrow{\toi} \sum_{k=1}^\infty x^{(k)} y^{(k)}  \quad\forall y \in \ell^q ) \]

%>=====< Question >=====<%

\question
Show that strong convergence implies weak convergence. Provide a counterexample for the
converse implication.

\subsection*{Solution}

\subsection{Strong convergence implies weak convergence}
$x_n \to x \implies x_n \wc x$
\begin{proof}
   $\forall L \in X^*:$
    \[
    |L(x_n)-L(x)| = |L(x_n-x)| \le \|L\|_{X^*}\overbrace{\|x_n-x\|}^{\to 0}
    \]
    \[
    \implies L(x_n) \xrightarrow{\toi} L(x)
    \]
\end{proof}\noindent
We claim that the converse is not true, for instance, let $X=\ell^2,\; \{e_n\}\subset\ell^2,\;e_n^{(k)}=\delta_{k,n}$
\[
e_n \wc 0 \iff T(e_n) \to T(0) = 0\quad \forall T \in (\ell^2)^*
\]
\[
\forall x \equiv x^{(j)} \in \ell^2,\; \sum e^{(j)} x^{(j) }= x^(n) \to 0 \implies e_n \wc 0
\]
but:
\[
\|e_n\|_2 = 1 \quad \forall n \in \N \implies e_n \nrightarrow 0
\]
