\documentclass[a4paper]{report}


\usepackage[T1]{fontenc} % codifica dei font per l'italiano
\usepackage[utf8]{inputenc} % lettere accentate da tastiera
\usepackage[italian]{babel} % lingua del documento
\usepackage[shortlabels]{enumitem}%per fare elenchi ad hoc
\usepackage{framed}% per riquadrare il testo
\usepackage{amsmath}% per avere le formule matematiche fatte bene
\usepackage{amssymb}% per avere le formule matematiche fatte bene
\usepackage{amsthm}%per fare i teoremi in modo ordinato
\usepackage[big]{layaureo} %per avere i margini più stretti
\usepackage{graphicx} %per inserire le immagini
\usepackage{mathtools}
\usepackage{mathabx}


\newcommand{\parallelsum}{\mathbin{\|}}%parallelo


\begin{document}

\begin{center}
\section*{Formulario Meccanica}
\end{center}

\subsection*{Definizioni}
\textbf{Vettore AB:}
\[
AB = A\longrightarrow B
\]
\textbf{Vincoli:}
\begin{itemize}
    \item \textbf{Olonomo:} vincolo in funzione solo delle coordinate.
    \item \textbf{Anolonomo:} vincolo anche in funzione delle derivate.
    \item \textbf{Unilatero:} vincolo con $\Phi > 0$
    \item \textbf{Bilatero:} vincolo con $\Phi\gtreqless 0 $
\end{itemize}
\textbf{Forze:}
\begin{itemize}
    \item \textbf{Attive:} Prescritte a priori da leggi
    \item \textbf{Reattive:} Reazione Vincolare, sconosciuta a priori
    \item \textbf{Interne}
    \item \textbf{Esterne}
\end{itemize}
\subsection*{Velocità}
\textbf{Atto di moto rigido:} 
\[
\vec{v}(P)=\vec{v}(Q)+\vec{\omega}\wedge \vec{QP}
\]
\textbf{Invariante cinematico: } 
\[
I=v(P)\cdot\omega    
\]
\textbf{Legge di composizione delle velocità: }
\[
\vec{v}_{ass}(P)=\vec{v}_{rel}(P) + \vec{v}_{trasc} = \vec{v}_{rel} +\vec{v}(Q) + \vec{\omega}\wedge \vec{QP}   
\]

\subsection*{Forze}
\textbf{Risultante: }
\[
\vec{R} = \sum_i \vec{f}_i    
\]
\textbf{Momento: }
\[
    \vec{M}_A=\sum_i \vec{AP}_i\wedge\vec{f}_i    
\]
\textbf{Legge cambiamento di polo: }
\[
\vec{M}_B=\vec{M}_A+\vec{R}\wedge \vec{AB}    
\]
\textbf{Invariante scalare: }
\[
I=\vec{M}_A\cdot \vec{R}
\]
\textbf{Reazione Vincolare per RSS: }
\[
    \mu\cdot\Phi_N \geq \Phi_T \longrightarrow \mu \geq \frac{\Phi_T}{\Phi_N}
\]


\subsection*{Statica}
\textbf{Equazioni Cardinali: }
\[
\vec{R}^{Est}=0 \quad \vec{M}^{Est}=0      
\]
\textbf{Principio dei Lavori Virtuali (PLV):}
In un sistema con vincoli ideali:
\[
\delta L^{att} = 0 \quad \forall \delta P     
\]
\textbf{Principio di Stazionarietà del Potenziale (PSP)}\\
Per forze attive conservative (es. elastica e gravitazionale):
\[
\exists U(q_k): Q_k = \frac{\partial U}{\partial q_k} = 0 \quad \forall q_k \iff \text{Equilibrio}
\]

\subsection*{Dinamica}
\textbf{Centro di Massa: }
\[
    x_G=\frac{\sum_i m_i x_i}{\sum_i m_i}=\frac{\int x\rho (x) dx}{\int  \rho (x) dx}    
\]
\textbf{Baricentro:}
\[
x(G)=\frac{\sum p_i x_i}{\sum p_i} = \frac{\int_V \rho(x) x dV }{\int_V \rho(x) dV}
\]
\textbf{Momento d'Inerzia: }
\[
I_a = mR^2 = \sum_i m_i R_i^2 = \int_V \rho (P) R^2(P) dV    
\]
Per un corpo piano:
\[
    I_z = I_x + I_y \text{ e l'asse z è sempre il principale}    
\]
\textbf{Teorema di Huygens: }\\
Con $\bar{\mu}$ passante per il baricentro
\[
    I_\mu = I_{\bar{\mu}}+ m \Delta^2 
\]
\textbf{Quantità di Moto:} 
\[
\underline{Q} = m \underline{v} = \sum_i m_i \underline{v}_i = \int_V \rho \underline{v} dV    
\]
Con $G$ centro di Massa:
\[
\underline{Q} = m \cdot \underline{v}(G)    
\]
\textbf{Momento delle quantità di Moto:}
\[
\underline{K}_A = m \underline{v}(G) \wedge \underline{GA} = \sum_i \underline{AP}_i \wedge m_i\underline{v}_i = \int_V \underline(AP) \wedge\rho \underline{v} dV
\]
Con $G$ centro di massa:
\[
   K_A =  m \vec{v}(G) \wedge GA \implies K_G \equiv 0
\]
Con moto rotatorio, A è il C.I.R. :
\[
    K_A = I_A \cdot \omega 
\]
\textbf{Legge del Cambiamento di Polo:}
\[
    \underline{K}_B = \underline{K}_A + \underline{Q}\wedge\underline{AB }
\]
Moto rototraslatorio, Q polo qualunque, G centro di massa:
\[
   K_Q = I_G \cdot \omega + mv(G) \wedge GQ
\]
\textbf{Energia Cinetica:}
\[
T = \frac{1}{2} m v^2 = \frac{1}{2} \sum m_i v_i^2 = \frac{1}{2} \int _V \rho(P) v^2(P) dV = \frac{1}{2} m v^2 + \frac{1}{2} I_\omega \omega^2
\]
\textbf{Teorema di K\"onig:}
\[
T_{ass} = \frac{1}{2}mv^2 + T_{rel} 
\]
\textbf{Energia Potenziale:}
\[
U = mgh - \frac{1}{2} k s^2    
\]
\textbf{Equazioni cardinali della Dinamica:}
\[
\vec{R}^{Est}=\frac{dQ}{dt}= m\cdot a(G) \quad \vec{M}_A = \frac{d}{dt}\vec{K}_A+\dot{A}\wedge Q
\]
\textbf{Potenza delle forze per Corpo Rigido:}
\[
\Pi = \sum-i \vec{f}_i \cdot \vec{v}_i = \vec{R}\cdot\vec{v}(Q)+\vec{M}_Q\cdot \omega   
\]
\textbf{Teorema dell'Energia Cinetica:}\\
\[
    \frac{d}{dt}T = \Pi^{tutte}
\]
Con vincoli ideali, bilateri e fissi:
\[
\frac{d}{dt}T=\Pi^{attive}    
\]
\textbf{Teorema di Conservazione dell'Energia Meccanica:}\\
Con vincoli ideali, bilateri e fissi, forze conservative
\[
  E=T-U=costante \text{(serve spesso una c.l. in t=0 )} \iff \Delta T - \Delta U = 0  
\]
\textbf{Teorema dell'Energia Cinetica per un Corpo Rigido:}
\[
    \frac{d}{dt}T = \Pi^{Est}
\]
\textbf{Equazioni di Langrange:}
\[
    L=T+U \longrightarrow \frac{d}{dt}\left(\frac{\partial L}{\partial \dot{q}_k}\right) - \frac{\partial L}{\partial q_k}=0
\]
\textbf{Momento Cinetico e Integrale del Moto:}
\[
    P_i = \frac{\partial L}{\partial \dot{q}_i} \text{ è detto momento cinetico} 
\]
Per Lagrange se:
\begin{align*}
&\frac{\partial L}{\partial q_k}=0 \text{ è detto 'Interale del Moto' } \implies \frac{d}{dt}\left(\frac{\partial L}{\partial \dot{q}_k}\right)=0 \implies \frac{\partial L}{\partial \dot{q}_k}=costante \text{  (Utile se in quiete per mettere in connesione due velocità)}\\
&\frac{\partial L}{\partial q_k}=costante \implies \frac{d}{dt}\left(\frac{\partial L}{\partial \dot{q}_k}\right)= costante\text{  (Utile per scoprire accelerazioni)} 
\end{align*}
\textbf{Forza e Potenziale Centrifughi:}
\[
F_C = m \omega^2 R \text{ e } U_C = \frac{1}{2} I\cdot \omega    
\]

















\end{document}
