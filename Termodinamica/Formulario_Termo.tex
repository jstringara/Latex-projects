%==== classe del documento ====
\documentclass[a4paper]{report}

%==== pacchetti ====
\usepackage[T1]{fontenc} % codifica dei font per l'italiano
\usepackage[utf8]{inputenc} % lettere accentate da tastiera
\usepackage[italian]{babel} % lingua del documento
\usepackage{framed}% per riquadrare il testo
\usepackage{amsmath}% per avere le formule matematiche fatte bene
\usepackage{amssymb}% per avere le formule matematiche fatte bene
\usepackage[big]{layaureo} %per avere i margini più stretti
\usepackage{graphicx} %per inserire le immagini
\usepackage{mathtools}
\usepackage{mathabx}
\usepackage{empheq}
\usepackage{tikz}
\usepackage{dirtytalk}
\usepackage{gensymb}


%==== new commands ====
\newcommand*\widefbox[1]{\fbox{\hspace{2em}#1\hspace{2em}}}


%==== path delle immagini ====
\graphicspath{ {./images/} }


%==== inizio documento ====
\begin{document}


%==== titolo ====
\begin{center}
\section*{Formulario di Termodinamica}
\end{center}   


%==== valore costanti fisiche ====
\subsection*{Costanti fisiche notevoli}
\[
   R = 8.314
\]

%==== Equazione di stato gai ideali ====
\subsection*{Equazione di stato dei gas ideali}
\[
   \boxed{PV=nRT} \iff \boxed{PV=mR^*T} \iff \boxed{Pv=R^*T} \text{ con } R^*=\frac{R}{M}
\]

%==== primo principio ====
\subsection*{Primo principio: Conservazione dell'Energia}
\[
    \boxed{\Delta U = Q^\leftarrow - L^\rightarrow} \iff \boxed{du = \delta q^\leftarrow - \delta l^\rightarrow}
\]
\textbf{Nota Bene: } $\Delta U_{ciclo} \equiv 0$

%==== Secondo principio ====
\subsection*{Secondo principio: Bilancio Entropico}
\begin{align*}
    \Aboxed{\Delta S  = & S_Q^\leftarrow + S_{IRR}} \longrightarrow \left\{\begin{array}{l}
        S_{IRR}\geq 0 \longrightarrow \left\{\begin{array}{l}
            S_{IRR} = 0 \text{ se reversibile}\\
            S_{IRR} > 0 \text{ se irreversibile}
        \end{array}\right.\\
        S_Q^\leftarrow \gtreqless 0 \longrightarrow \left\{\begin{array}{l}
            S_Q =  0 \iff Q=0\\
            S_Q > 0 \iff Q^\leftarrow \\
            S_Q < 0 \iff Q^\rightarrow
        \end{array}\right.
    \end{array}\right. \\
    &\downarrow \\
    & S_Q^\leftarrow = \int \frac{dQ^\leftarrow_{REV}}{T}
\end{align*}
\textbf{N.B.} $\rightsquigarrow$ Sia $U$ che $S$ sono grandezze estensive:
\begin{align*}
    U_{Sistema} &= \sum U_{Sottosistemi} \\
    S_{Sistema} &= \sum S_{Sottosistemi}
\end{align*}

%==== Lavoro ====
\subsection*{Lavoro}
\[
  \boxed{\delta L = PdV} \longrightarrow \boxed{L = \int_i^f PdV} \longrightarrow \left\{\begin{array}{l}
      L>0 \rightarrow \text{ Ciclo Diretto}\\
      L<0 \rightarrow \text{ Ciclo Inverso}
  \end{array}\right.  
\]

%==== Capacità Termica ====
\subsection*{Capacità Termica}
\[
  \boxed{c_x = \left(\frac{\delta Q^\leftarrow}{dT}\right)_x}
\]

%==== Calore Specifico ====
\subsection*{Calore Specifico}
\[
  \boxed{c = \frac{c_x}{m}=\left(\frac{\delta q^\leftarrow}{dT}\right)_x} \longrightarrow \left\{\begin{array}{l}
    \boxed{c_V = \left(\frac{\partial u}{\partial T}\right)_V }\text{ a V costante}\\
    \boxed{c_P = \left(\frac{\partial h}{\partial T}\right)_P }\text{ a P costante}    
  \end{array}\right.  
\]

%==== Entalpia ====
\subsection*{Entalpia}
\[
  \boxed{h = u + Pv} \iff \boxed{H = U+PV}
\]

%==== Gas Ideali ====
\subsection*{Gas Ideali}
Per Gas Ideali possiamo suppore $c_v=c_v(T)$ e $c_p=c_p(T)$ e costanti per intervalli limitati di T.
\begin{itemize}
    \item \textbf{GP mononatomico}: $\boxed{c_v = \frac{35}{2}R^*, \quad c_p = \frac{5}{2}R^*}$
    \item \textbf{GP biatomico (o poliatomico lineare)}: $\boxed{c_v = \frac{5}{2}R^*, \quad c_p = \frac{7}{2}R^*}$
    \item \textbf{GP poliatomico non lineare}: $\boxed{c_v = \frac{6}{2}R^*, \quad c_p = \frac{8}{2}R^*}$
\end{itemize}


%==== Calore ====
\subsection*{Calore}
\[
    Q = \Delta U + L \xrightarrow[L=\int P dV]{\Delta U = mc_v\Delta T} \boxed{mc_v\Delta T + \int P dV}
\]

%==== Politropiche ====
\subsection*{Politropiche}
Categoriziamo le trasformazioni al variare di $\boxed{n = \frac{c_x-c_p}{c_x-c_v} }$ :
\[
    \boxed{PV^n=\text{ costante}},\boxed{TV^{n-1}=\text{ costante}}, \boxed{PT^{\frac{n}{n-1}}=\text{ costante}}
\]
\begin{align*}
    & \rightarrow : \text{ isoterma } PV= \text{ cost}\rightarrow n=1 \\
    & \rightarrow : \text{ isocora } V= \text{ cost}\rightarrow n=\pm \infty \quad (c_x=c_v)\\
    & \rightarrow : \text{ isobara } P= \text{ cost}\rightarrow n= 0 \quad (c_x=c_p)\\
    & \rightarrow : \text{ adiabatica } PV^n= \text{ cost}\rightarrow n> 1, \quad n=k=\frac{c_p}{c_v}
\end{align*}

\begin{align*}
    \rightsquigarrow & n \neq 1 \implies \boxed{ L^\rightarrow = \frac{P_1V_1}{n-1}\left[1-\left(\frac{V_1}{V_2}\right)^{n-1}\right]=\frac{P_1V_1}{n-1}\left[1-\left(\frac{P_1}{P_2}\right)^{n-1}\right]}\\
    & n = 1 \implies \boxed{L^\rightarrow = P_1V_1\ln\left(\frac{V_2}{V_1}\right)=P_1V_1\ln\left(\frac{P_1}{P_2}\right)=R^*T\ln\left(\frac{V_2}{V_1}\right)}\\
\end{align*}
\begin{align*}
    \rightsquigarrow \Delta S & = s_2-s_1 = c_v \ln\left(\frac{T_2}{T_1}\right)\\
    \Delta S & = s_2-s_1 = c_p \ln\left(\frac{T_2}{T_1}\right)\\
    \Delta S & = s_2-s_1 = c_v \ln\left(\frac{P_2}{P_1}\right)
\end{align*}


%==== Macchina Termica ====
\subsection*{Macchina Termica}
\begin{align*}
    \textbf{Macchina Motrice:} &\left\{\begin{array}{l}
        \square\left\{\begin{array}{l}
            \text{Sistemi Chiusi}\\
            \text{Serbatoi a T costante}\\
            \text{Macchina Ideale }\rightarrow S_{IRR}
        \end{array}\right]\implies\boxed{\eta=\frac{L}{Q}}=\boxed{1-\frac{Q_F}{Q_C}}=\boxed{1-\frac{T_F}{T_C}}\\
        \square\left\{\begin{array}{l}
            \text{Sistemi Chiusi}\\
            \text{Serbatoi a T costante}\\
            \text{Macchina Non Ideale }
        \end{array}\right]\implies \boxed{\eta = 1-\frac{T_F}{T_C}-\frac{T_F\cdot S_{IRR}}{Q_C}}
    \end{array}\right.\\
    \textbf{Macchina Operatrice}&\left\{\begin{array}{l}
        \square\left\{\begin{array}{l}
            \text{Sistemi Chiusi}\\
            \text{Serbatoi a T costante}\\
            \text{100\% Reversibile }
        \end{array}\right]\implies\left\{\begin{array}{l}
            \overset{\text{Pompa di Calore}}{\boxed{COP_{PDC}=\frac{Q_C}{L}=\frac{T_C}{T_C-T_F}}}\\
            \overset{\text{Macchina Frigorifera}}{\boxed{COP_F=\frac{Q_F}{L}=\frac{T_F}{T_C-T_F}}}
        \end{array}\right.\\
        \square\left\{\begin{array}{l}
            \text{Sistemi Chiusi}\\
            \text{Serbatoi a T costante}\\
            \text{Macchina Non Ideale }
        \end{array}\right]\implies \left\{\begin{array}{l}
            \boxed{COP_{PDC}=\frac{T_C}{T_C-T_F+\frac{T_C\cdot T_F \cdot S_{IRR}}{Q_C}}}\\
            \boxed{COP_F = \frac{T_F}{T_C-T_F+\frac{T_CT_FS_{IRR}}{Q_F}}}
        \end{array}\right.
    \end{array}\right.
\end{align*}
\textbf{N.B.} $\boxed{COP_{PDC}=COP_F+1}$ e $\boxed{COP_{PDC}>1}$

%==== Sistemi Aperti ====
\subsection*{Sistemi Aperti}
\subsubsection*{Bilancio di Massa}
\[
    \frac{dm}{dt}=\sum_i \dot{m}_i \text{ con }(\dot{m}=\overset{\overset{\text{Densità}}{\uparrow}}{\rho} \cdot \underset{\underset{\text{Velocità}}{\downarrow}}{w} \cdot \overset{\overset{Sezione}{\uparrow}}{\Omega})
\]
\subsubsection*{Bilancio di Energia}
\[
    \frac{dE}{dt}=\sum_i \dot{E}^{\leftarrow}_i=\sum_i \dot{m}_i\left(h+gz+\frac{w^2}{2}\right)+\dot{Q}^{\leftarrow}-\dot{L}^{\rightarrow}
\]
\subsubsection*{Bilancio di Entropia}
\[
    \frac{dS}{dt}=\sum_i \dot{S}_i= \sum_i \dot{m}_is_i+\dot{S}_Q+\dot{S}_{IRR}
\]
\subsubsection*{Regime Stazionario}
I bilanci diventano:
\begin{itemize}
    \item $\boxed{\dot{m}_i^\leftarrow = -\dot{m}_u^\leftarrow}$
    \item $\boxed{\dot{m}\left[h_i-h_u+g(z_i-z_u)+\frac{w_i^2}{2}-\frac{w_u^2}{2}\right] + \dot{Q}^\leftarrow - \dot{L}^\rightarrow = 0}\xrightarrow[dw=0]{dz=0}\boxed{\dot{m}\Delta h = \dot{Q}-\dot{L}}$
    \item $\boxed{\dot{m}(s_i-s_u)+\dot{S}_Q+\dot{S}_{IRR}=0}\xrightarrow[dw=0]{dz=0}\boxed{\dot{m}\Delta S = \dot{S}_Q+\dot{S}_{IRR}}$
\end{itemize}


%==== Macchine Aperte ====
\subsection*{Macchine Aperte}
Adiabatico + $dz=0$ + $dw=0$:
\begin{align*}
    &\text{Turbina }\longrightarrow \text{ Motrice } (\dot{L}>0)\\
    &\left.\begin{array}{l}
        \text{Compressore}\\
        \text{Pompa}
    \end{array} \right\}\longrightarrow \text{ Operatrici }(\dot{L}<0)
\end{align*}


%==== Scambiatore di Calore ====
\subsection*{Scambiatore di Calore}
Nessuno scambio di $\dot{L}$ + $dz=0$ + $dw=0$


%==== Diffusore e Ugello ====
\subsection*{Diffusore e Ugello}
Nessuno scambio di $\dot{L}$ + $dz=0$ + $dw=0$ e:
\begin{itemize}
    \item \textbf{Diffusore}$\longrightarrow (w\downarrow)$
    \item  \textbf{Ugello}$\longrightarrow (w\uparrow)$
\end{itemize}


%==== Valvola di Laminazione ====
\subsection*{Valvola di Laminazione}
Nessuno scambio di $\dot{L}$ + $dz=0$ + $dw=0$


%==== Rendimento Isentropico Macchina Aperta ====
\subsection*{Rendimento Isentropico Macchina Aperta}
\begin{align*}
    &\boxed{\eta_{\begin{tiny}\begin{array}{l}
        \text{Turbina}\\
        \text{Is}
    \end{array}\end{tiny}} = \frac{\dot{L}_{Reale}}{\dot{L}_{Ideale}} = \frac{h_1-h_2'}{h_1-h_2} }\\
    &\boxed{\eta_{\begin{tiny}\begin{array}{l}
        \text{Compr/Pompa}\\
        \text{Is}
    \end{array}\end{tiny}} = \frac{\dot{L}_{Ideale}}{\dot{L}_{Reale}} = \frac{h_1-h_2}{h_1-h_2'} }
\end{align*}


%==== Lavoro per Sistema Aperto ====
\subsection*{Lavoro per Sistema Aperto}
\[
    \boxed{L_{REV} = -\int v dP} \iff \boxed{\dot{L}_{REV} = -\int V dP}
\]


%==== Rendimento di Secondo Principio ====
\subsection*{Rendimento di Secondo Principio}
\[
    \eta_{II\degree P} = \frac{\eta_{Reale}}{\eta_{Ideale}}    
\]


%==== Sistemi Eterogenei ====
\subsection*{Sistemi Eterogenei $\longrightarrow$ Bifase}
\[
    \boxed{m=m_\alpha+m_\beta}, \quad \boxed{E=E_\alpha+E_\beta}, \quad \boxed{\rho = \frac{m_\alpha \rho_\alpha + m_\beta \rho_\beta}{m_\alpha + m_\beta}}
\]
\subsubsection*{Frazione Massica}
\[
    \boxed{\chi_\alpha=\frac{m_\alpha}{m}} \rightsquigarrow \boxed{ \chi_\alpha + \chi_\beta = 1}
\]
\subsubsection*{Transizione di Fase}
Succede a \textbf{Pressione Costante} $\implies\boxed{dh= \delta q}$
\subsubsection*{Fasi}
\[
    \left\{ \begin{array}{l}
        \text{Liquido sottoraffreddato}\\
        \text{Liquido saturo} \rightarrow\text{ prossimo ad evaporare}
        \text{Miscela satura liquido-vapore}\\
        \text{Vapore saturo}\rightarrow\text{ prossimo ad evaporare}\\
        \text{Vapore surriscaldato}
    \end{array}\right.   
\]
\subsubsection*{Sistemi Liquido-Vapore}
\[
    \text{Titolo di Vapore} = \boxed{x_V=\frac{m_V}{m}}, \quad \text{Titolo di Liquido} = \boxed{x_L = 1 - x_V}
\]
Scegliamo $x=x_V$:
\[
    \implies \left\{\begin{array}[]{l}
        \boxed{ u = x\cdot u_V + (1-x)u_L}\\
        \boxed{ s = x\cdot s_V + (1-x)s_L}\\
        \boxed{ h = x\cdot h_V + (1-x)h_L}\\
        \boxed{ v = x\cdot v_V + (1-x)v_L}
    \end{array}
    \right.
\]


\subsection*{Miscele di Gas Ideali}
\[
    \text{Frazione Massica: }\boxed{m_m = \sum_i m_i} \rightarrow \boxed{x_i = \frac{m_i}{m_m}}, \quad \text{Frazione Molare: }\boxed{N_m=\sum_i N_i}\rightarrow\boxed{y_i = \frac{N_i}{N_m}}
\]
\subsubsection*{Formule}
\begin{itemize}
    \item \textbf{Legge di Dalton:} $P_m = \sum_i P_i(T_m, V_m)$
    \item \textbf{Legge di Amagat:} $V_m = \sum_i V_i(T_m, P_m)$
    \item \textbf{Notiamo:} $\left\{\begin{array}{l}
    P_i = y_iP_m \\
    V_i = y_iV_m
    \end{array}\right.$
    \item \textbf{Inoltre:} $\phi \in \{s,u,h,c_v,c_p\} \quad \phi_m = \sum_i (x_i\cdot\phi_i )$
    \item $\boxed{\Delta s_i = c_p\ln\left(\frac{T_{2i}}{T_{1i}}\right)-R^*\ln\left(\frac{P_{2i}}{P_1i}\right)}$
\end{itemize}


%==== Aria Umida ====
\subsection*{Aria Umida}
Miscela ideale: aria secca + vapor d'acqua, considero vapore e aria come \textbf{Gas Perfetti}
\subsubsection*{Umidità Assoluta}
\[
   \boxed{x = \frac{m_v}{m_{as}} = 0.622 \cdot \frac{P_V}{P-P_V}}
\]
\subsubsection*{Umidità Relativa}
\[
   \boxed{UR = \frac{m_v}{m_{Vsat@T}} = \frac{P_v}{P_{Vsat@T}}}
\]
\textbf{Notiamo:}
\[
    \boxed{x = \frac{0,622 \cdot UR \cdot P_{Vsat@T}}{P-UR\cdot P_{Vsat@T}} } \quad \boxed{UR=\frac{x\cdot P}{(0,622+x)P_{Vsat@T}} }
\]
\subsubsection{Entalpia}
\[
    \boxed{H=H_A+H_V}\xrightarrow{\text{divido per } M_A}\boxed{h=h_A+xh_V}
\]
\textbf{Inoltre:}
\begin{itemize}
    \item Sotto la curva di Saturazione abbia un comportamento \textbf{Isotermobarico}
    \item \textbf{Temperatura di Rugiada:} T alla quale l'aria raffreddata a T costante inizia a condensare
\end{itemize}


%==== Exergia ====
\subsection*{Exergia}
\begin{align*}
    &\boxed{Ex_Q = Q(1-\frac{T_\infty}{T})} \intertext{ (Lavoro massimo ottenibile da Q alla temperatura T mediante una macchina ciclica che evolve reversibilmente fino a $T_\infty$)}\\
    &\boxed{Ex_{QC} = L_{Reale}+T_\infty \Delta S_{IRR} + Ex_{QF}}\\
    &\boxed{\eta_{Ex}=\frac{L_{Reale}}{Ex_{QC}-Ex_{QF}} = \frac{L_{Reale}}{L_{Reale}+T_\infty \Delta S_{IRR}} }\\
    &\boxed{\Pi_{Ex}=1-\eta_{Ex}} \say{Perdita Exergetica}
\end{align*}


%==== Formalismo Termodinamico ====
\subsection*{Formalismo Termodinamico}
\subsubsection*{Potenziali Termodinamici}
\[
    \left\{\begin{array}{l}
        \text{Helmotz }= u-Ts\\
        \text{Entalpia }= u+Pv\\
        \text{Gibbs }= u-Ts+Pv
    \end{array}\right.
\]
\begin{itemize}
    \item Equazione Fondamentale $\longrightarrow\boxed{du=Tds-Pdv}$
    \item Entalpia $\longrightarrow\boxed{dh=Tds+vdP}$
    \item Funzione di Helmotz $\longrightarrow\boxed{df=-sdT-Pdv}$
    \item Funzione di Gibbs $\longrightarrow\boxed{dg=-sdT+vdP}$
\end{itemize}
\subsubsection*{Relazioni}


\tikzset{every picture/.style={line width=0.75pt}} %set default line width to 0.75pt        

\begin{tikzpicture}[x=0.75pt,y=0.75pt,yscale=-1,xscale=1]
%uncomment if require: \path (0,300); %set diagram left start at 0, and has height of 300

%Shape: Ellipse [id:dp39490302622802265] 
\draw   (44.95,39.76) .. controls (44.95,33.72) and (49.87,28.82) .. (55.95,28.82) .. controls (62.02,28.82) and (66.94,33.72) .. (66.94,39.76) .. controls (66.94,45.81) and (62.02,50.7) .. (55.95,50.7) .. controls (49.87,50.7) and (44.95,45.81) .. (44.95,39.76) -- cycle ;

%Shape: Ellipse [id:dp8073887245703708] 
\draw   (44.95,135.27) .. controls (44.95,129.23) and (49.87,124.33) .. (55.95,124.33) .. controls (62.02,124.33) and (66.94,129.23) .. (66.94,135.27) .. controls (66.94,141.31) and (62.02,146.21) .. (55.95,146.21) .. controls (49.87,146.21) and (44.95,141.31) .. (44.95,135.27) -- cycle ;

%Shape: Ellipse [id:dp19338323486890663] 
\draw   (140.02,40.36) .. controls (140.02,34.31) and (144.94,29.41) .. (151.01,29.41) .. controls (157.08,29.41) and (162,34.31) .. (162,40.36) .. controls (162,46.4) and (157.08,51.3) .. (151.01,51.3) .. controls (144.94,51.3) and (140.02,46.4) .. (140.02,40.36) -- cycle ;

%Shape: Ellipse [id:dp27538174359337075] 
\draw   (140.02,135.27) .. controls (140.02,129.23) and (144.94,124.33) .. (151.01,124.33) .. controls (157.08,124.33) and (162,129.23) .. (162,135.27) .. controls (162,141.31) and (157.08,146.21) .. (151.01,146.21) .. controls (144.94,146.21) and (140.02,141.31) .. (140.02,135.27) -- cycle ;

%Straight Lines [id:da931179762374658] 
\draw    (66.64,39.76) -- (139.72,39.76) ;
%Straight Lines [id:da7652264392332884] 
\draw    (151.01,51.3) -- (151.01,124.03) ;
%Straight Lines [id:da9318568422642115] 
\draw    (67.09,135.27) -- (140.17,135.27) ;
%Straight Lines [id:da9552237810027575] 
\draw    (55.95,51.15) -- (55.95,123.89) ;
%Straight Lines [id:da6508537217049795] 
\draw    (143.58,127.58) -- (65.98,50.34) ;
\draw [shift={(64.56,48.93)}, rotate = 404.87] [color={rgb, 255:red, 0; green, 0; blue, 0 }  ][line width=0.75]    (10.93,-3.29) .. controls (6.95,-1.4) and (3.31,-0.3) .. (0,0) .. controls (3.31,0.3) and (6.95,1.4) .. (10.93,3.29)   ;
%Straight Lines [id:da8826738241497376] 
\draw    (63.37,127.58) -- (140.98,50.34) ;
\draw [shift={(142.39,48.93)}, rotate = 495.13] [color={rgb, 255:red, 0; green, 0; blue, 0 }  ][line width=0.75]    (10.93,-3.29) .. controls (6.95,-1.4) and (3.31,-0.3) .. (0,0) .. controls (3.31,0.3) and (6.95,1.4) .. (10.93,3.29)   ;


% Text Node
\draw (49.95,31.26) node [anchor=north west][inner sep=0.75pt]   [align=left] {S};
% Text Node
\draw (50.45,126.77) node [anchor=north west][inner sep=0.75pt]   [align=left] {T};
% Text Node
\draw (145.01,126.77) node [anchor=north west][inner sep=0.75pt]   [align=left] {V};
% Text Node
\draw (145.01,31.86) node [anchor=north west][inner sep=0.75pt]   [align=left] {P};
% Text Node
\draw (97.18,21.81) node [anchor=north west][inner sep=0.75pt]    {$h$};
% Text Node
\draw (152.73,80.07) node [anchor=north west][inner sep=0.75pt]    {$g$};
% Text Node
\draw (97.63,134.18) node [anchor=north west][inner sep=0.75pt]    {$f$};
% Text Node
\draw (42.82,79.92) node [anchor=north west][inner sep=0.75pt]    {$u$};


\end{tikzpicture}
\subsubsection*{Coefficienti}
\begin{itemize}
    \item Calore specifico a P costante $\longrightarrow c_P = T \left(\frac{\partial S}{\partial T}\right)_P$
    \item Calore specifico a V costante $\longrightarrow c_V= T \left(\frac{\partial S}{\partial T}\right)_V$
    \item Coefficiente di dilatazione Isobaro $\longrightarrow \beta_V = \frac{1}{V}\left(\frac{\partial V}{\partial T}\right)_P$
    \item Coefficiente di comprimibilità Isotermo $\longrightarrow k_T= -\frac{1}{V} \left(\frac{\partial V}{\partial P}\right)_T$
    \item Coefficiente di comprimibilità Isoentropico $\longrightarrow k_S= -\frac{1}{V} \left(\frac{\partial V}{\partial P}\right)_S$
\end{itemize}
\textbf{Note:}\\ 
Coefficiente di Joule-Thompson $\rightarrow\boxed{\mu = \frac{v(\beta_v\cdot T-1)}{c_P}}$ e $\boxed{k=\frac{c_P}{c_V}=\frac{k_T}{k_S}}$



































\end{document}
